\biblebook{Physician}

PHYSICIAN, HEAL THYSELF!

Psychiatrist and surgeon, he had lost his way until

he realized that God, not he, was the Great Healer.

       I AM A PHYSICIAN, licensed to practice in a western state. I am also an alcoholic. In two ways I may be a little different from other alcoholics. First, we all hear at A.A. meetings about those who have lost everything, those who have been in jail, those who have been in prison, those who have lost their families, those who have lost their income. I never lost any of it. I never was on skid row. I made more money the last year of my drinking than I ever made before in my whole life. My wife never hinted that she would leave me. Everything that I touched from grammar school on was successful. I was president of my grammar school student body. I was president of all my classes in high school and in my last year I was president of that student body. I was president of each class in the University, and president of that student body. I was voted the man most likely to succeed. The same thing occurred in medical school. I belong to more medical societies and honor societies than men ten to twenty years my senior.

Mine was the skid row of success. The physical skid row in any city is miserable. The skid row of success is just as miserable.

The second way in which, perhaps, I differ from

some other alcoholics is this; many alcoholics state that they didn’t particularly like the taste of alcohol, but that they liked the effect. I loved alcohol! I used to like to get it on my fingers so I could lick them and get another taste. I had a lot of fun drinking. I enjoyed it immensely. And then one ill-defined day, one day that I can’t recall, I stepped across the line that alcoholics know so well, and from that day on drinking was miserable. When a few drinks made me feel good before I went over that line, those same drinks now made me wretched. In an attempt to get over that feeling, there was a quick onslaught of a greater number of drinks, and then all was lost. Alcohol failed to serve the purpose.

On the last day I was drinking I went up to see a friend who had had a good deal of trouble with alcohol, and whose wife had left him a number of times. He had come back, however, and he was on this program. In my stupid way I went up to see him with the idea in the back of my mind that I would investigate Alcoholics Anonymous from a medical standpoint. Deep in my heart was the feeling that maybe I could get some help here. This friend gave me a pamphlet, and I took it home and had my wife read it to me. There were two sentences in it that struck me. One said, “Don’t feel that you are a martyr because you stopped drinking,” and this hit me between the eyes. The second one said, “Don’t feel that you stop drinking for anyone other than yourself,” and this hit me between the eyes. After my wife had read this to me, I said to her, as I had said many times in desperation, “I have got to do something.” She’s a good-natured soul and said, “I wouldn’t worry about it; probably

something will happen.” And then we went up the side of a hill where we have a little barbecue area to make the fire for the barbecue, and on the way up, I thought to myself—I’ll go back down to the kitchen and refill this drink. And just then something did happen.

The thought came to me— This is the last one! I was well into the second fifth by this time. And as that thought came to me, it was as though someone had reached down and taken a heavy overcoat off my shoulders, for that was the last one.

About two days later I was called by a friend of mine from Nevada City—he’s a brother of my wife’s closest friend. He said, “Earl?” and I said, “Yes.” He said, “I’m an alcoholic, what do I do?” And I gave him some idea of what you do, and so I made my first Twelfth Step call before I ever came into the program. The satisfaction I got from giving him a little of what I had read in those pamphlets far surpassed any feeling that I had ever had before in helping patients.

So I decided that I would go to my first meeting. I was introduced as a psychiatrist. (I belong to the American Psychiatric Society, but I don’t practice psychiatry as such. I am a surgeon.)

As someone in A.A. said to me once upon a time, there is nothing worse than a confused psychiatrist.

I will never forget the first meeting that I attended. There were five people present, including myself. At one end of the table sat our community butcher. At the other side of the table sat one of the carpenters in our community, and at the further end of the table sat the man who ran the bakery, while on one side sat my friend who was a mechanic. I recall, as I walked into

that meeting, saying to myself, “Here I am, a Fellow of the American College of Surgeons, a Fellow of the International College of Surgeons, a diplomat of one of the great specialty boards in these United States, a member of the American Psychiatric Society, and I have to go to the butcher, the baker and the carpenter to help make a man out of me!”

Something else happened to me. This was such a new thought that I got all sorts of books on Higher Powers, and I put a Bible by my bedside, and I put a Bible in my car. It is still there. And I put a Bible in my locker at the hospital. And I put a Bible in my desk. And I put a Big Book by my night stand, and I put a “Twelve Steps and Twelve Traditions” in my locker at the hospital, and I got books by Emmet Fox, and I got books by God-knows-who, and I got to reading all these things. And the first thing you know I was lifted right out of the A.A. group, and I floated higher, and higher, and even higher, until I was way up on a pink cloud which is known as Pink Seven, and I felt miserable again. So I thought to myself, I might just as well be drunk as feel like this.

I went to Clark, the community butcher, and I said, “Clark, what is the matter with me? I don’t feel right. I have been on this program for three months and I feel terrible.” And he said, “Earl, why don’t you come on over and let me talk to you for a minute.” So he got me a cup of coffee, and a piece of cake, and sat me down and said, “Why, there’s nothing wrong with you. You’ve been sober for three months, been working hard. You’ve been doing all right.” But then he said, “Let me say something to you. We have here in this community an organization which helps people,

and this organization is known as Alcoholics Anonymous. Why don’t you join it?” I said, “What do you think I’ve been doing?” “Well,” he said, “you’ve been sober, but you’ve been floating way up on a cloud somewhere. Why don’t you go home and get the Big Book and open it at page seventy and see what it says?” So I did. I got the Big Book and I read it, and this is what it said: “Rarely have we seen a person fail who has thoroughly followed our path.” The word “thoroughly” rang a bell. And then it went on to say: “Half measures availed us nothing. We stood at the turning point.” And the last sentence was “We asked His protection and care with complete abandon.”

“Complete abandon”; “Half measures availed us nothing”; “Thoroughly follow our path”; “Completely give oneself to this simple program”—rang in my swelled head.

In 1935, as a physician, I went into psychoanalysis to get relief. I spent five and a half years in psychoanalysis and proceeded to become a drunk. I don’t mean that in any sense as a derogatory statement about psychotherapy; it’s a very great tool, not too potent, but a great tool. I would do it again.

I tried every gimmick that there was to get some peace of mind, but it was not until I was brought to my alcoholic knees, when I was brought to a group in my own community with the butcher, the baker, the carpenter and the mechanic, who were able to give me the Twelve Steps, that I was finally given some semblance of an answer to the last half of the First Step. So, after taking the first half of the First Step, and very gingerly admitting myself to Alcoholics Anonymous, something happened. And then I thought to

myself: “Imagine an alcoholic admitting anything!” But I made my admission just the same.

The Third Step said: “Made a decision to turn our will and our lives over to the care of God as we understood Him.” Now they asked us to make a decision! We’ve got to turn the whole business over to some joker we can’t even see! And this chokes the alcoholic. Here he is powerless, unmanageable, in the grip of something bigger than he is, and he’s got to turn the whole business over to someone else! It fills the alcoholic with rage. We are great people. We can handle anything. And so one gets to thinking to oneself, “Who is this God? Who is this fellow we are supposed to turn everything over to? What can He do for us that we can’t do for ourselves?” Well, I don’t know who He is, but I’ve got my own idea.

For myself, I have an absolute proof of the existence of God. I was sitting in my office one time after I had operated on a woman. It was a long four or five hour operation, a large surgical procedure, and she was on her ninth or tenth post-operative day. She was doing fine, she was up and around, and that day her husband phoned me and said, “Doctor, thanks very much for curing my wife,” and I thanked him for his felicitations, and he hung up. And then I scratched my head, and said to myself, “What a fantastic thing for a man to say, that I cured his wife. Here I am down at my office behind my desk and there she is out at the hospital. I am not even there, and if I was there the only thing I could do would be to give her moral support, and yet he thanks me for curing his wife.” I thought to myself—what is curing that woman? Yes, I put in those stitches. The Great Boss had given me

diagnostic and surgical talent, and He has loaned it to me to use for the rest of my life. It doesn’t belong to me. He has loaned it to me and I did my job, but that ended nine days ago. What healed those tissues, those tissues that I closed, what healed them? I didn’t. This to me is the proof of the existence of a Somethingness greater than I am. I couldn’t practice medicine without the Great Physician. All I do in a very simple way, is to help Him cure my patients.

Shortly after I was starting to work on the program I realized that I was not a good father; I wasn’t a good husband, but, oh, I was a good provider. I never robbed my family of anything. I gave them everything, except the greatest thing in the world, and that is peace of mind. So I went to my wife and asked her, wasn’t there something that she and I could do to somehow get together, and she turned on her heel and looked me squarely in the eye, and said, “You don’t care anything about my problem,” and I could have smacked her, but I said to myself, “Grab on to your serenity!”

She left, and I sat down and crossed my hands, and looked up and said, “For God’s sake, help me.” And then a silly, simple thought came to me. I didn’t know anything about being a father; I don’t know how to come home and work week-ends like other husbands; I don’t know how to entertain my family. But I remembered that every night after dinner my wife would get up and do the dishes. Well, I could do the dishes. So I went to her and said, “There’s only one thing I want in my whole life, and I don’t want any commendation; I don’t want any credit; I don’t want anything from you or Janey for the rest of your life

except one thing; and that is, the opportunity to do anything you want always, and I would like to start off by doing the dishes.” And now I am doing the darn dishes every night!

Doctors have been notoriously unsuccessful in helping alcoholics. They have contributed fantastic amounts of time and work to our problem, but they aren’t able, it seems, to arrest either your alcoholism or mine.

And the clergy have tried hard to help us, but we haven’t been helped. And the psychiatrist has had thousands of couches, and has put you and me on them many, many times, but he hasn’t helped us very much, though he has tried hard; and we owe the clergy and the doctor and the psychiatrist a deep debt of gratitude, but they haven’t helped our alcoholism, except in a rare few instances. But—Alcoholics Anonymous has helped.

What is this power that A.A. possesses? This curative power? I don’t know what it is. I suppose the doctor might say, “This is psychosomatic medicine.” I suppose the psychiatrist might say, “This is benevolent interpersonal relations.” I suppose others would say, “This is group psychotherapy.”

To me it is God.


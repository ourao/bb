\biblebook{Women}


\begin{biblechapter}
\verseWithHeading{WOMEN SUFFER TOO}

Despite great opportunities, alcohol nearly ended

her life. Early member, she spread the word among

women in our pioneering period.

      WHAT WAS I saying . . . From far away, as if in a delirium, I had heard my own voice—calling someone “Dorothy,” talking of shops, of jobs . . . the words came clearer . . . this sound of my own voice frightened me as it got closer . . . and suddenly, there I was, talking of I knew not what, to someone I’d never seen before this very moment. Abruptly I stopped speaking. Where was I?

I’d waked up in strange rooms before, fully dressed on a bed or a couch; I’d waked up in my own room, in or on my own bed, not knowing what hour or day it was, afraid to ask . . . but this was different. This time I seemed to be already awake, sitting upright in a big easy chair, in the middle of an animated conversation with a perfectly strange young woman, who didn’t appear to think it strange. She was chatting on, pleasantly and comfortably.

Terrified, I looked around. I was in a large, dark, rather poorly furnished room—the living room of a basement flat. Cold chills started chasing up and down my spine; my teeth were chattering; my hands were shaking so I tucked them under to keep them from flying away. My fright was real enough, but it

didn’t account for these violent reactions. I knew what they were, all right—a drink would fix them. It must have been a long time since I had my last drink—but I didn’t dare ask this stranger for one. I must get out of here. In any case I must get out of here before I let slip my abysmal ignorance of how I came to be here, and she realized that I was stark, staring mad. I was mad—I must be.

The shakes grew worse and I looked at my watch—six o’clock. It had been one o’clock when I last remembered looking. I’d been sitting comfortably in a restaurant with Rita, drinking my sixth martini and hoping the waiter would forget about the lunch order—at least long enough for me to have a couple more. I’d only had two with her, but I’d managed four in the fifteen minutes I’d waited for her, and of course I’d had the usual uncounted swigs from the bottle as I painfully got up and did my slow spasmodic dressing. In fact I had been in very good shape at one o’clock—feeling no pain. What could have happened? That had been in the center of New York, on noisy 42nd Street . . . this was obviously a quiet residential section. Why had “Dorothy” brought me here? Who was she? How had I met her? I had no answers, and I dared not ask. She gave no sign of recognizing anything wrong, but what had I been doing for those lost five hours? My brained whirled. I might have done terrible things, and I wouldn’t even know it!

Somehow I got out of there and walked five blocks past brownstone houses. There wasn’t a bar in sight, but I found the subway station. The name on it was unfamiliar and I had to ask the way to Grand Central. It took three-quarters of an hour and two changes to

get there—back to my starting point. I had been in the remote reaches of Brooklyn.

That night I got very drunk, which was usual, but I remembered everything, which was very unusual. I remembered going through what my sister assured me was my nightly procedure of trying to find Willie Seabrook’s name in the telephone book. I remembered my loud resolution to find him and ask him to help me get into that “Asylum” he had written about. I remembered asserting that I was going to do something about this, that I couldn’t go on . . . I remembered looking longingly at the window as an easier solution, and shuddering at the memory of that other window, three years before, and the six agonizing months in a London hospital ward. I remembered filling the Peroxide bottle in my medicine chest with gin, in case my sister found the bottle I hid under the mattress. And I remembered the creeping horror of the interminable night, in which I slept for short spells and woke dripping with cold sweat and shaken with utter despair, to drink hastily from my bottle and mercifully pass out again, “You’re mad, you’re mad, you’re mad!” pounded through my brain with each returning ray of consciousness, and I drowned the refrain with drink.

That went on for two more months before I landed in a hospital and started my slow fight back to normalcy. It had been going on like that for over a year. I was thirty-two years old.

When I look back on that last horrible year of constant drinking I wonder how I survived it either physically or mentally. For there were of course periods of clear realization of what I had become,

attended by memories of what I had been, what I had expected to be. And the contrast was pretty shattering. Sitting in a Second Avenue bar, accepting drinks from anyone who offered, after my small stake was gone; or sitting at home alone, with the inevitable glass in my hand, I would remember, and remembering, I would drink faster, seeking speedy oblivion. It was hard to reconcile this ghastly present with the simple facts of the past.

My family had money—I had never known denial of any material desire. The best boarding schools and a finishing school in Europe had fitted me for the conventional role of debutante and young matron. The times in which I grew up (the Prohibition era immortalized by Scott Fitzgerald and John Held Jr.) had taught me to be gay with the gayest; my own inner urges led me to outdo them all. The year after coming out, I married. So far, so good—all according to plan, like thousands of others. But then the story became my own. My husband was an alcoholic—I had only contempt for those without my own amazing capacity—the outcome was inevitable. My divorce coincided with my father’s bankruptcy, and I went to work, casting off all allegiances and responsibilites to any other than myself. For me, work was only a different means to the same end, to be able to do exactly what I wanted to do.

For the next ten years I did just that. For greater freedom and excitement I went abroad to live. I had my own business, successful enough for me to indulge most of my desires. I met all the peple I wanted to meet; I saw all the places I wanted to see; I did all the things I wanted to do—and I was increasingly miser-

able. Headstrong and willful, I rushed from pleasure to pleasure, and found the returns diminishing to the vanishing point. Hangovers began to assume monstrous proportions and the morning drink became an urgent necessity. “Blanks” were more frequent, and I seldom knew how I’d got home. When my friends suggested that I was drinking too much—they were no longer my friends. I moved from group to group—then from place to place—and went on drinking. With a creeping insidiousness, drink had become more important than anything else. It no longer gave me pleasure—it merely dulled the pain—but I had to have it. I was bitterly unhappy. No doubt I had been an exile too long—I should go home to America. I did. And to my surprise, my drinking grew worse.

When I entered a sanitarium for prolonged and intensive psychiatric treatment, I was convinced that I was having a serious mental breakdown. I wanted help, and I tried to cooperate. As the treatment progressed I began to get a picture of myself, of the temperament that had caused me so much trouble. I had been hypersensitive, shy, idealistic. My inability to accept the harsh realities of life had resulted in a disillusioned cynic, clothed in a protective armor against the world’s misunderstanding. That armor had turned into prison walls, locking me in loneliness—and fear. All I had left was an iron determination to live my own life in spite of the alien world—and here I was, an inwardly frightened, outwardly defiant woman, who desperately needed a prop to keep going.

Alcohol was that prop, and I didn’t see how I could live without it. When My doctor told me I should never touch a drink again, I couldn’t afford to believe

him. I had to persist in my attempts to get straightened out enough to be able to use the drinks I needed, without their turning on me. Besides, how could he understand? He wasn’t a drinking man, he didn’t know what it was to need a drink, nor what a drink could do for one in a pinch. I wanted to live, not in a desert, but in a normal world; and my idea of a normal world was among people who drank—teetotallers were not included. And I was sure that I couldn’t be with people who drank, without drinking. In that I was correct: I couldn’t be comfortable with any kind of people without drinking. I never had been.

Naturally, inspite of my good intentions, in spite of my protected life behind sanitarium walls, I several times got drunk, and was astounded . . . and badly shaken.

That was the point at which my doctor gave me the book “Alcoholics Anonymous” to read. The first chapters were a revelation to me. I wasn’t the only person in the world who felt and behaved like this! I wasn’t mad or vicious—I was a sick person. I was suffering from an actual disease that had a name and symptoms like diabetes or cancer or TB—and a disease was respectable, not a moral stigma! But then I hit a snag. I couldn’t stomach religion, and I didn’t like the mention of God or any of the other capital letters. If that was the way out, it wasn’t for me. I was an intellectual and I needed an intellectual answer, not an emotional one. I told my doctor so in no uncertain terms. I wanted to learn to stand on my own two feet, not to change one prop for another, and an intangible and dubious one at that. And so on and on, for several weeks, while I grudgingly plowed through some more

of the offending book, and felt more and more hopeless about myself.

Then the miracle happened—to me! It isn’t always so sudden with everyone, but I ran into a personal crisis which filled me with a raging and righteous anger. And as I fumed helplessly and planned to get good and drunk and show them, my eye caught a sentence in the book lying open on my bed: “We cannot live with anger.” The walls crumpled—and the light streamed in. I wasn’t trapped. I wasn’t helpless. I was free, and I didn’t have to drink to “show them.” This wasn’t “religion”—this was freedom! Freedom from anger and fear, freedom to know happiness and love.

I went to a meeting to see for myself this group of freaks or bums who had done this thing. To go into a gathering of people was the sort of thing that all my life, from the time I left my private world of books and dreams to meet the real world of people and parties and jobs, had left me feeling an uncomfortable outsider, needing the warming stimulus of drinks to join in. I went trembling into a house in Brooklyn filled with strangers . . . and I found I had come home at last, to my own kind. There is another meaning for the Hebrew word that in the King James version of the Bible is translated “salvation.” It is: “to come home.” I had found my salvation. I wasn’t alone any more.

That was the beginning of a new life, a fuller life, a happier life than I had ever known or believed possible. I had found friends, understanding friends who often knew what I was thinking and feeling better than I knew myself, and didn’t allow me to retreat into

my prison of loneliness and fear over a fancied slight or hurt. Talking things over with them, great floods of enlightenment showed me myself as I really was and I was like them. We all had hundreds of character traits, of fears and phobias, likes and dislikes, in common. Suddenly I could accept myself, faults and all, as I was—for weren’t we all like that? And, accepting, I felt a new inner comfort, and the willingness and strength to do something about the traits I couldn’t live with.

It didn’t stop there. They knew what to do about those black abysses that yawned ready to swallow me when I felt depressed, or nervous. There was a concrete program, designed to secure the greatest possible inner security for us long-time escapists. The feeling of impending disaster that had haunted me for years began to dissolve as I put into practice more and more of the Twelve Steps. It worked!

An active member of A.A. since 1939, I feel myself a useful member of the human race at last. I have something to contribute to humanity, since I am peculiarly qualified, as a fellow-sufferer, to give aid and comfort to those who have stumbled and fallen over this business of meeting life. I get my greatest thrill of accomplishment from the knowledge that I have played a part in the new happiness achieved by countless others like myself. The fact that I can work again and earn my living, is important, but secondary. I believe that my once over-weening self-will has finally found its proper place, for I can say many times daily, “Thy will be done, not mine” . . . and mean it.

\end{biblechapter}


\biblebook{Concepts}

%\bbChapterPreamble % not from 2nd ed??
\bbHeadingRaw{THE TWELVE CONCEPTS (SHORT FORM) append. \romNum{7}}


\begin{biblechapter}
\verseWithHeading{History}
    A.A.’s Twelve Steps are principles for personal recovery.
\verse The Twelve Traditions ensure the unity of the Fellowship.
\verse Written by co-founder Bill W. in 1962, 
    the Twelve Concepts for World Service 
    provide a group of related principles 
    to help ensure that various elements of A.A.’s service structure 
    remain responsive and responsible to those they serve.
The “short form” of the Concepts, which follows, was approved by the 1971 General Service Conference.
\end{biblechapter}


\begin{biblechapter}
\verseWithHeading{Concepts}
I. Final responsibility 
    and ultimate authority 
    for A.A. world services 
    should always reside in the collective conscience 
    of our whole Fellowship.

\verse II.The General Service Conference of A.A. has become,
    for nearly every practical purpose, 
    the active voice and the effective conscience 
    of our whole Society in its world affairs.

\verse III. To insure effective leadership, 
    we should endow each element of A.A.—the 
    Conference, the General Service Board 
    and its service corporations, staffs, committees,
    and executives—with a traditional “Right of Decision.”

\verse IV. At all responsible levels, 
    we ought to maintain a traditional “Right of Participation,” 
    allowing a voting representation in reasonable proportion 
    to the responsibility that each must discharge.

\verse V. Throughout our structure, 
    a traditional “Right of Appeal” ought to prevail, 
    so that minority opinion will be heard 
    and personal grievances receive careful consideration.

\verse VI. The Conference recognizes that the chief initiative 
    and active responsibility in most world service matters
    should be exercised by the trustee members of the Conference 
    acting as the General Service Board.

\verse VII. The Charter and Bylaws of the General Service Board 
    are legal instruments, 
    empowering the trustees to manage and conduct world service affairs.
    The Conference Charter is not a legal document; 
    it relies upon tradition and the A.A. purse for final effectiveness.

\verse VIII. The trustees are the principal planners and administrators 
    of overall policy and finance.
    They have custodial oversight of the separately incorporated 
    and constantly active services, 
    exercising this through their ability 
    to elect all the directors of these entities.

\verse IX. Good service leadership at all levels is indispensable 
    for our future functioning and safety.
    Primary world service leadership, 
    once exercised by the founders, 
    must necessarily be assumed by the trustees.

\verse X. Every service responsibility 
    should be matched by an equal service authority, 
    with the scope of such authority well defined.

\verse XI. The trustees should always have the best possible committees, 
    corporate service directors, executives, staffs, and consultants.
    Composition, qualifications, induction procedures, 
    and rights and duties will always be matters of serious concern.

\verse XII. The Conference shall observe the spirit of A.A. tradition,
    taking care that it never becomes 
    the seat of perilous wealth or power;
    that sufficient operating funds and reserve 
    be its prudent financial principle;
    that it place none of its members in a position 
    of unqualified authority over others; 
    that it reach all important decisions by 
    discussion, vote, and, whenever possible, by substantial unanimity;
    that its actions never be personally punitive 
    nor an incitement to public controversy;
    that it never perform acts of government, and that, 
    like the Society it serves, 
    it will always remain democratic in thought and action.
\end{biblechapter}


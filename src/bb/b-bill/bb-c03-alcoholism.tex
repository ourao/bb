
\biblebook{Alcoholism}

\bbHeading{BB. MORE ABOUT ALCOHOLISM p. 30-43}
\bbChapterPreamble


\begin{biblechapter}
\verseWithHeading{Unwilling}
    MOST OF us have been unwilling to admit we were real alcoholics. 
\verse No person likes to think he is bodily and mentally 
    different from his fellows.  
\verse Therefore, it is not surprising that our drinking careers 
    have been characterized by countless vain attempts 
    to prove we could drink like other people.  
\verse The idea that somehow, someday 
    he will control and enjoy his drinking 
    is the great obsession of every abnormal drinker.  
\verse The persistence of this illusion is astonishing.  
\verse Many pursue it into the gates of insanity or death.

\verse We learned that we had to fully concede to our innermost selves 
    that we were alcoholics. 
\verse This is the first step in recovery. 
\verse The delusion that we are like other people, 
    or presently may be, has to be smashed.
\end{biblechapter}


\begin{biblechapter}
\verseWithHeading{A Progressive Illness}
    We alcoholics are men and women 
    who have lost the ability to control our drinking. 
\verse We know that no real alcoholic ever recovers control. 
\verse All of us felt at times that we were regaining control, 
    but such intervals-usually brief-were inevitably followed 
    by still less control, 
    which led in time to pitiful and incomprehensible demoralization. 
\verse We are convinced to a man that alcoholics of our type 
    are in the grip of a progressive illness. 
\verse Over any considerable period we get worse, never better.

\verse We are like men who have lost their legs; they never grow new ones. 
\verse Neither does there appear to be any kind of treatment which will make alcoholics of our kind like other men. 
\verse We have tried every imaginable remedy. 
\verse In some instances there has been brief recovery, followed always by a still worse relapse. 
\verse Physicians who are familiar with alcoholism 
    agree there is no such thing as making 
    a normal drinker out of an alcoholic. 
\verse Science may one day accomplish this, but it hasn't done so yet.
\end{biblechapter}


\begin{biblechapter}
\verseWithHeading{Failure of The Will}
    Despite all we can say, 
    many who are real alcoholics 
    are not going to believe they are in that class. 
\verse By every form of self-deception and experimentation, 
    they will try to prove themselves exceptions to the rule, 
    therefore nonalcoholic. 
\verse If anyone who is showing inability to control his drinking 
    can do the right-about-face and drink like a gentleman, 
    our hats are off to him. 
\verse Heaven knows, we have tried hard enough and long enough 
    to drink like other people!

\verse Here are some of the methods we have tried: 
\verse Drinking beer only, 
    limiting the number of drinks, 
    never drinking alone, 
    never drinking in the morning, 
    drinking only at home, 
    never having it in the house, 
\verse never drinking during business hours, 
    drinking only at parties, 
    switching from scotch to brandy, 
    drinking only natural wines, 
\verse agreeing to resign if ever drunk on the job, 
    taking a trip, 
    not taking a trip, 
    swearing off forever (with and without a solemn oath), 
\verse taking more physical exercise, 
    reading inspirational books, 
    going to health farms and sanitariums, 
    accepting voluntary commitment to asylums 
    -we could increase the list ad infinitum. 
\end{biblechapter}


\begin{biblechapter}
\verseWithHeading{Drinking Experiments}
    We do not like to pronounce any individual as an alcoholic, 
    but you can quickly diagnose yourself. 
\verse Step over to the nearest bar-room and try some controlled drinking. 
\verse Try to drink and stop abruptly. 
\verse Try it more than once. 
\verse It will not take long for you to decide, 
    if you are honest with yourself about it. 
\verse It may be worth a bad case of jitters 
    if you get a full knowledge of your condition.
\verse Though there is no way of proving it, 
\verse we believe that early in our drinking careers most of us could have stopped drinking. 
\verse But the difficulty is that few alcoholics have enough desire to stop while there is yet time. 
\verse We have heard of a few instances where people, who showed definite signs of alcoholism, 
\verse were able to stop for a long period because of an overpowering desire to do so. 
\verse Here is one.
\end{biblechapter}


\begin{biblechapter}
\verseWithHeading{A Tragic Story}
    A man of thirty was doing a great deal of spree drinking. 
\verse He was very nervous in the morning after these bouts 
    and quieted himself with more liquor. 
\verse He was ambitious to succeed in business, 
    but saw that he would get nowhere if he drank at all. 
\verse Once he started, he had no control whatever. 
\verse He made up his mind that 
    until he had been successful in business and had retired, 
    he would not touch another drop. 
\verse An exceptional man, he remained bone dry for twenty-five years 
    and retired at the age of fifty-five, 
    after a successful and happy business career. 
\verse Then he fell victim to a belief which practically every alcoholic has
    -that his long period of sobriety and self-discipline 
    had qualified him to drink as other men. 
\verse Out came his carpet slippers and a bottle. 
\verse In two months he was in a hospital, puzzled and humiliated. 
\verse He tried to regulate his drinking for a while, 
\verse making several trips to the hospital meantime. 
\verse Then, gathering all his forces, 
    he attempted to stop altogether and found he could not. 
\verse Every means of solving his problem which money could buy was at his disposal. 
\verse Every attempt failed. 
\verse Though a robust man at retirement, 
\verse he went to pieces quickly and was dead within four years.

\verse This case contains a powerful lesson. 
\verse Most of us have believed 
    that if we remained sober for a long stretch, 
    we could thereafter drink normally. 
\verse But here is a man who at fifty-five years 
    found he was just where he had left off at thirty. 
    We have seen the truth demonstrated again and again: 
\verse \emph{"Once an alcoholic, always an alcoholic."}
\verse Commencing to drink after a period of sobriety, 
    we are in a short time as bad as ever. 
\verse If we are planning to stop drinking, 
    there must be no reservation of any kind, 
    nor any lurking notion that someday we will be immune to alcohol.

\verse Young people may be encouraged by this man's experience 
    to think that they can stop, 
    as he did, on their own will power. 
\verse We doubt if many of them can do it, 
    because none will really want to stop, 
    and hardly one of them, 
    because of the peculiar mental twist already acquired, 
    will find he can win out. 
\verse Several of our crowd, men of thirty or less, 
    had been drinking only a few years, 
    but they found themselves as helpless as those 
    who had been drinking twenty years.
\end{biblechapter}


\begin{biblechapter}
\verseWithHeading{Gravely}
    To be gravely affected, 
    one does not necessarily have to drink a long time 
    nor take the quantities some of us have. 
\verse This is particularly true of women. 
\verse Potential female alcoholics often turn into the real thing 
    and are gone beyond recall in a few years. 
\verse Certain drinkers, 
    who would be greatly insulted if called alcoholics, 
    are astonished at their inability to stop. 
\verse We, who are familiar with the symptoms, 
    see large numbers of potential alcoholics 
    among young people everywhere. 
\verse But try and get them to see it!  

\verse As we look back, 
    we feel we had gone on drinking many years beyond the point 
    where we could quit on our will power. 
\verse If anyone questions whether he has entered this dangerous area, 
    let him try leaving liquor alone for one year. 
\verse If he is a real alcoholic and very far advanced, 
    there is scant chance of success. 
\verse In the early days of our drinking 
    we occasionally remained sober for a year or more, 
    becoming serious drinkers again later. 
\verse Though you may be able to stop for a considerable period, 
    you may yet be a potential alcoholic. 
\verse We think few, to whom this book will appeal, 
    can stay dry anything like a year. 
\verse Some will be drunk the day after making their resolutions; 
    most of them within a few weeks.
\end{biblechapter}


\begin{biblechapter}
\verseWithHeading{Relapse}
    For those who are unable to drink moderately 
    the question is how to stop altogether. 
\verse We are assuming, of course, that the reader desires to stop. 
\verse Whether such a person can quit upon a nonspiritual basis 
    depends upon the extent to which 
    he has already lost the power to choose 
    whether he will drink or not. 
\verse Many of us felt that we had plenty of character. 
\verse There was a tremendous urge to cease forever. 
\verse Yet we found it impossible. 
\verse This is the baffling feature of alcoholism as we know it-this 
    utter inability to leave it alone, 
    no matter how great the necessity or the wish.

\verse How then shall we help our readers determine, 
    to their own satisfaction, 
    whether they are one of us? 
\verse The experiment of quitting for a period of time will be helpful, 
    but we think we can render an even greater service 
    to alcoholic sufferers 
    and perhaps to the medical fraternity. 
\verse So we shall describe some of the mental states that 
    precede a relapse into drinking, 
    for obviously this is the crux of the problem.

\verse What sort of thinking dominates an alcoholic who 
    repeats time after time the desperate experiment of the first drink? 
\verse Friends who have reasoned with him after a spree which 
    has brought him to the point of divorce or bankruptcy 
    are mystified when he walks directly into a saloon. 
\verse Why does he? 
\verse Of what is he thinking?
\end{biblechapter}


\begin{biblechapter}
\verseWithHeading{Jim}
    Our first example is a friend we shall call Jim. 
\verse This man has a charming wife and family. 
\verse He inherited a lucrative automobile agency. 
\verse He had a commendable World War record. 
\verse He is a good salesman. 
\verse Everybody likes him. 
\verse He is an intelligent man, normal so far as we can see, 
    except for a nervous disposition. 
\verse He did no drinking until he was thirty-five. 
\verse In a few years he became so violent when intoxicated 
    that he had to be committed. 
\verse On leaving the asylum he came into contact with us.

\verse We told him what we knew of alcoholism and the answer we had found. 
\verse He made a beginning. 
\verse His family was re-assembled, 
    and he began to work as a salesman 
    for the business he had lost through drinking. 
\verse All went well for a time, 
    but he failed to enlarge his spiritual life. 
\verse To his consternation, 
    he found himself drunk half a dozen times in rapid succession. 
\verse On each of these occasions we worked with him, 
    reviewing carefully what had happened. 
\verse He agreed he was a real alcoholic and in a serious condition. 
\verse He knew he faced another trip to the asylum if he kept on. 
\verse Moreover, he would lose his family for whom he had a deep affection.

\verse Yet he got drunk again. 
\verse We asked him to tell us exactly how it happened. 
\verse This is his story:

\emph{
    "I came to work on Tuesday morning. 
\verse I remember I felt irritated 
    that I had to be a salesman for a concern I once owned. 
\verse I had a few words with the boss, but nothing serious. 
\verse Then I decided to drive into the country 
    and see one of my prospects for a car. 
\verse On the way I felt hungry 
    so I stopped at a roadside place where they have a bar. 
\verse I had no intention of drinking. 
\verse I just thought I would get a sandwich. 
\verse I also had the notion that 
    I might find a customer for a car at this place, 
    which was familiar for I had been going to it for years. 
\verse I had eaten there many times during the months I was sober. 
\verse I sat down at a table and ordered a sandwich and a glass of milk. 
\verse Still no thought of drinking. 
\verse I ordered another sandwich and decided to have another glass of milk."
}

\emph{
\verse"Suddenly the thought crossed my mind 
    that if I were to put an ounce of whiskey in my milk 
    it couldn't hurt me on a full stomach. 
\verse I ordered a whiskey and poured it into the milk. 
\verse I vaguely sensed I was not being any too smart, 
    but felt reassured as I was taking the whiskey on a full stomach. 
\verse The experiment went so well that I ordered another whiskey 
    and poured it into more milk. 
\verse That didn't seem to bother me so I tried another."
}

\verse Thus started one more journey to the asylum for Jim. 
\verse Here was the threat of commitment, 
    the loss of family and position, 
    to say nothing of that intense mental and physical suffering 
    which drinking always caused him. 
\verse He had much knowledge about himself as an alcoholic. 
\verse Yet all reasons for not drinking were easily pushed aside 
    in favor of the foolish idea that he could take whiskey 
    if only he mixed it with milk!
\end{biblechapter}


\begin{biblechapter}
\verseWithHeading{Insanity}
    Whatever the precise definition of the word may be, 
    we call this plain insanity. 
\verse How can such a lack of proportion, 
    of the ability to think straight, 
    be called anything else?

\verse You may think this an extreme case. 
\verse To us it is not far-fetched, 
    for this kind of thinking 
    has been characteristic of every single one of us. 
\verse We have sometimes reflected more than Jim did upon the consequences. 
\verse But there was always the curious mental phenomenon 
    that parallel with our sound reasoning 
    there inevitably ran some insanely trivial excuse 
    for taking the first drink. 
\verse Our sound reasoning failed to hold us in check. 
\verse The insane idea won out. 
\verse Next day we would ask ourselves, 
    in all earnestness and sincerity, 
    how it could have happened.

\verse In some circumstances we have gone out deliberately to get drunk, 
    feeling ourselves justified by 
    nervousness, anger, worry, depression, jealousy or the like. 
\verse But even in this type of beginning we are obliged to admit 
    that our justification for a spree was insanely insufficient 
    in the light of what always happened. 
\verse We now see that when we began to drink deliberately, 
    instead of casually, 
    there was little serious 
    or effective thought during the period of premeditation 
    of what the terrific consequences might be.
\end{biblechapter}


\begin{biblechapter}
\verseWithHeading{Jaywalkers}
    Our behavior is as absurd and incomprehensible 
    with respect to the first drink 
    as that of an individual with a passion, say, for jay-walking. 
\verse He gets a thrill out of skipping in front of fast-moving vehicles. 
\verse He enjoys himself for a few years in spite of friendly warnings. 
\verse Up to this point you would label him as a foolish chap 
    having queer ideas of fun. 
\verse Luck then deserts him 
    and he is slightly injured several times in succession. 
\verse You would expect him, if he were normal, to cut it out. 
\verse Presently he is hit again and this time has a fractured skull. 
\verse Within a week after leaving the hospital 
    a fast-moving trolley car breaks his arm. 
\verse He tells you he has decided to stop jay-walking for good, 
    but in a few weeks he breaks both legs.

\verse On through the years this conduct continues, 
    accompanied by his continual promises to be careful 
    or to keep off the streets altogether. 
\verse Finally, he can no longer work, 
    his wife gets a divorce and he is held up to ridicule. 
\verse He tries every known means 
    to get the jay-walking idea out of his head. 
\verse He shuts himself up in an asylum, hoping to mend his ways. 
\verse But the day he comes out he races in front of a fire engine, 
    which breaks his back. 
\verse Such a man would be crazy, wouldn't he?

\verse You may think our illustration is too ridiculous. 
\verse But is it? 
\verse We, who have been through the wringer, 
    have to admit if we substituted alcoholism for jay-walking, 
    the illustration would fit us exactly. 
\verse However intelligent we may have been in other respects, 
    where alcohol has been involved, we have been strangely insane. 
\verse It's strong language-but isn't it true?
\end{biblechapter}


\begin{biblechapter}
\verseWithHeading{Halfway There}
    Some of you are thinking: 
\verse "Yes, what you tell us is true, but it doesn't fully apply. 
\verse We admit we have some of these symptoms, 
    but we have not gone to the extremes you fellows did, 
    nor are we likely to, 
    for we understand ourselves so well after what you have told us 
    that such things cannot happen again. 
\verse We have not lost everything in life through drinking 
    and we certainly do not intend to. 
\verse Thanks for the information."

\verse That may be true of certain nonalcoholic people who, 
    though drinking foolishly and heavily at the present time, 
    are able to stop or moderate, 
    because their brains and bodies have not been damaged as ours were. 
\verse But the actual or potential alcoholic, 
    with hardly an exception, 
    will be absolutely unable to stop drinking 
    on the basis of self-knowledge. 
\verse This is a point we wish to emphasize and re-emphasize, 
    to smash home upon our alcoholic readers 
    as it has been revealed to us out of bitter experience. 
\verse Let us take another illustration.
\end{biblechapter}


\begin{biblechapter}
\verseWithHeading{Fred}
    Fred is partner in a well known accounting firm. 
\verse His income is good, he has a fine home, 
    is happily married and the father of promising children of college age. 
\verse He has so attractive a personality 
    that he makes friends with everyone. 
\verse If ever there was a successful business man, it is Fred. 
\verse To all appearance he is a stable, well balanced individual. 
\verse Yet, he is alcoholic. 
\verse We first saw Fred about a year ago in a hospital 
    where he had gone to recover from a bad case of jitters. 
\verse It was his first experience of this kind, 
    and he was much ashamed of it. 
\verse Far from admitting he was an alcoholic, 
    he told himself he came to the hospital to rest his nerves. 
\verse The doctor intimated strongly 
    that he might be worse than he realized. 
\verse For a few days he was depressed about his condition. 
\verse He made up his mind to quit drinking altogether. 
\verse It never occurred to him that perhaps he could not do so, 
    in spite of his character and standing. 
\verse Fred would not believe himself an alcoholic, 
    much less accept a spiritual remedy for his problem. 
\verse We told him what we knew about alcoholism. 
\verse He was interested and conceded that he had some of the symptoms, 
    but he was a long way from admitting 
    that he could do nothing about it himself. 
\verse He was positive that this humiliating experience, 
    plus the knowledge he had acquired, 
    would keep him sober the rest of his life. 
\verse Self-knowledge would fix it.

\verse We heard no more of Fred for a while. 
\verse One day we were told that he was back in the hospital. 
\verse This time he was quite shaky. 
\verse He soon indicated he was anxious to see us. 
\verse The story he told is most instructive, 
    for here was a chap absolutely convinced he had to stop drinking, 
    who had no excuse for drinking, 
    who exhibited splendid judgment 
    and determination in all his other concerns, 
    yet was flat on his back nevertheless.
\end{biblechapter}


\begin{biblechapter}
\verseWithHeading{Fred's Testimony}
    Let him tell you about it: 

\emph{
\verse "I was much impressed with what you fellows said about alcoholism, 
    and I frankly did not believe 
    it would be possible for me to drink again. 
\verse I rather appreciated your ideas 
    about the subtle insanity which precedes the first drink, 
    but I was confident it could not happen to me after what I had learned. 
\verse I reasoned I was not so far advanced as most of you fellows, 
    that I had been usually successful 
    in licking my other personal problems, 
    and that I would therefore be successful where you men failed. 
\verse I felt I had every right to be self-confident, 
    that it would be only a matter of exercising my will power 
    and keeping on guard.
}

\emph{
\verse "In this frame of mind, 
    I went about my business and for a time all was well. 
\verse I had no trouble refusing drinks, 
    and began to wonder 
    if I had not been making too hard work of a simple matter. 
\verse One day I went to Washington 
    to present some accounting evidence to a government bureau. 
\verse I had been out of town before during this particular dry spell, 
    so there was nothing new about that. 
\verse Physically, I felt fine. 
\verse Neither did I have any pressing problems or worries. 
\verse My business came off well, 
    I was pleased and knew my partners would be too. 
\verse It was the end of a perfect day, not a cloud on the horizon."
}

\emph{
\verse "I went to my hotel and leisurely dressed for dinner. 
\verse As I crossed the threshold of the dining room, 
    the thought came to mind 
    that it would be nice to have a couple of cocktails with dinner. 
\verse That was all. 
\verse Nothing more. 
\verse I ordered a cocktail and my meal. 
\verse Then I ordered another cocktail. 
\verse After dinner I decided to take a walk. 
\verse When I returned to the hotel 
    it struck me a highball would be fine before going to bed, 
    so I stepped into the bar and had one. 
\verse I remember having several more that night and plenty next morning. 
\verse I have a shadowy recollection of being in an airplane 
    bound for New York 
    and of finding a friendly taxicab driver at the landing field 
    instead of my wife. 
\verse The driver escorted me about for several days. 
\verse I know little of where I went or what I said and did. 
\verse Then came the hospital 
    with unbearable mental and physical suffering."
}

\verseWithHeading{Admittance}
    \emph{"As soon as I regained my ability to think, I went carefully over that evening in Washington. 
\verse Not only had I been off guard, I had made no fight whatever against the first drink. 
\verse This time I had not thought of the consequences at all. 
\verse I had commenced to drink as carelessly as though the cocktails were ginger ale. 
\verse I now remembered what my alcoholic friends had told me, 
    how they prophesied that if I had an alcoholic mind, 
    the time and place would come-I would drink again. 
\verse They had said that though I did raise a defense, 
    it would one day give way 
    before some trivial reason for having a drink. 
\verse Well, just that did happen and more, 
    for what I had learned of alcoholism did not occur to me at all. 
\verse I knew from that moment that I had an alcoholic mind. 
\verse I saw that will power and self-knowledge 
    would not help in those strange mental blank spots. 
\verse I had never been able to understand people who said 
    that a problem had them hopelessly defeated. 
\verse I knew then. 
\verse It was a crushing blow."
}

\verseWithHeading{Powerlessness}
    \emph{"Two of the members of Alcoholics Anonymous came to see me. 
\verse They grinned, which I didn't like so much, 
    and then asked me if I thought myself alcoholic 
    and if I were really licked this time. 
\verse I had to concede both propositions. 
\verse They piled on me heaps of evidence to the effect 
    that an alcoholic mentality, 
    such as I had exhibited in Washington, was a hopeless condition. 
\verse They cited cases out of their own experience by the dozen. 
\verse This process snuffed out the last flicker of conviction 
    that I could do the job myself."
}

\verseWithHeading{Turning Over To God}
    \emph{"Then they outlined the spiritual answer 
        and program of action which a hundred of them 
        had followed successfully. 
\verse Though I had been only a nominal churchman, 
    their proposals were not, 
    intellectually, hard to swallow. 
\verse But the program of action, 
    though entirely sensible, 
    was pretty drastic. 
\verse It meant I would have to throw 
    several lifelong conceptions out of the window. 
\verse That was not easy. 
\verse But the moment I made up my mind to go through with the process, 
    I had the curious feeling that my alcoholic condition was relieved, 
    as in fact it proved to be.
}

\verseWithHeading{Radical Transformation}
    \emph{"Quite as important was the discovery that 
        spiritual principles would solve all my problems. 
\verse I have since been brought into a way of living 
    infinitely more satisfying and, 
    I hope, more useful than the life I lived before. 
\verse My old manner of life was by no means a bad one, 
    but I would not exchange its best moments for the worst I have now. 
\verse I would not go back to it even if I could."
}

\verse Fred's story speaks for itself. 
\verse We hope it strikes home to thousands like him. 
\verse He had felt only the first nip of the wringer. 
\verse Most alcoholics have to be pretty badly mangled 
    before they really commence to solve their problems.
\end{biblechapter}


\begin{biblechapter}
\verseWithHeading{Spiritual Experience}
    Many doctors and psychiatrists agree with our conclusions. 
\verse One of these men, 
    staff member of a world-renowned hospital, 
    recently made this statement to some of us: 

\emph{
\verse "What you say about the general hopelessness 
    of the average alcoholic's plight is, in my opinion, correct.  
\verse As to two of you men, whose stories I have heard, 
    there is no doubt in my mind that you were 100\% hopeless, 
    apart from divine help.  
\verse Had you offered yourselves as patients at this hospital, 
    I would not have taken you, 
    if I had been able to avoid it. 
\verse People like you are too heartbreaking. 
\verse Though not a religious person, 
    I have profound respect for the spiritual approach 
    in such cases as yours. 
\verse For most cases, there is virtually no other solution."
}

\verse Once more: 
    The alcoholic at certain times has no effective mental defense 
    against the first drink. 
\verse Except in a few rare cases, 
    neither he nor any other human being can provide such a defense. 
\verse His defense must come from a Higher Power.
\end{biblechapter}


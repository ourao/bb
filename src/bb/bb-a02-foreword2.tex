\biblebook{2 Foreword}

Extract from Alcoholics Anonymous 2nd Ed.,© AA World Services, Inc., 
    used here for scholarly analysis and commentary.
The B.B. 2nd Ed. 'Foreword To Second Edition', p. xv-xxi

\begin{biblechapter}
\verseWithHeading{Second Ed}
    Figures given in this foreword describe the
    Fellowship as it was in 1955.

\verseWithHeading{Miracles}
    Since the original Foreword to this book was written in 1939, 
    a wholesale miracle has taken place. 
\verse Our earliest printing voiced the hope 
    “that every alcoholic who journeys 
    will find the Fellowship of Alcoholics Anonymous at his destination. 
    Already,”
    continues the early text 
    “twos and threes and fives of us have sprung up in other communities.”

\verseWithHeading{Comes Of Age}
\verse Sixteen years have elapsed between our first printing of this book 
    and the presentation in 1955 of our second edition.
\verse In that brief space, 
    Alcoholics Anonymous has mushroomed into nearly 6,000 groups 
    whose membership is far above 150,000 recovered alcoholics.
\verse Groups are to be found in each of the United States 
    and all of the provinces of Canada.
\verse A.A. has flourishing communities 
    in the British Isles, the Scandinavian countries, South Africa, 
    South America, Mexico, Alaska, Australia and Hawaii. 
\verse All told, promising beginnings have been made 
    in some 50 foreign countries and U. S. possessions. 
\verse Some are just now taking shape in Asia. 
\verse Many of our friends encourage us by saying that 
    this is but a beginning, 
    only the augury of a much larger future ahead.

\verseWithHeading{Bill Meets Silkworth} 
    The spark that was to flare into the first A.A. group 
    was struck at Akron, Ohio, in June 1935, 
    during a talk between a New York stockbroker and an Akron physician. 
\verse Six months earlier, 
    the broker had been relieved of his drink obsession 
    by a sudden spiritual experience, %% xv
    following a meeting with an alcoholic friend 
    who had been in contact with the Oxford Groups of that day. 
\verse He had also been greatly helped by 
    the late Dr. William D. Silkworth, 
    a New York specialist in alcoholism who 
    is now accounted no less than a medical saint by A.A. members, 
    and whose story of the early days of our Society 
    appears in the next pages. 
\verse From this doctor, 
    the broker had learned the grave nature of alcoholism. 
\verse Though he could not accept all the tenets of the Oxford Groups, 
    he was convinced of the need for moral inventory, 
    confession of personality defects, 
    restitution to those harmed, 
    helpfulness to others, 
    and the necessity of belief in and dependence upon God.

\verseWithHeading{Bob meets Bill} 
    Prior to his journey to Akron, 
    the broker had worked hard with many alcoholics 
    on the theory that only an alcoholic could help an alcoholic, 
    but he had succeeded only in keeping sober himself. 
\verse The broker had gone to Akron on a business venture 
    which had collapsed, 
    leaving him greatly in fear that he might start drinking again. 
\verse He suddenly realized that in order to save himself 
    he must carry his message to another alcoholic. 
\verse That alcoholic turned out to be the Akron physician.

\verseWithHeading{Radical Willingness}
\verse This physician had repeatedly tried spiritual means 
    to resolve his alcoholic dilemma but had failed. 
\verse But when the broker gave him Dr. Silkworth’s description 
    of alcoholism and its hopelessness, 
    the physician began to pursue the spiritual remedy for his malady 
    with a willingness he had never before been able to muster.
\verse He sobered, never to drink again 
    up to the moment of his death in 1950. 
\verse This seemed to prove that one alcoholic could affect another 
    as no nonalcoholic could. %% xvi
\verse It also indicated that strenuous work, 
    one alcoholic with another, 
    was vital to permanent recovery.

\verseWithHeading{The First Group} 
    Hence the two men set to work almost frantically 
    upon alcoholics arriving in the ward of the Akron City Hospital. 
\verse Their very first case, a desperate one, 
    recovered immediately and became A.A. number three.
\verse He never had another drink.
\verse This work at Akron continued through the summer of 1935.
\verse There were many failures, 
    but there was an occasional heartening success.
\verse When the broker returned to New York in the fall of 1935, 
    the first A.A. group had actually been formed, 
    though no one realized it at the time.


\verseWithHeading{The Fellowship}
\verse A second small group promptly took shape at New York, 
    to be followed in 1937 with the start of a third at Cleveland. 
\verse Besides these, there were scattered alcoholics 
    who had picked up the basic ideas in Akron or New York 
    who were trying to form groups in other cities. 
\verse By late 1937, 
    the number of members having substantial sobriety time behind them 
    was sufficient to convince the membership that 
    a new light had entered the dark world of the alcoholic.

\verseWithHeading{The Big Book}
    It was now time, the struggling groups thought, 
    to place their message and unique experience before the world. 
\verse This determination bore fruit in the spring of 1939 
    by the publication of this volume. 
\verse The membership had then reached about 100 men and women. 
\verse The fledgling society, 
    which had been nameless, 
    now began to be called Alcoholics Anonymous, 
    from the title of its own book. 
\verse The flying-blind period ended 
    and A.A. entered a new phase of its pioneering time.

\verseWithHeading{Approvals & A.A. World Services}
    With the appearance of the new book a great deal began to happen. 
\verse Dr. Harry Emerson Fosdick, 
    the noted clergyman, 
    reviewed it with approval. 
\verse In the fall of 1939 Fulton Oursler, 
    then editor of Liberty, 
    printed a piece in his magazine, 
    called “Alcoholics and God.” 
\verse This brought a rush of 800 frantic inquiries 
    into the little New York office 
    which meanwhile had been established. 
\verse Each inquiry was painstakingly answered; 
    pamphlets and books were sent out. 
\verse Businessmen, traveling out of existing groups, 
    were referred to these prospective newcomers. 
\verse New groups started up and it was found, 
    to the astonishment of everyone, 
    that A.A.’s message could be transmitted in the mail 
    as well as by word of mouth. 
\verse By the end of 1939 it was estimated that 
    800 alcoholics were on their way to recovery.


\verseWithHeading{Growing Publicity}
\verse In the spring of 1940, John D. Rockefeller, Jr. 
    gave a dinner for many of his friends 
    to which he invited A.A. members to tell their stories. 
\verse News of this got on the world wires; 
    inquiries poured in again 
    and many people went to the bookstores to get the book 
    “Alcoholics Anonymous.’’ 
\verse By March 1941 the membership had shot up to 2,000. 
\verse Then Jack Alexander wrote a feature article 
    in the Saturday Evening Post 
    and placed such a compelling picture of A.A. before the general public 
    that alcoholics in need of help really deluged us. 
\verse By the close of 1941, A.A. numbered 8,000 members. 
\verse The mushrooming process was in full swing.
\verse A.A. had become a national institution.

\verseWithHeading{Problems Of Leadership}
\verse Our Society then entered a fearsome and exciting adolescent period. 
\verse The test that it faced was this:
\verse Could these large numbers of erstwhile erratic alcoholics 
    successfully meet and work together? 
\verse Would there be quarrels over membership, leadership, and money? 
\verse Would there be strivings for power and prestige? 
\verse Would there be schisms which would split A.A. apart? 
\verse Soon A.A. was beset by these very problems on every side 
    and in every group. 
\verse But out of this frightening and at first disrupting experience 
    the conviction grew that A.A.’s had to hang together or die separately.
\verse We had to unify our Fellowship or pass off the scene.

\verseWithHeading{Traditions & Concepts}
\verse As we discovered the principles by which 
    the individual alcoholic could live, 
    so we had to evolve principles by which the A.A. groups 
    and A.A. as a whole 
    could survive and function effectively. 
\verse It was thought that no alcoholic man or woman 
    could be excluded from our Society; 
    that our leaders might serve but never govern; 
    that each group was to be autonomous 
    and there was to be no professional class of therapy.
\verse There were to be no fees or dues; 
    our expenses were to be met by our own voluntary contributions.
\verse There was to be the least possible organization, 
    even in our service centers. 
\verse Our public relations were to be based upon attraction 
    rather than promotion. 
\verse It was decided that all members ought to be anonymous 
    at the level of press, radio, TV and films. 
\verse And in no circumstances should we give endorsements, 
    make alliances, 
    or enter public controversies.

\verseWithHeading{First AA Conference}
\verse This was the substance of A.A.’s Twelve Traditions, 
    which are stated in full on page 561 of this book.
\verse Though none of these principles had the force of rules or laws, 
    they had become so widely accepted by 1950 
    that they were confirmed by our first 
    International Conference held at Cleveland. 
\verse Today the remarkable unity of A.A. 
    is one of the greatest assets that our Society has.

\verseWithHeading{Public Acceptance}
    While the internal difficulties 
    of our adolescent period were being ironed out, 
    public acceptance of A.A. grew by leaps and bounds.
\verse For this there were two principal reasons: 
    the large numbers of recoveries, 
    and reunited homes.
\verse These made their impressions everywhere.
\verse Of alcoholics who came to A.A. and really tried, 
    50\% got sober at once and remained that way;
    25\% sobered up after some relapses, 
    and among the remainder, 
    those who stayed on with A.A. showed improvement.
\verse Other thousands came to a few A.A. meetings 
    and at first decided they didn’t want the program.
\verse But great numbers of these—about two out of three—began 
    to return as time passed.

\verseWithHeading{Advocations}
\verse Another reason for the wide acceptance of A.A. 
    was the ministration of friends - 
    friends in medicine, religion, and the press, 
    together with innumerable others 
    who became our able and persistent advocates.
\verse Without such support, 
    A.A. could have made only the slowest progress.
\verse Some of the recommendations of A.A.’s early medical 
    and religious friends 
    will be found further on in this book.

\verseWithHeading{Opinion On Religion} 
    Alcoholics Anonymous is not a religious organization.
\verse Neither does A.A. take any particular medical point of view, 
    though we cooperate widely with the men of medicine 
    as well as with the men of religion.
\verse Alcohol being no respecter of persons, 
    we are an accurate cross section of America, 
    and in distant lands, 
    the same democratic evening-up process is now going on.
\verse By personal religious affiliation, 
    we include Catholics, Protestants, Jews, Hindus, 
    and a sprinkling of Moslems and Buddhists. 
\verse More than 15\% of us are women.

\verseWithHeading{Hope} 
    At present, 
    our membership is pyramiding 
    at the rate of about twenty per cent a year.
\verse So far, 
    upon the total problem of several million actual 
    and potential alcoholics in the world, 
    we have made only a scratch.
\verse In all probability, 
    we shall never be able to touch more than 
    a fair fraction of the alcohol problem 
    in all its ramifications.
\verse Upon therapy for the alcoholic himself, 
    we surely have no monopoly. 
\verse Yet it is our great hope that 
    all those who have as yet found no answer 
    may begin to find one in the pages of this book 
    and will presently join us on the high road to a new freedom.
\end{biblechapter} %% xxi



\biblebook{Shown}


\begin{biblechapter}

HE HAD TO BE SHOWN

“Who is convinced against his will is of the same

opinion still.” But not this man.

      I WAS THE OLDEST of three children, and my father was an alcoholic. One of the earliest memories that I have is of a bottle sitting on his desk with a skull and crossbones and marked “Poison.” At that time, as I remember, he had promised never to take another drink. Of course he did. I can also remember that he was a salesman and a very good one. When he was uptown—we were living in the little town of Moscow—I went up to try to get some money from him to buy groceries. He wouldn’t give me any money for the groceries, but he did take me across the street and buy me a bag of candy, which I later took back and traded for a loaf of bread. I was not more than six at that time.

My father died in 1901 when I was eight years old and I was in the second or third grade at school. I immediately quit school and went to work, and from that time until I was high school age there was never a return to school. I always built up in my own mind the great things that I was going to do, and in fact I accomplished about fifty per cent of them and then lost interest. That continued through my entire life. When I was sixteen years old, my mother remarried and I was given the opportunity of going back to school. I

went into the high school grades, but having missed all the intermediate grades, I didn’t get along too well, so I developed the habit of going back to school just long enough for the football season and then quitting.

There was always a tremendous drive and ambition to become a great guy, because I think I recognized inwardly that I didn’t have any special talents. At a comparatively early age, I can remember being jealous of my brother. He did things much better than I did because he applied himself and learned how to do them, and I never applied myself. Whether I could have done as well as he, I don’t know.

I was married at the age of nineteen to a grand girl and had good business prospects. I had bought a piece of ground in Cuyahoga Falls and cut it into lots and had a profit of approximately $40,000 and that was a lot of money in those days. With that profit, I built a number of houses, but then I neglected them. I wouldn’t put sufficient time on them. Consequently, my labor bills ran up. I lost money, and then just fooled away a large part of the profit.

When I was eighteen, at the end of high school, the high school team had a banquet at a well-known roadhouse outside of Akron. We boys drove out in somebody’s car and went to the bar on the way to the dining room and I, in an effort to impress the other boys that I was city-bred, having lived in Scranton and Cleveland, asked them if they didn’t want to drink. They looked at one another queerly and, finally, one of them allowed he’d have a beer and they all followed him, each of them saying he’d have a beer too. I ordered a martini, extra dry. I didn’t even know what a martini looked like, but I had heard a man down the

bar order one. That was my first drink. I kept watching the man down the bar to see what he did with a contraption like that, and he just smelled of his drink and set it down again, so I did the same. He took a couple of puffs of a cigarette and I took a couple of puffs of my cigarette. He tossed off half of his martini; I tossed off half of mine and it nearly blew the top of my head off. It irritated my nostrils; I choked, I didn’t like it. There was nothing about that drink that I liked. But I watched him, and he tossed off the rest of his, so I tossed off the rest of mine. He ate his olive and I ate mine. I didn’t even like the olive. It was repulsive to me from every standpoint. I drank nine martinis in less than an hour.

Twenty-two years later, Doctor Bob told me that what I had done was like turning a switch and setting up a demand for more alcohol in my system. I didn’t know that then. I had no more reason to drink those martinis than a jackrabbit. At that particular time, the boys put me on a shutter and took me out to the shed, and I lay in the car while they enjoyed their banquet. That was the first time I ever drank hard liquor. Blackout drinking at once. I had no pleasure out of the drinking at all. All of a sudden I found myself guzzling. Right then I determined that never so long as I lived would I have anything more to do with martinis. They acted on me like the beating of a club.

I think it was probably more than a year before I had anything more to do with liquor. I was opening up these lots that I spoke of. I had a crew of men working there and I wanted them to work Sunday afternoons so that I could sell lots on Sunday. I went over and bought a jug of hard cider and a gallon of

wine that I gave these fellows to drink. When they got through the day’s work, part was left which I proceeded to drink. During the day, looking over the contracts and money in my pocket, I found that I had sold six lots that I couldn’t even remember, and didn’t even know the people I had sold them to. I had to look in the telephone book later to find out who these people were. Another blackout. Wine and hard cider.

I early discovered that if I drank anything, I was not accountable for what happened. I decided that I couldn’t drink. Anyhow, I recognized the fact that I couldn’t drink like normal people, but I tried hard and kept on trying for twenty-two years.

I sold three lots to an elderly lady in Cleveland. I came to Cleveland with the deeds to these lots and to pick up my money. She paid me in cash. The next morning I woke up in jail in Cleveland and the jailor had $1,175 of my money in an envelope. I didn’t remember anything that had happened. This was six or eight months after the last drinking episode.

Then I got married. (As I’ve said, I was nineteen). I felt having gotten married, I was an adult and one of the first things I did was to buy two cases of whiskey with no idea of drinking it. (I might say right here that never in my life did I ever intend to get drunk. I never had any desire to get drunk. So I consciously thought). I was a very young married man, having his whiskey in the cupboard over the sink, and when I helped my wife with the dishes at night I would take a cup of tea and spike it with whiskey. I could get through an evening with just a couple of snorts.

This was a regular occurrence for a little while. Eventually, there would be a ball game, or a show, or

some sort of special occasion to celebrate and I would turn up drunk. About that period, too, came increasing procrastination and the avoidance of responsibilities. I would put off doing anything that I could until the next day and, consequently, everything would pile up and then there would be the blackout.

At the end of this selling of lots, just prior to to World War I, I got into the crude rubber business, and six months later there was only one broker and myself left in Akron. So in spite of anything that I might do, I prospered, being one of only two brokers in the rubber center of the world.

I found, however, that when I would leave Akron to go to Chicago, I would get drunk. As long as there was everyday business, I could drink occasionally and didn’t always get drunk. I was a periodic. A big event of any kind precipitated heavy drinking. It had long since become a serious problem. I was prone to do everything on a big scale. I can well remember sitting with seven dollors in my pocket, planning on giving my family a hundred or two hundred dollors, when I made it next year. But I didn’t do a thing about giving them any part of the seven bucks I had in my pocket.

The rubber prosperity went on for about six years—1916 to 1922. It fell apart in the twenties. Every company in the country, except Firestone, was reorganized at that time. I was always able to skate along the fringe of big money. I made a point of knowing important people. I could work a deal up to where all I had to do was to go ahead with it; all the planning had been done, all the financing had been done, but then I’d say, “Nuts to it!” and walk away. Near success, only near. I figured the only difference

between me and a millionaire was that I hadn’t the strength, or that he got the breaks and I didn’t.

Akron was really on the boom in those days, 1919-1920; expansion was terrific. I optioned a piece of land just off East Market Street to put up a three-hundred suite apartment. One hundred for unmarried women at one end, one hundred for men at the other, and one hundred for married couples in the middle. In the basement were to be dry-cleaning facilities, a barbershop, a pool room, a grocery shop and everything. I had contracted for half of it, at least verbally, and the contractors were taking half of the second mortgage bond. At that particular stage, I lost interest in it, sold the option for $5,000 and forgot the whole deal. Another time, I had a rubber pool project. My idea was to have all the companies pool their funds and buy rubber when rubber was cheap and then put it in a pool. When rubber reached a certain low point, they would draw on the rubber out of the pool and buy. With the big companies, and with the amount of money we could have gotten and the promises I had, it could have been done. I worked along until I had really big names in rubber on a tentative contract, and then I neglected to go through with it.

To my mind, drinking didn’t have anything to do with not going through with things. I don’t know whether I drank to cover up being a failure, or whether I drank and then missed the deals. I was able to rationalize it anyway. I can well remember over a long period of years when I thought I was the only person in the world who knew that sooner or later I was going to get drunk. I can remember occasions when friends recommended me for positions or busi-

ness opportunities that I wouldn’t take because I felt that at some future date I’d get drunk and they would be hurt.

In the meantime, the domestic situation was not getting along too well. We had two children, a boy and a girl, and when the boy was about twelve, we broke up the marriage. That was at my suggestion. I can remember telling the poor little soul that I could probably quit drinking if I wasn’t married to her, and told her that, after all, I didn’t like restraint! I didn’t like having to come home at a certain time; I didn’t like this, I didn’t like that, and I think the poor girl acually divorced me to help me stop drinking! Naturally, what little restraint I had exercised before was gone now, and my drinking became worse.

Long since I had come to believe I was insane because I did so many things I didn’t want to do. I didn’t want to neglect my children. I loved them, I think, as much as any parent. But I did neglect them. I didn’t want to get into fights, but I did get into fights. I didn’t want to get arrested, but I did get arrested. I didn’t want to jeopardize the lives of innocent people by driving an automobile while intoxicated, but I did. I quite naturally came to the conclusion that I must be insane. My big job was to keep other people from finding it out. I can remember well thinking that I would quit drinking, really go to work hard, apply myself eight hours a day, five days a week, make a lot of money, and then I could start drinking again. That was the reverse of my former pattern of fearing to go to work because I might drink! Always at the end of these dreams was drinking. Now I attempted to quit. I think this was about 1927. I was divorced in 1925

or 1926. I determined that I wouldn’t drink. I remember one occasion when I did not drink for three hundred and sixty-four days, but I didn’t quite make the year.

Another time, I had gone around to Max R. trying to get a job driving one of his trucks. He had known something of my drinking pattern, and he asked me what I was doing about it. I told him that I had not had a drink for ninety days and that I had come to the conclusion that I was one of those individuals who couldn’t drink. So, knowing that, I had determined that so long as I lived I would never take another drink. On that statement and the fact that I had been sober for ninety days, he gave me a job selling lots in an allotment he had. I was moved in as Sales Manager and had four men working under me. At the end of about four months, I not only had good looking clothes—and I might say that at the time I first talked to Max I didn’t have a suit of clothes I could bend over in real sudden; now I had six suits of clothes. I had an automobile. Everything in the world a man could possibly want, and I was driving from Akron to Cleveland, having just been to the bank and discovered that I had approximately $5,000. I drove towards Cleveland wondering why I found myself in such a changed set of conditions as compared with those of six months before. I came to the conclusion it was because I hadn’t been drunk. And I hadn’t been drunk because I hadn’t taken a drink. And I then and there said a prayer, if you please. An offer of appreciation for not having had a drink for those few months and then and there, without anybody promising me anything or

threatening me, I made a solemn vow that never so long as I live would I take another drink.

(My mother and father were Catholics and I had been baptized, but at the end of my instructions for Confirmation I had not gone to church, and then when my mother remarried, she married a Protestant and the whole religious angle was forgotten. So I had never had any lasting contact with any kind of religion.)

So I was driving to Cleveland when I made this solemn promise never to drink again. That was at three-thirty in the afternoon. At three-thirty the next morning, I was in Champlain Street Station, in jail for driving while intoxicated and insulting an officer; and the suit of which I was so proud was in such shape that the turnkey had to get me a pair of trousers to go into court in the next morning. I had run into a man I always drank with. Whenever this man and I met—I didn’t know his name then nor do I know it now—we would always get drunk. I had run into him, and he looked real prosperous; his face and eyes looked clear and he started to compliment me on my good front and how well I looked, and I said, “I haven’t had a drink for nine months.” He said, “Well, I haven’t had a drink for three months.” And we stood there for twenty minutes, telling each other how much better we were, how much better we looked, how much better off we were financially, mentally, physically, morally, and in every way, shape and manner. And then we both realized we should go. We shook hands, and he hung onto my hand for a moment and said, “Tell you what I’ll do for old times sake. I’ll buy you one drink, and if you suggest a second one, I’ll poke

you right in the nose.” And I think we calculated, or I did, that there wasn’t anybody who knew that I wasn’t drinking. I could take one drink and get right back on the wagon. Nobody would know it, so I agreed to have the one drink. We went into a bootleg joint and I don’t remember leaving the place. I was picked up at two-thirty that morning with my car smashed up by a street-car because I had run into a big concrete safety zone, and the street-car had run into me, and they took me out through the roof—there’s where I lost the suit. I had lost a hundred dollors I had in my pocket, and lost a wristwatch too. I lost the car, of course. But more important, I lost my sobriety. And I continued to drink, on and off then, until every dollor I had in the world was gone again and I was right back living at my sister’s, getting my cigaretts by calling her grocer and telling him to put in a couple of cartons with her order, exactly as I had before I started to work at Max’s.

In 1932, some friends of mine advised me that I might try Christian Science, which had done a lot for some of their friends. So I started to investigate Science through some friends of mine who were Readers in the church. I accepted their help, and it was helpful. I quit drinking immediately. The circumstances under which I reached these people were very odd because I was led there through things that I said when I couldn’t even remember speaking. I told somebody that I was going down to get Christian Science and they took me down, but I don’t remember saying that. Yet I wound up at this place. I attended their meetings every Sunday and Wednesday for about nine months. If there was a lecture on the subject within

a hundred miles of Akron, I attended. Then I started to miss meetings because it was raining or snowing or something else. Pretty soon, I wasn’t going at all, and was avoiding those people who had been so kind to me. I avoided them rather than explain why they weren’t seeing me. My sobriety continued for another six months.

At the end of fifteen months, I tried the beer experiment. After drinking one glass of beer at the end of my work period for about five days, I thought I’d better find out whether I really had the stuff licked. So I didn’t have a beer one night, and as I drove home I was breaking my arm patting myself on the back because I had proved I could lick liquor. I had proved that liquor was not my master. I had avoided a drink this time. So having licked it, there was no reason why I shouldn’t have a drink, and I stopped in before I got home and had one. Then I got into the habit of having beers, and decided that a drink of whiskey was not any worse; so I would get the one drink of whiskey but, on second thought, I decided that as long as I was only going to have one, I might as well make it a double-header. So I had one double-header every night for about two or three weeks. I didn’t drink very long at a time. I think the longest drunk I was ever on was eleven days, but usually only two days with a complete blackout for a day, and then backing off by drinking as long as I could get anything.

This Christian Science experience with a sobriety of fifteen months was in 1932. Then I started drinking again, with possibly a little more restraint, periods a little bit longer than they had been before, but substantially the same pattern. During the latter part of

the Christian Science experience I had gotten a job and was working at Firestone. I was bouncing along and not doing to badly. There were times when I got to drinking, and I had been warned by Firestone that they wouldn’t stand for this much longer, so, clearly, they were conscious of the fact that I drank too much and too often.

To show you the point to which this obsession went, there came an occasion when I had spent a most delightful week-end, and at nine o’clock on Sunday night I was on my way home, and I thought I would get a drink. I went into a bar, and there I got into a fight. I was arrested and taken to jail where I was beaten up by two or three fellows who were already in there and whom I tried to boss. I was badly beaten. I tried to conduct a kangeroo court and hit them with a broomstick. I had a broken nose, a fractured cheekbone, and was black from the lower part of my face up into my hair. I was black and blue, with my lips all swollen, when they roused us to go into court in the morning. I looked so terrible in the court that the judge suggested that I get a continuance and let me sign all the papers to go to a hospital and to a doctor. I went downstairs and there was the grizzly old veteran police officer in charge of the property desk, and as he gave me the stuff, he asked, “Are you going out in the street that way?” I said, “I’m certainly not going to stay here!” I had white trousers on, white shoes and a white shirt that was streaked with blood. He said, “Well, why don’t you take a cab?” I said, “Allright, call me a cab,” as though I was talking to a bellboy. He did call me a cab and when I got into the cab, I said, “Drive me to a liquor store.” We drove to

a store in North Hill and I sent him in with what money I had to get a quart. He brought the quart out and I took a good swig. When I got home I had to give him a check for the taxi fare. I drank some more and slept through the day. At night, I woke up and the folks with whom I roomed were home by then. I offered them a drink, and they came to the bottom of the stairs and I stuck my face around the top of the stairs and the good woman fainted, just looking at me. So they decided that I should have a doctor. They called a doctor and it happened that they called one I knew. He came in and took a look at me and sent me to the hospital.

When I had been in the hospital ten days, Sister Ignatia, who has played such a part in the development of A.A. in this area, stuck her head in the door one morning and announced, looking at me quizzically, that they might be able to make something human out of my face after all. And at the end of fourteen days, they let me out. Three days later I went to work. The next day, they called me in for an examination, and the doctor wouldn’t let me continue working and pardoned me from the plant for ten days because he said my eye had been injured. So I was barred from the plant for ten days, and during that ten day period I was drunk twice, showing how little control these restraints had on me.

Shortly after that, my brother, who had then become associated with a group of people and had stopped drinking, urged me to attend meetings with him. Naturally, I wanted no part of any meetings. I explained to my sister that some of the people he was meeting with had been in hospitals. I couldn’t afford

to be found with those people, but I said I would certainly pay his dues if it would keep him from drinking. But me, I wanted no part of it! I didn’t have any need of such an association!

One morning, after I had been on a binge for a couple days, I awoke to find my brother and Doctor Bob in my room talking to me about not drinking. My only thought that day was getting a drink, and how to get rid of those clowns was my big problem. They asked me if I would take some medicine, and I promised that I would if they got me a drink. So Paul was dispatched and brought back a pint. I got two drinks, each of them a quarter of that pint, in me, and was talking along with these people, but I felt that sooner or later they were going to have me cornered because they were smarter than I was and the drink was beginning to take effect; but as I reached for the third drink Bob said, “Listen, Buster, you promised to take some medicine if we got you a drink. Now we got you the drink, but you haven’t taken the medicine.” I agreed with him and told him in no uncertain terms That I never broke my word in my life. I told him I’d take the medicine and I would take it, but I hadn’t told him when, and thereupon I got away with the third drink. I then began asking a lot of questions of both my brother and Dr. Bob about how this worked, and I suppose I was becoming more glassy-eyed all the while, for eventually I said to Bob, “Your all dried up. You’re never going to want another drink, are you?”; and this answer of his is very important to those of us who are victims of alcoholism. He said, “So long as I’m thinking as I’m thinking now, and so long as I’m doing the things I’m doing now, I don’t

believe I’ll ever take another drink.” And I said, “Well, what about Paul, have you got him all dried up?” He siad Paul would have to answer for himself. So Paul repeated substantially what Dr. Bob had said. And I said, “Now you want to dry me up. I’m not going to want another drink?”  “Well,” the doctor said, “we have hopes in that direction.” I said, “In that case, there’s no use of wasting this,” and I got the last of that pint. A few minutes later, Dr. Bob left, leaving with my brother some medicine I should get. Paul measured the medicine out, but he figured that with my track record that little bit wouldn’t be enough, so he doubled it and added a few drops more and then gave it to me. I immediately became unconscious. This was on Thursday. I regained consciousness on Sunday. I had taken five and a half ounces of paraldehyde. Because it effected so strenuously, they felt that hospitalization was indicated and I awoke in a hospital.

On Sunday when I came to, it was a bad, wet, snowy day in February, 1937, and Paul and Doc and a lot of the other fellows were in Cleveland on business. The people in the group hadn’t been around that day; part of my family was in Florida and the rest of them weren’t speaking to me, so I spent a very lonesome day and by evening I was feeling very sorry for myself. It was getting pretty dark and I hadn’t turned on any lights, when some big fellow stepped inside the door and flipped on the light switch. I said, “Look, Bub, if I want those lights on, I’ll turn them on.” I’ll never forget, he never even hesitated and I had never seen him before in my life. He took off his hat and his overcoat, and he said, “You don’t look very good. How are

you feeling?” I said, “How do you suppose? I’m feeling terrible.” He said, “Maybe you need a little drink.” That was the smartest man I’d met in months. I thought he had it in his pocket, so I said, “You got some?” He said, “No, call the nurse.” And he got me a drink. Then he started to talk to me about his drinking experiences, what his drinking had cost him, how much he had drunk and where, things like that, and I remember I was quite bored because I had never seen the guy before and had no interest at all in what, where and when he drank. The man turned out to be Bill D., a very early member of A.A., and I couldn’t tell you a word of what he said. Not one experience registered with me. When he left, I realized from his story that as a drinker I was just a panty-waist. I knew I could quit because he had quit; he hadn’t had a drink for over a year. The important thing was that he was happy. He was released, relieved from his alcoholism and was happy and contented because of it. That I shall never forget.

The next day, others from the group came in to see me. I remember well one fellow, Joe, walking nearly three miles through slush, wet and snow to come to the hospital to see a man that he had never seen before in his life, and that impressed me very much. He walked to the hospital to save bus fare and did it gladly in order to be helpful to an individual he had never even seen. There were only seven or eight people in the group before me and they all visited me during my period in the hospital. The very simple program they advised me to follow was that I should ask to know God’s will for me for that one day, and then, to the best of my ability, to follow that, and at night to ex-

press my gratefulness to God for the things that had happened to me during the day. When I left the hospital I tried this for a day and it worked, for a week and it worked, and for a month, and it worked—and then for a year and it still worked. It has continued to work now for nearly eighteen years.

\end{biblechapter}

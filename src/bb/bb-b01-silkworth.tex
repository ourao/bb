\biblebook{Silkworth}

Extract from Alcoholics Anonymous 2nd Ed.,© AA World Services, Inc., 
    used here for scholarly analysis and commentary.
The B.B. 2nd Ed. 'The Doctor's Opinion', p. xxiii-xxx


\begin{biblechapter}
\verseWithHeading{Medical Testimony}
    WE OF Alcoholics Anonymous believe that the reader will be interested 
    in the medical estimate of the plan of recovery described in this book. 
\verse Convincing testimony must surely come from medical men 
    who have had experience with the sufferings of our members 
    and have witnessed our return to health. 
\verse A well known doctor, 
    chief physician at a nationally prominent hospital 
    specializing in alcoholic and drug addiction, 
    gave Alcoholics Anonymous this letter:
\end{biblechapter}


\begin{biblechapter}
\verseWithHeading{Silkworth}
    \emph{To Whom It May Concern:

\verse I have specialized in the treatment of alcoholism for many years.

\verseWithHeading{Bill}
    In late 1934 I attended a patient who, 
    though he had been a competent business man of good earning capacity, 
    was an alcoholic of a type I had come to regard as hopeless.

\verse In the course of his third treatment 
    he acquired certain ideas concerning a possible means of recovery. 
\verse As part of his rehabilitation 
    he commenced to present his conceptions to other alcoholics, 
    impressing upon them that they must do likewise with still others. 
\verse This has become the basis of a rapidly growing fellowship 
    of these men and their families. 
\verse This man and over one hundred others appear to have recovered.

\verseWithHeading{Powerless}
    I personally know scores of cases who were of the type 
    with whom other methods had failed completely.

\verseWithHeading{Medical Importance}
    These facts appear to be of extreme medical importance; 
    because of the extraordinary possibilities 
    of rapid growth inherent in this group 
    they may mark a new epoch in the annals of alcoholism. 
\verse These men may well have a remedy for thousands of such situations.

\verse You may rely absolutely on anything they say about themselves.
    %\comment{extreme}

\verse Very truly yours, 

William D. Silkworth, M.D.
}
\end{biblechapter}


\begin{biblechapter}
\verseWithHeading{Statement Request; Disease Model}
    The physician who, at our request, gave us this letter, 
    has been kind enough to enlarge upon his views 
    in another statement which follows.
\verse In this statement he confirms 
    what we who have suffered alcoholic torture must believe-that 
    the body of the alcoholic is quite as abnormal as his mind.
\verse It did not satisfy us to be told 
    that we could not control our drinking 
    just because we were maladjusted to life, 
    that we were in full flight from reality, 
    or were outright mental defectives. 
\verse These things were true to some extent, 
    in fact, to a considerable extent with some of us. 
\verse But we are sure that our bodies were sickened as well. 
\verse In our belief, any picture of the alcoholic 
    which leaves out this physical factor 
    is incomplete.

\verseWithHeading{Allergy}
    The doctor's theory that we have an allergy to alcohol interests us. 
\verse As laymen, our opinion as to its soundness may, 
    of course, mean little. 
\verse But as ex-problem drinkers, 
    we can say that his explanation makes good sense. 
\verse It explains many things for which we cannot otherwise account.

\verseWithHeading{Hospitalisation}
    Though we work out our solution on the spiritual 
    as well as an altruistic plane, 
    we favor hospitalization for the alcoholic who is very jittery 
    or befogged. 
\verse More often than not, 
    it is imperative that a man's brain be cleared before he is approached, 
    as he has then a better chance of understanding and accepting 
    what we have to offer.

\verse The doctor writes:
\end{biblechapter}

\emph{
\begin{biblechapter}
\verseWithHeading{More from Silkworth}
    The subject presented in this book seems to me 
    to be of paramount importance to those afflicted 
    with alcoholic addiction.

\verse I say this after many years' experience as Medical Director 
    of one of the oldest hospitals in the country 
    treating alcoholic and drug addiction.

\verse There was, therefore, a sense of real satisfaction 
    when I was asked to contribute a few words on a subject 
    which is covered in such masterly detail in these pages.

\verseWithHeading{Moral Psychology}
    We doctors have realized for a long time 
    that some form of moral psychology 
    was of urgent importance to alcoholics, 
    but its application presented difficulties beyond our conception. 
\verse What with our ultra-modern standards, 
    our scientific approach to everything, 
    we are perhaps not well equipped to apply the powers of good 
    that lie outside our synthetic knowledge.

\verse Many years ago one of the leading contributors to this book 
    came under our care in this hospital and while here 
    he acquired some ideas which he put into practical application at once.

\verseWithHeading{Twelve}
    Later, he requested the privilege 
    of being allowed to tell his story to other patients here 
    and with some misgiving, we consented. 
\verse The cases we have followed through have been most interesting; 
    in fact, many of them are amazing.
\verse The unselfishness of these men as we have come to know them, 
    the entire absence of profit motive, 
    and their community spirit, 
    is indeed inspiring to one who has labored long and wearily 
    in this alcoholic field. 
\verse They believe in themselves, 
    and still more in the Power which 
    pulls chronic alcoholics back from the gates of death.

\verseWithHeading{Medical Detox}
    Of course an alcoholic ought to be freed 
    from his physical craving for liquor, 
    and this often requires a definite hospital procedure, 
    before psychological measures can be of maximum benefit.
\end{biblechapter}


\begin{biblechapter}
\verseWithHeading{Allergy Hypothesis}
    We believe, and so suggested a few years ago, 
    that the action of alcohol on these chronic alcoholics 
    is a manifestation of an allergy; 
    that the phenomenon of craving is limited to this class 
    and never occurs in the average temperate drinker. 
\verse These allergic types can never safely use alcohol 
    in any form at all; 
    and once having formed the habit 
    and found they cannot break it, 
    once having lost their self-confidence, 
    their reliance upon things human, 
    their problems pile up on them 
    and become astonishingly difficult to solve.

\verse Frothy emotional appeal seldom suffices. 
\verse The message which can interest and hold these alcoholic people must have depth and weight. 
\verse In nearly all cases, their ideals must be grounded in a power greater than themselves, 
    if they are to re-create their lives.

\verse If any feel that as psychiatrists directing a hospital for alcoholics we appear somewhat sentimental, 
    let them stand with us a while on the firing line, 
    see the tragedies, the despairing wives, the little children; 
    let the solving of these problems become a part of their daily work, 
    and even of their sleeping moments, 
    and the most cynical will not wonder that we have accepted and encouraged this movement. 
\verse We feel, after many years of experience, 
    that we have found nothing which has contributed more to the rehabilitation of these men 
    than the altruistic movement now growing up among them.

\verse Men and women drink essentially because they like the effect produced by alcohol. 
\verse The sensation is so elusive that, while they admit it is injurious, 
    they cannot after a time differentiate the true from the false. 
\verse To them, their alcoholic life seems the only normal one. 
\verse They are restless, irritable and discontented, 
    unless they can again experience the sense of ease and comfort 
    which comes at once by taking a few drinks-drinks which they see others taking with impunity. 
\verse After they have succumbed to the desire again, as so many do, 
    and the phenomenon of craving develops, 
    they pass through the well-known stages of a spree, emerging remorseful, 
    with a firm resolution not to drink again. 
\verse This is repeated over and over, 
    and unless this person can experience an entire psychic change there is very little hope of his recovery.

\verse On the other hand-and strange as this may seem to those who do not understand-
    once a psychic change has occurred, 
    the very same person who seemed doomed, 
    who had so many problems he despaired of ever solving them, 
    suddenly finds himself easily able to control his desire for alcohol, 
    the only effort necessary being that required to follow a few simple rules.

\verse Men have cried out to me in sincere and despairing appeal: 
\verse "Doctor, I cannot go on like this! 
    I have everything to live for! 
    I must stop, but I cannot! 
    You must help me!"

\verse Faced with this problem, 
    if a doctor is honest with himself, 
    he must sometimes feel his own inadequacy. 
\verse Although he gives all that is in him, it often is not enough. 
\verse One feels that something more than human power is needed to produce the essential psychic change. 
\verse Though the aggregate of recoveries resulting from psychiatric effort is considerable, 
    we physicians must admit we have made little impression upon the problem as a whole. 
\verse Many types do not respond to the ordinary psychological approach.
\end{biblechapter}


\begin{biblechapter}
\verseWithHeading{Craving}
    I do not hold with those who believe that alcoholism is entirely a problem of mental control. 
\verse I have had many men who had, for example, 
    worked a period of months on some problem or business deal which was to be settled on a certain date, 
    favorably to them. 
\verse They took a drink a day or so prior to the date, 
    and then the phenomenon of craving at once became paramount to all other interests 
    so that the important appointment was not met. 
\verse These men were not drinking to escape; 
    they were drinking to overcome a craving beyond their mental control.

\verse There are many situations which arise out of the phenomenon of craving 
    which cause men to make the supreme sacrifice rather than continue to fight.
\end{biblechapter}


\begin{biblechapter}
\verseWithHeading{Types}
    The classification of alcoholics seems most difficult, 
    and in much detail is outside the scope of this book. 
\verse There are, of course, the psychopaths who are emotionally unstable. 
\verse We are all familiar with this type. 
\verse They are always "going on the wagon for keeps." 
\verse They are over-remorseful and make many resolutions, but never a decision.

\verse There is the type of man who is unwilling to admit that he cannot take a drink. 
\verse He plans various ways of drinking. 
\verse He changes his brand or his environment. 
\verse There is the type who always believes that 
    after being entirely free from alcohol for a period of time 
    he can take a drink without danger. 
\verse There is the manic-depressive type, who is, perhaps, 
    the least understood by his friends, 
    and about whom a whole chapter could be written.

\verse Then there are types entirely normal in every respect 
    except in the effect alcohol has upon them. 
\verse They are often able, intelligent, friendly people.

\verse All these, and many others, have one symptom in common: 
    they cannot start drinking without developing the phenomenon of craving. 
\verse This phenomenon, as we have suggested, 
    may be the manifestation of an allergy which differentiates these people, 
    and sets them apart as a distinct entity. 
\verse It has never been, by any treatment with which we are familiar, permanently eradicated. 
\verse The only relief we have to suggest is entire abstinence.

\verse This immediately precipitates us into a seething caldron of debate. 
\verse Much has been written pro and con, but among physicians, 
    the general opinion seems to be that most chronic alcoholics are doomed.
\end{biblechapter}


\begin{biblechapter}
\verseWithHeading{There is a Solution}
    What is the solution? 
\verse Perhaps I can best answer this by relating one of my experiences.

\verse About one year prior to this experience 
    a man was brought in to be treated for chronic alcoholism. 
\verse He had but partially recovered from a gastric hemorrhage 
    and seemed to be a case of pathological mental deterioration. 
\verse He had lost everything worth while in life 
    and was only living, one might say, to drink. 
\verse He frankly admitted and believed that for him there was no hope. 
\verse Following the elimination of alcohol, there was found to be no permanent brain injury. 
\verse He accepted the plan outlined in this book. 
\verse One year later he called to see me, and I experienced a very strange sensation. 
\verse I knew the man by name, and partly recognized his features, 
    but there all resemblance ended. 
\verse From a trembling, despairing, nervous wreck, 
    had emerged a man brimming over with self-reliance and contentment. 
\verse I talked with him for some time, 
    but was not able to bring myself to feel that I had known him before. 
\verse To me he was a stranger, and so he left me. 
\verse A long time has passed with no return to alcohol.

\verse When I need a mental uplift, 
    I often think of another case brought in by a physician prominent in New York City. 
\verse The patient had made his own diagnosis, 
    and deciding his situation hopeless, 
    had hidden in a deserted barn determined to die. 
\verse He was rescued by a searching party, 
    and, in desperate condition, brought to me. 
\verse Following his physical rehabilitation, 
    he had a talk with me in which he frankly stated he thought the treatment a waste of effort, 
    unless I could assure him, which no one ever had, 
    that in the future he would have the "will power" to resist the impulse to drink.

\verse His alcoholic problem was so complex, 
    and his depression so great, 
    that we felt his only hope would be through what we then called "moral psychology," 
    and we doubted if even that would have any effect.

\verse However, he did become "sold" on the ideas contained in this book. 
\verse He has not had a drink for a great many years. 
\verse I see him now and then and he is as fine a specimen of manhood as one could wish to meet.

\verse I earnestly advise every alcoholic to read this book through, 
    and though perhaps he came to scoff, 
    he may remain to pray.

\verse William D. Silkworth, M.D.
\end{biblechapter}
}

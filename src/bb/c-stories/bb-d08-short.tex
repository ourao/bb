\biblebook{Short}

\bbHeading{BB. HE SOLD HIMSELF SHORT}
\bbChapterPreamble


\begin{biblechapter}
\verseWithHeading{Foreword}
    But he found that there was a Higher Power 
    which had more faith in him than he had in himself.
Thus, A.A. was born in Chicago.
\end{biblechapter}


\begin{biblechapter}
\verseWithHeading{Intro}
    I GREW up in a small town outside Akron, Ohio, 
    where the life was typical of any average small town.
\verse I was very much interested in athletics, 
    and because of this and parental influence, 
    I didn’t drink or smoke in either grade or high school.

\verse All of this changed when I went to college.
\verse I had to adapt to new associations and associates, 
    and it seemed to be the smart thing to drink and smoke.
\verse I confined drinking to weekends 
    and drank normally in college 
    and for several years thereafter.

\verse After I left school, I went to work in Akron, living at home with my parents. Homelife was again a restraining influence. When I drank, I hid it from my folks out of respect for their feelings. This continued until I was twenty-seven. Then I started traveling, with the United States and Canada as my territory, and with so much freedom and with an unlimited expense account, I was soon drinking every night and kidding myself that it was all part of the job. I know now that sixty percent of the time I drank alone without benefit of customers.

In 1930, I moved to Chicago. Shortly thereafter, aided by the Depression, I found that I had a great

deal of spare time and that a little drink in the morning helped. By 1932, I was going on two or three day benders. That same year, my wife became fed up with my drinking around the house and called my dad in Akron to come and pick me up. She asked him to do something about me because she couldn’t. She was thoroughly disgusted.

This was the beginning of five years of bouncing back and forth between my home in Chicago and Akron to sober up. It was a period of binges coming closer and closer together and being of longer duration. Once dad came all the way to Florida to sober me up, after a hotel manager called him and said that if he wanted to see me alive he better get there fast. My wife could not understand why I would sober up for dad but not for her. They went into a huddle and dad explained that he simply took my pants, shoes and money away, so that I could get no liquor and had to sober up.

One time my wife decided to try this too. After finding every bottle that I had hidden around the apartment, she took away my pants, my shoes, my money and my keys, threw them under the bed in the back bedroom and slip-locked our door. By one a.m. I was desperate. I found some wool stockings, some white flannels that had shrunk to my knees, and an old jacket. I jimmied the front door so that I could get back in, and walked out. I was hit by an icy blast. It was February with snow and ice on the ground and I had a four block walk to the nearest cab stand, but I made it. On my ride to the nearest bar, I sold the driver on how misunderstood I was by my wife and what an unreasonable person she was. By the time we

reached the bar, he was willing to buy me a quart with his own money. Then when we got back to the apartment, he was willing to wait two or three days until I got my health back to be paid off for the liquor and fare. I was a good salesman. My wife could not understand the next morning why I was drunker than the night before when she took my bottles.

After a particularly bad Christmas and New Year’s holiday, dad picked me up again early in January, 1937, to go through the usual sobering up routine. This consisted of walking the floor for three or four days and nights until I could take nourishment. This time he had a suggestion to offer. He waited until I was completely sober and the day before I was to head back for Chicago, he told me of a small group of men in Akron who apparently had the same problem that I had but were doing something about it. He said they were sober, happy and had their self-respect back as well as the respect of their neighbors. He mentioned two of them that I had known through the years and suggested that I talk with them. But I had my health back and besides, I reasoned, they were much worse than I would ever be. Why, even a year ago, I had seen Howard, an ex-doctor, mooching a dime for a drink. I could not possibly be that bad. I would at least have asked for a quarter! So I told dad that I would lick it on my own, that I would drink nothing for a month and after that only beer.

Several months later dad was back in Chicago to pick me up again, but this time my attitude was entirely different. I could not wait to tell him that I wanted help, that if these men in Akron had anything

I wanted it, and would do anything to get it. I was completely licked by alcohol.

I can still remember very distinctly getting into Akron at eleven p.m. and routing this same Howard out of bed to do something about me. He spent two hours with me that night telling me his story. He said he had finally learned that drinking was a fatal illness made up of an allergy plus an obsession, and once the drinking had passed from habit to obsession, we were completely hopeless, and could look forward only to spending the balance of our lives in mental institutions or to death.

He laid great stress on the progression of his attitude toward life and people, and most of his attitudes had been very similar to mine. I thought at times that he was telling my story! I had thought that I was completely different from other people, that I was beginning to become a little balmy, even to the point of withdrawing more and more from society and wanting to be alone with my bottle.

Here was a man with essentially the same outlook on life, except that he had done something about it. He was happy, getting a kick out of life and people, and beginning to get his medical practice back again. As I look back on that first evening I realize that I began to hope, then, for the first time; and I felt that if he could regain these things, perhaps it would be possible for me too.

The next afternoon and evening, two other men visited me and each told me his story and the things that they were doing to try to recover from this tragic illness. They had that certain something that seemed to glow, a peace and a serenity combined with happi-

ness. In the next two or three days the balance of this handful of men contacted me, encouraged me, and told me how they were trying to live this program of recovery and the fun they were having doing it.

Then and then only, after a thorough indoctrination by eight or nine individuals, was I allowed to attend my first meeting. This first meeting was held in the living room of a home and was led by Bill D., the first man that Bill W. and Dr. Bob had worked with successfully.

The meeting consisted of perhaps eight or nine alcoholics and seven or eight wives. It was different from the meetings now held. The big A.A. book had not been written and there was no literature except various religious pamphlets. The program was carried on entirely by word of mouth.

The meeting lasted an hour and closed with the Lord’s Prayer. After it was closed we all retired to the kitchen and had coffee and doughnuts and more discussion until the small hours of the morning.

I was terribly impressed by this meeting and the quality of happiness these men displayed, despite their lack of material means. In this small group, during the Depression, there was no one who was not hard up.

I stayed in Akron two or three weeks on my initial trip trying to absorb as much of the program and philosophy as possible. I spent a great deal of time with Dr. Bob, whenever he had the time to spare, and in the homes of two or three other people, trying to see how the family lived the program. Every evening we would meet at the home of one of the members and have coffee and doughnuts and spend a social evening.

The day before I was due to go back to Chicago, a

Wednesday and Dr. Bob’s afternoon off, he had me down to the office and we spent three or four hours formally going through the Six-Step program as it was at that time. The six steps were:

1. Complete deflation.

2. Dependence and guidance from a Higher Power.

3. Moral inventory.

4. Confession.

5. Restitution.

6. Continued work with other alcoholics.

Dr. Bob led me through all of these steps. At the moral inventory, he brought up some of my bad personality traits or character defects, such as selfishness, conceit, jealousy, carelessness, intolerance, ill-temper, sarcasm and resentments. We went over these at great length and then he finally asked me if I wanted these defects of character removed. When I said yes, we both knelt at his desk and prayed, each of us asking to have these defects taken away.

This picture is still vivid. If I live to be a hundred, it will always stand out in my mind. It was very impressive and I wish that every A.A. could have the benefit of this type of sponsorship today. Dr. Bob always emphasized the religious angle very strongly, and I think it helped. I know it helped me. Dr. Bob then led me through the restitution step, in which I made a list of all of the persons I had harmed, and worked out ways and means of slowly making restitution.

I made several decisions at that time. One of them was that I would try to get a group started in Chicago; the second was that I would have to return to Akron to attend meetings at least every two months until I

did get a group started in Chicago; third, I decided I must place this program above everything else, even my family, because if I did not maintain my sobriety I would lose my family anyway. If I did not maintain my sobriety, I would not have a job. If I did not maintain my sobriety, I would have no friends left. I had few enough at that time.

The next day I went back to Chicago and started a vigorous campaign among my so-called friends or drinking companions. Their answer was always the same: If they should need it at any time they would surely get in touch with me. I went to a minister and a doctor that I still knew and they, in turn, asked me how long I had been sober. When I told them six weeks, they were polite and said that they would contact me in case they had anyone with an alcoholic problem.

Needless to say, it was a year or more before they did contact me. On my trips back to Akron to get my spirits recharged and to work with other alcoholics, I would ask Dr. Bob about this delay and wonder just what was wrong with me. He would invariably reply, “When you are right and the time is right, Providence will provide. You must always be willing and continue to make contacts.”

A few months after I made my original trip to Akron I was feeling pretty cocky, and I didn’t think my wife was treating me with proper respect, now that I was an outstanding citizen. So I set out to get drunk deliberately, just to teach her what she was missing. A week later, I had to get an old friend from Akron to spend two days sobering me up. That was my lesson, that one could not take the moral inventory and

then file it away; that the alcoholic has to continue to take inventory every day if he expects to get well and stay well. That was my only slip. It taught me a valuable lesson. In the summer of 1938, almost a year from the time I made my original contact with Akron, the man for whom I was working and who knew about the program, approached me and asked if I could do anything about one of his salesmen who was drinking very heavily. I went to the sanitarium where this chap was incarcerated and found to my surprise that he was interested. He had been wanting to do something about his drinking for a long time, but did not know how. I spent several days with him, but I did not feel adequate to pass the program on to him alone. So I suggested that he take a trip to Akron for a couple of weeks, which he did, living with one of the A.A. families there. When he returned, we had practically daily meetings from that time on.

A few months later one of the men who had been in touch with the group in Akron came to Chicago to live, and then there were three of us who continued to have informal meetings quite regularly.

In the spring of 1939, the Big Book was printed, and we had two inquiries from the New York office because of a fifteen-minute radio talk that was made. Neither one of the two was interested for himself, one being a mother who wanted to do something for her son. I suggested to her that she should see the son’s minister or doctor, and that perhaps he would recommend the A.A. program.

The doctor, a young man, immediately took fire with the idea, and while he did not convince the son, he turned over two prospects who were anxious for

the program. The three of us did not feel up to the job, and after a few meetings we convinced the prospects that they too should go to Akron where they could see an older group in action.

In the meantime, another doctor in Evanston became convinced that the program had possibilities, and turned over a woman to us to do something about. The girl was full of enthusiasm and she made the trip to Akron too. Immediately on her return we began to have formal meetings once a week, in the autumn of 1939, and we have continued to do this and to expand ever since.

Occasionally, it is accorded to a few of us to watch something fine grow from a tiny kernel into something of gigantic goodness. Such has been my privilege, both nationally and in my home city. From a mere handful in Akron we have spread throughout the world. From a single member in the Chicago area, commuting to Akron we now exceed six thousand.

These last eighteen years have been the happiest of my life, trite though that statement may seem. Fifteen of those years I would not have enjoyed had I continued drinking. Doctors told me before I stopped that I had only three years at the outside to live.

This latest part of my life has had a purpose, not in great things accomplished but in daily living. Courage to face each day has replaced the fears and uncertainties of earlier years. Acceptance of things as they are has replaced the old impatient champing at the bit to conquer the world. I have stopped tilting at windmills, and instead have tried to accomplish the little daily tasks, unimportant in themselves, but tasks that are an integral part of living fully.

      Where derision, contempt and pity were once shown me, I now enjoy the respect of many people. Where once I had casual acquaintances, all of whom were fair weather friends, I now have a host of friends who accept me for what I am. And over my A.A. years I have made many real, honest, sincere friendships that I shall always cherish.

I’m rated as a modestly successful man. My stock of material goods isn’t great. But I have a fortune in friendships, courage, self assurance and honest appraisal of my own abilities. Above all, I have gained the greatest thing accorded to any man, the love and understanding of a gracious God, who has lifted me from the alcoholic scrap-heap to a position of trust where I have been able to reap the rich rewards that come from showing a little love for others and from serving them as I can.
\end{biblechapter}

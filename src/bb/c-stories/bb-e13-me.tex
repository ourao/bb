\biblebook{Me}


\bbChapterPreamble


\bbHeading{ME AN ALCOHOLIC?}


\begin{biblechapter}
\verseWithHeading{Foreword}
    Barleycorn’s wringer squeezed this author—but 
    he escaped quite whole.
\end{biblechapter}


\begin{biblechapter}
\verseWithHeading{Intro}
    WHEN I TRY to reconstruct what my life was like “before,” 
    I see a coin with two faces.

\verse One, the side I turned to myself and the world, 
    was respectable—even, in some ways, 
    distinguished.
\verse I was father, husband, taxpayer, home owner.
\verse I was club-man, athlete, artist, musician, 
    author, editor, aircraft pilot and world traveler.
\verse I was listed in “Who’s Who in America” as an American who, 
    by distinguished achievement, had arrived.
\verse I was, so far as anybody could tell, quite a lad.

\verse The other side of the coin was sinister, baffling. I was inwardly unhappy most of the time. There would be times when the life of respectability and achievement seemed insufferably dull—I had to break out. This I would do by going completely “bohemian” for a night, getting drunk and rolling home with the dawn. Next day remorse would be on me like a tiger. I’d claw my way back to “respectability” and stay there—until the inevitable next time.

The insidiousness of alcoholism is an appalling thing. In all the twenty-five years of my drinking there were only a few occasions when I took a morning drink. My binges were one-night stands only. Once or twice, during my early drinking, I carried it over into the second day, and only once, that I can remember, did it continue into the third. I was never drunk on the job, never missed a day’s work, was seldom rendered totally ineffective by a hangover and kept my liquor expenses well within my adequate budget. I continued to advance in my chosen field. How could such a man possibly be called an alcoholic? Whatever the root of my unhappiness might turn out to be, I thought, it could not possibly be booze.

Of course I drank. Everybody did, in the set which I regarded as the apex of civilization. My wife loved to drink, and we tied on many a hooter in the name of marital bliss. My associates, and all the wits and literary lights I so much admired, also drank. Evening cocktails were as standard as morning coffee, and I suppose my average daily consumption ran a little more or less than a pint. Even on my rare (at first) binge nights it never ran much over a quart.

How easy it was, in the beginning, to forget that those binges ever happened! After a day or two of groveling remorse I’d come up with an explanation. “The nervous tension had piled up and just had to spill over.” Or, “My physical plant had got a little run down and the stuff rushed right to my head.” Or, “I got to talking and forgot how many I was taking and it hit me.” Always we’d emerge with a new formula for avoiding future trouble. “You’ve got to space your drinks and take plenty of water in between,” or “coat the stomach with a little olive oil,” or “drink anything but those damn martinis.” Weeks would go by without further trouble, and I’d be assured I’d at last hit on the right formula. The binge had been just “one of those things.” After a month it seemed unlikely that

it ever really happened. After three months it was forgotten. Intervals between binges were, at first, eight months.

My growing inward unhappiness was a very real thing, however, and I knew that something would have to be done about it. A friend had found help in psychoanalysis. After a particularly ugly one-nighter, my wife suggested I try it, and I agreed. Educated child of the scientific age that I was, I had complete faith in the science of the mind. It would be a sure cure and also an adventure. How exciting to learn the inward mysteries that govern the behavior of people, how wonderful to know, at last, all about myself! To cut a long story short, I spent seven years and ten thousand dollars on my psychiatric adventure, and emerged in worse condition than ever.

To be sure, I learned many fascinating things, and many things that were to prove helpful later. I learned what a devastating effect it can have on a child to coddle him and build him up, and then turn and beat him savagely, as had happened to me. I came to understand the intricate processes of projection, by which we cast into our adult world the images of the horrors of our childhood. Under the skilled guidance of an expert practitioner I wallowed in the world’s individual and collective mental agony.

Meanwhile, I was getting worse, both as regards my inward misery and my drinking. My daily alcoholic consumption remained about the same through all this, with perhaps a slight increase, and my binges remained one-nighters. But they were occurring with alarming frequency. In seven years the intervals between them decreased from eight months to ten days!

And they were growing uglier. One night I barely made my downtown club; if I’d had to go another fifty feet I’d have collapsed in the gutter. On another occasion I arrived home covered with blood. I’d deliberately smashed a window. With all this, it was becoming increasingly hard to maintain my front of distinction and respectability to the world. My personality was stretched almost to splitting in the effort; schizophrenia stared me in the face, and one night I was in a suicidal despair.

My outward professional life looked fine, on the surface. I was now head of a publishing venture in which nearly a million dollars had been invested. My opinions were quoted in Time and Newsweek with pictures, and I addressed the public by radio and television. It was a fantastic structure, built on a crumbling foundation. It was tottering and it had to fall. It did.

After my last binge I came home and smashed my dining room furniture to splinters, kicked out six windows and two balustrades. When I woke up sober, my handiwork confronted me. It is impossible for me to reproduce my despair. I can only list a few of its elements.

I’d had absolute faith in science, and only in science. “Knowledge is power,” I’d always been taught. Now I had to face up to the fact that knowledge of this sort, applied to my individual case, was not power. Science could take my mind apart expertly, but it couldn’t seem to put it together again. I crawled back to my analyst, not so much because I had faith in him, but because I had nowhere else to turn.

After talking with him for a time I heard myself saying, “Doc, I think I’m an alcoholic.”

“Yes,” he said, surprisingly, “you are.”

“Then why in God’s name haven’t you told me so, during all these years?”

“Two reasons,” he said. “First, I couldn’t be sure. The line between a heavy drinker and an alcoholic is not always clear. It wasn’t until just lately that, in your case, I could draw it. Second, you wouldn’t have believed me even if I had told you.”

I had to admit to myself that he was right. Only through being beaten down by my own misery would I ever have accepted the term “alcoholic” as applied to myself. Now, however, I accepted it fully. I knew from my general reading that alcoholism was irreversible and fatal. And I knew that somewhere along the line I’d lost the power to stop drinking. “Well, Doc,” I said, “what are we going to do?”

“There’s nothing I can do,” he said, “and nothing medicine can do. However, I’ve heard of an organization called Alcoholics Anonymous that has had some success with people like you. They make no guarantees and are not always successful. But if you want to, you’re free to try them. It might work.”

Many times in the intervening years I have thanked God for that man, a man who had the courage to admit failure, a man who had the humility to confess that all the hard-won learning of his profession could not turn up the answer. I looked up an A.A. meeting and went there—alone.

Here I found an ingredient that had been lacking in any other effort I had made to save myself. Here was—power! Here was power to live to the end of any given day, power to have the courage to face the next day, power to have friends, power to help people, power to be sane, power to stay sober. That was November, 1947. It is now past November, 1954, and I haven’t had a drink during those seven years. More over, I am deeply convinced that so long as I continue to strive, in my bumbling way, toward the principles I first encountered in the earlier chapters of this book, this remarkable power will continue to flow through me. What is this power? With my A.A. friends, all I can say is that it’s a power greater than myself. If pressed further, all I can do is follow the psalmist who said it long, long before me: “Be still, and know that I am God.”

My story has a happy ending, but not of the conventional kind. I had a lot more hell to go through. But what a difference there is between going through hell without a power greater than one’s self, and with it! As might have been predicted, my teetering tower of worldly success collapsed. My alcoholic associates fired me, took control, and ran the enterprise into bankruptcy. My alcoholic wife took up someone else, divorced me, and took with her all my remaining property. The most terrible blow of my life befell me after I’d found sobriety through A.A. Perhaps the single flicker of decency that shone through the fog of my drinking days was a clumsy affection for my two children, a boy and a girl. One night my son, when he was only sixteen, was suddenly and tragically killed. The Higher Power was on deck to see me through, sober. I think He’s on hand to see my son through, too. I think He’s on hand to see all of us through whatever may come to us.

\verse There have been some wonderful things, too.
\verse My new wife and I don’t own any property to speak of, 
    and the flashy successes of another day are no longer mine.
\verse But we have a baby who, 
    if you’ll pardon a little post-alcoholic sentimentality, 
    is right out of Heaven.
\verse My work is on a much deeper and more significant level 
    than it ever was before, 
    and I am today a fairly creative, relatively sane human being.
\verse And should I have more bad times, 
    I know that I’ll never again have to go through them alone.
\end{biblechapter}

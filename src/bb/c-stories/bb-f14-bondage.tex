\biblebook{Bondage}

\bbHeading{BB. FREEDOM FROM BONDAGE}
\bbChapterPreamble


\begin{biblechapter}
\verseWithHeading{Foreword}
    Young when she joined, 
    this A.A. believes her serious drinking 
    was the result of even deeper defects.
\verse She here tells how she was set free.
\end{biblechapter}


\begin{biblechapter}
\verseWithHeading{Intro}
    THE MENTAL TWISTS that led up to my drinking 
    began many years before I ever took a drink 
    for I am one of those whose history proves conclusively 
    that my drinking was “a symptom of a deeper trouble.”

\verse Through my efforts to get down to “causes and conditions,” 
    I stand convinced that my emotional illness 
    has been present from my earliest recollection.
\verse I never did react normally to any emotional situation.

\verse The medical profession would probably tell me I was conditioned for alcoholism by the things that happened to me in my childhood. And I am sure they would be right as far as they go, but A.A. has taught me I am the result of the way I reacted to what happened to me as a child. What is much more important to me, A.A. has taught me that through this simple program I may experience a change in this reaction pattern that will indeed allow me to “match calamity with serenity.”

I am an only child, and when I was seven years old my parents separated very abruptly. With no explanation at all, I was taken from my home in Florida to my grandparents’ home in the middle west. My mother went to a nearby city to go to work, and my father, being an alcoholic, simply went. My grandparents were strangers to me and I remember being lonely, and terrified and hurt.

In time I concluded that the reason I was hurt was because I loved my parents, and I concluded too that if I never allowed myself to love anybody or anything I could never be hurt again. It became second nature for me to remove myself from anything or anybody I found myself growing fond of.

I grew up believing that one had to be totally self-sufficient, for one never dared to depend on another human being. I thought that life was a pretty simple thing; you simply made a plan for your life, based upon what you wanted, and then you needed only the courage to go after it. I thought I knew exactly what I wanted out of life and I thought I knew exactly how to get it.

In my late teens I became aware of emotions I’d not counted on; restlessness, anxiety, fear and insecurity. The only kind of security I knew anything about at that time was material security and I decided that all these intruders would vanish immediately if I only had a lot of money. The solution seemed very simple. With cold calculation I set about to marry a fortune, and I did. The only thing this changed, however, was my surroundings, and it was soon apparent that I could have the same uncomfortable emotions with an unlimited checking account that I could on a working girl’s salary. It was impossible for me to say at this point, “Maybe there is something wrong with my philosophy,” and I certainly couldn’t say, “Maybe there is something wrong with me.” It was not difficult to convince myself that my unhappiness was the fault of the man I had married, and I divorced him at the end of a year.

I was married and divorced again before I was twenty-three years old, this time to a prominent band leader—a man that many women wanted. I thought this would give me ego-strength, make me feel wanted and secure, and alleviate my fears, but again nothing changed inside me.

The only importance in all of this lies in the fact that at twenty-three I was just as sick as I was at thirty-three, when I came into A.A., but at that time I apparently had no place to go because I had no drinking problem. Had I been able to explain to a psychiatrist the feelings of futility, loneliness and lack of purpose, that had come with my deep sense of personal failure at this second divorce, I seriously doubt that the good doctor could have convinced me that my basic problem was a spiritual hunger, but A.A. has shown me this was the truth. And if I had been able to turn to the church at that time I’m sure they could not have convinced me my sickness was within myself, nor could they have shown me the need for self-analysis that A.A. has shown me is vital if I am to survive. So I had no place to go. Or so it seemed to me.

I looked around me at people who seemed happy and tried to analyze their happiness, and it seemed to me that without exception these people had something or somebody they loved very much. I didn’t have the courage to love; I was not even sure I had the capacity. Fear of rejection and its ensuing pain were not to be risked, and I turned away from myself once more for the answer, this time to the drinks I had always refused before, and in alcohol I found a false courage.

I wasn’t afraid of anything or anybody after I learned about drinking, for it seemed right from the beginning that with liquor I could always retire to my little private world where nobody could get at me to hurt me. It seems only fitting that when I did finally fall in love it was with an alcoholic, and for the next ten years I progressed as rapidly as is humanly possible into what I believed to be hopeless alcoholism.

During this time, our country was engaged in a second World War and my husband was one of the first to go overseas. My reaction to this was identical in many respects to my reaction to my parents leaving me when I was seven. Apparently I’d grown physically at the customary rate of speed, and I had acquired an average amount of intellectual training in the intervening years, but there had been no emotional maturity at all. I realize now that this phase of my development had been arrested by my obsession with self, and my egocentricity had reached such proportions that adjustment to anything outside my personal control was impossible for me. I was immersed in self-pity and resentment, and the only people who would support this attitude or who I felt understood me at all were the people I met in bars and the ones who drank as I did. It became more and more necessary to escape from myself, for my remorse and shame and humiliation when I was sober were almost unbearable. The only way existence was possible was through rationalizing every sober moment and drinking myself into complete oblivion as often as I could.

My husband eventually returned, but it was not

long until we realized we could not continue our marriage. By this time I was such a past master at kidding myself that I had convinced myself I had sat out a war and waited for this man to come home and, as my resentment and self-pity grew, so did my alcoholic problem.

The last three years of my drinking, I drank on my job. The amount of will power exercised to control my drinking during working hours, diverted into a constructive channel, would have made me President, and the thing that made the will power possible was the knowledge that as soon as my day was finished I could drink myself into oblivion. Inside, though, I was scared to death, for I new that the time was coming (and it couldn’t be too remote) when I would be unable to hold that job. Maybe I wouldn’t be able to hold any job, or maybe (and this was my greatest fear) I wouldn’t care whether I had a job or not. I knew it didn’t make any difference where I started, the inevitable end would be skid row. The only reality I was able to face had been forced upon me by its very repetition—I had to drink; and I didn’t know there was anything in the world that could be done about it.

About this time, I met a man who had three motherless children and it seemed that might be a solution to my problem. I had never had a child and this had been a satisfactory excuse many times for my drinking. It seemed logical to me that if I married this man and took the responsibility for these children that they would keep me sober. So I married again. This caused the rather cryptic comment from one of my A.A. friends, when I told my story after coming into the program, “that I had always been a cinch for the program, for I had always been interested in mankind, but that I was just taking them one man at a time.”

The children kept me sober for darn near three weeks, and then I went on (please God) my last drunk. I’ve heard it said many times in A.A., “There is just one good drunk in every alcoholic’s life, and that’s the one that brings us into A.A.,” and I believe it. I was drunk for sixty days around the clock and it was my intention, literally, to drink myself to death. I went to jail for the second time during this period for being drunk in an automobile. I was the only person I’d ever known personally who had ever been in jail, and I guess it is most significant that the second time was less humiliating than the first had been.

Finally, in desperation, my family appealed to a doctor for advice and he suggested A.A. The people who came knew immediately I was in no condition to absorb anything of the program, and I was put in a sanitarium to be defogged so that I could make a sober decision about this for myself. It was here that I realized for the first time that as a practicing alcoholic I had no rights. Society can do anything it chooses to do with me when I am drunk and I can’t lift a finger to stop it, for I forfeit my rights through the simple expedient of becoming a menace to myself and to the people around me. With deep shame came the knowledge too that I had lived with no sense of social obligation nor had I known the meaning of moral responsibility to my fellow man.

I attended my first A.A. meeting on July 25th, 1947, and it is with deep gratitude that I’m able to say I’ve not had a drink since that time, and that I take no sedation or narcotics, for this program is to me one of complete sobriety and I no longer need to escape reality. One of the truly great things A.A. has taught me is that reality too has two sides; I had only known the grim side before the program, but now I had a chance to learn about the pleasant side as well.

The A.A. members who sponsored me told me in the beginning that I would not only find a way to live without having a drink, but that I would find a way to live without wantingto drink, if I would do these simple things. They said if you want to know how this program works, take the first word of your question—the H is for honesty, the O is for open-mindedness and the W is for willingness; these our Book calls the essentials of recovery. They suggested that I study the A.A. book and try to take the Twelve Steps according to the explanation in the Book, for it was their opinion that the application of these principles in our daily lives will get us sober and keep us sober. I believe this, and I believe too that it is equally impossible to practice these principles to the best of our ability, a day at a time, and still drink, for I don’t think the two things are compatible.

I had no problem admitting I was powerless over alcohol, and I certainly agreed that my life had become unmanageable. I had only to reflect on the contrast between the plans I made so many years ago for my life with what really happened to know I couldn’t manage my life drunk or sober. A.A. taught me that willingness to believe was enough for a beginning. It’s been true in my case, nor could I quarrel with “restore us to sanity,” for my actions drunk or sober, before A.A., were not those of a sane person. My desire to be honest with myself made it necessary

for me to realize that my thinking was irrational. It had to be or I could not have justified my erratic behavior as I did. I’ve been benefited from a dictionary definition I found that reads: “rationalization is giving a socially acceptable reason for socially unacceptable behavior, and socially unacceptable behavior is a form of insanity.”

A.A. has given me serenity of purpose, the opportunity to be of service to God and to the people about me, and I am serene in the infallibility of these principles that provide the fulfillment of my purpose.

A.A. has taught me that I will have peace of mind in exact proportion to the peace of mind I bring into the lives of the other people, and it has taught me the true meaning of the admonition “happy are ye who know these things and do them.” For the only problems I have now are those I create when I break out in a rash of self-will.

I’ve had many spiritual experiences since I’ve been in the program, many that I didn’t recognize right away, for I’m slow to learn and they take many guises. But one was so outstanding that I like to pass it on whenever I can in the hope that it will help someone else as it has me. As I said earlier, self-pity and resentment were my constant companions and my inventory began to look like a thirty-three year diary, for I seemed to have a resentment against everybody I had ever known. All but one “responded to the treatment” suggested in the Steps immediately, but this one posed a problem.

It was against my mother and it was twenty-five years old. I had fed it, fanned it and nurtured it as one might a delicate child, and it had become as much

a part of me as my breathing. It had provided me with excuses for my lack of education, my marital failures, personal failures, inadequacy, and of course, my alcoholism and, though I really thought I had been willing to part with it, now I knew I was reluctant to let it go.

One morning, however, I realized I had to get rid of it, for my reprieve was running out, and if I didn’t get rid of it I was going to get drunk—and I didn’t want to get drunk any more. In my prayers that morning I asked God to point out to me some way to be free of this resentment. During the day a friend of mine brought me some magazines to take to a hospital group I was interested in, and I looked through them and a “banner” across the front of one featured an article by a prominent clergyman in which I caught the word “resentment.”

He said, in effect: “If you have a resentment you want to be free of, if you will pray for the person or the thing that you resent, you will be free. If you will ask in prayer for everything you want for yourself to be given to them, you will be free. Ask for their health, their prosperity, their happiness, and you will be free. Even when you don’t really want it for them, and your prayers are only words and you don’t mean it, go ahead and do it anyway. Do it every day for two weeks and you will find you have come to mean it and to want it for them, and you will realize that where you used to feel bitterness and resentment and hatred, you now feel compassionate understanding and love.”

It worked for me then, and it has worked for me many times since, and it will work for me every time I am willing to work it. Sometimes I have to ask first for the willingness, but it too always comes. And because it works for me, it will work for all of us. As another great man says, “The only real freedom a human being can ever know is doing what you ought to do because you want to do it.”

This great experience that released me from the bondage of hatred and replaced it with love is really just another affirmation of the truth I know: I get everything I need in Alcoholics Anonymous—everything I need I get—and when I get what I need I invariably find that it was just what I wanted all the time.
\end{biblechapter}

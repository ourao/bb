\biblebook{European}

\bbHeading{BB. THE EUROPEAN DRINKER}
\bbChapterPreamble


\begin{biblechapter}
\verseWithHeading{Foreword}
Beer and wine were not the answer.
\end{biblechapter}


\begin{biblechapter}
\verseWithHeading{Intro}
    I WAS born in Europe, in Alsace to be exact, 
    shortly after it had become German 
    and practically grew up with “good Rhine wine” of song and story.
\verse My parents had some vague ideas of making a priest out of me 
    and for some years I attended the Franciscan school 
    at Basle, Switzerland, 
    just across the border, 
    about six miles from my home.
\verse But, although I was a good Catholic, 
    the monastic life had little appeal for me.

\verse Very early I became apprenticed to harness-making and acquired considerable knowledge of upholstering. My daily consumption of wine was about a quart, but that was common where I lived. Everybody drank wine. And it is true that there was no great amount o f drunkenness. But I can remember, in my teens, that there were a few characters who caused the village heads to nod pityingly and sometimes in anger as they paused to say, “That sot, Henri” and “Ce pauvre Jules,” who drank too much. They were undoubtedly the alcoholics of our village.

Military service was compulsory and I did my stretch with the class of my age, goose-stepping in German barracks and taking part in the Boxer Rebellion in China, my first time at any great distance from home. In foreign parts many a soldier who has been

abstemious at home learns to use new and potent drinks. So I indulged with my comrades in everything the Far East had to offer. I cannot say, however, that I acquired any craving for hard liquor as a result. When I got back to Germany I settled down to finish my apprenticeship, drinking the wine of the country as usual.

Many friends of my family had emigrated to America, so at twenty-four I decided that the United States offered me the opportunity I was never likely to find in my native land. I came directly to a growing industrial city in the middle west, where I have lived practically ever since. I was warmly welcomed by friends of my youth who had preceded me. For weeks after my arrival I was feted and entertained in the already large colony of Alsatians in the city, among the Germans in their saloons and clubs. I early decided that the wine of America was very inferior stuff and took up beer instead.

Fond of singing, I joined a German singing society which had good club headquarters. There I sat in the evenings, enjoying with my friends our memories of the “old country,” singing the old songs we all knew, playing simple card games for drinks and consuming great quantities of beer.

At that time I could go into any saloon, have one or two beers, walk out and forget about it. I had no desire whatever to sit down at a table and stay a whole morning or afternoon drinking. Certainly at that time I was one of those who “can take it or leave it alone.” There had never been any drunkards in my family. I came of good stock, of men and women who drank wine all their lives as a beverage, and while

they occasionally got drunk at special celebrations, they were up and about their business the next day.

Prohibition came. Having regard for the law of the land, I resigned myself to the will of the national legislators and quit drinking altogether, not because I had found it harmful, but because I couldn’t get what I was accustomed to drink. You can all remember that in the first few months after the change, a great many men, who had formerly been used to a few beers every day or an occasional drink of whiskey, simply quit all alcoholic drinks. For the great majority of us, however, that condition didn’t last. We saw very early that Prohibition wasn’t going to work. It wasn’t very long before home-brewing was an institution and men began to search ferverishly for old recipe books on wine-making.

But I hardly tasted anything for two years and started in business for myself, founding a mattress factory which is today an important industrial enterprise in our city. I was doing very well with that and general upholstering work, and there was every indication that I would be financially independent by the time I reached middle age. By this time I was married and was paying for a home. Like most immigrants I Wanted to be somebody and have something and I was very happy and contented as I felt success crown my efforts. I missed the old social times, of course, but had no definite craving even for beer.

Successful home-brewers among my friends began to invite me to their homes. I decided that if these fellows could make it I would try it myself, and so I did. It wasn’t very long until I had developed a pretty

good brew with uniformity and plenty of authority. I knew the stuff I was making was a lot stronger than I had been used to, but never suspected that steady drinking of it might develop a taste for something even stronger still.

It wasn’t long before the bootlegger was an established institution in this, as in other towns. I was doing well in business, and in going around town I was frequently invited to have a drink in a speakeasy. I condoned my domestic brewing along with the bootleggers and their business. More and more I formed the habit of doing some of my business in the speakeasy, and after a time I did not need that as an excuse. The “speaks” usually sold whiskey. Beer was too bulky and it couldn’t be kept in a jug under the counter ready to be dumped when John Law came around. I was now forming an entirely new drinking technique. Before long I had a definite taste for hard liquor, knew nausea and headaches I had never known before, but as in the old days, I suffered them out. Gradually, however, I’d suffer so much that I simply had to have the morning-after drink.

I became a periodic drinker. I was eased out of the business I had founded and was reduced to doing general upholstery in a small shop at the back of my house. My wife upbraided me often and plenty when she saw that my “periodics” were gradually losing me what business I could get. I began to bring bottles in. I had them hidden away in the house and all over my shop in careful concealment. I had all the usual experiences of the alcoholic, for I was certainly one by this time. Sometimes, after sobering up after a bout of several weeks, I would righteously resolve to quit.

With a great deal of determination, I would throw out full pints—pour them out and smash the bottles—firmly resolved never to take another drink of the stuff. I was going to straighten up.

In four or five days I would be hunting all over the place, at home and in my workshop, for the bottles I had destroyed, cursing myself for being a fool. My “periodics” became more frequent until I reached the point where I wanted to devote all my time to drinking, working as little as possible, and then only when the necessity of my family demanded it. As soon as I had satisfied that, what I earned as an upholsterer went for liquor. I would promise to have jobs done and never do them. My customers lost confidence in me to the point where I retained what business I had only because I was a well-trained and reputedly fine craftsman. “Best in the business, when he’s sober,” my customers would say, and I still had a following who would give me work though they deplored my habits, because they knew the job would be well done when they eventually got it.

I had always been a good Catholic, possibly not so devoted as I should have been, but fairly regular in my attendance at services. I had never doubted the existence of the Supreme Being, but now I began to absent myself from my church where I had formerly been a member of the choir. Unfortunately, I had no desire to consult my priest about my drinking. In fact I was scared to talk to him about it, for I feared the kind of talk he would give me. Unlike many other Catholics who frequently take pledges for definite periods—a year, two years or for good, I never had any desire to “take a pledge” before the priest. And yet,

realizing at last that liquor really had me, I wanted to quit. My wife wrote away for advertised cures for the liquor habit and gave them to me in coffee. I even got them myself and tried them. None of the various cures of this kind were any good.

Then occured the event that saved me. An alcoholic who was a doctor came to see me. He didn’t talk like a preacher at all. In fact his language was perfectly suited to my understanding. He had no desire to know anything, except whether I was definite about my desire to quit drinking. I told him with all the sincerity at my command that I did. Even then he went into no great detail about how he and a crowd of alcoholics, with whom he associated, had mastered their difficulty. Instead he told me that some of them wanted to talk to me and would be over to see me.

This doctor had imparted his knowledge to just a few other men at that time—not more than four or five—they now number more than seventy persons.* And, because as I have discovered since, it is part of the “treatment” that these men be sent to see and talk with alcoholics who want to quit, he kept them busy. He had already imbued them with his own spirit until they were ready and willing at all times to go where sent, and as a doctor he well knew that this mission and duty would strengthen them as it later helped me. The visits from these men impressed me at once. Where preaching and prayers had touched me very little, I immediately desired further knowledge of these men.

I could see they were sober. The third man who came to see me had been one of the greatest business-

* Written in 1939.

getters his company had ever employed. From the top of the heap in a few years he had skidded to becoming a shuffling customer, still entering the better barrooms but welcomed by neither mine host nor his patrons. His own business was practically gone, he told me, when he discovered the answer.

“You’ve been trying man’s ways and they always fail,” he told me. “You can’t win unless you try God’s way.”

I had never heard of the remedy expressed in just this language. In a few sentences he made God seem personal to me, explained Him as a being who was interested in me, the alcoholic, and that all I needed to do was to be willing to follow His way; that as long as I followed it I would be able to overcome my desire for liquor.

Well, there I was, willing to try it, but I didn’t know how, except in a vague way. I knew somehow that it meant more than just going to church and living a moral life. If that was all, then I was a little doubtful that it was the answer I was looking for.

He went on talking and told me that he had found the plan has a basis of love, and the practice of Christ’s injunction, “Love thy neighbor as thyself.” Taking that as a foundation, he reasoned that if a man followed that rule he could not be selfish. I could see that. And he further said that God could not accept me as a sincere follower of His Divine Law unless I was ready to be thoroughly honest about it.

That was perfectly logical. My church taught that. I had always known that, in theory. We talked, too, about personal morals. Every man has his problem of this kind, but we didn’t discuss it very much. My visi-

tor well knew, that as I tried to follow God I would get to studying these things out for myself.

That day I gave my will to God and asked to be directed. But I have never thought of that as something to do and then forget about. I very early came to see that there had to be a continual renewal of that simple deal with God; that I had perpetually to keep the bargain. So I began to pray; to place my problems in God’s hands.

For a long time I kept on trying, in a pretty dumb way at first, I know, but very earnestly. I didn’t want to be a fake. And I began putting in practice what I was learning every day. It wasn’t very long until my doctor friend sent me to tell another alcoholic what my experience had been. This duty together with my weekly meetings with my fellow alcoholics, and my daily renewal of the contract I originally made with God, have kept me sober when nothing else ever did.

I have been sober for many years now. The first few months were hard. Many things happened; business trials, little worries, and feelings of general despondency came near driving me to the bottle, but I made progress. As I go along I seem to get strength daily to be able to resist more easily. And when I get upset, cross-grained and out of tune with my fellow man I know that I am out of tune with God. Searching where I have been at fault, it is not hard to discover and get right again, for I have proven to myself and to many others who know me that God can keep a man sober if he will let Him.
\end{biblechapter}

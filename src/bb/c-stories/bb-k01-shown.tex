\biblebook{Shown}


\bbChapterPreamble


\bbHeading{HE HAD TO BE SHOWN}


\begin{biblechapter}
\verseWithHeading{Foreword}
    “Who is convinced against his will is of the same opinion still.”
\verse But not this man.
\end{biblechapter}


\begin{biblechapter}
\verseWithHeading{Alcoholic Lineage}
    I WAS THE OLDEST of three children, 
    and my father was an alcoholic.
\verse One of the earliest memories that I have 
    is of a bottle sitting on his desk 
    with a skull and crossbones and marked “Poison.”
\verse At that time, as I remember, 
    he had promised never to take another drink.
\verse Of course he did.
\verse I can also remember that he was a salesman 
    and a very good one.
\verse When he was uptown—we were living in the little town of 
    Moscow—I went up to try to get some money from him 
    to buy groceries.
\verse He wouldn’t give me any money for the groceries, 
    but he did take me across the street 
    and buy me a bag of candy, 
    which I later took back 
    and traded for a loaf of bread.
\verse I was not more than six at that time.

\verse My father died in 1901 when I was eight years old 
    and I was in the second or third grade at school.
\verse I immediately quit school and went to work, 
    and from that time until I was high school age 
    there was never a return to school.
\verse I always built up in my own mind 
    the great things that I was going to do, 
    and in fact I accomplished about fifty per cent of them 
    and then lost interest.
\verse That continued through my entire life.
\verse When I was sixteen years old, 
    my mother remarried 
    and I was given the opportunity of going back to school.
\verse I went into the high school grades, 
    but having missed all the intermediate grades, 
    I didn’t get along too well, 
    so I developed the habit of going back to school 
    just long enough for the football season and then quitting.

\verse There was always a tremendous drive and ambition 
    to become a great guy, 
    because I think I recognized inwardly 
    that I didn’t have any special talents.
\verse At a comparatively early age, 
    I can remember being jealous of my brother.
\verse He did things much better than I did 
    because he applied himself 
    and learned how to do them, 
    and I never applied myself.
\verse Whether I could have done as well as he, I don’t know.
\end{biblechapter}


\begin{biblechapter}
\verseWithHeading{Married}
\verse I was married at the age of nineteen to a grand girl 
    and had good business prospects.
\verse I had bought a piece of ground in Cuyahoga Falls 
    and cut it into lots 
    and had a profit of approximately \$40,000 
    and that was a lot of money in those days.
\verse With that profit, I built a number of houses, 
    but then I neglected them. 
\verse I wouldn’t put sufficient time on them. 
\verse Consequently, my labor bills ran up.
\verse I lost money, 
    and then just fooled away a large part of the profit.
\end{biblechapter}


\begin{biblechapter}
\verseWithHeading{Mimetic Temptation}
\verse When I was eighteen, 
    at the end of high school, 
    the high school team had a banquet 
    at a well-known roadhouse outside of Akron.
\verse We boys drove out in somebody’s car 
    and went to the bar on the way to the dining room 
    and I, in an effort to impress the other boys 
    that I was city-bred, 
    having lived in Scranton and Cleveland, 
    asked them if they didn’t want to drink. 
\verse They looked at one another queerly and, finally, 
    one of them allowed he’d have a beer 
    and they all followed him, 
    each of them saying he’d have a beer too.
\verse I ordered a martini, extra dry.
\verse I didn’t even know what a martini looked like, 
    but I had heard a man down the bar order one. 
\verse That was my first drink.
\verse I kept watching the man down the bar 
    to see what he did with a contraption like that, 
    and he just smelled of his drink and set it down again, 
    so I did the same.
\verse He took a couple of puffs of a cigarette 
    and I took a couple of puffs of my cigarette.
\verse He tossed off half of his martini; 
    I tossed off half of mine 
    and it nearly blew the top of my head off.
\verse It irritated my nostrils; 
    I choked, I didn’t like it.
\verse There was nothing about that drink that I liked.
\verse But I watched him, and he tossed off the rest of his, 
    so I tossed off the rest of mine.
\verse He ate his olive and I ate mine.
\verse I didn’t even like the olive.
\verse It was repulsive to me from every standpoint.
\verse I drank nine martinis in less than an hour.

\verse Twenty-two years later, 
    Doctor Bob told me that what I had done 
    was like turning a switch 
    and setting up a demand for more alcohol in my system.
\verse I didn’t know that then.
\verse I had no more reason to drink those martinis than a jackrabbit.
\verse At that particular time, 
    the boys put me on a shutter 
    and took me out to the shed, 
    and I lay in the car while they enjoyed their banquet.
\verse That was the first time I ever drank hard liquor.
\verse Blackout drinking at once.
\verse I had no pleasure out of the drinking at all.
\verse All of a sudden I found myself guzzling.
\verse Right then I determined that never so long as I lived 
    would I have anything more to do with martinis.
\verse They acted on me like the beating of a club.

\verse I think it was probably more than a year 
    before I had anything more to do with liquor.
\verse I was opening up these lots that I spoke of.
\verse I had a crew of men working there 
    and I wanted them to work Sunday afternoons 
    so that I could sell lots on Sunday.
\verse I went over and bought a jug of hard cider 
    and a gallon of wine that I gave these fellows to drink.
\verse When they got through the day’s work, 
    part was left which I proceeded to drink.
\verse During the day, looking over the contracts and money in my pocket, 
    I found that I had sold six lots that I couldn’t even remember, 
    and didn’t even know the people I had sold them to.
\verse I had to look in the telephone book later to find out who these people were.
\verse Another blackout.
\verse Wine and hard cider.

\verse I early discovered that if I drank anything, 
    I was not accountable for what happened.
\verse I decided that I couldn’t drink.
\verse Anyhow, I recognized the fact that 
    I couldn’t drink like normal people, 
    but I tried hard 
    and kept on trying for twenty-two years.

\verse I sold three lots to an elderly lady in Cleveland.
\verse I came to Cleveland with the deeds to these lots 
    and to pick up my money.
\verse She paid me in cash.
\verse The next morning I woke up in jail in Cleveland 
    and the jailor had \$1,175 of my money in an envelope.
\verse I didn’t remember anything that had happened.
\verse This was six or eight months after the last drinking episode.

\verse Then I got married.
\verse (As I’ve said, I was nineteen).
\verse I felt having gotten married, 
    I was an adult 
    and one of the first things I did 
    was to buy two cases of whiskey with no idea of drinking it.
\verse (I might say right here 
    that never in my life did I ever intend to get drunk. 
    I never had any desire to get drunk.
    So I consciously thought).
\verse I was a very young married man, 
    having his whiskey in the cupboard over the sink, 
    and when I helped my wife with the dishes at night 
    I would take a cup of tea and spike it with whiskey.
\verse I could get through an evening with just a couple of snorts.

\verse This was a regular occurrence for a little while.
\verse Eventually, there would be a ball game, or a show, 
    or some sort of special occasion to celebrate 
    and I would turn up drunk.
\verse About that period, too, 
    came increasing procrastination 
    and the avoidance of responsibilities.
\verse I would put off doing anything that I could until the next day 
    and, consequently, everything would pile up 
    and then there would be the blackout.

\verse At the end of this selling of lots, 
    just prior to World War I, 
    I got into the crude rubber business, 
    and six months later there was only one broker and myself left in Akron.
\verse So in spite of anything that I might do, 
    I prospered, 
    being one of only two brokers in the rubber center of the world.

\verse I found, however, 
    that when I would leave Akron to go to Chicago, 
    I would get drunk.
\verse As long as there was everyday business, 
    I could drink occasionally 
    and didn’t always get drunk.
\verse I was a periodic.
\verse A big event of any kind precipitated heavy drinking.
\verse It had long since become a serious problem.
\verse I was prone to do everything on a big scale.
\verse I can well remember sitting with seven dollors in my pocket, 
    planning on giving my family a hundred or two hundred dollors, 
    when I made it next year. 
\verse But I didn’t do a thing about giving them any part 
    of the seven bucks I had in my pocket.

\verse The rubber prosperity went on for about six years—1916 to 1922.
\verse It fell apart in the twenties.
\verse Every company in the country, 
    except Firestone, was reorganized at that time.
\verse I was always able to skate along the fringe of big money.
\verse I made a point of knowing important people.
\verse I could work a deal up to where all I had to do 
    was to go ahead with it; 
    all the planning had been done, 
    all the financing had been done, 
    but then I’d say, 
    “Nuts to it!” 
    and walk away.
\verse Near success, only near.
\verse I figured the only difference between me and a millionaire 
    was that I hadn’t the strength, 
    or that he got the breaks and I didn’t.

\verse Akron was really on the boom in those days, 1919-1920; 
    expansion was terrific. 
\verse I optioned a piece of land just off East Market Street 
    to put up a three-hundred suite apartment. 
\verse One hundred for unmarried women at one end, 
    one hundred for men at the other, 
    and one hundred for married couples in the middle.
\verse In the basement were to be dry-cleaning facilities, 
    a barbershop, 
    a pool room, 
    a grocery shop and everything. 
\verse I had contracted for half of it, at least verbally, 
    and the contractors were taking half of the second mortgage bond.
\verse At that particular stage, 
    I lost interest in it, 
    sold the option for \$5,000 
    and forgot the whole deal.
\verse Another time, I had a rubber pool project.
\verse My idea was to have all the companies pool their funds 
    and buy rubber when rubber was cheap 
    and then put it in a pool.
\verse When rubber reached a certain low point, 
    they would draw on the rubber out of the pool and buy.
\verse With the big companies, 
    and with the amount of money we could have gotten 
    and the promises I had, 
    it could have been done.
\verse I worked along until I had really big names in rubber 
    on a tentative contract, 
    and then I neglected to go through with it.

\verse To my mind, 
    drinking didn’t have anything to do with not going through with things.
\verse I don’t know whether I drank to cover up being a failure, 
    or whether I drank and then missed the deals.
\verse I was able to rationalize it anyway.
\verse I can well remember over a long period of years 
    when I thought I was the only person in the world 
    who knew that sooner or later I was going to get drunk.
\verse I can remember occasions when friends recommended me for positions 
    or business opportunities that I wouldn’t take 
    because I felt that at some future date 
    I’d get drunk and they would be hurt.

\verse In the meantime, 
    the domestic situation was not getting along too well.
\verse We had two children, a boy and a girl, 
    and when the boy was about twelve, 
    we broke up the marriage.
\verse That was at my suggestion.
\verse I can remember telling the poor little soul 
    that I could probably quit drinking if I wasn’t married to her, 
    and told her that, after all, 
    I didn’t like restraint!
\verse I didn’t like having to come home at a certain time; 
    I didn’t like this, I didn’t like that, 
    and I think the poor girl acually divorced me to help me stop drinking!
\verse Naturally, what little restraint I had exercised before 
    was gone now, 
    and my drinking became worse.

\verse Long since I had come to believe I was insane 
    because I did so many things I didn’t want to do.
\verse I didn’t want to neglect my children.
\verse I loved them, I think, as much as any parent.
\verse But I did neglect them.
\verse I didn’t want to get into fights, but I did get into fights.
\verse I didn’t want to get arrested, but I did get arrested.
\verse I didn’t want to jeopardize the lives of innocent people 
    by driving an automobile while intoxicated, but I did.
\verse I quite naturally came to the conclusion that I must be insane.
\verse My big job was to keep other people from finding it out.
\verse I can remember well thinking that I would quit drinking, 
    really go to work hard, 
    apply myself eight hours a day, 
    five days a week, 
    make a lot of money, 
    and then I could start drinking again.
\verse That was the reverse of my former pattern of fearing to go to work 
    because I might drink!
\verse Always at the end of these dreams was drinking.
\verse Now I attempted to quit.
\verse I think this was about 1927.
\verse I was divorced in 1925 or 1926.
\verse I determined that I wouldn’t drink.
\verse I remember one occasion when I did not drink 
    for three hundred and sixty-four days, 
    but I didn’t quite make the year.
\end{biblechapter}


\begin{biblechapter}
\verseWithHeading{Max R}
\verse Another time, 
    I had gone around to Max R. 
    trying to get a job driving one of his trucks.
\verse He had known something of my drinking pattern, 
    and he asked me what I was doing about it.
\verse I told him that I had not had a drink for ninety days 
    and that I had come to the conclusion 
    that I was one of those individuals who couldn’t drink.
\verse So, knowing that, 
    I had determined that so long as I lived 
    I would never take another drink.
\verse On that statement 
    and the fact that I had been sober for ninety days, 
    he gave me a job selling lots in an allotment he had.
\verse I was moved in as Sales Manager 
    and had four men working under me.
\verse At the end of about four months, 
    I not only had good looking clothes—and 
    I might say that at the time I first talked to Max 
    I didn’t have a suit of clothes I could bend over in real sudden; 
    now I had six suits of clothes.
\verse I had an automobile.
\verse Everything in the world a man could possibly want, 
    and I was driving from Akron to Cleveland, 
    having just been to the bank 
    and discovered that I had approximately \$5,000.
\verse I drove towards Cleveland 
    wondering why I found myself in such a changed set of conditions 
    as compared with those of six months before.
\verse I came to the conclusion it was because I hadn’t been drunk.
\verse And I hadn’t been drunk because I hadn’t taken a drink.
\verse And I then and there said a prayer, if you please.
\verse An offer of appreciation for not having had a drink 
    for those few months 
    and then and there, 
    without anybody promising me anything or threatening me, 
    I made a solemn vow that never so long as I live 
    would I take another drink.

\verse (My mother and father were Catholics and I had been baptized, 
    but at the end of my instructions for Confirmation 
    I had not gone to church, 
    and then when my mother remarried, 
    she married a Protestant and the whole religious angle was forgotten.
\verse So I had never had any lasting contact with any kind of religion.)

\verse So I was driving to Cleveland when I made this solemn promise 
    never to drink again. 
\verse That was at three-thirty in the afternoon.
\verse At three-thirty the next morning, 
    I was in Champlain Street Station, 
    in jail for driving while intoxicated 
    and insulting an officer; 
    and the suit of which I was so proud was in such shape 
    that the turnkey had to get me a pair of trousers 
    to go into court in the next morning.
\verse I had run into a man I always drank with.
\verse Whenever this man and I met—I 
    didn’t know his name then nor do I know it now—we 
    would always get drunk.
\verse I had run into him, and he looked real prosperous; 
    his face and eyes looked clear 
    and he started to compliment me on my good front 
    and how well I looked, and I said, 
    “I haven’t had a drink for nine months.”
\verse He said, “Well, I haven’t had a drink for three months.”
\verse And we stood there for twenty minutes, 
    telling each other how much better we were, 
    how much better we looked, 
    how much better off we were 
    financially, mentally, physically, morally, 
    and in every way, shape and manner.
\verse And then we both realized we should go.
\verse We shook hands, and he hung onto my hand for a moment and said, 
    “Tell you what I’ll do for old times sake.
    I’ll buy you one drink, 
        and if you suggest a second one, 
        I’ll poke you right in the nose.”
\verse And I think we calculated, or I did, 
    that there wasn’t anybody who knew that I wasn’t drinking. 
\verse I could take one drink and get right back on the wagon.
\verse Nobody would know it, 
    so I agreed to have the one drink.
\verse We went into a bootleg joint 
    and I don’t remember leaving the place.
\verse I was picked up at two-thirty that morning 
    with my car smashed up by a street-car 
    because I had run into a big concrete safety zone, 
    and the street-car had run into me, 
    and they took me out through the roof—there’s 
    where I lost the suit.
\verse I had lost a hundred dollors I had in my pocket, 
    and lost a wristwatch too.
\verse I lost the car, of course.
\verse But more important, I lost my sobriety.
\verse And I continued to drink, on and off then, 
    until every dollor I had in the world was gone again 
    and I was right back living at my sister’s, 
    getting my cigaretts by calling her grocer 
    and telling him to put in a couple of cartons with her order, 
    exactly as I had before I started to work at Max’s.

\verse In 1932, 
    some friends of mine advised me that I might try Christian Science, 
    which had done a lot for some of their friends.
\verse So I started to investigate Science through some friends of mine 
    who were Readers in the church.
\verse I accepted their help, and it was helpful.
\verse I quit drinking immediately.
\verse The circumstances under which I reached these people were very odd 
    because I was led there through things that I said 
    when I couldn’t even remember speaking.
\verse I told somebody that I was going down to get Christian Science 
    and they took me down, 
    but I don’t remember saying that.
\verse Yet I wound up at this place.
\verse I attended their meetings every Sunday and Wednesday 
    for about nine months.
\verse If there was a lecture on the subject 
    within a hundred miles of Akron, 
    I attended.
\verse Then I started to miss meetings 
    because it was raining or snowing or something else. 
\verse Pretty soon, I wasn’t going at all, 
    and was avoiding those people who had been so kind to me.
\verse I avoided them rather than explain why they weren’t seeing me.
\verse My sobriety continued for another six months.

\verse At the end of fifteen months, 
    I tried the beer experiment.
\verse After drinking one glass of beer 
    at the end of my work period for about five days, 
    I thought I’d better find out whether I really had the stuff licked.
\verse So I didn’t have a beer one night, 
    and as I drove home I was breaking my arm patting myself on the back 
    because I had proved I could lick liquor.
\verse I had proved that liquor was not my master.
\verse I had avoided a drink this time.
\verse So having licked it, 
    there was no reason why I shouldn’t have a drink, 
    and I stopped in before I got home and had one.
\verse Then I got into the habit of having beers, 
    and decided that a drink of whiskey was not any worse; 
    so I would get the one drink of whiskey but, on second thought, 
    I decided that as long as I was only going to have one, 
    I might as well make it a double-header.
\verse So I had one double-header every night for about two or three weeks.
\verse I didn’t drink very long at a time.
\verse I think the longest drunk I was ever on was eleven days, 
    but usually only two days with a complete blackout for a day, 
    and then backing off by drinking as long as I could get anything.

\verse This Christian Science experience 
    with a sobriety of fifteen months was in 1932.
\verse Then I started drinking again, 
    with possibly a little more restraint, 
    periods a little bit longer than they had been before, 
    but substantially the same pattern.
\verse During the latter part of the Christian Science experience 
    I had gotten a job and was working at Firestone.
\verse I was bouncing along and not doing to badly.
\verse There were times when I got to drinking, 
    and I had been warned by Firestone 
    that they wouldn’t stand for this much longer, 
    so, clearly, they were conscious of the fact 
    that I drank too much and too often.
\end{biblechapter}


\begin{biblechapter}
\verseWithHeading{Final Orders}
\verse To show you the point to which this obsession went, 
    there came an occasion when I had spent a most delightful week-end, 
    and at nine o’clock on Sunday night I was on my way home, 
    and I thought I would get a drink.
\verse I went into a bar, and there I got into a fight.
\verse I was arrested and taken to jail 
    where I was beaten up by two or three fellows 
    who were already in there and whom I tried to boss.
\verse I was badly beaten.
\verse I tried to conduct a kangeroo court 
    and hit them with a broomstick.
\verse I had a broken nose, 
    a fractured cheekbone, 
    and was black from the lower part of my face 
    up into my hair.
\verse I was black and blue, with my lips all swollen, 
    when they roused us to go into court in the morning.
\verse I looked so terrible in the court 
    that the judge suggested that I get a continuance 
    and let me sign all the papers to go to a hospital and to a doctor.
\verse I went downstairs 
    and there was the grizzly old veteran police officer 
    in charge of the property desk, 
    and as he gave me the stuff, he asked, 
    “Are you going out in the street that way?”
\verse I said, “I’m certainly not going to stay here!”
\verse I had white trousers on, 
    white shoes and a white shirt that was streaked with blood.
\verse He said, “Well, why don’t you take a cab?”
\verse I said, “Allright, call me a cab,”
    as though I was talking to a bellboy.
\verse He did call me a cab and when I got into the cab, I said,
    “Drive me to a liquor store.”
\verse We drove to a store in North Hill 
    and I sent him in with what money I had to get a quart.
\verse He brought the quart out and I took a good swig.
\verse When I got home I had to give him a check for the taxi fare.
\verse I drank some more and slept through the day.
\verse At night, I woke up 
    and the folks with whom I roomed were home by then.
\verse I offered them a drink,
    and they came to the bottom of the stairs 
    and I stuck my face around the top of the stairs 
    and the good woman fainted, just looking at me.
\verse So they decided that I should have a doctor.
\verse They called a doctor 
    and it happened that they called one I knew.
\verse He came in and took a look at me and sent me to the hospital.

\verse When I had been in the hospital ten days, 
    Sister Ignatia, 
    who has played such a part in the development of A.A. in this area, 
    stuck her head in the door one morning and announced, 
    looking at me quizzically, 
    that they might be able to 
    make something human out of my face after all.
\verse And at the end of fourteen days, they let me out.
\verse Three days later I went to work.
\verse The next day, they called me in for an examination, 
    and the doctor wouldn’t let me continue working 
    and pardoned me from the plant for ten days 
    because he said my eye had been injured.
\verse So I was barred from the plant for ten days, 
    and during that ten day period I was drunk twice, 
    showing how little control these restraints had on me.

\verse Shortly after that, my brother, 
    who had then become associated with a group of people 
    and had stopped drinking, 
    urged me to attend meetings with him.
\verse Naturally, I wanted no part of any meetings.
\verse I explained to my sister 
    that some of the people he was meeting with had been in hospitals.
\verse I couldn’t afford to be found with those people, 
    but I said I would certainly pay his dues 
    if it would keep him from drinking.
\verse But me, I wanted no part of it!
\verse I didn’t have any need of such an association!
\end{biblechapter}


\begin{biblechapter}
\verseWithHeading{Bob}
\verse One morning, 
    after I had been on a binge for a couple days, 
    I awoke to find my brother and Doctor Bob 
    in my room talking to me about not drinking.
\verse My only thought that day was getting a drink, 
    and how to get rid of those clowns was my big problem.
\verse They asked me if I would take some medicine, 
    and I promised that I would if they got me a drink.
\verse So Paul was dispatched and brought back a pint.
\verse I got two drinks, 
    each of them a quarter of that pint, in me, 
    and was talking along with these people, 
    but I felt that sooner or later 
    they were going to have me cornered 
    because they were smarter than I was 
    and the drink was beginning to take effect; 
    but as I reached for the third drink Bob said, 
\verse “Listen, Buster, 
    you promised to take some medicine if we got you a drink.
\verse Now we got you the drink, 
        but you haven’t taken the medicine.”
\verse I agreed with him and told him in no uncertain terms 
    That I never broke my word in my life.
\verse I told him I’d take the medicine and I would take it, 
    but I hadn’t told him when, 
    and thereupon I got away with the third drink.
\verse I then began asking a lot of questions 
    of both my brother and Dr. Bob about how this worked, 
    and I suppose I was becoming more glassy-eyed all the while, 
    for eventually I said to Bob, 
\verse “Your all dried up. 
\verse You’re never going to want another drink, are you?”; 
\verse and this answer of his is very important to those of us 
    who are victims of alcoholism.
\verse He said, 
    “So long as I’m thinking as I’m thinking now, 
    and so long as I’m doing the things I’m doing now, 
    I don’t believe I’ll ever take another drink.”
\verse And I said, 
    “Well, what about Paul, have you got him all dried up?”
\verse He said Paul would have to answer for himself.
\verse So Paul repeated substantially what Dr. Bob had said.
\verse And I said, 
    “Now you want to dry me up.
    I’m not going to want another drink?”
\verse “Well,” the doctor said, 
    “we have hopes in that direction.”
\verse I said, 
    “In that case, there’s no use of wasting this,” 
    and I got the last of that pint.
\verse A few minutes later, Dr. Bob left, 
    leaving with my brother some medicine I should get.
\verse Paul measured the medicine out, 
    but he figured that with my track record 
    that little bit wouldn’t be enough, 
    so he doubled it and added a few drops more 
    and then gave it to me.
\verse I immediately became unconscious.
\verse This was on Thursday.
\verse I regained consciousness on Sunday.
\verse I had taken five and a half ounces of paraldehyde.
\verse Because it effected so strenuously, 
    they felt that hospitalization was indicated 
    and I awoke in a hospital.

\verse On Sunday when I came to, 
    it was a bad, wet, snowy day in February, 1937, 
    and Paul and Doc and a lot of the other fellows 
    were in Cleveland on business.
\verse The people in the group hadn’t been around that day; 
    part of my family was in Florida 
    and the rest of them weren’t speaking to me, 
    so I spent a very lonesome day 
    and by evening I was feeling very sorry for myself.
\verse It was getting pretty dark and I hadn’t turned on any lights, 
    when some big fellow stepped inside the door 
    and flipped on the light switch.
\verse I said, 
    “Look, Bub, if I want those lights on, 
    I’ll turn them on.”
\verse I’ll never forget, he never even hesitated 
    and I had never seen him before in my life.
\verse He took off his hat and his overcoat, and he said, 
    “You don’t look very good. How are you feeling?”
\verse I said, 
    “How do you suppose? 
    I’m feeling terrible.”
\verse He said, 
    “Maybe you need a little drink.”
\verse That was the smartest man I’d met in months.
\verse I thought he had it in his pocket, so I said, 
    “You got some?”
\verse He said, 
    “No, call the nurse.”
\verse And he got me a drink.
\verse Then he started to talk to me about his drinking experiences, 
    what his drinking had cost him, 
    how much he had drunk and where, 
    things like that, 
    and I remember I was quite bored 
    because I had never seen the guy before 
    and had no interest at all in what, where and when he drank.
\verse The man turned out to be Bill D., 
    a very early member of A.A., 
    and I couldn’t tell you a word of what he said.
\verse Not one experience registered with me.
\verse When he left, 
    I realized from his story that as a drinker 
    I was just a panty-waist.
\verse I knew I could quit because he had quit; 
    he hadn’t had a drink for over a year.
\verse The important thing was that he was happy.
\verse He was released, 
    relieved from his alcoholism 
    and was happy and contented because of it.
\verse That I shall never forget.

\verse The next day, 
    others from the group came in to see me.
\verse I remember well one fellow, Joe, 
    walking nearly three miles through slush, wet and snow 
    to come to the hospital to see a man 
    that he had never seen before in his life, 
    and that impressed me very much.
\verse He walked to the hospital to save bus fare 
    and did it gladly in order to be helpful 
    to an individual he had never even seen.
\verse There were only seven or eight people in the group before me 
    and they all visited me during my period in the hospital.
\verse The very simple program they advised me to follow 
    was that I should ask to know God’s will for me for that one day, 
    and then, to the best of my ability, 
    to follow that, 
    and at night to express my gratefulness to God 
    for the things that had happened to me during the day.
\verse When I left the hospital I tried this for a day and it worked, 
    for a week and it worked, 
    and for a month, and it worked—and 
    then for a year and it still worked.
\verse It has continued to work now for nearly eighteen years.
\end{biblechapter}

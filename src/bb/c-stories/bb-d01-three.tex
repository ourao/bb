\biblebook{Three}


\bbChapterPreamble


\bbHeading{ALCOHOLIC ANONYMOUS NUMBER THREE}


\begin{biblechapter}
\verseWithHeading{Foreword}
    Pioneer member of Akron’s Group No. 1, 
    the first A.A. group in the world.
\verse He kept the faith; 
    therefore, he and countless others found a new life.
\end{biblechapter}


\begin{biblechapter}
\verseWithHeading{Intro}
    ONE OF FIVE children, 
    I was born on a Kentucky farm in Carlyle County.
\verse My parents were well-to-do people and there marriage was a happy one.
\verse My wife, a Kentucky girl, 
    came with me to Akron 
    where I completed my course in law at the Akron Law School.

\verse My case is rather unusual in one respect. There were no childhood episodes of unhappiness to account for my alcoholism. I had, seemingly, just a natural affinity for grog. My marriage was happy and, as I have said, I never had any of the reasons, conscious or unconscious, which are often given for drinking. Yet, as my record shows, I did become an extremely serious case.

Before my drinking had cut me down completely, I achieved a considerable measure of success, having been a City Councilman for five years and a financial director of Kenmore, a suburb later taken into the city itself. But, of course, this all went up the spout with my increased drinking. So, at the time Dr. Bob and Bill came along I had about run out my strength.

The first time I became intoxicated I was eight years old. This was no fault of my father or my mother,

as they were both very much opposed to drinking. A couple of hired hands were cleaning out the barn on the farm and I would ride to and fro on the sled, and while they were loading I would drink hard cider out of a barrel in the barn. On the return trip, after two or three loads, I passed out and had to be carried to the house. I remember that my father kept whiskey around the house for medical purposes and entertainment, and I would drink from this when no one was about and then water it to keep my parents from knowing I was drinking.

This continued until I enrolled in our state university and, at the end of the four years, I realized that I was a drunk. Morning after morning I would awake sick and with terrible jitters, but there was always a flask of liquor sitting on the table beside my bed. I would reach over and get this and take a shot and in a few moments get up and take another, shave and eat my breakfast, slip a half pint of liquor in my hip pocket, and go on to school. Between classes I would run down to the wash room, take enough to steady ny nerves and then go on to the next class. This was in 1917.

I left the university in the latter part of my senior year and enlisted in the army. At the time, I called it patriotism. Later, I realized that I was running from alcohol. It did help to a certain extent, since I got in places where I could not obtain anything to drink, and so broke the habitual drinking.

Then Prohibition came into effect, and the facts that the stuff obtainable was so horrible and sometimes deadly, and that I had married and had a job which I had to look after, helped me for a period of some three

or four years, although I would get drunk every time I could get hold of enough to drink to get started. My wife and I belonged to some bridge clubs and they began to make wine and serve it. However, after two or three trials, I found this was not satisfactory because they did not serve enough to satisfy me. So I would refuse to drink. This problem was soon solved, however, as I began to take my bottle along with me and hide it in the bathroom or in the shrubbery outside.

As time went on my drinking became progressively worse. Away from my office two or three weeks at a time; horrible days and nights when I would lie on the floor of my home, lying awake and reaching over to get the bottle, taking a drink and going back into oblivion.

During the first six months of 1935, I was hospitalized eight times for intoxication and shackled to the bed two or three days before I even knew where I was.

On June 26, 1935, I came to in the hospital and to say I was discouraged is to put it mildly. Each of the seven times that I had left this hospital in the last six months, I had come out fully determined in my own mind that I would not get drunk again—for at least six or eight months. It hadn’t worked out that way and I didn’t know what the matter was and did not know what to do.

I was moved into another room that morning and there was my wife. I thought to myself, “Well, she is going to tell me this is the end,” and I certainly couldn’t blame her and did not intend to try to justify myself. She told me that she had been talking to a couple of fellows about drinking. I resented this very

much, until she informed me that they were a couple of drunks just as I was. That wasn’t so bad, to tell it to another drunk.

She said “You are going to quit.” That was worth a lot even though I did not believe it. Then she told me that these two drunks she had been talking to had a plan whereby they thought they could quit drinking, and part of that plan was that they tell it to another drunk. This was going to help them stay sober. All the other people that had talked to me wanted to help me, and my pride prevented me from listening to them, and caused only resentment on my part, but I felt as if I would be a real stinker if I did not listen to a couple of fellows for a short time, if that would cure them. She also told me that I could not pay them even if I wanted to and had the money, which I did not.

They came in and began to give me instruction in the program which later became known as Alcoholics Anonymous. There was not much of it at the time.

I looked up and there were two great big fellows over six foot tall, very likable looking. (I knew afterwards that the two who came in were Bill W. and Doctor Bob.) Before very long we began to relate some incidents of our drinking, and, naturally, pretty soon, I realized both of them knew what they were talking about because you can see things and smell things when you’re drunk, that you can’t other times, and, if I had thought they didn’t know what they were talking about, I wouldn’t have been willing to talk to them at all.

After a while, Bill said, “Well, now, you’ve been talking a good long time, let me talk a minute or two.” So, after hearing some more of my story, he turned

around and said to Doc—I don’t think he knew I heard him, but I did—he said, “Well, I believe he’s worth saving and working on.” They said to me, “Do you want to quit drinking? It’s none of our business about your drinking. We’re not up here trying to take any of your rights or privileges away from you, but we have a program whereby we think we can stay sober. Part of that program is that we take it to someone else, that needs it and wants it. Now, if you don’t want it, we’ll not take up your time, and we’ll be going and looking for someone else.”

The next thing they wanted to know was if I thought I could quit of my own accord, without any help, if I could just walk out of the hospital and never take another drink. If I could, that was wonderful, that was just fine, and they would very much appreciate a person who had that kind of power, but they were looking for a man that knew he had a problem, and knew that he couldn’t handle it himself and needed outside help. The next question, they wanted to know was if I believed in a Higher Power. I had no trouble there because I had never actually ceased to believe in God, and had tried lots of times to get help but hadn’t succeeded. The next thing they wanted to know was would I be willing to go to this Higher Power and ask for help, calmly and without any reservations.

They left this with me to think over, and I lay there on that hospital bed and went back over and reviewed my life. I thought of what liquor had done to me, the opportunities that I had discarded, the abilities that had been given to me and how I had wasted them, and I finally came to the conclusion, that if I didn’t want

to quit, I certainly ought to want to, and that I was willing to do anything in the world to stop drinking.

I was willing to admit to myself that I had hit bottom, that I had gotten hold of something that I didn’t know how to handle by myself. So, after reviewing these things and realizing what liquor had cost me, I went to this Higher Power which, to me, was God, without any reservation, and admitted that I was completely powerless over alcohol, and that I was willing to do anything in the world to get rid of the problem. In fact, I admitted that from now on I was willing to let God take over, instead 0f me. Each day I would try to find out what His will was, and try to follow that, rather than trying to get Him to always agree that the things I thought of myself were the things best for me. So, when they came back, I told them.

One of the fellows, I think it was Doc, said, “Well, you want to quit?” I said, “Yes, Doc, I would like to quit, at least for five, six, or eight months, until I get things straightened up, and begin to get the respect of my wife and some other people back, and get my finances fixed up and so on.” And they both laughed very heartily, and said, “Thats better than you’ve been doing, isn’t it?” Which of course was true. They said, “We’ve got some bad news for you. It was bad news for us, and it will probably be bad news for you. Weather you quit six days, months, or years, if you go out and take a drink or two you’ll end up in the hospital tied down, just like you have been in these past six months. You are an alcoholic.” As far as I know that was the first time I had ever paid any attention to that word. I figured I was a drunk. And they said, ” No, you have a disease, and it doesn’t make any

difference how long you do without it, after a drink or two you’ll end up just like you are now.” That certainly was real disheartening news, at the time.

The next question they asked was, “You can quit twenty-four hours, can’t you?” I said, “Sure, yes, anybody can do that, for twenty-four hours.” They said, “That’s what we’re talking about. Just twenty-four hours at a time.” That sure did take a load off of my mind. Every time I’d start thinking about drinking, I would think of the long, dry years ahead without having a drink; but this idea of twenty-four hours, that it was up to me from then on, was a lot of help.

(At this point, the Editors intrude just long enough to supplement Bill D.’s account, that of the man on the bed, with that of Bill W., the man who sat by the side of the bed.) Says Bill W.:
Nineteen years ago last summer, Dr. Bob and I saw him (Bill D.) for the first time. Bill lay on his hospital bed and looked at us in wonder.

Two days before this, Dr. Bob had said to me, “If you and I are going to stay sober, we had better get busy.” Straightway, Bob called Akron’s City Hospital and asked for the nurse on the receiving ward. He explained that he and a man from New York had a cure for alcoholism. did she have an alcoholic customer on whom it could be tried? Knowing Bob of old, she jokingly replied, “Well, Doctor, I suppose you’ve already tried it yourself?”

Yes, she did have a customer—a dandy. He just arrived in D.T.’s. Had blacked the eyes of two nurses, and now they had him strapped down tight. Would this one do? After prescribing medicines, Dr. Bob ordered, “Put him in a private room. We’ll be down as soon as he clears up.”

Bill didn’t seem too impressed. Looking sadder than ever, he wearily ventured, “Well, this is wonderful for you fellows, but it can’t be for me. My case is so terrible that I’m scared to go out of this hospital at all. You don’t have

to sell me religion, either. I was at one time a deacon in the church and I still believe in God. But I guess He doesn’t believe much in me.”
Then Dr. Bob said, “Well, Bill, maybe you’ll feel better tomarrow. Wouldn’t you like to see us again?”

“Sure I would,” replied Bill, “Maybe it won’t do any good, but I’d like to see you both, anyhow. You certainly know what you are talking about.”

Looking in later we found Bill with his wife, Henrietta. Eagerly he pointed to us saying, “These are the fellows I told you about; they are the ones who understands.”

Bill then related how he had lain awake nearly all night. Down in the pit of his depression, new hope had somehow been born. The thought flashed through his mind, “If they can do it, I can do it!”  Over and over he said this to himself. Finally, out of his hope, there burst conviction. Now he was sure. Then came a great joy. At length peace stole over him and he slept.

Before our visit was over, Bill suddenly turned to his wife and said, “Go fetch my clothes, dear. We’re going to get up and get out of here.”  Bill D. walked out of that hospital a free man never to drink again.

A.A.’s Number One Group dates from that very day.

(Bill D. now continues his story.)

       It was in the next two or three days after I had first met Doc and Bill, That I finally came to a decision to turn my will over to God and to go along with this program the best that I could. Their talk and action had instilled me with a certain amount of confidence, although I was not too absolutely certain. I wasn’t afraid that the program wouldn’t work, but I still was doubtful whether I would be able to hang on to the program, but I did come to the conclusion that I was willing to put everything I had into it, with God’s power, and that I wanted to do just that. As soon as I had done that I did feel a great release. I knew that
I had a helper that I could rely upon, who wouldn’t fail me. If I could stick to Him and listen, I would make it. Then I remember when the boys came back, that I told them, “I have gone to this Higher Power and I have told Him that I am willing to put His world first, above everything. I have already done it, and I am willing to do it again here in the presence of you or I am willing to say it any place, anywhere in the world from now on and not be ashamed of it.” And this, as I said, certainly gave me a lot of confidence, seemed to take a lot of the burden off me.

I remember telling them too that it was going to be awfully tough, because I did some other things, smoked cigarettes and played penny ante poker, sometimes bet on the horse races and they said, “Don’t you think you’re having more trouble with this drinking than with anything else at the present time? Don’t you believe you are going to have all you can do to get rid of that?” I said, “Yes,” reluctantly, “I probably will.” They said, “Let’s forget about those other things, that is, trying to eliminate them all at once, and concentrate on the drink.” Of course, we had talked over quite a number of failings that I had and made a sort of an inventory, which wasn’t to difficult, because I had an awful lot of things wrong that were very apparent to me, because I knew all about them. Then they said, “There is one more thing. You should go out and take this program to somebody else that needs it and wants it.”

Of course, by this time, my business was practically non-existent. I didn’t have any. Naturally, for quite a time, I wasn’t too well physically, either. It took me a year, or a year and a half to get to feeling physically

well, and it was rather tough, but I soon found folks whose friendship I had once had, and I found, after I had been sober for quite some little time, that these people began to act like they had in previous years, before I had gotten so bad, so that I didn’t pay too awful much attention to financial gains. I spent most of my time trying to get back these friendships, and to make some recompense towards my wife, whom I had hurt a lot.

It would be hard to estimate how much A.A. has done for me. I really wanted the program and I wanted to go along with it. I noticed that the others seemed to have such a release, a happiness, a something that I thought a person ought to have. I was trying to find the answer. I knew there was even more, something that I hadn’t got, and I remember one day, a week or two after I had come out of the hospital, Bill was over to my house talking to my wife and me. We were eating lunch, and I was listening and trying to find out why they had this release that they seemed to have. Bill looked across at my wife, and said to her, “Henrietta, the Lord has been so wonderful to me, curing me of this terrible disease, that I just want to keep talking about it and telling people.”

I thought, “I think I have the answer.” Bill was very, very grateful that he had been released from this terrible thing and he had given God the credit for having done it, and he’s so grateful about it he wants to tell other people about it. That sentence, “The Lord has been so wonderful to me, curing me of this terrible disease, that I just want to keep telling people about it,” has been a sort of a golden text for the A.A. program and for me.

       Of course, as time went on, and I began to get my health back and began to be so I didn’t have to hide from people all the time, it’s just been wonderful. I still go to meetings, because I like to go. I meet the people that I like to talk to. Another reason that I go is that I’m still grateful for the good years that I’ve had. I’m so grateful for both the program and the people in it that I still want to go, and then probably the most wonderful thing that I learned from the program—I’ve seen this in the ‘A.A. Grapevine’ a lot of times, and I’ve had people say it to me personally, and I’ve heard people get up in meetings and make the same statement: The statement is, “I came into A.A. solely for the purpose of sobriety, but it has been through A.A. that I have found God.”

I feel that is about the most wonderful thing that a person can do.

\end{biblechapter}

\biblebook{Hawk}

THE NEWS HAWK

This newsman covered life from top to bottom; but

he ended up, safely enough, in the middle.

 

      WITH NOTHING but a liberal arts education, very definitely estranged from my family and already married, soon after graduation from college I became a bookmaker’s clerk on the British racing circuits, far better off financially than the average professional man. I moved in a gay crowd in the various “pubs” and sporting clubs. My wife traveled with me, but with a baby coming I decided to settle in a large city where I got a job with a commission agent which is a polite term for a hand-book operator. My job was to collect bets and betting-slips in the business section, a lucrative spot. My boss, in his way, was “big business.” Drinking was all in the day’s work.

One evening, the book, after checking up, was very definitely in the red for plenty through a piece of studied carelessness on my part, and my boss, very shrewd and able, fired me with a parting statement to the effect that once was enough. With a good stake I sailed for New York. I knew I was through among the English “bookies.”

Tom Sharkey’s brawling bar in 14th Street and the famous wine-room at the back were headquarters for me. I soon ran through my stake. Some college friends got me jobs when I finally had to go to work,

but I didn’t stick to them. I wanted to travel. Making my way to Pittsburg, I met other former friends and got a job in a large factory where peacemakers were making good money. My fellow-workers were mostly good Saturday night drinkers and I was right with them. Young and able to travel with the best of them, I managed to hold my job and keep my end up in the barrooms.

I quit the factory and got a job on a small newspaper, going from that to a Pittsburgh daily, long ago defunct. Following a big drunk on that sheet, where I was doing leg-work and rewrite, a feeling of nostalgia made me buy a ticket for Liverpool and I returned to Britain.

During my visit there, renewing acquaintance with former friends, I soon spent most of my money. I wanted to roam again and through relatives got a supercargo job on an Australian packet which allowed me to visit my people in Australia where I was born. But I didn’t stay long. I was soon back in Liverpool. Coming out of a pub near the Cunard pier I saw the Lusitania standing out in the middle of the mersey. She had just come in and was scheduled to sail into days.

In my mind’s eye I saw Broadway again and Tom Sharkey’s bar; the roar of the subway was in my ears. Saying goodbye to my wife and baby, I was treading Manhattan’s streets in a little more than a week. Again I spent my bankroll, by no means as thick as the one I had when I first saw the skyline of Gotham. I was soon broke, this time without trainfare to go anywhere. I got my first introduction to “riding the rods and making a blind.”

      In my eary twenties the hardships of hobo life did not discourage me, but I had no wish to become just a tramp. Forced to detrain from an empty gondola on the other side of Chicago by a terrific rainstorm which drenched me to the skin, I hit the first factory building I saw for a job. That job began a series of brief working spells, each one ending in a “drunk” and the urge to travel. My migrations extended for over a year as far west as Omaha. Drifting back to Ohio, I landed on a small newspaper and later was impressed into the direction of boy-welfare work at the local “Y.” I stayed sober for four years except for a one-night carousal in Chicago. I stayed so sober that I used to keep a quart of medicinal whiskey in my bureau which I used to taper off the occasional newspaper alcoholics who were sent to see me.

Lots of times, vain-gloriously, I used to take the bottle out, look at it and say, “I’ve got you licked.”

The war was getting along. Curious about it, feeling I was missing something, absolutely without any illusions about the aftermath, with no pronounced feeling of patriotism, I joined up with a Canadian regiment, serving a little over two years. Slight casualties, complicated however by a long and serious illness, were my only mishaps. Remarkably enough, I was a very abstemious soldier. My four years of abstinence had something to do with it, but soldiering is a tough enough game for a sober man, and I had no yen for full-pack slogging through mud with a cognac or vin rouge hangover.

Discharged in 1919, I really made up for my dry spell. Quebec, Toronto, Buffalo, and finally Pittsburgh, were the scenes of man-sized drunks until I

had gone through my readjusted discharge pay, a fair sum.

I again became a reporter on a Pittsburgh daily. I applied for a publicity job and got it. My wife came over from Scotland and we started housekeeping in a large Ohio city.

The new job lasted five years. Every encouragement was given me with frequent salary increases, but the sober times between “periods” became shorter. I myself could see deterioration in my work from being physically and mentally affected by liquor, although I had not yet reached the point where all I wanted was more to drink. Successive Monday morning hangovers, which despite mid-week resolutions to do better, came with unfailing regularity, eventually caused me to quit my job. Washington, D.C. and news-gathering agency work followed, along with many parties. I couldn’t stand the pace. My drinking was never the spaced doses of the careful tippler; it was always gluttonous.

Returning to the town I had left three months before, I became editor of a monthly magazine, soon had additional publicity and advertising accounts and the money rolled in. The strain of overwork led me to the bottle again. My wife made several attempts to get me to stop and I had the usual visits from persons who would always ask me “Why?”—as if I knew! Offered the job of advertising manager for an eastern automotive company, I move to Philadelphia to begin life anew. In three months John Barleycorn had me kicked out.

I did six years of newspaper advertising, and trade journal work with many, many drunks of drab and

dreary hue woven into the pattern of my life. I visited my family just once in that time. An old avocation, the collecting of first editions, rare books and Americana, fascinated me between times. I had some financial success through no ability of my own and, when jobless and almost wiped out in 1930, I began to trade and sell my collections; much of the proceeds went to keep my apartment stocked with liquor and almost every night saw me helpless to bed.

I tried to help myself. I even began to go the rounds of the churches. I listened to famous ministers—found nothing. I began to know the inside of jails and work-houses. My family would have nothing to do with me, in fact couldn’t because I couldn’t spare any of my money, which I needed for drink, to support them. My last venture, a book shop, was hastened to closed doors by my steady intoxication. Then I had an idea.

Loading a car with good old books to sell to collectors, librarians, universities and historical societies, I started out to travel the country. I stayed sober during the trip except for an occasional bottle of beer because funds barely met expenses. When I hit Houston, Texas, I found employment in a large bookstore. Need I say that in a very short time I was walking along a prairie highway with arm extended and thumb pointed? In the two succeeding years I held ten different jobs ranging from newspaper copy-desk and rewrite, to traffic director for an oil field equipment company. Always, in between, there were intervals of being broke, riding freights and hitch-hiking interminable distances from one big town to another in three states. Now on a new job I was always thinking

about payday and how much liquor I could buy and the pleasure I could have.

I knew I was a drunkard. Enduring all hangover hells that every alcoholic experiences, I made the usual resolutions. My thoughts sometimes turned to the idea that three must be a remedy. I have stood listening to street-corner preachers tell how they beat the game. They seemed to be happy in their fashion, they and their little groups of supporters, but always pride of intellect stopped me from seeking what they evidently had. Sniffing at emotional religion, I walked away. I was an honest agnostic, but definitely not a hater of the church or its adherents. What philosophy I had was thoroughly paganistic—all my life was devoted to a search for pleasure. I wanted to do nothing except what it pleased me to do when I wanted to do it.

Federal Theatre in Texas gave me an administrative job which I held for a year, only because I worked hard and productively when I worked, and because my very tolerant chief ascribed my frequent lapses to a bohemian temperament. When it was closed through Washington edict I began with Federal Writers in San Antonio. In those days my system was always to drink up my last pay check and believe that necessity would bring the next job. A friend who knew I would soon be broke mounted guard over me when I left my job of writing the histories of Texas cities and put me aboard a bus for the town I had left almost five years before.

In five years a good many persons had forgotten that I had been somewhat notorious. I arrived drunk, but I promised my wife I would keep sober, and I

knew I could get work if I did. Of course I didn’t keep sober. My wife and family stood by me for ten weeks and then, quite justifiably, ejected me. I managed to maintain myself with odd jobs, did ten weeks in a social rescue institution, and at length wound up in a second-hand bookstore in an adjacent town as manager. While there I was called to the hospital in my home town to see a former partner who had insisted that I visit him. I found my friend was there for alcoholism and now he was insisting that he had found the only cure. I listened to him, rather tolerantly. I noticed a Bible on his table and it amazed me. I had never known him to be anything but a good healthy pagan with a propensity for getting into liquor jams and scrapes. As he talked I gathered vaguely, (because he was a faltering beginner then, just as I am now) that to be relieved of alcoholism I would have to be different.

Some days later, after he had been discharged, a stranger came into my shop in the nearby town. He introduced himself and began to tell me about a bunch of some sixty former drinkers and drunkards who met once a week, and he invited me to go with him to the next meeting. I thanked him, pleaded business engagements and promised I’d go with him at some future date.

“Anyhow, I’m on the wagon now,” I said. “I’m doing a job I like and it’s quiet where I live, practically no temptations. I don’t feel bothered about liquor.”

He looked at me quizzically. He knew too well that didn’t mean a thing, just as I knew in my heart that it would be only a question of time—a few days, a week, or even a month, it was inevitable—till I would be off

on another bender. The time came just a week later. As I look back on the events of two months, I can clearly see that I had been circling around, half-afraid of encountering the remedy for my situation, half wanting it, deferring fulfillment of my promise to get in touch with the doctor I had heard about. An accident while drunk laid me low for about three weeks. As soon as I could get up and walk I started to drink again and kept it up until my friend of the hospital, who, in his first try at the new way of life had stubbed his toe in Chicago but had come back to the town to take counsel and make a new start, picked me up and got me into a hospital.

I had been drinking heavily from one state of semicoma to another and it was several days before I got “defogged,” but subconsciously I was in earnest about wanting to quit liquor forever. It was no momentart emotionalism bornof self-pity in a maudlin condition. I was seeking something and I was ready to learn. I did not need to be told that my efforts were and would be unavailing if I did not get help. The doctor who came to see me almost at once did not assail me with any new doctrines; he made sure that I had a need and that I wanted to have that need filled, and little by little I learned how my need could be met. The story of Alcoholics Anonymous fascinated me. Singly and in groups of two or three, they came to visit me. Some of them I had known for years, good two-fisted drinkers who had disappeared from their former haunts. I had missed them myself from the barrooms of the town.

There were businessmen, professional men, and factory workers. All sorts were represented and their

relation of experiences and how they had found the only remedy, added to their human existence as sober men, laid the foundation of a very necessary faith. Indeed, I was beginning to see that I would require implicit faith, like a small child, if I was going to get anywhere. The big thing was that these men were all sober and evidently had something I didn’t have. Whatever it was, I wanted it.

I left the hospital on a meeting night. I was greeted warmly, honestly, and with a true ring of sincerity by everyone present. That night I was taken home by a former alcoholic and his wife. They did not show me to my room and wish me a good night’s rest. Instead, over coffee cups, this man and his wife told me what had been done for them. They were earnest and obviously trying to help me on the road I had chosen. They will never know how much their talk with me helped. The hospitality of their home and their fine fellowship were freely mine.

I had never, since the believing days of childhood, been able to conceive an authority directing the universe. But I had never been a flippant, wise-cracking sneerer at the few persons I had met who had impressed me as Christian men and women, or at any institution whose sincerity of purpose I could see. No conviction was necessary to establish my status as a miserable failure at managing my own life. I began to read the Bible daily and to go over a simple devotional exercise as a way to begin each day. Gradually I began to understand.

I cannot say that my taste for liquor has entirely disappeared. It has been that way with some, but it has not been with me and may never be. Neither can

I honestly say that I have forgotten the “fleshpots of Egypt.” I haven’t. But I can remember the urge of the Prodigal Son to return to his Father.

Formerly, in the acute mental and physical pain during the remorseful periods succeeding each drunk, I found my recollection of the misery I had gone through a bolsterer of resolution and afterward, perhaps, a deterrent for a time. But in those days I had no one to whom I might take my troubles. Today I have. Today I have Someone who will always hear me; I have a warm fellowship among men who understand my problems; I have tasks to do and am glad to do them, to see others who are alcoholics and to help them in any way I can to become sober men. I took my last drink in 1937.




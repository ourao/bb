\biblebook{Vicious}


\bbChapterPreamble


\bbHeading{THE VICIOUS CYCLE}


\begin{biblechapter}
\verseWithHeading{Foreword}
    How it finally broke a Southerner’s obstinacy 
    and destined this salesman to start A.A. at Philadelphia.
\end{biblechapter}


\begin{biblechapter}
\verseWithHeading{Intro}
    JANUARY 8, 1938—that was my D-Day; 
    the place, Washington, D.C.
\verse This last real merry-go-round 
    had started the day before Christmas 
    and I had really accomplished a lot in those fourteen days.
\verse First, my new wife had walked out, 
    bag, baggage and furniture; 
    then the apartment landlord 
    had thrown me out of the empty apartment 
    and the finish was the loss of another job.
\verse After a couple of days in dollar hotels and one night in the pokey, 
    I finally landed on my mother’s doorstep—shaking apart, 
    with several days’ beard and, 
    of course, broke as usual.
\verse Many of these same things had happened to me many times before, 
    but this time they had all descended together.
\verse For me, this was It.

\verse Here I was, thirty-nine years old and a complete washout.
\verse Nothing had worked.
\verse Mother would take me in only if I would stay locked 
    in a small storeroom and give her my clothes and shoes.
\verse We had played this game before.
\verse That is the way Jackie found me, 
    lying on a cot in my skivvies, 
    with hot and cold sweats, 
    pounding heart and that awful itchy scratchiness all over.
\verse Somehow I had always managed to avoid D.T.’s.

I had not asked for help and seriously doubt that I would have, but Fitz, an old school friend of mine,

had persuaded Jackie to call on me. Had he come two or three days later I think I would have thrown him out, but he hit when I was open for anything.

Jackie arrived about seven in the evening and talked until three a.m. I don’t remember much of what he said, but I did realize that here was another guy exactly like me; he had been in the same laughing academies and the same jails, known the same loss of jobs, same frustrations, same boredom and the same loneliness. If anything, he had known all of them even better and more often than I. Yet he was happy, relaxed, confident and laughing. That night for the first time in my life I really let down my hair and admitted my general loneliness. Jackie told me about a group of fellows in New York, of whom my old friend Fitz was one, who had the same problem I had, and who by working together to help each other were not now drinking and were happy like himself. He said something about God or a Higher Power, but I brushed that off—that was for the birds, not for me. Little more of our talk stayed in my memory, but I do know I slept the rest of that night, while before I had never known what a real night’s sleep was.

This was my introduction to this “understanding fellowship” although it was to be more than a year later before our Society was to bear the name Alcoholics Anonymous. All of us in A.A. know the tremendous happiness that is in our sobriety, but there are also tragedies. My sponsor, Jackie, was one of these. He brought in many of our original members, yet he himself could not make it and died of alcoholism. The lesson of his death still remains with me, yet I often wonder what would have happened if somebody

else had made that first call on me. So I always say that as long as I remember January 8th that is how long I will remain sober.

The age-old question in A.A. is which came first, the neurosis or the alcoholism. I like to think I was fairly normal before alcohol took over. My early life was spent in Baltimore where my father was a physician and a grain merchant. My family lived in very prosperous circumstances, and while both my parents drank, sometimes too much, neither was an alcoholic. Father was a very well-integrated person, and while mother was highstrung and a bit selfish and demanding, our home life was reasonably harmonious. There were four of us children, and although both of my brothers later became alcoholic—one died of alcoholism—my sister has never taken a drink in her life.

Until I was thirteen I attended public schools, with regular promotions and average grades. I have never shown any particular talents nor have I had any really frustrating ambitions. At thirteen I was packed off to a very fine Protestant boarding school in Virginia, where I stayed four years, graduating without any special achievements. In sports I made the track and tennis teams; I got along well with the other boys and had a fairly large circle of acquaintances but no intimate friends. I was never homesick and was always pretty self-sufficient.

However, here I probably took my first step towards my coming alcoholism by developing a terrific aversion to all churches and established religions. At this school we had Bible readings before each meal, and church services four times on Sunday, and I became so rebellious at this that I swore I would never join or

go to any church except for weddings and funerals.

At seventeen I entered the University, really to satisfy my father who wanted me to study medicine there as he had. That is where I had my first drink and I still remember it, for every “first” drink afterwards did exactly the same trick—I could feel it go right through every bit of my body and down to my very toes. But each drink after the “first” seemed to become less effective and after three or four they all seemed like water. I was never a hilarious drunk; the more I drank the quieter I got, and the drunker I got the harder I fought to stay sober. So it is clear that I never had any fun out of drinking—I would be the soberest-seeming one in the crowd and all of a sudden I was the drunkest. Even that first night I blacked out, which leads me to believe that I was an alcoholic from my very first drink. The first year in college I just got by in my studies, and that year I majored in poker and drinking. I refused to join any fraternity, as I wanted to be a free lance, and that year my drinking was confined to one-night stands, once or twice a week. The second year my drinking was more or less restricted to weekends, but I was nearly kicked out for scholastic failure.

In the spring of 1917, in order to beat being fired from school, I became “patriotic” and joined the Army. I am one of the lads who came out of the service with a lower rank than when I went in. I had been to OTC the previous summer, so I went into the Army as a sergeant, but I came out a private, and you really have to be unusual to do that. In the next two years I washed more pans and peeled more potatoes than any other doughboy. In the Army, I became a periodic

alcoholic—the periods always coming whenever I could make the opportunity. However, I did manage to keep out of the guardhouse. My last bout in the Army lasted from November 5th to 11th, 1918. We heard by wireless on the 5th that the Armistice would be signed the next day (this was a premature report) so I had a couple of cognacs to celebrate; then I hopped a truck and went AWOL. My next conscious memory was in Bar le Duc, many miles from base. It was November 11th and bells were ringing and whistles blowing for the real Armistice. There I was, unshaven, clothes torn and dirty with no recollection of wandering all over France but, of course, a hero to the local French. Back at camp, all was forgiven because it was the End, but in the light of what I have since learned I know I was a confirmed alcoholic at nineteen.

With the war over, and back in Baltimore with the folks, there were several small jobs for three years and then I went to work soliciting as one of the first ten employees of a new national finance company. What an opportunity I shot to pieces there! This company now does a volume of over three billion dollars annually. Three years later, at twenty-five, I opened and operated their Philadelphia office and was earning more than I ever have since. I was the fair-haired boy all right, but two years later I was blacklisted as an irresponsible drunk. It doesn’t take long.

My next job was in sales promotion for an oil company in Mississippi where I promptly became high man and got lots of pats on the back. Then I turned two company cars over in a short time and Bingo—fired again. Oddly enough, the big shot who

fired me from this company was one of the first men I met when I later joined the New York A.A. Group. He had also gone all the way through the wringer and had been dry two years when I saw him again.

After the oil job blew up, I went back to Baltimore and mother, my first wife having said a permanent “Goodbye.” Then came a sales job with a national tire company. I re-organized their city sales policy and eighteen months later, when I was thirty, they offered me the branch managership. As part of this promotion, they sent me to their national convention in Atlantic City to tell the big wheels how I’d done it. At this time I was holding what drinking I did down to weekends, but I hadn’t had a drink at all in a month. I checked into my hotel room and then noticed a placard tucked under the glass on the bureau stating “There will be positively NO drinking at this convention” signed by the president of the company. That did it! Who, me? The Big Shot? The only salesman invited to talk at the convention? The man who was going to take over one of the biggest branches come Monday? I’d show ’em who was boss! No one in that company saw me again—ten days later I wired my resignation.

As long as things were tough and the job a challenge I could always manage to hold on pretty well, but as soon as I learned the combination, got the puzzle under control and the boss to pat me on the back, I was gone again. Routine jobs bored me, but I would take on the toughest one I could find, work day and night until I had it under control; then it would become tedious and I’d lose all interest in it. I could never be bothered with the follow-through and would

invariably reward myself for my efforts with that “first” drink.

After the tire job came the thirties, the Depression and the down hill road. In the eight years before A.A. found me, I had over forty jobs—selling and traveling—one thing after another, and the same old routine. I’d work like mad for three or four weeks without a single drink, save my money, pay a few bills and then “reward” myself with alcohol. Then broke again, hiding out in cheap hotels all over the country, one night jail stands here and there, and always that horrible feeling “What’s the use—nothing is worth while.” Every time I blacked out, and that was every time I drank, there was always that gnawing fear, “What did I do this time?” Once I found out. Many alcoholics have learned they can bring their bottle to a cheap movie theater and drink, sleep, wake up and drink again in the darkness. I had repaired to one of these one morning with my jug and when I left late in the afternoon, I picked up a newspaper on the way home. Imagine my surprise when I read in a page one “box” that I had been taken from the theater unconscious around noon that day, removed by ambulance to a hospital and stomach-pumped and then released. Evidently I had gone right back to the movie with a bottle, stayed there several hours and started home with no recollection of what had happened.

The mental state of the sick alcoholic is beyond description. I had no resentments against individuals—the whole world was all wrong. My thoughts went round and round with “What’s it all about anyhow? People have wars and kill each other; they struggle and cut each other’s throats for success and what does

anyone get out of it? Haven’t I been successful, haven’t I accomplished extraordinary things in business? What do I get out of it? Everything’s all wrong and the hell with it.” For the last two years of my drinking I prayed on every drink that I wouldn’t wake up again. Three months before I met Jackie I had made my second feeble try at suicide.

This was the background that made me willing to listen on January 8th. After being dry two weeks and sticking close to Jackie, all of a sudden I found I had become the sponsor of my sponsor, for he was suddenly taken drunk. I was startled to learn he had only been off the booze a month or so himself when he brought me the message! However, I made an SOS call to the New York Group, whom I hadn’t met yet, and they suggested we both come there. This we did the next day, and what a trip! I really had a chance to see myself from a non-drinking point of view. We checked into the home of Hank, the man who had fired me eleven years before in Mississippi, and there I met Bill, our founder. Bill had then been dry three years and Hank, two. At the time, I thought them just a swell pair of screwballs, for they were not only going to save all the drunks in the world but also all the so-called normal people! All they talked of that first weekend was God, and how they were going to straighten out Jackie’s and my life. In those days we really took each other’s inventories firmly and often. Despite all this, I did like these new friends because, again, they were like me. They had also been periodic big shots who had goofed out repeatedly at the wrong time, and they also knew how to split one paper match into three separate matches. (This is very use-

ful knowledge in places where matches are prohibited.) They, too, had taken a train to one town and had wakened hundreds of miles in the opposite direction, never knowing how they got there. The same old routines seemed to be common to us all. During that first weekend I decided to stay in New York and take all they gave out with, except the “God stuff.” I knew they needed to straighten out their thinking and habits, but I was all right, I just drank too much. Just give me a good front and a couple of bucks and I’d be right back in the big time. I’d been dry three weeks, had the wrinkles out and had sobered up my sponsor all by myself!

Bill and Hank had just taken over a small automobile polish company and they offered me a job—ten dollars a week and keep at Hank’s house. We were all set to put DuPont out of business.

At that time the group in New York was composed of about twelve men who were working on the principle of every drunk for himself; we had no real formula and no name. We would follow one man’s ideas for a while, decide he was wrong and switch to another’s method. But we were staying sober as long as we kept and talked together. There was one meeting a week at Bill’s home in Brooklyn, and we all took turns there spouting off about how we had changed our lives overnight, how many drunks we had saved and straightened out, and last, but not least, how God had touched each of us personally on the shoulder. Boy, what a circle of confused idealists! Yet we all had one really sincere purpose in our hearts and that was not to drink. At our weekly meeting I was a menace to serenity those first few months, for I took

every opportunity to lambaste that “spiritual angle” as we called it, or anything else that had any tinge of theology. Much later I discovered the elders held many prayer meetings hoping to find a way to give me the heave-ho, but at the same time stay tolerant and spiritual. They did not seem to be getting an answer, for here I was staying sober and selling lots of auto polish, on which they were making one thousand percent profit. So I rocked along my merry independent way until June, when I went out selling auto polish in New England. After a very good week, two of my customers took me to lunch on Saturday. We ordered sandwiches and one man said, “Three beers.” I let mine sit. After a bit, the other man said, “Three beers.” I let that sit too. Then it was my turn—I ordered “Three beers,” but this time it was different, I had a cash investment of thirty cents and, on a ten dollar a week salary, that’s a big thing. So I drank all three beers, one after the other, said, “I’ll be seeing you, boys,” and went around the corner for a bottle. I never saw either of them again.

I had completely forgotten the January 8th when I found the fellowship, and I spent the next four days wandering around New England half drunk, by which I mean I couldn’t get drunk and I couldn’t get sober. I tried to contact the boys in New York, but telegrams bounced right back and when I finally got Hank on the telephone he fired me right then. This was when I really took my first good look at myself. My loneliness was worse than it had ever been before, for now even my own kind had turned against me. This time it really hurt, more than any hangover ever had. My brilliant agnosticism vanished, and I saw for the first

time that those who really believed, or at least honestly tried to find a Power greater than themselves, were much more composed and contented than I had ever been, and they seemed to have a degree of happiness which I had never known.

Peddling off my polish samples for expenses, I crawled back to New York a few days later in a very chastened frame of mind. When the others saw my altered attitude they took me back in, but for me they had to make it tough; if they hadn’t I don’t think I ever would have stuck it out. Once again, there was the challenge of a tough job, but this time I was determined to follow through. For a long time the only Higher Power I could concede was the power of the group, but this was far more than I had ever recognized before, and it was at least a beginning. It was also an ending, for never since June 16th, 1938, have I had to walk alone.

Around this time our big A.A. book was being written and it all became much simpler; we had a definite formula which some sixty of us agreed was the middle course for all alcoholics who wanted sobriety, and that formula has not been changed one iota down through the years. I don’t think the boys were completely convinced of my personality change, for they fought shy of including my story in the book, so my only contribution to their literary efforts was my firm conviction, being still a theological rebel, that the word God should be qualified with the phrase “as we understand him”— for that was the only way I could accept spirituality.

After the book appeared we all became very busy in our efforts to save all and sundry, but I was still

actually on the fringes of A.A. While I went along with all that was done and attended the meetings, I never took an active job of leadership until February 1940. Then I got a very good position in Philadelphia and quickly found I would need a few fellow alcoholics around me if I was to stay sober. Thus I found myself in the middle of a brand new group. When I started to tell the boys how we did it in New York and all about the spiritual part of the program, I found they would not believe me unless I was practicing what I preached. Then I found that as I gave in to this spiritual or personality change I was getting a little more serenity. In telling newcomers how to change their lives and attitudes, all of a sudden I found I was doing a little changing myself. I had been too self-sufficient to write a moral inventory, but I discovered in pointing out to the new man his wrong attitudes and actions that I was really taking my own inventory, and that if I expected him to change I would have to work on myself too. This change has been a long, slow process for me, but through these latter years the dividends have been tremendous.

In June, 1945, with another member, I made my first—and only—Twelfth Step call on a female alcoholic and a year later I married her. She has been sober all the way through and for me that has been good. We can share in the laughter and tears of our many friends, and most important, we can share our A.A. way of life and are given a daily opportunity to help others.

In conclusion, I can only say that whatever growth or understanding has come to me, I have no wish to graduate. Very rarely do I miss the meetings of my

neighborhood A.A. group, and my average has never been less than two meetings a week. I have served on only one committee in the past nine years, for I feel that I had my chance the first few years and that newer members should fill the jobs. They are far more alert and progressive than we floundering fathers were, and the future of our fellowship is in their hands. We now live in the West and are very fortunate in our area A.A.; it is good, simple and friendly, and our one desire is to stay in A.A. and not on it. Our pet slogan is “Easy Does It.”

And I still say that as long as I remember that January 8th in Washington, that is how long, by the grace of God as I understand Him, I will retain a happy sobriety.
\end{biblechapter}

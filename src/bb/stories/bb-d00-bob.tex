\biblebook{Bob}


\bbChapterPreamble


\bbHeading{DOCTOR BOB'S NIGHTMARE}


\begin{biblechapter}
\verseWithHeading{Foreword}
    A Co-founder of Alcoholics Anonymous.
\verse The birth of our Society 
    dates from his first day of permanent sobriety, 
    June 10, 1935.

\verse To 1950, the year of his death, 
    he carried the A.A. message  
    to  more  than  5,000  alcoholic  men  and women, 
    and to all these he gave his medical services
    without thought of charge.

\verse In this prodigy of service, 
    he was well assisted by Sister Ignatia 
    at St. Thomas Hospital in Akron, Ohio,
    one of the greatest friends our Fellowship will ever know.
\end{biblechapter}


\begin{biblechapter}
\verseWithHeading{Intro}
    I WAS born in a small New England village 
    of about seven thousand souls.
\verse The general moral standard was, as I recall it, 
    far above the average.
\verse No beer or liquor was sold in the neighborhood, 
    except at the State liquor agency 
    where perhaps one might procure a pint 
    if he could convince the agent that he really needed it.
\verse Without this proof 
    the expectant purchaser would be forced to depart 
    empty handed with none of what I later came to believe 
    was the great panacea for all human ills.
\verse Men who had liquor shipped in 
    from Boston or New York by express 
    were looked upon with great distrust and disfavor 
    by most of the good townspeople.
\verse The town was well supplied with churches and schools 
    in which I pursued my early educational activities.

\verse My father was a professional man of recognized ability and both my father and mother were most active in church affairs. Both father and mother were considerably above the average in intelligence.

Unfortunately for me I was the only child, which perhaps engendered the selfishness which played such an important part in bringing on my alcoholism.

From childhood through high school I was more or less forced to go to church, Sunday School and evening service, Monday night Christian Endeavor and sometimes to Wednesday evening prayer meeting. This had the effect of making me resolve that when I was free from parental domination, I would never again darken the doors of a church. This resolution I kept steadfastly for the next forty years, except when circumstances made it seem unwise to absent myself.

After high school came four years in one of the best colleges in the country where drinking seemed to be a major extra-curricular activity. Almost everyone seemed to do it. I did it more and more, and had lots of fun without much grief, either physical or financial. I seemed to be able to snap back the next morning better than most of my fellow drinkers, who were cursed (or perhaps blessed) with a great deal of morning-after nausea. Never once in my life have I had a headache, which fact leads me to believe that I was an alcoholic almost from the start. My whole life seemed to be centered around doing what I wanted to do, without regard for the rights, wishes, or privileges of anyone else; a state of mind which became more and more predominant as the years passed. I was graduated with “summa cum laude” in the eyes of the drinking fraternity, but not in the eyes of the Dean.

The next three years I spent in Boston, Chicago, and Montreal in the employ of a large manufacturing con-

cern, selling railway supplies, gas engines of all sorts, and many other items of heavy hardware. During these years, I drank as much as my purse permitted, still without paying too great a penalty, although I was beginning to have morning jitters at times. I lost only a half day’s work during these three years.

My next move was to take up the study of medicine, entering one of the largest universities in the country. There I took up the business of drinking with much greater earnestness than I had previously shown. On account of my enormous capacity for beer, I was elected to membership in one of the drinking societies, and soon became one of the leading spirits. Many mornings I have gone to classes, and even though fully prepared, would turn and walk back to the fraternity house because of my jitters, not daring to enter the classroom for fear of making a scene should I be called on for recitation.

This went from bad to worse until Sophomore spring when, after a prolonged period of drinking, I made up my mind that I could not complete my course, so I packed my grip and went South and spent a month on a large farm owned by a friend of mine. When I got the fog out of my brain, I decided that quitting school was very foolish and that I had better return and continue my work. When I reached school, I discovered the faculty had other ideas on the subject. After much argument they allowed me to return and take my exams, all of which I passed creditably. But they were much disgusted and told me they would attempt to struggle along without my presence. After many painful discussions, they finally gave me my credits and I

migrated to another of the leading universities of the country and entered as a Junior that Fall.

There my drinking became so much worse that the boys in the fraternity house where I lived felt forced to send for my father, who made a long journey in the vain endeavor to get me straightened around. This had little effect however for I kept on drinking and used a great deal more hard liquor than in former years.

Coming up to final exams I went on a particularly strenuous spree. When I went in to write the examinations, my hand trembled so I could not hold a pencil. I passed in at least three absolutely blank books. I was, of course, soon on the carpet and the upshot was that I had to go back for two more quarters and remain absolutely dry, if I wished to graduate. This I did, and proved myself satisfactory to the faculty, both in deportment and scholastically.

I conducted myself so creditably that I was able to secure a much coveted internship in a western city, where I spent two years. During these two years I was kept so busy that I hardly left the hospital at all. Consequently, I could not get into any trouble.

When those two years were up, I opened an office downtown. Then I had some money, all the time in the world, and considerable stomach trouble. I soon discovered that a couple of drinks would alleviate my gastric distress, at least for a few hours at a time, so it was not at all difficult for me to return to my former excessive indulgence.

By this time I was beginning to pay very dearly physically and, in hope of relief, voluntarily incarcerated myself at least a dozen times in one of the

local sanitariums. I was between Scylla and Charybdis now, because if I did not drink my stomach tortured me, and if I did, my nerves did the same thing. After three years of this, I wound up in the local hospital where they attempted to help me, but I would get my friends to smuggle me a quart, or I would steal the alcohol about the building, so that I got rapidly worse.

Finally my father had to send a doctor out from my home town who managed to get me back there some way and I was in bed about two months before I could venture out of the house. I stayed about town a couple of months more and returned to resume my practice. I think I must have been thoroughly scared by what had happened, or by the doctor, or probably both, so that I did not touch a drink again until the country went dry.

With the passing of the Eighteenth Amendment I felt quite safe. I knew everyone would buy a few bottles, or cases, of liquor as their exchequers permitted, and it would soon be gone. Therefore it would make no great difference, even if I should do some drinking. At that time I was not aware of the almost unlimited supply the government made it possible for us doctors to obtain, neither had I any knowledge of the bootlegger who soon appeared on the horizon. I drank with moderation at first, but it took me only a relatively short time to drift back into the old habits which had wound up so disastrously before.

During the next few years, I developed two distinct phobias. One was the fear of not sleeping, and the other was the fear of running out of liquor. Not being

a man of means, I knew that if I did not stay sober enough to earn money, I would run out of liquor. Most of the time, therefore, I did not take the morning drink which I craved so badly, but instead would fill up on large doses of sedatives to quiet the jitters, which distressed me terribly. Occasionally, I would yield to the morning craving, but if I did, it would be only a few hours before I would be quite unfit for work. This would lessen my chances of smuggling some home that evening, which in turn would mean a night of futile tossing around in bed followed by a morning of unbearable jitters. During the subsequent fifteen years I had sense enough never to go to the hospital if I had been drinking, and very seldom did I receive patients. I would sometimes hide out in one of the clubs of which I was a member, and had the habit at times of registering at a hotel under a fictitious name. But my friends usually found me and I would go home if they promised that I should not be scolded.

If my wife were planning to go out in the afternoon, I would get a large supply of liquor and smuggle it home and hide it in the coal bin, the clothes chute, over door jambs, over beams in the cellar and in cracks in the cellar tile. I also made use of old trunks and chests, the old can container, and even the ash container. The water tank on the toilet I never used, because that looked too easy. I found out later that my wife inspected it frequently. I used to put eight or twelve ounce bottles of alcohol in a fur lined glove and toss it onto the back airing porch when winter days got dark enough. My bootlegger had hidden alcohol at the back steps where I could get it at my convenience. Sometimes I would bring it in my

pockets, but they were inspected, and that became too risky. I used also to put it up in four ounce bottles and stick several in my stocking tops. This worked nicely until my wife and I went to see Wallace Beery in “Tugboat Annie,” after which the pant-leg and stocking racket were out!

I will not take space to relate all my hospital or sanitarium experiences.

During all this time we became more or less ostracized by our friends. We could not be invited out because I would surely get tight and my wife dared not invite people in for the same reason. My phobia for sleeplessness demanded that I get drunk every night, but in order to get more liquor for the next night, I had to stay sober during the day, at least up to four o’ clock. This routine went on with few interruptions for seventeen years. It was really a horrible nightmare, this earning money, getting liquor, smuggling it home, getting drunk, morning jitters, taking large doses of sedatives to make it possible for me to earn more money, and so on ad nauseam. I used to promise my wife, my friends, and my children that I would drink no more—promises which seldom kept me sober even through the day, though I was very sincere when I made them.

For the benefit of those experimentally inclined, I should mention the so-called beer experiment. When beer first came back, I thought that I was safe. I could drink all I wanted of that. It was harmless; nobody ever got drunk on beer. So I filled the cellar full, with the permission of my good wife. It was not long before I was drinking at least a case and a half a day. I put on thirty pounds weight in about two

months, looked like a pig, and was uncomfortable from shortness of breath. It then occurred to me that after one was all smelled up with beer nobody could tell what had been drunk, so I began to fortify my beer with straight alcohol. Of course, the result was very bad, and that ended the beer experiment.

About the time of the beer experiment I was thrown in with a crowd of people who attracted me because of their seeming poise, health, and happiness. They spoke with great freedom from embarrassment, which I could never do, and they seemed very much at ease on all occasions and appeared very healthy. More than these attributes, they seemed to be happy. I was self conscious and ill at ease most of the time, my health was at the breaking point, and I was thoroughly miserable. I sensed they had something I did not have, from which I might readily profit. I learned that it was something of a spiritual nature, which did not appeal to me very much, but I thought it could do no harm. I gave the matter much time and study for the next two and a half years, but still got tight every night nevertheless. I read everything I could find, and talked to everyone who I thought knew anything about it.

My wife became deeply interested and it was her interest that sustained mine, though I at no time sensed that it might be an answer to my liquor problem. How my wife kept her faith and courage during all those years, I’ll never know, but she did. If she had not, I know I would have been dead a long time ago. For some reason, we alcoholics seem to have the gift of picking out the world’s finest women. Why they

should be subjected to the tortures we inflicted upon them, I cannot explain.

About this time a lady called up my wife one Saturday afternoon, saying she wanted me to come over that evening to meet a friend of hers who might help me. It was the day before Mother’s Day and I had come home plastered, carrying a big potted plant which I set down on the table and forthwith went upstairs and passed out. The next day she called again. Wishing to be polite, though I felt very badly, I said, “Let’s make the call,” and extracted from my wife a promise that we would not stay over fifteen minutes.

We entered her house at exactly five o’ clock and it was eleven fifteen when we left. I had a couple of shorter talks with this man afterward, and stopped drinking abruptly. This dry spell lasted for about three weeks; Then I went to Atlantic City to attend several days’ meeting of a National Society of which I was a member. I drank all the Scotch they had on the train and bought several quarts on my way to the hotel. This was on Sunday. I got tight that night, stayed sober Monday till after the dinner and then proceeded to get tight again. I drank all I dared in the bar, and then went to my room to finish the job. Tuesday I started in the morning, getting well organized by noon. I did not want to disgrace myself, so I then checked out. I bought some more liquor on the way to the depot. I had to wait some time for the train. I remember nothing from then on until I woke up at a friend’s house, in a town near home. These good people notified my wife, who sent my newly-made friend over to get me. He came and got me home and to bed, gave

me a few drinks that night, and one bottle of beer the next morning.

That was June 10, 1935, and that was my last drink. As I write nearly six years have passed.

The question which might naturally come into your mind would be: “what did the man do or say that was different from what others had done or said?” It must be remembered that I had read a great deal and talked to everyone who knew, or thought they knew anything about the subject of alcoholism. But this was a man who had experienced many years of frightful drinking, who had had most all the drunkard’s experiences known to man, but who had been cured by the very means I had been trying to employ, that is to say, the spiritual approach. He gave me information about the subject of alcoholism which was undoubtedly helpful. Of far more importance was the fact that he was the first living human with whom I bad ever talked, who knew what he was talking about in regard to alcoholism from actual experience. In other words, be talked my language. He knew all the answers, and certainly not because he had picked them up in his reading.

It is a most wonderful blessing to be relieved of the terrible curse with which I was afflicted. My health is good and I have regained my self-respect and the respect of my colleagues. My home life is ideal and my business is as good as can be expected in these uncertain times.

I spend a great deal of time passing on what I learned to others who want and need it badly. I do it for four reasons:

1. Sense of duty.
2. It is a pleasure.

3. Because in so doing I am paying my debt to the man who took time to pass it on to me.

4. Because every time I do it I take out a little more insurance for myself against a possible slip.

Unlike most of our crowd, I did not get over my craving for liquor much during the first two and one-half years of abstinence. It was almost always with me. But at no time have I been anywhere near yielding. I used to get terribly upset when I saw my friends drink and knew I could not, but I schooled myself to believe that though I once had the same privilege, I had abused it so frightfully that it was withdrawn. So it doesn’t behoove me to squawk about it, for after all, nobody ever used to throw me down and pour any liquor down my throat.

If you think you are an atheist, an agnostic, a skeptic, or have any other form of intellectual pride which keeps you from accepting what is in this book, I feel sorry for you. If you still think you are strong enough to beat the game alone, that is your affair. But if you really and truly want to quit drinking liquor for good and all, and sincerely feel that you must have some help, we know that we have an answer for you. It never fails if you go about it with one half the zeal you have been in the habit of showing when getting another drink.

Your Heavenly Father will never let you down!

\end{biblechapter}

\biblebook{Kingdom}


\bbChapterPreamble


\bbHeading{THE KEYS OF THE KINGDOM}

\begin{biblechapter}
\verseWithHeading{Foreword}
    This worldly lady helped to develop A.A. in Chicago
    and thus passed her keys to many.
\end{biblechapter}


\begin{biblechapter}
\verseWithHeading{Intro}
    A LITTLE MORE than fifteen years ago, 
    through a long and calamitous series of shattering experiences, 
    I found myself being helplessly propelled 
    toward total destruction.
\verse I was without power to change the course my life had taken.
\verse How I had arrived at this tragic impasse 
    I could not have explained to anyone.
\verse I was thirty-three years old and my life was spent.
\verse I was caught in a cycle of alcohol and sedation 
    that was proving inescapable and consciousness had become intolerable.

\verse I was a product of the post-war prohibition era of the roaring ’20’s. That age of the flapper and the “It” girl, speakeasies and the hip flask, the boyish bob and the drugstore cowboy, John Held Jr. and F. Scott Fitzgerald, all generously sprinkled with a patent pseudo-sophistication. To be sure, this had been a dizzy and confused interval, but most everyone else I knew had emerged from it with both feet on the ground and a fair amount of adult maturity.

Nor could I blame my dilemma on my childhood environment. I couldn’t have chosen more loving and conscientious parents. I was given every advantage in a well-ordered home. I had the best schools, summer camps, resort vacations and travel. Every reason-

able desire was possible of attainment for me. I was strong and healthy and quite athletic.

I experienced some of the pleasure of social drinking when I was sixteen. I definitely liked it, everything about it, the taste, the effects; and I realize now that a drink did something for me or to me that was different from the way it affected others. It wasn’t long before any party without drinks was a dud for me.

I was married at twenty, had two children and was divorced at twenty-three. My broken home and broken heart fanned my smoldering self-pity into a fair-sized bonfire and this kept me well supplied with reasons for having another drink, and then another.

At twenty-five I had developed an alcoholic problem. I began making the rounds of the doctors in the hope that one of them might find some cure for my accumulating ailments, preferably something that could be removed surgically.

Of course the doctors found nothing. Just an unstable woman, undisciplined, poorly adjusted and filled with nameless fears. Most of them prescribed sedatives and advised rest and moderation.

Between the ages of twenty-five and thirty I tried everything. I moved a thousand miles away from home to Chicago and a new environment. I studied art; I desperately endeavored to create an interest in many things, in a new place among new people. Nothing worked. My drinking habits increased in spite of my struggle for control. I tried the beer diet, the wine diet, timing, measuring, and spacing of drinks. I tried them mixed, unmixed, drinking only when gay, only when depressed. And still by the time I was thirty years old I was being pushed around with

a compulsion to drink that was completely beyond my control. I couldn’t stop drinking. I would hang on to sobriety for short intervals, but always there would come the tide of an overpowering necessity to drink and, as I was engulfed in it, I felt such a sense of panic that I really believed I would die if I didn’t get that drink inside.

Needless to say, this was not pleasurable drinking. I had long since given up any pretense of the “social” cocktail hour. This was drinking in sheer desperation, alone and locked behind my own door. Alone in the relative safety of my home because I knew I dare not risk the danger of blacking out in some public place or at the wheel of a car. I could no longer gage my capacity and it might be the second or the tenth drink that would erase my consciousness.

The next three years saw me in sanitariums, once in a ten day coma, from which I very nearly did not recover, in and out of hospitals or confined at home with day and night nurses. By now I wanted to die, but had lost the courage even to take my life. I was trapped, and for the life of me I did not know how or why this had happened to me. And all the while my fear fed a growing conviction that before long it would be necessary for me to be put away in some institution. People didn’t behave this way outside of an asylum. Heartsickness, shame, and fear, fear bordering on panic, and no complete escape any longer except in oblivion. Certainly now, anyone would have agreed that only a miracle could prevent my final breakdown. But how does one get a prescription for a miracle?

For about one year, prior to this time, there was one doctor who had continued to struggle with me. He

had tried everything from having me attend daily mass at six a.m. to performing the most menial labor for his charity patients. Why he bothered with me as long as he did I shall never know, for he knew there was no answer for me in medicine and he, like all doctors of his day, had been taught that the alcoholic was incurable and should be ignored. Doctors were advised to attend patients who could be benefited by medicine. With the alcoholic, they could only give temporary relief and in the last stages not even that. It was a waste of the doctors’ time and the patients’ money. Nevertheless, there were a few doctors who saw alcoholism as a disease and felt that the alcoholic was a victim of something over which he had no control. They had a hunch that there must be an answer for these apparently hopeless ones, somewhere. Fortunately for me, my doctor was one of the enlightened.

And then, in the spring of 1939, a very remarkable book was rolled off a New York press with the title “Alcoholics Anonymous.” However, due to financial difficulties the whole printing was, for a while, held up and the book received no publicity, nor, of course, was it available in the stores, even if one knew it existed. But somehow my good doctor heard of this book and also he learned a little about the people responsible for its publication. He sent to New York for a copy, and after reading it he tucked it under his arm and called on me. That call marked the turning point in my life.

Until now, I had never been told that I was an alcoholic. Few doctors will tell a hopeless patient that there is no answer for him or for her. But this day my doctor gave it to me straight and said, “People like

you are pretty well known to the medical profession. Every doctor gets his quota of alcoholic patients. Some of us struggle with these people because we know that they are really very sick, but we also know that short of some miracle, we are not going to help them except temporarily, and that they will inevitably get worse and worse until one of two things happens. Either they die of acute alcoholism or they develop wet brains and have to be put away permanently.”

He further explained that alcohol was no respecter of sex or background, but that most of the alcoholics he had encountered had better than average minds and abilities. He said the alcoholic seemed to possess a native acuteness and usually excelled in his field, regardless of environmental or educational advantages.

“We watch the alcoholic performing in a position of responsibility and we know that because he is drinking heavily and daily he has cut his capacities by fifty percent, and still he seems able to do a satisfactory job. And we wonder how much further this man could go if his alcoholic problem could be removed and he could throw one hundred percent of his abilities into action. But, of course,” he continued, “eventually the alcoholic loses all of his capacities as his disease gets progressively worse, and this is a tragedy that is painful to watch; the disintegration of a sound mind and body.”

Then he told me there was a handful of people in Akron and New York who had worked out a technique for arresting their alcoholism. He asked me to read the book “Alcoholics Anonymous,” and then he wanted me to talk with a man who was experiencing success with his own arrestment. This man could tell

me more. I stayed up all night reading that book. For me it was a wonderful experience. It explained so much I had not understood about myself and, best of all, it promised recovery if I would do a few simple things and be willing to have the desire to drink removed. Here was hope. Maybe I could find my way out of this agonizing existence. Perhaps I could find freedom and peace and be able once again to call my soul my own.

The next day I received a visit from Mr. T., a recovered alcoholic. I don’t know what sort of person I was expecting, but I was very agreeably surprised to find Mr. T. a poised, intelligent, well groomed and mannered gentleman. I was immediately impressed with his graciousness and charm. He put me at ease with his first few words. Looking at him it was hard to believe he had ever been as I was then.

However, as he unfolded his story for me, I could not help but believe him. In describing his suffering, his fears, his many years of groping for some answer to that which always seemed to remain unanswerable, he could have been describing me, and nothing short of experience and knowledge could have afforded him that much insight! He had been dry for two and a half years and had been maintaining his contact with a group of recovered alcoholics in Akron. Contact with this group was extremely important to him. He told me that eventually he hoped such a group would develop in the Chicago area, but that so far this had not been started. He thought it would be helpful for me to visit the Akron group and meet many like himself.

By this time, with the doctor’s explanation, the revelations contained in the book, and the hope-inspiring

interview with Mr. T., I was ready and willing to go into the interior of the African jungles, if that was what it took, for me to find what these people had.

So I went to Akron, and also to Cleveland, and I met more recovered alcoholics. I saw in these people a quality of peace and serenity that I knew I must have for myself. Not only were they at peace with themselves, but they were getting a kick out of life such as one seldom encounters, except in the very young. They seemed to have all the ingredients for successful living. Philosophy, faith, a sense of humor (they could laugh at themselves), clear-cut objectives, appreciation—and most especially appreciation and sympathetic understanding for their fellow man. Nothing in their lives took precedence over their response to a call for help from some alcoholic in need. They would travel miles and stay up all night with someone they had never laid eyes on before and think nothing of it. Far from expecting praise for their deeds, they claimed the performance a privilege and insisted that they invariably received more than they gave. Extraordinary people!

I didn’t dare hope I might find for myself all that these people had found, but if I could acquire some small part of their intriguing quality of living—and sobriety—that would be enough.

Shortly after I returned to Chicago, my doctor, encouraged by the results of my contact with A.A., sent us two more of his alcoholic patients. By the latter part of September 1939, we had a nucleus of six and held our first official group meeting.

I had a tough pull back to normal good health. It has been so many years since I had not relied on some

artificial crutch, either alcohol or sedatives. Letting go of everything at once was both painful and terrifying. I could never have accomplished this alone. It took the help, understanding and wonderful companionship that was given so freely to me by my “ex-alkie” friends. This and the program of recovery embodied in the Twelve Steps. In learning to practice these steps in my daily living I began to acquire faith and a philosophy to live by. Whole new vistas were opened up for me, new avenues of experience to be explored, and life began to take on color and interest. In time, I found myself looking forward to each new day with pleasurable anticipation.

A.A. is not a plan for recovery that can be finished and done with. It is a way of life, and the challenge contained in its principles is great enough to keep any human being striving for as long as he lives. We do not, cannot, outgrow this plan. As arrested alcoholics, we must have a program for living that allows for limitless expansion. Keeping one foot in front of the other is essential for maintaining our arrestment. Others may idle in a retrogressive groove without too much danger, but retrogression can spell death for us. However, this isn’t as rough as it sounds, as we do become grateful for the necessity that makes us toe the line, for we find that we are more than compensated for a consistent effort by the countless dividends we receive.

A complete change takes place in our approach to life. Where we used to run from responsibility, we find ourselves accepting it with gratitude that we can successfully shoulder it. Instead of wanting to escape some perplexing problem, we experience a thrill of challenge in the opportunity it affords for another ap-

plication of A.A. techniques, and we find ourselves tackling it with surprising vigor.

The last fifteen years of my life have been rich and meaningful. I have had my share of problems, heartaches and disappointments, because that is life, but also I have known a great deal of joy, and a peace that is the handmaiden of an inner freedom. I have a wealth of friends and, with my A.A. friends, an unusual quality of fellowship. For, to these people, I am truly related. First, through mutual pain and despair, and later through mutual objectives and new-found faith and hope. And, as the years go by, working together, sharing our experiences with one another, and also sharing a mutual trust, understanding and love—without strings, without obligation—we acquire relationships that are unique and priceless.

There is no more “aloneness,” with that awful ache, so deep in the heart of every alcoholic that nothing, before, could ever reach it. That ache is gone and never need return again.

Now there is a sense of belonging, of being wanted and needed and loved. In return for a bottle and a hangover, we have been given the Keys of the Kingdom.
\end{biblechapter}

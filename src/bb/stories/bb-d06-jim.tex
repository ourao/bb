\biblebook{Jim}


\bbChapterPreamble


\bbHeading{JIM’S STORY}


\begin{biblechapter}
\verseWithHeading{Foreword}
    This physician, 
    the originator of A.A.’s first colored group, 
    but badly caught in the toils, 
    tells of his release 
    and of how freedom came as he worked among his own people.
\end{biblechapter}


\begin{biblechapter}
\verseWithHeading{Intro}
    I WAS BORN in a little town in Virginia 
        in an average religious home.
\verse My father, a negro, was a country physician.
\verse I remember in my early youth my mother dressed me 
    just as she did my two sisters, 
    and I wore curls until I was six years of age.
\verse At that time I started school 
    and that’s how I got rid of the curls.
\verse I found that even then I had fears and inhibitions.
\verse We lived just a few doors from the First Baptist Church 
    and when they had funerals 
    I remember very often asking my mother 
    whether the person was good or bad 
    and whether they were going to heaven or hell.
\verse I was about six then.

\verse My mother had been recently converted and, actually, had become a religious fanatic. That was her main neurotic manifestation. She was very possessive with us children. Another thing that mother drilled into me was a very Puritanical point of view as to sex relations, and as to motherhood and womanhood. I’m sure my ideas as to what life should be like were quite different from that of the average person with whom I associated. Later on in life that took its toll. I realize that now.

About this time an incident took place in grade school that I have never forgotten because it made me realize that I was actually a physical coward. During recess we were playing basketball and I had accidentally tripped a fellow just a little larger than I was. He took the basketball and smashed me in the face with it. That was enough provocation to fight but I didn’t fight, and I realized after recess why I didn’t. It was fear. That hurt and disturbed me a great deal.

Mother was of the old school and figured that anyone I associated with should be of the proper type. Of course, in my day, times had changed; she just hadn’t changed with the times. I don’t know whether it was right or wrong, but at least I know that people weren’t thinking the same. We weren’t even permitted to play cards in our home, but father would give us just a little toddy with whiskey and sugar and warm water now and then. We had no whiskey in the house, other than my father’s private stock. I’ve never seen him drunk in my life, although he’d take a shot in the morning and usually one in the evening, and so did I; but for the most part he kept his whiskey in his office. The only time that I have ever seen my mother take anything alcoholic was around Christmas time, when she would drink some eggnog or light wine.

I remember in my first year in high school, that mother suggested that I do not join the cadet unit. She got a medical certificate so that I should not have to join it. I don’t know whether she was a pacifist or whether she just thought that in the event of another war it would have some bearing on my joining up.

About then too, I realized that my point of view on the opposite sex wasn’t entirely like that of most of the

boys I knew. For that reason, I believe, I married at a much younger age than I would have, had it not been for my home training. My wife and I have been married for some thirty years now. She was the first girl that I ever took out. I had quite a heartache about her then because she wasn’t the type of girl that my mother wanted me to marry. In the first place, she had been married before; I was her second husband. My mother resented it so that the first Christmas after our marriage, which was in May of 1923, she didn’t even invite us to dinner. After our first child came my parents both became allies, but in later days, after I became an alcoholic, they both turned against me.

My father had come out of the South and had suffered a great deal down there. He wanted to give me the very best and he thought that nothing but being a doctor would suffice. On the other hand, I believe that I’ve always been medically inclined, though I have never been able to see medicine quite as the average person sees it. I do surgery because that’s something that you can see; it’s more tangible. But I can remember in post graduate days, and during internship, that very often I’d go to a patient’s bed and start a process of elimination and then, very often, I’d wind up guessing. That wasn’t the way it was with my father. I think, with him it possibly was a gift—intuitive diagnosis. Father, through the years, had built up a very good mail order business because, at that time, there wasn’t too much money in medicine.

I don’t think I suffered too much as far as the racial situation was concerned because I was born into it and knew nothing other than that. A man wasn’t actually mistreated, though if he was he could only resent it.

He could do nothing about it. On the other hand, I got quite a different picture farther south. Economic conditions had a great deal to do with it, because I’ve often heard my father say that his mother would take one of the old time flour sacks and cut a hole through the bottom and two corners of it and there you’d have a gown. Of course, when father finally came to Virginia to work his way through school, he resented the southern “cracker,” as he often called them, so much that he didn’t even go back to his mother’s funeral. He said he never wanted to set foot in the deep south again and he didn’t.

I went to elementary and high school in Washington, D.C. and then to Howard University. My internship was in Washington. I never had too much trouble in school. I was able to get my work out. All my troubles arose when I was thrown socially among groups of people. As far as school was concerned, I made fair grades throughout.

This was around 1935, and it was about this time that I actually started drinking. During the years 1930 to 1935, due to the Depression and its aftermath, business went from bad to worse. I had my own medical practice in Washington at that time, but the practice slackened and the mail order business started to fall off. Dad, due to having spent most of his time in a small Virginia town, didn’t have any too much money, and the money he had saved and the property he had acquired were in Washington. He was in his late fifties, and all that he had undertaken fell upon my shoulders at his death in 1928. For the first couple of years it wasn’t too bad because the momentum kept things going. But when things became crucial,

everything started going haywire and I started going haywire with them. At this point I believe I had only been intoxicated on maybe three or four occasions, and certainly whiskey was no problem to me.

My father had purchased a restaurant, which he felt would take up some of my spare time, and that’s how I met Vi. She came in for her dinner. I’d known her five or six months. To get rid of me one evening, she decided to go to the movies, she and another friend. A very good friend of mine who owned a drug store across the street from us, came by only about two hours later and said that he had seen Vi down town. I said that she told me she was going to the movies, and I became foolishly disturbed about it and, as things snowballed, I decided to go out and get drunk. That’s the first time I was ever really drunk in my life. The fear of the loss of Vi and the feeling that, though she had the right to do as she pleased, she should have told me the truth about it, upset me. That was my trouble. I thought that all women should be perfect.

I don’t think I actually started to drink pathologically until approximately 1935. About that time I had lost practically all my property except the place we were living in. Things had just gone from bad to worse. It meant that I had to give up a lot of the things that I had been accustomed to and that wasn’t the easiest thing in the world for me. I think that was basically the thing that started me drinking in 1935. I started drinking alone then. I’d go into my home with a bottle, and I remember clearly how I would look around to see if Vi was watching. Something should have told me then that things were haywire. I can remember her watching. There came a time when

she spoke to me about it, and I would say that I had a bad cold or that I wasn’t feeling well. That went on for maybe two months, and then she got after me again about drinking. At that time the repeal whiskies were back, and I’d go to the store and buy my whiskey and take it to my office and put it under the desk, first in one place and then in another, and there soon was an accumulation of empty bottles. My brother-in-law was living with us at that time and I said to Vi, “Maybe the bottles are brother’s. I don’t know. Ask him about it. I don’t know anything about the bottles.” I actually wanted a drink, besides feeling that I had to have a drink. From that point on, it’s just the average drinker’s story.

I got to the place where I’d look forward to the week-end’s drinking and pacify myself by saying that the week-ends were mine, that it didn’t interfere with my family or with my business if I drank on the week-ends. But the week-ends stretched on into Mondays, and the time soon came when I drank every day. My practice at that juncture was just barely getting us a living.

A peculiar thing happened in 1940. That year, on a Friday night, a man whom I had known for years came to my office. My father had treated him many years prior to this. This man’s wife had been suffering for a couple of months and when he came in he owed me a little bill. I filled a prescription for him. The following day, Saturday, he came back and said, “Jim, I owe you for that prescription last night. I didn’t pay you.” I thought, “I know you didn’t pay me, because you didn’t get a prescription.” He said, “Yes. You know the prescription that you gave me for my wife

last night.” Fear gripped me then, because I could remember nothing about it. It was the first blackout I had to recognize as a blackout. The next morning I carried another prescription to this man’s house and exchanged it for the bottle his wife had. Then I said to my wife, “Something has to be done.” I took that bottle of medicine and gave it to a very good friend of mine who was a pharmacist and had it analyzed, and the bottle was perfectly all right. But I knew at that point that I couldn’t stop and I knew that I was a danger to myself and to others.

I had a long talk with a psychiatrist, but nothing came of that, and I had also, just about that time, talked with a minister for whom I had a great deal of respect. He went into the religious side and told me that I didn’t attend church as regularly as I should, and that he felt, more or less, that this was responsible for my trouble. I rebelled against this, because just about the time that I was getting ready to leave high school, a revelation came to me about God and it made things very complicated for me. The thought came to me that if God, as my mother said, was a vengeful God, he couldn’t be a loving God. I wasn’t able to comprehend it. I rebelled, and from that time on I don’t think I attended church more than a dozen times.

After this incident in 1940, I sought some other means of livelihood. I had a very good friend who was in the Government service, and I went to him about a job. He got me one. I worked for the Government about a year and still maintained my evening office practice when the Government agencies were decentralized. Then I went South, because they told me that the particular county I was going to in North

Carolina was a dry county. I thought that this would be a big help to me. I would meet some new faces and be in a dry county.

But I found that after I got to North Carolina it wasn’t any different. The State was different, but I wasn’t. Nevertheless, I stayed sober there about six months, because I knew that Vi was to come later and bring the children. We had two girls and a boy at that time. Something happened. Vi had secured work in Washington. She also was in the Government service. I started inquiring where I could get a drink and, of course, I found that it wasn’t hard. I think whiskey was cheaper there than it was in Washington. Matters got worse all the time until finally they got so bad that I was reinvestigated by the Government. Being an alcoholic, slick, and having some good sense left, I survived the investigation. Then I had my first bad stomach hemorrhage. I was out of work for about four days. I got into a lot of financial difficulties too. I borrowed five hundred dollars from the bank and three hundred from the loan shop, and I drank that up pretty fast. Then I decided that I’d go back to Washington, which I did.

My wife received me graciously, although she was living in a one room with a kitchen affair. She’d been reduced to that. I promised that I was going to do the right thing. We were now both working in the same agency. I continued to drink. I got drunk one night in October, went to sleep in the rain and woke up with pneumonia. Nevertheless, we continued to work together and I continued to drink, but I guess, deep down within her heart as well as within mine, we both knew I couldn’t stop drinking. Vi thought

I didn’t want to stop. We had several fights and on one or two occasions I struck her with my fist. She decided that she didn’t want any more of that. So she went to court and talked it over with the judge. They cooked up a plan whereby she didn’t have to be molested by me if she didn’t want to be.

I went back to my mother’s for a few days until things cooled off, because the District Attorney had put out a summons for me to come to see him in his office. A policeman came to the door and asked for James S., but there wasn’t any James S. there. He came back several times. Within about ten days I got locked up for being drunk and this same policeman was in the station house as I was being booked. I had to put up a three hundred dollar bond because he was carrying the same summons around in his pocket for me. So I went down to talk to the District Attorney, and the arrangement was made that I would go home to stay with my mother, and that meant that Vi and I were separated. I continued to work and continued to go to lunch with Vi, and none of our acquaintances on the job knew that we had separated. Very often we rode to and from work together, but being separated really galled me deep down.

The November following, I took a few days off after pay day to celebrate my birthday on the 25th of the month. As usual I got drunk and lost the money. Someone had taken it from me. That was the usual pattern. I sometimes gave it to my mother and then I’d go back and hound her for it. I was just about broke. I guess I had five or ten dollars in my pocket. Anyhow, on the 24th, after drinking all day on the 23rd, I must have decided I wanted to see my wife

and have some kind of reconciliation or at least talk with her. I don’t remember whether I went by streetcar, whether I walked or went in a taxicab. The one thing I can remember now was that Vi was on the corner of 8th and L, and I remember vividly that she had an envelope in her hand. I remember talking to her, but what happened after that I don’t know. What actually happened was that I had taken a penknife and stabbed Vi three times with it. Then I left and went home to bed. Around eight or nine o’clock there came two big detectives and a policeman to arrest me for assault; and I was the most amazed person in the world when they said I had assaulted someone, and especially that I had assaulted my wife. I was taken to the station house and locked up. The next morning I went up for arraignment. Vi was very kind and explained to the jury that I was basically a fine fellow and a good husband, but that I drank too much and that she thought I had lost my mind and felt that I should be committed to an asylum. The judge said that if she felt that way he would confine me for thirty days’ examination and observation. There was no observation. There might have been some investigation. The closest I came to a psychiatrist during that time was an intern who came to take blood tests. After the trial, I got big-hearted again and felt that I should do something in payment for Vi’s kindness to me; so I left Washington and went to Seattle to work. I was there about three weeks, and then I got restless and started to tramp across the country, here and there, until I finally wound up in Pennsylvania, in a steel mill.

I worked in the steel mill for possibly two months,

and then I became disgusted with myself and decided to go back home. I think the thing that galled me was that just after Easter I had drawn my salary for two weeks’ work and had decided that I was going to send some money to Vi; and above all else I was going to send my baby daughter an Easter outfit. But there happened to be a liquor store between the post office and the mill, and I stopped to get that one drink. Of course the kid never got the Easter outfit. I got very little out of the two hundred, that I drew on that pay day.

I knew I wasn’t capable of keeping the bulk of the money myself, so I gave it to a white fellow who owned the bar which I frequented. He kept the money for me, but I worried him to death for it. Finally, I broke the last one hundred dollar bill the Saturday before I left. I got out of that bill one pair of shoes, and the rest of that money was blown. I took the last to buy my railroad ticket.

I’d been home about a week or ten days when one of my friends asked if I could repair one of his electrical outlets. Thinking only of two or three dollars to buy some whiskey, I did the job and that’s how I met Ella G., who was responsible for my coming into A.A. I went to this friend’s shop to repair his electrical outlet and I noticed this lady. She continued to watch me, although she didn’t say anything. Finally she said, “Isn’t your name Jim S.?” I said, “Yes.” Then she told me who she was. She was Ella G. When I had known her years before she was rather slender, but at this time she weighed as much as she does now, which is up around in the two hundreds or very close to it. I had not recognized her, but as soon as she said

who she was I remembered her right away. She didn’t say anything about A.A. or Charlie G. my sponsor at that time, but she did ask about Vi, and I told her Vi was working and how she could locate her. It was around noon, a day or two later, when the telephone rang and it was Ella. She asked me if I would let someone come up and talk to me concerning a business deal. She never mentioned anything about my whiskey drinking because if she had I would have told her, “No” right then. I asked her just what this deal was, but she wouldn’t say. She said, “He has something of interest, if you will see him.” I told her that I would. She asked me one other thing. She asked me if I would try to be sober if I possibly could. So I put forth some effort that day to try to stay sober if I could, though my sobriety was just a daze.

About seven that evening my sponsor walked in, Charlie G. He didn’t seem too much at ease in the beginning. I guess I felt, and he sensed it, that I wanted him to hurry up and say what he had to say and get out. Anyhow, he started talking about himself. He started telling me how much trouble he had, and I said to myself, “I wonder why this guy is telling me all his troubles. I have troubles of my own.” Finally, he brought in the angle of whiskey. He continued to talk and I to listen. After he’d talked half an hour, I still wanted him to hurry up and get out so I could go and get some whiskey before the liquor store closed. But as he continued to talk, I realized that this was the first time I had met a person who had the same problems I did and who, I sincerely believe, understood me as an individual. I knew my wife didn’t because I had been sincere in all my promises to her

as well as to my mother and to my close friends, but the urge to take that drink was more powerful than anything else.

After Charlie had talked a while, I knew that this man had something. In that short period he built within me something that I had long since lost, which was hope. When he left I walked with him to the streetcar line, which was just about a half a block, but there were two liquor stores, one on each corner from my home. I put Charlie on the car and when I left him I passed both of those liquor stores without even thinking about them.

The following Sunday we met at Ella G.’s. It was Charlie and three or four others. That was the next meeting, and the first meeting of a colored group in A.A., so far as I know. We held some two or three meetings at Ella’s home and from there we held some two or three at her mother’s home. Then Charlie or someone in the group suggested that we try to get a place in a church or hall to hold meetings. I approached several ministers and all of them thought it was a very good idea, but they never relinquished any space. So, finally, I went to the YMCA and they graciously permitted us to use a room at two dollars a night. At that time we had our meetings on Friday nights. Of course, it wasn’t very much of a meeting in the beginning; most of the time it was just Vi and myself. But, finally, we got one or two to come in and stick and from there, of course, we started to grow.

I haven’t mentioned it, but Charlie, my sponsor, was white, and when we got our group started we got help from other white groups in Washington. They came, many of them, and stuck by us and told us how

to hold meetings. They taught us a great deal about Twelve Step work too. Indeed, without their aid we couldn’t possibly have gone on. They saved us endless time and lost motion. And, not only that, but they gave us financial help. Even when we were paying that two dollars a night, they often paid it for us because our collection was so small.

At this time I wasn’t working. Vi was taking care of me and I was devoting all my time to the building of that group. I worked at that alone for six months. I just gathered up this and that alcoholic, because, in the back of my mind, I wanted to save all the world. I had found this new “something,” and I wanted to give it to everyone who had a problem. We didn’t save the world, but we did manage to help some individuals.

That’s my story of what A.A. has done for me.

\end{biblechapter}

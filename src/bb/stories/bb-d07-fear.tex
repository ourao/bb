\biblebook{Fear}


\bbChapterPreamble


\bbHeading{THE MAN WHO MASTERED FEAR}


\begin{biblechapter}
\verseWithHeading{Foreword}
    He spent eighteen years in running away; 
    and then found he didn’t have to run.
\verse So he started A.A. in Detroit.
\end{biblechapter}


\begin{biblechapter}
\verseWithHeading{Intro}
    FOR EIGHTEEN YEARS, from the age of twenty-one to thirty-nine, fear governed my life. By the time I was thirty I had found that alcohol dissolved fear. For a little while. In the end I had two problems instead of one: Fear and alcohol.

I came from a good family. I believe the sociologists would call it “upper middle class.” By the time I was twenty-one I had had six years of life in foreign countries, spoke three languages fluently, and had attended college for two years. A low ebb in the family fortunes necessitated my going to work when I was twenty. I entered the business world with every confidence that success lay ahead of me. I had been brought up to believe this, and I had shown during my ‘teens considerable enterprise and imagination about earning money. To the best of my recollection I was completely free from any abnormal fears. Vacations from school and from work, spelled “travel” to me—and I traveled with gusto. During my first year out of college I had endless dates, and went to countless dances, balls and dinner parties.

Suddenly all this changed. I underwent a shattering nervous breakdown. Three months in bed. Three more months of being up and around the house for brief periods and in bed the rest of the time. Visits

from friends which lasted over fifteen minutes exhausted me. A complete checkup at one of the best hospitals revealed nothing. I heard for the first time an expression which I was to grow to loathe: “There is nothing organically wrong.” Psychiatry might have helped, but psychiatrists had not penetrated the middle west.

Spring came. I went for my first walk. Half a block from my house, I tried to turn the corner. Fear froze me in my tracks, but the instant I turned back toward home this paralyzing fear left me. This was the beginning of an unending series of such experiences. I told our family doctor, an understanding man who gave hours of his time trying to help me, about this experience. He told me that it was imperative that I walk around the entire block, cost me what it might in mental agony. I carried out his instructions. When I reached a point directly back of our house, where I could have cut through a friend’s garden, I was almost overpowered by the desire to get home, but I made the whole journey. Probably only a few readers of this story will be able, from personal experiences of their own, to understand the exhilaration and sense of accomplishment which I felt after finishing this seemingly simple assignment.

The details of the long road back to something resembling normal living—the first short streetcar ride, the purchase of a used bike which enabled me to widen the narrow horizon of life, the first trip downtown—I will not dwell on. I got an easy, part-time job selling printing for a small neighborhood printer. This widened the scope of my activities. A year later I was able to buy a Model T roadster and take a better

job with a downtown printer. From this job and the next one with another printer I was courteously dismissed. I simply did not have the pep to do hard, “cold-turkey” selling. I switched into real estate brokerage and property management work. Almost simultaneously, I discovered that cocktails in the late afternoon and highballs in the evening relieved the many tensions of the day. This happy combination of pleasant work and alcohol lasted for five years. Of course, the latter ultimately killed the former, but of this, more anon.

All this changed when I was thirty years old. My parents died, both in the same year, leaving me, a sheltered and somewhat immature man, on my own. I moved into a “bachelor hall.” These men all drank on Saturday nights, and enjoyed themselves. My pattern of drinking became very different from theirs. I had bad nervous headaches, particularly at the base of my neck. Liquor relieved these. At last I discovered alcohol as a cure-all. I joined their Saturday night parties and enjoyed myself too. But—I also stayed up week nights after they had retired and drank myself into bed. My thinking about drinking had undergone a great change. Liquor had become a crutch on the one hand and a means of retreat from life on the other.

The ensuing nine years were the Depression years, both nationally and personally. With the bravery born of desperation, and abetted by alcohol, I married a young and lovely girl. Our marriage lasted four years. At least three of those four years must have been a living hell for my wife, because she had to watch the man she loved disintegrate morally, mentally and

financially. The birth of a baby boy did nothing toward staying the downward spiral. When she finally took the baby and left, I locked myself in the house and stayed drunk for a month.

The next two years were simply a long drawn out process of less and less work and more and more whiskey. I ended up, homeless, jobless, penniless and rudderless, as the problem guest of a close friend whose family was out of town. Haunting me through each day’s stupor—and there were eighteen or nineteen such days in this man’s home—was the thought: “Where do I go when his family comes home?” When the day of their return was almost upon me, and suicide was the only answer I had been able to think of, I went into Ralph’s room one evening and told him the truth. He was a man of considerable means and he might have done what many men would have done in such a case. He might have handed me fifty dollars and said that I ought to pull myself together and make a new start. I have thanked God many times in the last sixteen years that that was just what he did not do!

Instead, he got dressed, took me out, bought me three or four double shots and put me to bed. The next day he turned me over to a couple who, although neither was an alcoholic, knew Dr. Bob and were willing to drive me to Akron where they would turn me over to his care. The only stipulation they made was this: I had to make the decision myself. What decision? The choice was limited. To go north into the empty pine country and shoot myself, or to go south in the faint hope that a bunch of strangers might help me with my drinking problem. Well, suicide was a last straw matter, and I had not drawn the last straw

yet. So I was driven to Akron the very next day by these Good Samaritans, and turned over to Dr. Bob and the then tiny Akron Group.

Here, while in a hospital bed, men with clear eyes, happy faces, and a look of assurance and purposefulness about them, came to see me and told me their stories. Some of these were hard to believe, but it did not require a giant brain to perceive that they had something I could use. How could I get it? It was simple, they said, and went on to explain to me in their own language the program of recovery and daily living which we know today as the Twelve Steps of A.A. Dr. Bob dwelt at length on how prayer had given him release, time and time again, from the nearly overpowering compulsion to take a drink. He it was who convinced me, because his own conviction was so real, that a Power greater than myself could help me in the crises of life, and that the means of communicating with this Power was simple prayer. Here was a tall, rugged, highly educated Yankee talking in a matter of course way about God and prayer. If he and these other fellows could do it, so could I.

When I got out of the hospital, I was invited to stay with Dr. Bob and his dear wife, Anne. I was suddenly and uncontrollably seized with the old, paralyzing panic. The hospital had seemed so safe. Now I was in a strange house, in a strange city, and fear gripped me. I shut myself in my room, which began to go around in circles. Panic, confusion and chaos were supreme. Out of this maelstrom just two coherent thoughts came to the surface; one, a drink would mean homelessness and death; two, I could no longer relieve the pressure of fear by starting home, as was once my

habitual solution to this problem, because I no longer had a home. Finally, and I shall never know how much later it was, one clear thought came to me: Try prayer. You can’t lose, and maybe God will help you—just maybe, mind you. Having no one else to turn to, I was willing to give Him a chance, although with considerable doubt. I got down on my knees for the first time in thirty years. The prayer I said was simple. It went something like this: “God, for eighteen years I have been unable to handle this problem. Please let me turn it over to you.”

Immediately a great feeling of peace descended upon me, intermingled with a feeling of being suffused with a quiet strength. I lay down on the bed and slept like a child. An hour later I awoke to a new world. Nothing had changed and yet everything had changed. The scales had dropped from my eyes and I could see life in its proper perspective. I had tried to be the center of my own little world, whereas God was the center of a vast universe of which I was perhaps an essential, but a very tiny, part.

It is well over sixteen years since I came back to life. I have never had a drink since. This alone is a miracle. It is, however, only the first of a series of miracles which have followed one another as a result of my trying to apply to my daily life the principles embodied in our Twelve Steps. I would like to sketch for you the highlights of these sixteen years of a slow but steady and satisfying upward climb.

Poor health and a complete lack of money necessitated my remaining with Dr. Bob and Anne for very close to a year. It would be impossible for me to pass over this year without mentioning my love for, and my

indebtedness to these two wonderful people who are no longer with us. They made me feel as if I were a part of their family, and so did their children. The example which they and Bill W., whose visits to Akron were fairly frequent, set me of service to their fellow men imbued me with a great desire to emulate them. Sometimes during that year I rebelled inwardly at what seemed like lost time, and at having to be a burden to these good people whose means were limited. Long before I had any real opportunity to give, I had to learn the equally important lesson of receiving graciously.

During my first few months in Akron I was quite sure that I never wanted to see my home town again. Too many economic and social problems would beset me there. I would make a fresh start somewhere else. After six months of sobriety I saw the picture in a different light: Detroit was the place I had to return to, not only because I must face the mess I had made there, but because it was there that I could be of the most service to A.A. In the spring of 1939, Bill stopped off in Akron on his way to Detroit on business. I jumped at the suggestion that I accompany him. We spent two days there together before he returned to New York. Friends invited me to stay on for as long as I cared to. I remained with them for three weeks, using part of the time in making many amends, which I had had no earlier opportunity of making.

The rest of my time was devoted to A.A. spadework. I wanted “ripe” prospects and I didn’t feel that I would get very far chasing individual drunks in and out of bars. So I spent much of my time calling on the people who I felt would logically come in contact

with alcoholic cases, doctors, ministers, lawyers and the personnel men in industry. I also talked A.A. to every friend who would listen, at lunch, at dinner, on street corners. A doctor tipped me off to my first prospect. I landed him and shipped him by train to Akron, with a pint of whiskey in his pocket to keep him from wanting to get off the train in Toledo! Nothing has ever to this day equaled the thrill of that first case.

Those three weeks left me completely exhausted and I had to return to Akron for three more months of rest. While there, two or three more “cash customers” (as Dr. Bob used to call them—probably because they had so little cash) were shipped in to us from Detroit. When I finally returned to Detroit to find work and to learn to stand on my own feet, the ball was already rolling, however slowly. But it took six more months of work and disappointments before a group of three men got together in my rooming house bedroom for their first A.A. meeting.

It sounds simple, but there were obstacles and doubts to overcome. I well remember a session I had with myself soon after I returned. It ran something like this: If I go around shouting from the rooftops about my alcoholism, it might very possibly prevent me from getting a good job. But—supposing that just one man died because I had, for selfish reasons, kept my mouth shut? No. I was supposed to be doing God’s will, not mine. His road lay clear before me, and I’d better quit rationalizing myself into any detours. I could not expect to keep what I had gained unless I gave it away.

The Depression was still on and jobs were scarce. My health was still uncertain. So I created a job for

myself selling women’s hosiery and men’s made-to-order shirts. This gave me the freedom to do A.A. work, and to rest for periods of two or three days when I became too exhausted to carry on. There was more than one occasion when I got up in the morning with just enough money for coffee and toast and the bus fare to carry me to my first appointment. No sale—no lunch. During that first year, however, I managed to make both ends meet, and to avoid ever going back to my old habit ends meet, and to avoid ever going back to my old habit pattern of borrowing money when I could not earn it. Here by itself was a great step forward.

During the first three months I carried on all these activities without a car, depending entirely on buses and streetcars—I, who had always to have a car at my immediate command. I, who had never made a speech in my life and who would have been frightened sick at the prospect, stood up in front of Rotary groups in different parts of the city and talked about Alcoholics Anonymous. I, carried away with the desire to serve A.A., gave what was probably one of the first radio broadcasts about A.A., living through a case of mike fright and feeling like a million dollars when it was all over. I lived through a week of the fidgets because I had agreed to address a group of alcoholic inmates in one of our state mental hospitals. There was the same reward—exhilaration at a mission accomplished. Do I have to tell you who gained the most out of all this?

Within a year of my return to Detroit, A.A. was a definitely established little group of about a dozen members and I too was established in a modest but steady job handling an independent dry-cleaning route of my own. I was my own boss. It took five years of A.A. living, and a substantial improvement in my

health, before I could take a full-time office job where someone else was boss.

This office job brought me face to face with a problem which I had sidestepped all my adult life, lack of training. This time I did something about it. I enrolled in a correspondence school which taught nothing but accounting. With this specialized training, and a liberal business education in the school of hard knocks, I was able to set up shop some two years later as an independent accountant. Seven years of work in this field brought an opportunity to affiliate myself actively with one of my clients, a fellow A.A. We complement each other beautifully, as he is a born salesman and my taste is for finance and management. At long last I am doing the kind of work I have always wanted to do, but never had the patience and emotional stability to train myself for. The A.A. program showed me the way to come down to earth, start from the bottom and work up. This represents another great change for me. In the long ago past I used to start at the top as president or treasurer and end up with the sheriff breathing down my neck.

So much for my business life. Obviously I have overcome fear to a sufficient degree to think in terms of success in business. With God’s help I am able, for one day at a time, to carry business responsibilities which, not many years ago, I would not have dreamed of assuming. But what about my social life? What about those fears which once paralyzed me to the point of my becoming a semi-hermit? What about my fear of travel?

It would be wonderful were I able to tell you that my confidence in God and my application of the

Twelve Steps to my daily living have utterly banished fear. But this would not be the truth. The most accurate answer I can give you is this: Fear has never again ruled my life since that day in September, 1938, when I found that a Power greater than myself could not only restore me to sanity but could keep me both sober and sane. Never in sixteen years have I dodged anything because I was afraid of it. I have faced life instead of running away from it.

Some of the things which used to stop me in my tracks from fear still make me nervous in the anticipation of their doing, but once I kick myself into doing them nervousness disappears and I enjoy myself. In recent years I have had the happy combination of time and money to travel occasionally. I am very apt to get into quite an uproar for a day or two before starting, but I do start, and once started, I have a swell time.

Have I ever wanted a drink during these years? Only once did I suffer from a nearly overpowering compulsion to take a drink. Oddly enough, the circumstances and surroundings were pleasant. I was at a beautifully set dinner table. I was in a perfectly happy frame of mind. I had been in A.A. a year, and the last thing in my mind was a drink. There was a glass of sherry at my place. I was seized with an almost uncontrollable desire to reach out for it. I shut my eyes and asked for help. In fifteen seconds or less, the feeling passed. There have also been numerous times when I have thought about taking a drink. Such thinking usually began with thoughts of the pleasant drinking of my youth. I learned early in my A.A. life that I could not afford to fondle such thoughts, as you might fondle a pet, because this particular pet could

grow into a monster. Instead, I quickly substitute one or another vivid scene from the nightmare of my later drinking.

Twenty-odd years ago I made a mess out of my one and only marriage. It was therefore not extraordinary that I should shy away from any serious thought of marriage for a great many years after joining A.A. Here was something requiring a greater willingness to assume responsibility, and a larger degree of cooperation and “give and take” than even business requires of one. However, I must have felt, deep down inside myself, that living the selfish life of a bachelor was only half living. By living alone you can pretty much eliminate grief from your life, but you also eliminate joy. At any rate the last great step toward a well rounded life still lay ahead of me. So six months ago I acquired a ready made family consisting of one charming wife, four grown children to whom I am devoted, and three grandchildren. Being an alcoholic, I couldn’t dream of doing anything by halves! My wife, a sister member in A.A., had been a widow nine years and I had been single eighteen years. The adjustments in such a case are difficult and take time, but we both feel that they are certainly worth it. We are both depending upon God and our use of the A.A. program to help us make a success of this joint undertaking.

It is undoubtedly too soon for me to say how much of a success I shall be as a husband in time to come. I do feel, though, that the fact that I finally grew up to a point where I could even tackle such a job is the apex of the story of a man who spent eighteen years running away from life.
\end{biblechapter}

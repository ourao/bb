\biblebook{Farm}

FROM FARM TO CITY

She tells how A.A. works when the going is rough.

A pioneer woman member of A.A.’s first Group.


      I COME FROM a very poor family in material things, with a fine Christian mother, but with no religious background. I was the oldest in a family of seven, and my father was an alcoholic. I was deprived of many of the things that we feel are important in life, such as education particularly, because of my father’s drinking. Mine was far from a happy childhood. I had none of those things that children should have to make them happy.

We moved in from the country at the age when girls want all sorts of nice things. I remember starting to city school, coming from a country school, and wanting so very, very much to be like the other girls and trying flour on my face for powder because I wasn’t able to have any real powder. I remember feeling that they were all making fun of me. I feared that I wasn’t dressed like the rest. I know that one of the outfits I had was a skirt and a very funny looking blouse that my mother had picked up at a rummage sale. I look back and remember these things because they made me very unhappy, and added to my feeling of inferiority at never being the same as other people.

At the age of sixteen, I was invited to spend the summer with an aunt and I, very delightedly, accepted

the invitation. It was a small town—Liberty, Indiana. When I came to my aunt, she knew that I had had an unhappy childhood, and she said, “Now, Ethel, you’re welcome to have boy friends in our home, but there are two boys in this town that I don’t want you to date, and one of them comes from a very fine family, one of the best. But he’s in all sorts of scrapes because he drinks too much.” Four months later, I married this guy. I’m sure his family felt that it was a marriage that—well, I was a girl from the wrong side of the tracks—definitely!

I felt that his family were accepting me because it was good sense. I could do something for their Russ. But they didn’t do anything for me to build up my ego. And Russ didn’t tell me he’d stop drinking, and he certainly didn’t stop. It went on and grew worse and worse. We had two daughters. I was sixteen when we were married and he was seven years older. I remember one instance when he took off and went down to Cincinnati and was gone for a week on a drunk.

Finally, it got so bad that I left him and went back home and took my two children with me. I didn’t see him for a year, or even hear from him. That was seven or eight years after we were married. I was still bitter because I felt that drink had completely ruined my childhood and my married life, and I hated everything pertaining to it. I was about twenty-five then, and I had never touched a drop.

I got a job in the woolen mills in Ravena—very hard work. I looked much older than I was, I was always large, and I went back to work in this job. I kept my children with me. At the end of a year, the children got a card from their father, which I still have and

cherish. He said, “Tell Mommy I still love her.” I had gone to an attorney to see about getting a divorce during that year.

Then he came into town on the bum. He had taken up light work, and he had a safety valve and a pair of spurs and the clothes on his back, and that was all. I welcomed him with open arms. I didn’t realize how I still felt about him. He told me that he would never drink again. And I believed him. As many times as he would tell me that, I still believed him. Partially so, anyway. He got a job and went back to work.

He stayed “dry” for thirteen years! Dr. Bob often said that it was a record for what he felt was a typical alcoholic.

We built up a splendid life. At the end of those thirteen years I never dreamed that he’d ever take another drink. I had never taken one. Our oldest daughter got married; they were living at our house. Our other daughter was in her last year of high school, and one night the new son-in-law and my husband went out to the prize fight. I never was concerned anymore, anywhere he went. He hardly ever went to anything like that without me. We were together all the time, but this night I got up and saw it was late. I heard my son-in-law coming upstairs, and I asked him where dad was. He had a very peculiar look on his face, and he said, “He’s coming.” He was coming, on his hands and knees, up the stairs. As I look back, I was very broken up about it. But I don’t believe now that it was with any deep feeling of resentment that I said to him, “The children are raised, and if this is the way you want it, this is the way we’ll have it. Where

you go, I’ll go, and what you drink, I’ll drink.” That’s when I started drinking.

We were the most congenial drinkers you ever saw. We never rowed or fought. We had the grandest time ever. We just loved it. We’d start out on the craziest trips. He’d always say, “Take me for a ride, Ma.” So, sometimes we’d end up in Charleston, West Virginia, or here or there, drinking all along the way. These vacations became quite something, and he always had two weeks vacation the first two weeks of every September.

One year we got as far as Bellaire, Ohio. We always started out on the Saturday before Labor Day. I’m pretty near afraid of Labor Day yet. One Sunday afternoon, the only time I ever got picked up for drunken driving, I got picked up in Bellaire. They threw us in the jail. I wasn’t nearly in the condition I had been in many times to be picked up. I really wasn’t very high. They called the Mayor in so we wouldn’t have to stay in there over the holiday. He took his one hundred and seventeen dollars and let us go, and we proceeded. That to me was the greatest humiliation, to think that I’d finally landed in jail. My husband said that I said, “Can you imagine them giving us that jail fare?” And he said, “What jail fare?” And I said, “Well, they brought a pitcher of coffee, and a sandwich wrapped up for me.” And he said, “That wasn’t jail fare. They didn’t give me anything to eat. Somebody must have taken pity on you and gone out and got it for you.” And another thing, it’s a wonder they didn’t throw us back in because I could become very dignified and sarcastic. As we left, and they were escorting us across the bridge into Wheeling, I,

with great dignity and sarcasm, told them if their wives were ever visiting in Akron, and they, too, were looking for their route signs as I was, that I hoped that I could extend to them the hospitality that had been shown to me in Bellaire.

The next time vacation time rolled around that was a bitter lesson to us. Of course, this year we were drinking heavier and heavier and we decided on staying home and being sensible, doing a little drinking, and painting the house. So, on that Saturday before Labor Day, I got drunk and set the house on fire—so we didn’t have to paint it. I think that was the last vacation before sobriety.

I hated myself worse and worse, and as I hated myself I became more defiant towards everything and everybody. We drank with exactly the same accord that we finally accepted A.A. We comforted each other.

My defiant attitude became worse. There was a very religious family that lived down the road from us, and we were on the same party line. I’d hear them on the phone having prayer meetings and so forth, that sort of talk over the phone, and it completely burnt me up. They used a sound truck some. It would stop out in front of our house, and I still believe those people sent it! They’d sit out there and play hymns and I’d be lying in there with a terrific hangover. If I’d had a gun I’d have shot the horns right off the thing, because it made me raving mad.

It was just about this time, in 1940, that we met up with A.A. Russ read a piece in the paper, and he kind of snickered, and said, “See here, where John D. has found something to keep him from drinking!” “What’s

that?” I said. “Oh, some darn thing they’ve got here in the paper about it.” We talked about it afterwards, and we felt that there might be some time we’d need it. It was a thought that there might be some hope for us.

One morning after a terrific drinking bout, I was in a little bar near our house, and I shook so that I was very much ashamed, because I was getting the shakes worse and worse. I sipped the drink off the bar because I couldn’t hold it in my hand, but I was still a lady, believe it or not, and I was deeply ashamed. There was a man watching, and I turned to him and said, with a defiant air I carried with me all the time, “If I don’t quit this I’m going to have to join that alcoholic business they’re talking about.” He said, “Sister, if you think you’re a screwball now, all you have to do is join up with that. I’ll get you the password, and I can find out where they meet because I know a guy that belongs. But they are the craziest bunch! They roll on the floor and holler, and pull their hair.” “Well, I’m nuts enough now,” I said to him. But right then the hope died that had been in my heart when we read about John D.

Time went on and the drinking got worse and worse, and I was in another barroom, down the road the other way, a small one, and I took my glass—that morning I’d been able to lift it from the bar—and I said to the woman behind the bar, “I wish I might never take another drop of that stuff. It’s killing me.” She said, “Do you really mean that?” I said, “Yes.” She said, “Well, you better talk to Jack.” (Jack was the owner of the place. We always tried to buy him a drink, and he always told us he had liquor trouble—couldn’t drink.)

She said to me, “You know, he used to own the Merry-Go-Round. He used to drink, and then he found something that started up in Akron that helped him quit drinking.” Right away, I saw it was the same outfit this other guy told me about, and then again hope died.

Finally, one morning, I got up and got in the car and cried all the way down to the M.’s—the people who owned the bar—and told her I was licked and wanted help. I thought, “No matter how crazy they are I’ll do anything they say to do.” I drove these three or four miles down the road only to find that Jack was out. (This was funny. They owned this joint, she ran it, and he sold for a brewery. That was his job. And he’d been dry a year. I don’t think Jack was hospitalized. I think his entry into A.A. was through spending some hours with Dr. Bob at his office. He brought many people into A.A. through his barroom.) Mrs. M. said she would send Jack over as quick as he came in.

He came with two cans of beer. He gave my husband one and me one about ten-thirty on the eighth day of May in 1941. He said, “There’s a doctor here in Akron. I’m going in to see him, and see what can be done.” Dr. Bob was in Florida, but Jack didn’t know that.

That was our last drink of anything alcoholic. That nasty little can of beer! At two-forty-five that morning I thought I would die. I lay across the bed on my stomach with nothing but pain and sickness. I was scared to death to call a doctor. I thought when people did what we did that they just locked them up. I

didn’t know that anything was ever done for them in a medical way. So I stayed awake.

Men from A.A. started coming out to the house the next day. I paced the floor with a bath towel around my shoulders, the perspiration running off me. An attorney sat at the side of the bed where I was lying, and he sat on the edge of his chair and looked as innocent as a baby. I thought, “That guy never could have been drunk.” He said, “This is my story,”—real prim. And I thought, “I bet he’s a sissy. I bet he never drank.” But he told a story of drinking that was amazing to me.

Jack brought the Saturday Post with Jack Alexander’s story. He said, “Read this.” Jack didn’t seem to have too much of the spiritual understanding. He said, “I think this will tell you more. This is based, really, on the Sermon on the Mount. Now, if you’ve got a Bible around…” One of our gifts from the family was a very lovely Bible, but we’d let the bulldog chew it because we weren’t too interested in it. I had a little Testament, which was very small print. When you have a hangover and can’t even sit still, try to read small print! Russ said, “Mother, if this tells us how to do it, you’ll have to read it.” And I’d try, but I couldn’t even see the letters. But it was so important that we do the things we were told to do! Jack said there was a meeting in Akron every Wednesday night and that it was very important that we go. Jack said, “Now you start and go to these meetings, and then you’ll find out all about it.” I don’t think that there was anything said about religion. I didn’t know anything about the Sermon on the Mount.

I had the Big Book (Alcoholics Anonymous) that

had been brought to me. Paul S. had just called me, and I remember he stressed reading the Big Book. I was reading it for all that was in it, and I said to Russ, “We can’t do this. We couldn’t begin to.” And Jim G. had such a wonderful sense of humor, and when he came I was in tears, and I told him, “I want to do this, but I can’t. This is too much. I could never go and make up to all the people I’ve done wrong to.” He said, “Let’s put the Big Book away again, and when you read it again, turn to the back and read some of the stories. Have you read those?” No, I was all interested in this part that told you how to do it. That was the only part I was interested in. And then he got us to laugh, which was what we needed. When we went to bed my sides ached, and I said to my husband, “I thought I would never laugh again, but I have laughed.”

“Well,” I said to dad, when the A.A. people kept coming with these lovely cars and looked so nice, “I suppose the neighbors say, ‘ Now those old fools must have up and died, but where’s the hearse?'”

On Wednesday night Jack M. said, “You meet me at the Ohio Edison Building, and I will take you to the meeting.” And we went down through the valley, and I remember ed reading about the Ku Klux Klan and how they burnt crosses, and I thought, “God alone knows what we are getting into this time!” I didn’t know what they were going to do because he didn’t tell us much. So we came to King’s School. And they introduced me to Miriam and Annabelle. They told Annabelle to take me under her wing, and I shall never forget how she sort of curled up her nose and said, “They tell me you drink too.” I often think how

that could have turned some people away, because there were no other women alcoholics there then. And I said, “Why sure, that’s what I’m here for.” And I was glad, and I have been ever since, that I said that. And I wasn’t resentful toward her, either.

There was a young fellow who led the meeting and that was a beautiful thing to me. He talked about his wife taking his little boy away from him because of his drinking, and how he got back together with them through A.A., and we began feeling grateful right then that all these things hadn’t been taken from us. They opened with a little prayer, and I thought it was very fine that we stood, all of us together, and closed with the Lord’s Prayer.

I’d like to say here how important it was to us then that we do all the little things that people said were important, because later when Russ was so sick that I had to hold him up, they had a meeting out at the house. When we closed the meeting with the Lord’s Prayer, Russ said, “Mother, help me stand.” This was after his illness. We were in A.A. three and a half years when he was taken from me. We had never missed a Wednesday night at King’s School for a year. We had that record.

I always feel that our God consciousness was a steady growth after we became associated with A.A. And we loved every minute of that association. We had big picnics out at the house with A.A. We had meetings at each other’s homes and, of course, that was a grand place for people to get together out there; they seemed to think so too.

I give a great deal of credit to Doc and Anne for changing our life. They spent at least an evening a

week in our home out there for weeks and weeks. Sometimes saying very little, but letting us say. Russ used to be very much pleased because he’d say, “I think Dr. Bob thoroughly enjoys coming out here. He can relax and it’s quiet.”

At that time they didn’t let us know that people ever had trouble. I mean slips. I remember sometime, it was possibly six months after we had been going steadily to King ‘s School, that we were coming home from a meeting and saw a car along the way, and a fellow in back drinking a bottle of beer. And Russ said, “I would have sworn that was Jack M.” The next morning his wife came dragging him in before Russ went to work, while I was getting breakfast. It had been Jack M. We wept and Russ didn’t go to work.

Jack had been sober about a year and a half. His wife was cussing him, raving at him, “I just brought him over to show you what kind of a guy he is! He wants to go to the hospital, and I’m not paying for the hospital again!” We were so mad at her because she talked to him that way. Russ said, “Don’t do another thing today but help him. Do something for him! If he thinks he needs to go to the hospital, I’ll pay for him.” She said, “He’s not going to the hospital, whether you pay for it or I pay for it, he’s not going!”

In the spiritual strength I had found, because of A.A., I finally felt that I had made a complete surrender, that I had really turned my life over that summer. I thought I had done that until Russ’ second collapse, and the doctor told me very candidly that he wasn’t long for this world. I knew then that I hadn’t made a complete surrender, because I tried to bargain with

the God I had found, and I said, “Anything but that! Don’t do that to me!”

Russ lived a year longer than they expected him to live, and in that year he was in bed for at least six months. I can’t express what A.A. meant to us during that year. Before the end finally came, I had, I guess, made the surrender because I finally had been able to say that I would not mind too much. And I realize that there was one salvation for me. Thank God I had no desire for a drink when he died.

There were two women in the St. Thomas Hospital at that time in a room. (Russ was buried on Friday, and on Sunday afternoon Hilda S. had invited me there to dinner Sunday night, and I didn’t think I could do it. I knew Doc and Anne were going to be there, and all of them thought it would be good for me, but the first thing I did was to go to St. Thomas and try to talk to those women.) I sat down on the side of one of their beds, and I started to weep, and I couldn’t stop, and I was so started to weep, and I couldn’t stop, and I was startled, and I apologized again and again for it. And that woman told me long after that was the surest proof to her that this program could work. If, on Sunday, I could be there, trying to think of something that would help her with this problem, then we must have something that could work. I felt it certainly must be very depressing to her that I should sit there by her bedside and cry.

I feel that one of the things that I still have to guard against is that I used to be set in my way about what I considered the old-time A.A. I have to tell myself, “Other things are progressing and A.A. must too.” We old-timers who get scattered and separated and then witness the construction of services to get in more

people and to make this thing function, we think that A.A. has changed, but the root of it hasn’t. We are older in A.A., and we’re older in years. It’s only natural that we don’t have the capacity to change, but we ought not to criticize those who have.

There’s another thing I would stress. I think it’s awfully hard on people, especially if they’re new people, to hear these long drawn – out talks. I don’t ever remember that I was bored myself when we first came in, and they came out to the house and talked to us about these things. I ate up every bit of it, because I wanted to find out how to stay sober.

Before I stop—I always was a great talker—I want to say that nobody will ever know how I miss Annie’s advice about things. I would get in the biggest dither about something. I hadn’t been in too long when one of the men’s wives called me one Sunday and told me she didn’t think I had any part of the program. Well, I wasn’t sure I did, and it was awful foggy, and I wept and asked her what she thought I ought to do about it. She said she didn’t know, but that I sure showed plain enough I didn’t have any part of it. I didn’t think I was going to get drunk right then, but I remember how comforting it was then, but I remember how comforting it was when I called Anne and told her. I was crying, and I said, “Alice says she knows I don’t have any part of the program.” She talked to me and laughed about it and got me all over it. Another thing that was helpful to me. I used to think I was cowardly because when things came out pertaining to the program that troubled me, I said to her many times, “Annie, am I being a coward because I lay those things away on the shelf and skip it?” She said, “I feel you’re just being wise. If it isn’t anything that’s

going to help you or anybody else, why should you become involved in it, and get all disturbed about it?”

So there you are. That’s my story. I know I’ve talked too long, but I always do. And, anyhow, if I went on for ten or a hundred times as long I couldn’t even begin to tell you all that A.A. has meant to me.

\biblebook{Gentleman}

\begin{biblechapter}


HE THOUGHT HE COULD DRINK

LIKE A GENTLEMAN

      I WAS BORN in Cleveland, Ohio, in 1889, the last child of a family of eight children. My parents were hard working people. My father was a railroad man and a Civil War veteran. I can remember that in my childhood, he was ill at ease with the children because he attempted to assert an army discipline that had been ground into him during his three and a half years of army service. The differences between my father and my sisters, who were school teachers, made an excellent environment for the type of child I was—that is, slick and cute enough to take advantage of any adult quarrel. In other words, I was always safe from the discipline of my father, and, having developed along that line, I had considerable difficulty in school. Rules were made for others, but not for me. Of course, it was always my aim to have my own way without being caught.

My mother was eighty-nine years old when she died, and I was a full-blown alcoholic at the time of her death. She was a woman devoted to her family and loyal to her husband, but quarrels did not make a happy environment for her. I had four brothers and three sisters. As I look back, all the brothers devel-

oped personality problems. The sisters seemed to remain unaffected. I seemed to react by developing a streak of varying meanness, which would cause me to do things to create excitement and to get attention. I very early sampled the effects of alcohol. In fact, on one occasion, I was picked up by the police and brought home. I was then about sixteen years old. I didn’t go to high school. I went to five grade schools, primarily because I was expelled for my conduct, but I eventually graduated from the eighth grade.

I was always interested in mechanics and after having about twenty miscellaneous jobs, lasting from one day to two weeks, I obtained a job as a toolmaker apprentice. Being intensely interested in the work, I changed my conduct sufficiently to master the job. I finished my apprenticeship and was moved into the drafting department. That was in Cleveland. As a draftsman, I worked for several large companies and gained a variety of experience. Not far from where I lived they built a new technical high school and one of the teachers sold me on the idea that I needed a little mechanical drawing if I were going to be a good toolmaker. I proceeded to take up the drawing and advanced rapidly in it. The school then obtained a job for me in the drafting department of another company. After I was on the drafting board about two years I decided I wanted a technical education. I was then about eighteen. I did not have a high school education so I went to night school to take the full high school course, which I finished in two years and nine months. I apparently was willing to subdue these personality disturbances in a tremendous drive to succeed. I had an objective. I could discipline myself but along

the way there would be festivities and occasions when I got drunk. Although, during this period I was not addicted to any pattern of alcohol consumption, when I did drink the drinking was pretty wild.

I then entered Case School and worked while I went to school and finished there. This was an engineering college. Following graduation, I was offered a pretty good job which I took. In the fall of the graduating year, I became involved in some litigation over the ownership of inventions and patents. This experience sent me to law school where I went at night and which course I completed in less than three years, taking the highest state bar examinations and passing them. The law school experience was not inspired by a desire to follow patent law, which has been my profession since: I went to law school primarily to learn the law of contracts following my own experience with litigation. A year later, after I completed the course in contracts, I quit the law school, and undertook some engineering work for a patent law firm on behalf of clients who were in diffculties and who desired to keep their troubles from their own engineering department. This work consumed a period of about two-thirds of a year, and worked out successfully so I decided to follow patent law. I went back to law school, and doubled up on the courses because I was then approaching thirty years of age and I wanted to get through as quickly as possible. I was supporting myself through all this education by being a toolmaker and a draftsman.

I married when I was twenty-eight years old, and started in law school after I was married. As a matter

of fact, I had two children at the time I was admitted to the bar.

I kept myself so busy that, outside of some school and group parties, I didn’t go overboard on drinking very much between the ages of twenty-five and thirty. My life was fairly well crowded and I didn’t seem to need any stimulants to keep me going. By the time I had completed law school I had picked up some experience in patent law, for I remained with the patent law firm and worked too in Washington where they found that I was a capable infringement investigator. In 1924, I had acquired enough clients of my own so that the firm made me a junior partner. My drinking career began about four years after I had moved up into partnership and had joined certain clubs, societies and so forth, and during which period we had Prohibition. I was then about thirty-seven or thirty-eight.

All during Prohibition, every alcoholic felt that he made the best hootch, regardless of how bad it was. I became a specialist in making elderberry blossom wine.

There had been some occasions—there was an automobile wreck, for instance—when I had police escort home but not to jail, all of which, instead of doing me a favor, did me harm because I was then full of self-esteem as to the progress I had made both professionally and financially. The first definite indications of an alcoholic pattern began to arise when I would go to New York on business and disappear, and wind up in Philadelphia or Boston for two or three day periods. I would have to return to New York and pick up my bills and bags. These periods became more frequent

and I resolved that when I became forty, which was very shortly, I was going on the wagon. Forty came and went and then the resolution was advanced to forty-one, forty-two, and so on in the usual pattern. I realized that I had a problem, although my realization was not very deep because my own conceit wouldn’t admit that I had any personality problems. I could not see why I couldn’t drink like a gentleman, and that was my primary ambition—until I landed in A.A. This pattern deepened and became worse. I became a constant drinker with a terrific fight to control the amount of my consumption each day.

My practice had advanced to the point where it could stand a lot of abuse and it got it. Whenever a situation arose that fast talk wouldn’t explain away, I simply withdrew. In other words, I fired the client before the client fired me. I was willful, full of will to do things I wanted to do and to get the things that I wanted to have.

Insofar as religion was concerned, I had Catholic training in my youth. I went to both Catholic and public schools. I never left the church, but I was a fringe member, and the thought simply never occurred to me that through the exercise of what I had I might find the answer to my problem, simply because I wouldn’t admit that I had a problem. The successful demonstration I had made of my life problems in other respects convinced me that some day I was going to be able to drink like a gentleman.

When I was about forty-seven, after indulging in all kinds of self-deception to control my drinking, I arrived at a period when I felt convinced I had to have so much alcohol every day and that the real problem

was to control how much. After two or three years of effort along this line, I reached the point of actual despair that I ever would be able to drink only a harmless amount each day. And then my thinking became calculation as to how much longer I had to live, how long the assets would last. By that time, I had one boy in college, another a senior in high school, and a daughter about twelve years old. My efficiency as a professional man was probably reduced to twenty-five percent of what it should have been.

I had two partners. They suffered from my conduct without saying anything, but the reason for this was that I still managed to hang onto a very substantial practice. They probably felt that it was useless, that surely I had enough intelligence to know what I was doing. They were wrong. They never raised the issue. In fact, as I look back, I have often thought that they probably concluded that they would put up with me for a couple of years, that I couldn’t live much longer, and that they would inherit whatever was left of the practice. That is not unusual.

As far as my home was concerned, I did not see then, though of course I can see now, that it was anything but a happy situation for my wife. My children had lost respect for me and, in fact, it was only after three or four years of sobriety that any of them ever said anything to me to indicate that I had recouped even a little of their respect.

I was forty-nine and a half years old when I was first approached about the Akron Group. It was not known to me as a group, but I later learned that my wife had known about it for nine months and had prayed constantly that I would stumble into Akron

some way or another. She knew that at that time any suggestion she made about my drinking would only build up a barrier, a consideration for which I am ever grateful. Had anyone undertaken to explain to me what the A.A. movement was, what it’s real function was, I probably would have been set back several years and I doubt if I would have survived at all.

So the story of my introduction into A.A. begins with the activities of my wife. She had a hairdresser who used to tell her about a brother-in-law who had been quite a drinker and about some doctor in Akron who had straightened him out. My wife didn’t tell me this, but one Sunday afternoon when Mary was trying to get the cobwebs out of my mind, Clarence and his sister-in-law, the hairdresser, called at the house. I was introduced to them and Clarence proceeded to put on his Twelve Step work. I was kind of shocked about anybody talking about themselves the way he did, and my impression was that the guy was a little “touched.” However, on a couple of other occasions, Clarence seemed to bob up at the last saloon that I would stop at on the way home every day. I resented it of course, and I offered Clarence his commission, whatever it might be, if he would please not bother me because I had arrived at the conclusion that he was a solicitor for some alcoholic institute. One evening I had gone out after dinner to take on a couple of double-headers and stayed a little later than usual, and when I came home Clarence was sitting on the davenport with Bill W. I do not recollect the specific conversation that went on but I believe I did challenge Bill to tell me something about A.A. and I do recall one other thing: I wanted to know what this was that

worked so many wonders, and hanging over the mantel was a picture of Gethsemane and Bill pointed to it and said, “There it is,” which didn’t make much sense to me. There was also some conversation about Dr. Bob and I must have said that I would go down to Akron with Bill in the morning.

The next morning, my wife came into my room and wakened me and said, “That man is downstairs and he said you said you’d go to Akron.” I said, “Did I say that?” She said. “Well, he wouldn’t be here if you didn’t say so.” And being a big-shot man of my word, I said, “Well, if I said so, then I’ll go.” That’s about the spirit in which I went to Akron. Bill bought me a drink or two on the way, and Dorothy S. came with us, and the three of us went over to the City Hospital. We drove my car and I left it down in the yard. Bill left me at the elevator and said, “Your room is so and so,” and I didn’t see him again for six months. The interne came along with a glassful of bleached lightning that put me away for about fifteen hours. I went into the hospital in April, 1939.

My experience in the hospital I considered to be terrific because Dr. Bob told me very quickly that medicine would have very little to do with it, outside of trying to restore my appetite for food. I had had no hospitalization up to this time because I would not call doctors when I was getting over a bad one. I would use barbiturates. In fact, the last three years of my drinking was a routine of barbiturates in the morning, so that I could stop shaking enough to shave, and then alcohol beginning about four-thirty or five o’clock, with a struggle not to take a drink at noon or during the day, because I had the idea that if I took

one drink, I would smell as though I had a pint.

Dr. Bob did not lay out the whole program. He startled me by informing me that he was an alcoholic, that he had found a way which so far enabled him to live without taking a drink, and that the main idea was to find a way how not to take that first drink. He told me that there were some other fellows that had tried this with success, and if I cared to see any of them he’d have them come in to see me. I believe every member of the Akron Group did come to see me, which impressed me terrifically, not so much because of the stories they told, but because they would take the time to come and talk to me without even knowing who I was. I did not know there was such a thing as group activity until I left the hospital. I left on a Wednesday afternoon, had dinner in Akron and then went to a house where I encountered my first meeting. I had attended several of these meetings before I discovered that all those who were there were not alcoholics. That is, it was sort of a mixed bunch of Oxford Groupers, who were interested in the alcoholic problem, and of alcoholics themselves. My reaction to those meetings was good. In fact, I never lost my faith, since I had been prepared by some conversations toward the end of my sojourn in the hospital with Dr. Bob, conversations pretty much along spiritual lines. There was one experience with Doc which made a terrific impression upon me. The afternoon that I was to leave the hospital, he came in to see me and asked me if I were willing to attempt to follow the program. I told him that I had no other intention. That was at the end of eight days in which I had had no liquor. He then pulled up his chair with one of his

knees touching mine and said, “Will you pray with me for your success?” And he then said a beautiful prayer. That was an experience that I have never forgotten, and many times in my own work with A.A. newcomers I feel kind of guilty because I haven’t done the same thing.

One of the things that came up repeatedly in the stories they told me was that once they had accepted the program, they never had a desire to take a drink. That was skeptically recieved by me when I first heard it, but after some twenty-eight or thirty fellows had come to see me, and pretty nearly all of them had said the same thing, I began to believe it. In my own experience I was so jubilant at finding myself sober, and I had so many things to catch up on, that a month went by before the thought even occurred to me. I had a genuine release right from the start. I’ve never had a desire to take a drink.

Doc dwelt on the idea that this was an illness, but Doc was pretty frank with me. He found that I had enough faith in the Almighty to be fairly frank. He pointed out to me that probably it was more of a moral or spiritual illness than it was a physical one.

We went to Akron for about six weeks and we did a lot of visiting among the people in Akron. There were, at that time, in the neighborhood of twelve or thirteen Cleveland members who had been sober anywhere from a year and a half to a couple of months. They had all been to Akron. It was finally decided to undertake the organization of a Cleveland group and toward the end of May, 1939, the first meeting was held in Cleveland in my home. At that meeting, there were a number of Akron people and all the Cleveland people.

       Professionally, after I was sober for a month or so I realized that I should school myself to dissolve the partnership I was in because I felt that I would never regain the respect of my partners no matter how long I was sober, and that I would be at a disadvantage. I still had enough practice to earn a good living if I would only work, so I resolved that in January of 1940 I would launch a patent law firm of my own.

Shortly after I came to this conclusion, I was importuned by another well-known patent law firm to help them out on some trial work because their trial man had had a heart attack and had been forbidden to go into the courtroom. Somewhere in one of the conversations, I mentioned that I was comtemplating forming a new firm. On hearing that, these people induced me to make the move immediately and join them as a senior partner, which I did. I found in the fall of 1939, that I was not mentally impaired, so far as trial work was concerned, and thereafter took up where I had left off when I was about forty-five years old. My physical health was badly shaken, but I began to pick up. In fact, after six months of living on food instead of on whiskey, I gained about thirty pounds.

I realized that there wasn’t anything I could say to place myself in a more favorable light with my children, that it was going to be a matter of time; for I also understood the intolerance of young people towards deficiencies in their elders. I believe though that it helped my family tremendously to have the A.A. meetings every week in my own home. My oldest child sometimes sat in on the meetings.

I had accepted Catholicism somewhat as an inheri-

tance. My education had been pretty much pagan—science. I resolved that if I were going to continue with the Catholic Church, I was going to know the roots of the doctrine, since those roots had caused me some confusion. So I enrolled at the university for night courses in religion, and I pursued those courses for a year. In summing up, I can say that A.A. has made me, I hope, a real Catholic.

\end{biblechapter}

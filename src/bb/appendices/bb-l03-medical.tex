\biblebook{Medical}


\bbChapterPreamble


\bbHeadingRaw{THE MEDICAL VIEW ON A.A append. \romNum{3}}


\begin{biblechapter}
\verseWithHeading{NY Medical Society}
    Since Dr. Silkworth's first endorsement of Alcoholics Anonymous, 
    medical societies and physicians throughout the world 
    have set their approval upon us.
\verse Following are excerpts from the comments of doctors 
    present at the actual meeting
    \footnote{1944}
    of the Medical Society 
    of the State of New York where a paper on A.A. was read:

\verseWithHeading{Kennedy}
    Dr. Foster Kennedy, neurologist: 

\emph{"This organization of Alcoholics Anonymous 
    calls on two of the greatest reservoirs of power known to man, 
    religion and that instinct for association with one's fellows...the 
    'herd instinct.'
\verse I think our profession must take appreciative cognizance
    of this great therapeutic weapon.
\verse If we do not do so, 
    we shall stand convicted of emotional sterility 
    and of having lost the faith that moves mountains, 
    without which medicine can do little."
}

\verseWithHeading{Collier}
    Dr. G. Kirby Collier, psychiatrist: 

\emph{"I have felt that A.A. is a group unto themselves 
    and their best results can be had under their own guidance, 
    as a result of their philosophy.
\verse Any therapeutic or philosophic procedure 
    which can prove a recovery rate of 50\% to 60\% 
    must merit our consideration."
}

\verseWithHeading{Tiebout}
    Dr. Harry M. Tiebout, psychiatrist: 
    
\emph{"As a psychiatrist, 
    I have thought a great deal about the relationship 
    of my speciality to A.A. 
    and I have come to the conclusion that our particular function 
    can very often lie in preparing the way for the patient 
    to accept any sort of treatment or outside help.
\verse I now conceive the psychiatrist's job to be the task of 
    breaking down the patient's inner resistance 
    so that which is inside him will flower, 
    as under the activity of the A.A. program."
}
\end{biblechapter}


\begin{biblechapter}
\verseWithHeading{Bauer \& AMA}
    Dr. W. W. Bauer, 
    broadcasting under the auspices of The American Medical Association 
    in 1946, over the NBC network, said, in part:

\verse\emph{"Alcoholics Anonymous are no crusaders; 
    not a temperance society.  
\verse They know that they must never drink.
\verse They help others with similar problems... 
\verse In this atmosphere the alcoholic often overcomes 
    his excessive concentration upon himself.
\verse Learning to depend upon a higher power 
    and absorb himself in his work with other alcoholics, 
    he remains sober day by day.
\verse The days add up into weeks, the weeks into months and years."
}
\end{biblechapter}


\begin{biblechapter}
\verseWithHeading{Stouffer}
    Dr. John F. Stouffer, Chief Psychiatrist, 
    Philadelphia General Hospital, 
    citing his experience with A.A., said:

\verse\emph{ "The alcoholics we get here at Philadelphia General 
    are mostly those who cannot afford private treatment, 
    and A.A. is by far the greatest thing we have been able to offer them.
\verse Even among those who occasionally land back in here again, 
    we observe a profound change in personality.
\verse You would hardly recognize them."
}
\end{biblechapter}


\begin{biblechapter}
\verseWithHeading{APA}
    The American Psychiatric Association requested, in 1949, 
    that a paper he prepared 
    by one of the older members of Alcoholics Anonymous 
    to be read at the Association's annual meeting of that year.
\verse This was done, 
    and the paper was printed in the American Journal of Psychiatry 
    for November, 1949.

\verse (This address is now available in pamphlet form 
    at nominal cost through most A.A. groups 
    or from Box 459, Grand Central Station, New York, NY 10163, 
    under the title "Three Talks to Medical Societies by Bill W."--
    formerly called "Bill on Alcoholism" 
    and earlier "Alcoholism the Illness.")
\end{biblechapter}

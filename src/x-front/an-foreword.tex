\begin{fmatterchapter}

\chapter*{Foreword}
%?? How to differentiate foreword from preface based on this work?
% Preface ought to be more biographical and personally motivated.

\begin{center}[ EARLY DRAFT STATE ]\end{center}

This work, preserving the Bigbook verbatim, 
    transforms its presentation 
    into a truly universal basic text.
You might wonder what does that mean and why.

Upon opening the Bigbook 
    one will find the authors stating on the very second paragraph 
    that the text itself is a 'basic text'.
The basic text has become the bedrock of the fellowship of Alcoholics.
Many self described 'Bigbook Study' groups 
    meet on a weekly basis to spiritually grow through it.
Every other group reads parts of it to open and close meetings.

The language found throughout, 
    including literal sentences from the forewords, 
    resembles what has become our meetings preamble.
The infamous 'How It Works' excerpt, 
    describing in summary our whole program, 
    has become the most common piece to be read after our preamble.
The daily reflections which is now often read after How It Works, 
    grew from the written reflections by our society's members, 
    submitted to the grapevine over the years, 
    based on excerpts from the basic text.
The last number of pages of Into Action 
    has become an alternative reading piece to How It Works,
    for eleventh step prayer and meditation groups.
The 12 promises from 'Into Action', 
    outlining what is in store for us upon following this spiritual program, 
    is now often read to close meetings.
Our slogans, signs and scrolls which are dotted about on display 
    during our meetings,
    mostly contain phrases from our basic text.

The book of Twelve Steps And Twelve Traditions 
    has also grown out of the basic text.
It was from a request to make a more updated version of the original Bigbook, whilst preserving the content of the text itself.
It is primarily a series of essays on each step, 
    and has become the basis for perhaps most meetings today.
The traditions contained in the appendix 
    are now often read during group conscience meetings.
The whole history of the fellowship has evolved around this basic text.

Members attempt to recall specific lines verbatim, from memory, 
    and try to recall approximate page numbers to those lines.
There is a great many interpretations and discussions 
    on developing spiritually through the text.
It is read repeatedly by millions of alcoholics day in day out.

We have come to breath our spiritual program 
    through the words on those pages.
Everywhere in our society, we hear and see them.
Even if a member never read the text directly, 
    they would absorb its message through simple participation.
If it were not for the words within 
    we would surely die a short life 
    or else live one of unbearable agony.

And many say it has come to resemble an outline 
    of a universal spiritual program 
    and has become a classic of western spiritual literature.
One which is applicable to not only alcoholics, 
    but which has become the cornerstone for most recovery fellowships,
    and which most everyone can take spiritual meaning from.
Some whisper it is perhaps the greatest spiritual development of the West 
    over the 20th century, with its author, Bill 
    being listed as the healer in the Times 100 most influential people 
    of the 20th century.
It may be safe to say this basic text will persist for centuries.

The principle limitation of our text, from being a truly basic text, 
    rests in it being bound to a limited and specific 
    presentation, publication and printing.
Of being rigidly put within defined page and chapter numbers 
    without a means to intuitively refer to the location of parts of it, 
    independent of lengthy chapter or page numbers.
Furthermore, there is no agreed upon way to directly refer 
    to the location of specific spiritual verses of meaning 
    contained within each chapter.
In other words, its a dense and lengthy piece of spiritual writing, 
    easy to get lost in, 
    with little apparatus to navigate.

The text is spiritually dense, 
    and the meaning associated in recall and discussion of the text, 
    is often lost in the lack of precision of referencing.
And mental recall and referencing are essential to our spiritual growth.
When it comes to our text, this is but another name for prayer and spiritual reading.

This work, preserving the basic text verbatim, 
    simply presents a means of transforming its presentation 
    into a truly universal basic text.

It is the year 2025, and our text is 5 years away 
    from its centurion anniversary.
This verbatim versification aims to set in motion works towards 
    spiritually materialising the claim 
    of the second paragraph of our text.
That "this book has become the basic text of our society".

%%%%%%%%%%%%%%%%%%%%%%%%%%%%%%%%%%%%%%%%%%%%%%%%%%%%%%%
%This work preserves verbatim the basic text to its content, 
%A form which will persist across centuries and millenia.
%In 1000 years from now, it will at the very least be a fundamental historical document.
%The earliest manuscript will be kept in a high status musuem.

%A verse of text is more spiritually digestible. The text is dense, and is read repeatedly to allow its truth to sink into the soul.
%When a text is presented in this manner it can fascillitate a deeper connection to our psyche.

%The generation of the text in verse has also been codified,
%    so to allow it to be cloned, 
%    so it can be set free spiritually to allow alcoholics to 

%become the foundation for ongoing commentarial and derivative works.

\end{fmatterchapter}

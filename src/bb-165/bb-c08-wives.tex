\biblebook{Wives}

Alcoholics Anonymous Bigbook 'To Wives'; Pages 104-121

\begin{biblechapter}
\verseWithHeading{Introduction}
WITH FEW EXCEPTIONS, our book thus far has spoken of men.  But what we have said applies quite as much to women.  Our activities in behalf of women who drink are on the increase.  There is every evidence that women regain their health as readily as men if they try our suggestions.

But for every man who drinks others are involved-the wife who trembles in fear of the next debauch; the mother and father who see their son wasting away.

Among us are wives, relatives and friends whose problem has been solved, as well as some who have not yet found a happy solution.  We want the wives of Alcoholics Anonymous to address the wives of men who drink too much.  What they say will apply to nearly everyone bound by ties of blood or affection to an alcoholic.

As wives of Alcoholics Anonymous, we would like you to feel that we understand as perhaps few can.  We want to analyze mistakes we have made.  We want to leave you with the feeling that no situation is too difficult and no unhappiness too great to be overcome.

We have traveled a rocky road, there is no mistake about that.  We have had long rendezvous with hurt pride, frustration, self-pity, misunderstanding and fear.  These are not pleasant companions.  We have been driven to maudlin sympathy, to bitter resentment.  Some of us veered from extreme to extreme, ever hoping that one day our loved ones would be themselves once more.

Our loyalty and the desire that our husbands hold up their heads and be like other men have begotten all sorts of predicaments.  We have been unselfish and self-sacrificing.  We have told innumerable lies to protect our pride and our husbands' reputations.  We have prayed, we have begged, we have been patient.  We have struck out viciously.  We have run away.  We have been hysterical.  We have been terror stricken.  We have sought sympathy.  We have had retaliatory love affairs with other men.

Our homes have been battle-grounds many an evening.  In the morning we have kissed and made up.  Our friends have counseled chucking the men and we have done so with finality, only to be back in a little while hoping, always hoping.  Our men have sworn great solemn oaths that they were through drinking forever.  We have believed them when no one else could or would.  Then, in days, weeks, or months, a fresh outburst.

We seldom had friends at our homes, never knowing how or when the men of the house would appear.  We could make few social engagements.  We came to live almost alone.  When we were invited out, our husbands sneaked so many drinks that they spoiled the occasion.  If, on the other hand, they took nothing, their self-pity made them killjoys.

There was never financial security.  Positions were always in jeopardy or gone.  An armored car could  not have brought the pay envelopes home.  The checking account melted like snow in June.

Sometimes there were other women.  How heartbreaking was this discovery; how cruel to be told they understood our men as we did not!

The bill collectors, the sheriffs, the angry taxi drivers, the policemen, the bums, the pals, and even the ladies they sometimes brought home-our husbands thought we were so inhospitable.  "Joykiller, nag, wet blanket"-that's what they said.  Next day they would be themselves again and we would forgive and try to forget.

We have tried to hold the love of our children for their father.  We have told small tots that father was sick, which was much nearer the truth than we realized.  They struck the children, kicked out door panels, smashed treasured crockery, and ripped the keys out of pianos.  In the midst of such pandemonium they may have rushed out threatening to live with the other woman forever.  In desperation, we have even got tight ourselves-the drunk to end all drunks.  The unexpected result was that our husbands seemed to like it.

Perhaps at this point we got a divorce and took the children home to father and mother.  Then we were severely criticized by our husband's parents for desertion.  Usually we did not leave.  We stayed on and on.  We finally sought employment ourselves as destitution faced us and our families.

We began to ask medical advice as the sprees got closer together.  The alarming physical and mental symptoms, the deepening pall of remorse, depression and inferiority that settled down on our loved ones-these things terrified and distracted us.  As animals on a treadmill, we have patiently and wearily climbed, falling back in exhaustion after each futile effort to reach solid ground.  Most of us have entered the final stage with its commitment to health resorts, sanitariums, hospitals, and jails.  Sometimes there were screaming delirium and insanity.  Death was often near.

Under these conditions we naturally made mistakes.  Some of them rose out of ignorance of alcoholism.  Sometimes we sensed dimly that we were dealing with sick men.  Had we fully understood the nature of the alcoholic illness, we might have behaved differently.

How could men who loved their wives and children be so unthinking, so callous, so cruel?  There could be no love in such persons, we thought.  And just as we were being convinced of their heartlessness, they would surprise us with fresh resolves and new attentions.  For a while they would be their old sweet selves, only to dash the new structure of affection to pieces once more.  Asked why they commenced to drink again, they would reply with some silly excuse, or none.  It was so baffling, so heartbreaking.  Could we have been so mistaken in the men we married?  When drinking, they were strangers.  Sometimes they were so inaccessible that it seemed as though a great wall had been built around them.

And even if they did not love their families, how could they be so blind about themselves?  What had become of their judgment, their common sense, their will power?  Why could they not see that drink meant ruin to them?  Why was it, when these dangers were pointed out that they agreed, and then got drunk again immediately?

These are some of the questions which race through the mind of every woman who has an alcoholic husband.  We hope this book has answered some of them.  Perhaps your husband has been living in that strange world of alcoholism where everything is distorted and exaggerated.  You can see that he really does love you with his better self.  Of course, there is such a thing as incompatibility, but in nearly every instance the alcoholic only seems to be unloving and inconsiderate; it is usually because he is warped and sickened that he says and does these appalling things.  Today most of our men are better husbands and fathers than ever before.

Try not to condemn your alcoholic husband no matter what he says or does.  He is just another very sick, unreasonable person.  Treat him, when you can, as though he had pneumonia.  When he angers you, remember that he is very ill.

There is an important exception to the foregoing.  We realize some men are thoroughly bad-intentioned, that no amount of patience will make any difference.  An alcoholic of this temperament may be quick to use this chapter as a club over your head.  Don't let him get away with it.  If you are positive he is one of this type you may feel you had better leave.  Is it right to let him ruin your life and the lives of your children?  Especially when he has before him a way to stop his drinking and abuse if he really wants to pay the price.

The problem with which you struggle usually falls within one of four categories:

One:  Your husband may be only a heavy drinker.  His drinking may be constant or it may be heavy only on certain occasions.  Perhaps he spends too much money for liquor.  It may be slowing him up mentally and physically, but he does not see it.  Sometimes he is a source of embarrassment to you and his friends.  He is positive he can handle his liquor, that it does him no harm, that drinking is necessary in his business.  He would probably be insulted if he were called an alcoholic.  This world is full of people like him.  Some will moderate or stop altogether, and some will not.  Of those who keep on, a good number will become true alcoholics after a while.

Two:  Your husband is showing lack of control, for he is unable to stay on the water wagon even when he wants to.  He often gets entirely out of hand when drinking.  He admits this is true, but is positive that he will do better.  He has begun to try, with or without your cooperation, various means of moderating or staying dry.  Maybe he is beginning to lose his friends.  His business may suffer somewhat.  He is worried at times, and is becoming aware that he cannot drink like other people.  He sometimes drinks in the morning and through the day also, to hold his nervousness in check.  He is remorseful after serious drinking bouts and tells you he wants to stop.  But when he gets over the spree, he begins to think once more how he can drink moderately next time.  We think this person is in danger.  These are the earmarks of a real alcoholic.  Perhaps he can still tend to business fairly well.  He has by no means ruined everything.  As we say among ourselves, "He wants to want to stop."

Three:  This husband has gone much further than husband number two.  Though once like number two he became worse.  His friends have slipped away, his home is a near-wreck and he cannot hold a position.  Maybe the doctor has been called in, and the weary round of sanitariums and hospitals has begun.  He admits he cannot drink like other people, but does not see why.  He clings to the notion that he will yet find a way to do so.  He may have come to the point where he desperately wants to stop but cannot.  His case presents additional questions which we shall try to answer for you.  You can be quite hopeful of a situation like this.

Four:  You may have a husband of whom you completely despair.  He has been placed in one institution after another.  He is violent, or appears definitely insane when drunk.  Sometimes he drinks on the way home from the hospital.  Perhaps he has had delirium tremens.  Doctors may shake their heads and advise you to have him committed.  Maybe you have already been obliged to put him away.  This picture may not be as dark as it looks.  Many of our husbands were just as far gone.  Yet they got well.

Let's now go back to husband number one.  Oddly enough, he is often difficult to deal with.  He enjoys drinking.  It stirs his imagination.  His friends feel closer over a highball.  Perhaps you enjoy drinking with him yourself when he doesn't go too far.  You have passed happy evenings together chatting and drinking before your fire.  Perhaps you both like parties which would be dull without liquor.  We have enjoyed such evenings ourselves; we had a good time.  We know all about liquor as a social lubricant.  Some, but not all of us, think it has its advantages when reasonably used.

The first principle of success is that you should never be angry.  Even though your husband becomes unbearable and you have to leave him temporarily, you should, if you can, go without rancor.  Patience and good temper are most necessary.

Our next thought is that you should never tell him what he must do about his drinking.  If he gets the idea that you are a nag or killjoy, your chance of accomplishing anything useful may be zero.  He will use that as an excuse to drink more.  He will tell you he is misunderstood.  This may lead to lonely evenings for you.  He may seek someone else to console him-not always another man.

Be determined that your husband's drinking is not going to spoil your relations with your children or your friends.  They need your companionship and your help.  It is possible to have a full and useful life, though your husband continues to drink.  We know women who are unafraid, even happy under these conditions.  Do not set your heart on reforming your husband.  You may be unable to do so, no matter how hard you try.

We know these suggestions are sometimes difficult to follow, but you will save many a heartbreak if you can succeed in observing them.  Your husband may come to appreciate your reasonableness and patience.  This may lay the groundwork for a friendly talk about his alcoholic problem.  Try to have him bring up the subject himself.  Be sure you are not critical during such a discussion.  Attempt instead, to put yourself in his place.  Let him see that you want to be helpful rather than critical.

When a discussion does arise, you might suggest he read this book or at least the chapter on alcoholism.  Tell him you have been worried, though perhaps needlessly.  You think he ought to know the subject better, as everyone should have a clear understanding of the risk he takes if he drinks too much.  Show him you have confidence in his power to stop or moderate.  Say you do not want to be a wet blanket; that you only want him to take care of his health.  Thus you may succeed in interesting him in alcoholism.

He probably has several alcoholics among his own acquaintances.  You might suggest that you both take an interest in them.  Drinkers like to help other drinkers.  Your husband may be willing to talk to one of them.

If this kind of approach does not catch your husband's interest, it may be best to drop the subject, but after a friendly talk your husband will usually revive the topic himself.  This may take patient waiting, but it will be worth it.  Meanwhile you might try to help the wife of another serious drinker.  If you act upon these principles, your husband may stop or moderate.

Suppose, however, that your husband fits the description of number two.  The same principles which apply to husband number one should be practiced.  But after his next binge, ask him if he would really like to get over drinking for good.  Do not ask that he do it for you or anyone else.  Just would he like to?

The chances are he would.  Show him your copy of this book and tell him what you have found out about alcoholism.  Show him that as alcoholics, the writers of the book understand.  Tell him some of the interesting stories you have read.  If you think he will be shy of a spiritual remedy, ask him to look at the chapter on alcoholism.  Then perhaps he will be interested enough to continue.

If he is enthusiastic your cooperation will mean a great deal.  If he is lukewarm or thinks he is not an alcoholic, we suggest you leave him alone.  Avoid urging him to follow our program.  The seed has been planted in his mind.  He knows that thousands of men, much like himself, have recovered.  But don't remind him of this after he had been drinking, for he may be angry.  Sooner or later, you are likely to find him reading the book once more.  Wait until repeated stumbling convinces him he must act, for the more you hurry him the longer his recovery may be delayed.

If you have a number three husband, you may be in luck.  Being certain he wants to stop, you can go to him with this volume as joyfully as though you had struck oil.  He may not share your enthusiasm, but he is practically sure to read the book and he may go for the program at once.  If he does not, you will probably not have long to wait.  Again, you should not crowd him.  Let him decide for himself.  Cheerfully see him through more sprees.  Talk about his condition or this book only when he raises the issue.  In some cases it may be better to let someone outside the family present the book.  They can urge action without arousing hostility.  If your husband is otherwise a normal individual, your chances are good at this stage.

You would suppose that men in the fourth classification would be quite hopeless, but that is not so.  Many of Alcoholics Anonymous were like that.  Everybody had given them up.  Defeat seemed certain.  Yet often such men had spectacular and powerful recoveries.

There are exceptions.  Some men have been so impaired by alcohol that they cannot stop.  Sometimes there are cases where alcoholism is complicated by other disorders.  A good doctor or psychiatrist can tell you whether these complications are serious.  In any event, try to have your husband read this book.  His reaction may be one of enthusiasm.  If he is already committed to an institution, but can convince you and your doctor that he means business, give him a chance to try our method, unless the doctor thinks his mental condition too abnormal or dangerous.  We make this recommendation with some confidence.  For years we have been working with alcoholics committed to institutions.  Since this book was first published, A.A. has released thousands of alcoholics from asylums and hospitals of every kind.  The majority have never returned.  The power of God goes deep!

You may have the reverse situation on your hands.  Perhaps you have a husband who is at large, but who should be committed.  Some men cannot or will not get over alcoholism.  When they become too dangerous, we think the kind thing is to lock them up, but of course a good doctor should always be consulted.  The wives and children of such men suffer horribly, but not more than the men themselves.

But sometimes you must start life anew.  We know women who have done it.  If such women adopt a spiritual way of life their road will be smoother.

If your husband is a drinker, you probably worry over what other people are thinking and you hate to meet your friends.  You draw more and more into yourself and you think everyone is talking about conditions at your home.  You avoid the subject of drinking, even with your own parents.  You do not know what to tell the children.  When your husband is bad, you become a trembling recluse, wishing the telephone had never been invented.

We find that most of this embarrassment is unnecessary.  While you need not discuss your husband at length, you can quietly let your friends know the nature of his illness.  But you must be on guard not to embarrass or harm your husband.

When you have carefully explained to such people that he is a sick person, you will have created a new atmosphere.  Barriers which have sprung up between you and your friends will disappear with the growth of sympathetic understanding.  You will no longer be self-conscious or feel that you must apologize as though your husband were a weak character.  He may be anything but that.  Your new courage, good nature and lack of self-consciousness will do wonders for you socially.

The same principle applies in dealing with the children.  Unless they actually need protection from their father, it is best not to take sides in any argument he has with them while drinking.  Use your energies to promote a better understanding all around.  Then that terrible tension which grips the home of every problem drinker will be lessened.

Frequently, you have felt obliged to tell your husband's employer and his friends that he was sick, when as a matter of fact he was tight.  Avoid answering these inquiries as much as you can.  Whenever possible, let your husband explain.  Your desire to protect him should not cause you to lie to people when they have a right to know were he is and what he is doing.  Discuss this with him when he is sober and in good spirits.  Ask him what you should do if he places you in such a position again.  But be careful not to be resentful about the last time he did so.

There is another paralyzing fear.  You may be afraid your husband will lose his position; you are thinking of the disgrace and hard times which will befall you and the children.  This experience may come to you.  Or you may already have had it several times.  Should it happen again, regard it in a different light.  Maybe it will prove a blessing!  It may convince your husband he wants to stop drinking forever.  And now you know that he can stop if he will!  Time after time, this apparent calamity has been a boon to us, for it opened up a path which led to the discovery of God.

We have elsewhere remarked how much better life is when lived on a spiritual plane.  If God can solve the age-old riddle of alcoholism, He can solve your problems too.  We wives found that, like everybody else, we were afflicted with pride, self-pity, vanity and all the things which go to make up the self-centered person; and we were not above selfishness or dishonesty.  As our husbands began to apply spiritual principles in their lives, we began to see the desirability of doing so too.

At first, some of us did not believe we needed this help.  We thought, on the whole, we were pretty good women, capable of being nicer if our husbands stopped drinking.  But it was a silly idea that we were too good to need God.  Now we try to put spiritual principles to work in every department of our lives.  When we do that, we find it solves our problems too; the ensuing lack of fear, worry and hurt feelings is a wonderful thing.  We urge you to try our program, for nothing will be so helpful to your husband as the radically changed attitude toward him which God will show you how to have.  Go along with your husband if you possibly can.

If you and your husband find a solution for the pressing problem of drink you are, of course, going to be very happy.  But all problems will not be solved at once.  Seed has started to sprout in a new soil, but growth has only begun.  In spite of your new-found happiness, there will be ups and downs.  Many of the old problems will still be with you.  This is as it should be.  

The faith and sincerity of both you and your husband will be put to the test.  These work-outs should be regarded as part of your education, for thus you will be learning to live.  You will make mistakes, but if you are in earnest they will not drag you down.  Instead, you will capitalize them.  A better way of life will emerge when they are overcome.

Some of the snags you will encounter are irritation, hurt feelings and resentments.  Your husband will sometimes be unreasonable and you will want to criticize.  Starting from a speck on the domestic horizon, great thunderclouds of dispute may gather.  These family dissensions are very dangerous, especially to your husband.  Often you must carry the burden of avoiding them or keeping them under control.  Never forget that resentment is a deadly hazard to an alcoholic.  We do not mean that you have to agree with your husband whenever there is an honest difference of opinion.  Just be careful not to disagree in a resentful or critical spirit.

You and your husband will find that you can dispose of serious problems easier than you can the trivial ones.  Next time you and he have a heated discussion, no matter what the subject, it should be the privilege of either to smile and say, "This is getting serious.  I'm sorry I got disturbed.  Let's talk about it later."  If your husband is trying to live on a spiritual basis, he will also be doing everything in his power to avoid disagreement or contention.

Your husband knows he owes you more than sobriety.  He wants to make good.  Yet you must not expect too much.  His ways of thinking and doing are the habits of years.  Patience, tolerance, understanding and love are the watchwords.  Show him these things in yourself and they will be reflected back to you from him.  Live and let live is the rule.  If you both show a willingness to remedy your own defects, there will be little need to criticize each other.

We women carry with us a picture of the ideal man, the sort of chap we would like our husbands to be.  It is the most natural thing in the world, once his liquor problem is solved, to feel that he will now measure up to that cherished vision.  The chances are he will not for, like yourself, he is just beginning his development.  Be patient.

Another feeling we are very likely to entertain is one of resentment that love and loyalty could not cure our husbands of alcoholism.  We do not like the thought that the contents of a book or the work of another alcoholic has accomplished in a few weeks that for which we struggled for years.  At such moments we forget that alcoholism is an illness over which we could not possibly have had any power.  Your husband will be the first to say it was your devotion and care which brought him to the point where he could have a spiritual experience.  Without you he would have gone to pieces long ago.  When resentful thoughts come, try to pause and count your blessings.  After all, your family is reunited, alcohol is no longer a problem and you and your husband are working together toward an undreamed-of future.

Still another difficulty is that you may become jealous of the attention he bestows on other people, especially alcoholics.  You have been starving for his companionship, yet he spends long hours helping other men and their families.  You feel he should now be yours.  The fact is that he should work with other people to maintain his own sobriety.  Sometimes he will be so interested that he becomes really neglectful.  Your house is filled with strangers.  You may not like some of them.  He gets stirred up about their troubles, but not at all about yours.  It will do little good if you point that out and urge more attention for yourself.  We find it a real mistake to dampen his enthusiasm for alcoholic work.  You should join in his efforts as much as you possibly can.  We suggest that you direct some of your thought to the wives of his new alcoholic friends.  They need the counsel and love of a woman who has gone through what you have.

It is probably true that you and your husband have been living too much alone, for drinking many times isolates the wife of an alcoholic.  Therefore, you probably need fresh interests and a great cause to live for as much as your husband.  If you cooperate, rather than complain, you will find that his excess enthusiasm will tone down.  Both of you will awaken to a new sense of responsibility for others. You, as well as your husband, ought to think of what you can put into life instead of how much you can take out.  Inevitably your lives will be fuller for doing so.  You will lose the old life to find one much better.

Perhaps your husband will make a fair start on the new basis, but just as things are going beautifully he dismays you by coming home drunk.  If you are satisfied he really wants to get over drinking, you need not be alarmed.  Though it is infinitely better that he have no relapse at all, as has been true with many of our men, it is by no means a bad thing in some cases.  Your husband will see at once that he must redouble his spiritual activities if he expects to survive.  You need not remind him of his spiritual deficiency-he will know of it.  Cheer him up and ask him how you can be still more helpful.

The slightest sign of fear or intolerance may lessen your husband's chance of recovery.  In a weak moment he may take your dislike of his high-stepping friends as one of those insanely trivial excuses to drink.

We never, never try to arrange a man's life so as to shield him from temptation.  The slightest disposition on your part to guide his appointments or his affairs so he will not be tempted will be noticed.  Make him feel absolutely free to come and go as he likes.  This is important.  If he gets drunk, don't blame yourself.  God has either removed your husband's liquor problem or He has not.  If not, it had better be found out right away.  Then you and your husband can get right down to fundamentals.  If a repetition is to be prevented, place the problem, along with everything else, in God's hands.

We realize that we have been giving you much direction and advice.  We may have seemed to lecture.  If that is so we are sorry, for we ourselves don't always care for people who lecture us.  But what we have related is based upon experience, some of it painful.  We had to learn these things the hard way.  That is why we are anxious that you understand, and that you avoid these unnecessary difficulties. 

So to you out there-who may soon be with us-we say "Good luck and God bless you!"

\end{biblechapter}

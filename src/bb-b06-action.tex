\biblebook{Action}

Alcoholics Anonymous Bigbook "Into Action"; Page 72-88

\begin{biblechapter}
\verseWithHeading{Introduction}

HAVING MADE our personal inventory, what shall we do about it?  We have been trying to get a new attitude, a new relationship with our Creator, and to discover the obstacles in our path.  We have admitted certain defects; we have ascertained in a rough way what the trouble is; we have put our finger on the weak items in our personal inventory.  Now these are about to be cast out.  This requires action on our part, which, when completed, will mean that we have admitted to God, to ourselves, and to another human being, the exact nature of our defects.  This brings us to the Fifth Step in the program of recovery mentioned in the preceding chapter.
This is perhaps difficult-especially discussing our defects with another person.  We think we have done well enough in admitting these things to ourselves.  There is doubt about that.  In actual practice, we usually find a solitary self-appraisal insufficient.  Many of us thought it necessary to go much further.  We will be more reconciled to discussing ourselves with another person when we see good reasons why we should do so.  The best reason first: If we skip this vital step, we may not overcome drinking.  Time after time newcomers have tried to keep to themselves certain facts about their lives.  Trying to avoid this humbling experience, they have turned to easier methods.  Almost invariably they got drunk.  Having persevered with the rest of the program, they wondered why they fell.  We think the reason is that they never completed their housecleaning.  They took inventory all right, but hung on to some of the worst items in stock.  They only thought they had lost their egoism and fear; they only thought they had humbled themselves.  But they had not learned enough of humility, fearlessness and honesty, in the sense we find it necessary, until they told someone else all their life story.
More than most people, the alcoholic leads a double life.  He is very much the actor.  To the outer world he presents his stage character.  This is the one he likes his fellows to see.  He wants to enjoy a certain reputation, but knows in his heart he doesn't deserve it.
The inconsistency is made worse by the things he does on his sprees.  Coming to his senses, he is revolted at certain episodes he vaguely remembers.  These memories are a nightmare.  He trembles to think someone might have observed him.  As fast as he can, he pushes these memories far inside himself.  He hopes they will never see the light of day.  He is under constant fear and tension-that makes for more drinking.
Psychologists are inclined to agree with us.  We have spent thousands of dollars for examinations.  We know but few instances where we have given these doctors a fair break.  We have seldom told them the whole truth nor have we followed their advice.  Unwilling to be honest with these sympathetic men, we were honest with no one else.  Small wonder many in the medical profession have a low opinion of alcoholics and their chance for recovery!
We must be entirely honest with somebody if we expect to live long or happily in this world.  Rightly and naturally, we think well before we choose the person or persons with whom to take this intimate and confidential step.  Those of us belonging to a religious denomination which requires confession must, and of course, will want to go to the properly appointed authority whose duty it is to receive it.  Though we have no religious connection, we may still do well to talk with someone ordained by an established religion.  We often find such a person quick to see and understand our problem.  Of course, we sometimes encounter people who do not understand alcoholics.
If we cannot or would rather not do this, we search our acquaintance for a close-mouthed, understanding friend.  Perhaps our doctor or psychologist will be the person.  It may be one of our own family, but we cannot disclose anything to our wives or our parents which will hurt them and make them unhappy.  We have no right to save our own skin at another person's expense.  Such parts of our story we tell to someone who will understand, yet be unaffected.  The rule is we must be hard on ourself, but always considerate of others.
Notwithstanding the great necessity for discussing ourselves with someone, it may be one is so situated that there is no suitable person available.  If that is so, this step may be postponed, only, however, if we hold ourselves in complete readiness to go through with it at the first opportunity.  We say this because we are very anxious that we talk to the right person.  It is important that he be able to keep a confidence; that he fully understand and approve what we are driving at; that he will not try to change our plan.  But we must not use this as a mere excuse to postpone.
When we decide who is to hear our story, we waste no time.  We have a written inventory and we are prepared for a long talk.  We explain to our partner what we are about to do and why we have to do it.  He should realize that we are engaged upon a life-and-death errand.  Most people approached in this way will be glad to help; they will be honored by our confidence.
We pocket our pride and go to it, illuminating every twist of character, every dark cranny of the past.  Once we have taken this step, withholding nothing, we are delighted.  We can look the world in the eye.  We can be alone at perfect peace and ease.  Our fears fall from us.  We begin to feel the nearness of our Creator.  We may have had certain spiritual beliefs, but now we begin to have a spiritual experience.  The feeling that the drink problem has disappeared will often come strongly.  We feel we are on the Broad Highway, walking hand in hand with the Spirit of the Universe.
Returning home we find a place where we can be quiet for an hour, carefully reviewing what we have done.  We thank God from the bottom of our heart that we know Him better.  Taking this book down from our shelf we turn to the page which contains the twelve steps.  Carefully reading the first five proposals we ask if we have omitted anything, for we are building an arch through which we shall walk a free man at last.  Is our work solid so far?  Are the stones properly in place?  Have we skimped on the cement put into the foundation?  Have we tried to make mortar without sand?
If we can answer to our satisfaction, we then look at Step Six.  We have emphasized willingness as being indispensable.  Are we now ready to let God remove from us all the things which we have admitted are objectionable?  Can He now take them all-every one?  If we still cling to something we will not let go, we ask God to help us be willing.
When ready, we say something like this: "My Creator, I am now willing that you should have all of me, good and bad.  I pray that you now remove from me every single defect of character which stands in the way of my usefulness to you and my fellows.  Grant me strength, as I go out from here, to do your bidding.  Amen."  We have then completed Step Seven.
Now we need more action, without which we find that "Faith without works is dead."  Let's look at Steps Eight and Nine.  We have a list of all persons we have harmed and to whom we are willing to make amends.  We made it when we took inventory.  We subjected ourselves to a drastic self-appraisal.  Now we go out to our fellows and repair the damage done in the past.  We attempt to sweep away the debris which has accumulated out of our effort to live on self-will and run the show ourselves.  If we haven't the will to do this, we ask until it comes.  Remember it was agreed at the beginning we would go to any lengths for victory over alcohol.
Probably there are still some misgivings.  As we look over the list of business acquaintances and friends we have hurt, we may feel diffident about going to some of them on a spiritual basis.  Let us be reassured.  To some people we need not, and probably should not emphasize the spiritual feature on our first approach.  We might prejudice them.  At the moment we are trying to put our lives in order.  But this is not an end in itself.  Our real purpose is to fit ourselves to be of maximum service to God and the people about us.  It is seldom wise to approach an individual, who still smarts from our injustice to him, and announce that we have gone religious.  In the prize ring, this would be called leading with the chin.  Why lay ourselves open to being branded fanatics or religious bores?  We may kill a future opportunity to carry a beneficial message.  But our man is sure to be impressed with a sincere desire to set right the wrong.  He is going to be more interested in a demonstration of good will than in our talk of spiritual discoveries.
We don't use this as an excuse for shying away from the subject of God.  When it will serve any good purpose, we are willing to announce our convictions with tact and common sense.  The question of how to approach the man we hated will arise.  It may be he has done us more harm than we have done him and, though we may have acquired a better attitude toward him, we are still not too keen about admitting our faults.  Nevertheless, with a person we dislike, we take the bit in our teeth.  It is harder to go to an enemy than to a friend, but we find it much more beneficial to us.  We go to him in a helpful and forgiving spirit, confessing our former ill feeling and expressing our regret.
Under no condition do we criticize such a person or argue.  Simply we tell him that we will never get over drinking until we have done our utmost to straighten out the past.  We are there to sweep off our side of the street, realizing that nothing worth while can be accomplished until we do so, never trying to tell him what he should do.  His faults are not discussed.  We stick to our own.  If our manner is calm, frank, and open, we will be gratified with the result.
In nine cases out of ten the unexpected happens.  Sometimes the man we are calling upon admits his own fault, so feuds of years' standing melt away in an hour.  Rarely do we fail to make satisfactory progress.  Our former enemies sometimes praise what we are doing and wish us well.  Occasionally, they will offer assistance.  It should not matter, however, if someone does throw us out of his office.  We have made our demonstration, done our part.  It's water over the dam.
Most alcoholics owe money.  We do not dodge our creditors.  Telling them what we are trying to do, we make no bones about our drinking; they usually know it anyway, whether we think so or not.  Nor are we afraid of disclosing our alcoholism on the theory it may cause financial harm.  Approached in this way, the most ruthless creditor will sometimes surprise us.  Arranging the best deal we can we let these people know we are sorry.  Our drinking has made us slow to pay.  We must lose our fear of creditors no matter how far we have to go, for we are liable to drink if we are afraid to face them.
Perhaps we have committed a criminal offense which might land us in jail if it were known to the authorities.  We may be short in our accounts and unable to make good.  We have already admitted this in confidence to another person, but we are sure we would be imprisoned or lose our job if it were known.  Maybe it's only a petty offense such as padding the expense account.  Most of us have done that sort of thing.  Maybe we are divorced, and have remarried but haven't kept up the alimony to number one.  She is indignant about it, and has a warrant out for our arrest.  That's a common form of trouble too.
Although these reparations take innumerable forms, there are some general principles which we find guiding.  Reminding ourselves that we have decided to go to any lengths to find a spiritual experience, we ask that we be given strength and direction to do the right thing, no matter what the personal consequences may be.  We may lose our position or reputation or face jail, but we are willing.  We have to be.  We must not shrink at anything.
Usually, however, other people are involved.  Therefore, we are not to be the hasty and foolish martyr who would needlessly sacrifice others to save himself from the alcoholic pit.  A man we know had remarried.  Because of resentment and drinking, he had not paid alimony to his first wife.  She was furious.  She went to court and got an order for his arrest.  He had commenced our way of life, had secured a position, and was getting his head above water.  It would have been impressive heroics if he had walked up to the Judge and said, "Here I am."
We thought he ought to be willing to do that if necessary, but if he were in jail he could provide nothing for either family.  We suggested he write his first wife admitting his faults and asking forgiveness.  He did, and also sent a small amount of money.  He told her what he would try to do in the future.  He said he was perfectly willing to go to jail if she insisted.  Of course she did not, and the whole situation has long since been adjusted.
Before taking drastic action which might implicate other people we secure their consent.  If we have obtained permission, have consulted with others, asked God to help and the drastic step is indicated we must not shrink.
This brings to mind a story about one of our friends.  While drinking, he accepted a sum of money from a bitterly-hated business rival, giving him no receipt for it.  He subsequently denied having received the money and used the incident as a basis for discrediting the man.  He thus used his own wrong-doing as a means of destroying the reputation of another.  In fact, his rival was ruined.
He felt that he had done a wrong he could not possibly make right.  If he opened that old affair, he was afraid it would destroy the reputation of his partner, disgrace his family and take away his means of livelihood.  What right had he to involve those dependent upon him?  How could he possibly make a public statement exonerating his rival?
After consulting with his wife and partner he came to the conclusion that is was better to take those risks than to stand before his Creator guilty of such ruinous slander.  He saw that he had to place the outcome in God's hands or he would soon start drinking again, and all would be lost anyhow.  He attended church for the first time in many years.  After the sermon, he quietly got up and made an explanation.  His action met wide-spread approval, and today he is one of the most trusted citizens of his town.  This all happened years ago.
The chances are that we have domestic troubles.  Perhaps we are mixed up with women in a fashion we wouldn't care to have advertised.  We doubt if, in this respect, alcoholics are fundamentally much worse than other people.  But drinking does complicate sex relations in the home.  After a few years with an alcoholic, a wife gets worn out, resentful and uncommunicative.  How could she be anything else?  The husband begins to feel lonely, sorry for himself.  He commences to look around in the night clubs, or their equivalent, for something besides liquor.  Perhaps he is having a secret and exciting affair with "the girl who understands."  In fairness we must say that she may understand, but what are we going to do about a thing like that?  A man so involved often feels very remorseful at times, especially if he is married to a loyal and courageous girl who has literally gone through hell for him.
Whatever the situation, we usually have to do something about it.  If we are sure our wife does not know, should we tell her?  Not always, we think.  If she knows in a general way that we have been wild, should we tell her in detail?  Undoubtedly we should admit our fault.  She may insist on knowing all the particulars.  She will want to know who the woman is and where she is.  We feel we ought to say to her that we have no right to involve another person.  We are sorry for what we have done and, God willing, it shall not be repeated.  More than that we cannot do; we have no right to go further.  Though there may be justifiable exceptions, and though we wish to lay down no rule of any sort, we have often found this the best course to take.
Our design for living is not a one-way street.  It is as good for the wife as for the husband.  If we can forget, so can she.  It is better, however, that one does not needlessly name a person upon whom she can vent jealousy.
Perhaps there are some cases where the utmost frankness is demanded.  No outsider can appraise such an intimate situation.  It may be that both will decide that the way of good sense and loving kindness is to let by-gones be by-gones.  Each might pray about it, having the other one's happiness uppermost in mind.  Keep it always in sight that we are dealing with that most terrible human emotion-jealousy.  Good generalship may decide that the problem be attacked on the flank rather than risk a face-to-face combat.
If we have no such complication, there is plenty we should do at home.  Sometimes we hear an alcoholic say that the only thing he needs to do is to keep sober.  Certainly he must keep sober, for there will be no home if he doesn't.  But he is yet a long way from making good to the wife or parents whom for years he has so shockingly treated.  Passing all understanding is the patience mothers and wives have had with alcoholics.  Had this not been so, many of us would have no homes today, would perhaps be dead.
The alcoholic is like a tornado roaring his way through the lives of others.  Hearts are broken.  Sweet relationships are dead.  Affections have been uprooted.  Selfish and inconsiderate habits have kept the home in turmoil.  We feel a man is unthinking when he says that sobriety is enough.  He is like the farmer who came up out of his cyclone cellar to find his home ruined.  To his wife, he remarked, "Don't see anything the matter here, Ma.  Ain't it grand the wind stopped blowin'?"
Yes, there is a long period of reconstruction ahead.  We must take the lead.  A remorseful mumbling that we are sorry won't fill the bill at all.  We ought to sit down with the family and frankly analyze the past as we now see it, being very careful not to criticize them.  Their defects may be glaring, but the chances are that our own actions are partly responsible.  So we clean house with the family, asking each morning in meditation that our Creator show us the way of patience, tolerance, kindliness and love.
The spiritual life is not a theory.  We have to live it.  Unless one's family expresses a desire to live upon spiritual principles we think we ought not to urge them.  We should not talk incessantly to them about spiritual matters.  They will change in time.  Our behavior will convince them more than our words.  We must remember that ten or twenty years of drunkenness would make a skeptic out of anyone.
There may be some wrongs we can never fully right.  We don't worry about them if we can honestly say to ourselves that we would right them if we could.  Some people cannot be seen-we send them an honest letter.  And there may be a valid reason for postponement in some cases.  But we don't delay if it can be avoided.  We should be sensible, tactful, considerate and humble without being servile or scraping.  As God's people we stand on our feet; we don't crawl before anyone.
If we are painstaking about this phase of our development, we will be amazed before we are half way through.  We are going to know a new freedom and a new happiness.  We will not regret the past nor wish to shut the door on it.  We will comprehend the word serenity and we will know peace.  No matter how far down the scale we have gone, we will see how our experience can benefit others.  That feeling of uselessness and self-pity will disappear.  We will lose interest in selfish things and gain interest in our fellows.  Self-seeking will slip away.  Our whole attitude and outlook upon life will change.  Fear of people and of economic insecurity will leave us.  We will intuitively know how to handle situations which used to baffle us.  We will suddenly realize that God is doing for us what we could not do for ourselves.
Are these extravagant promises?  We think not.  They are being fulfilled among us-sometimes quickly, sometimes slowly.  They will always materialize if we work for them.
This thought brings us to Step Ten, which suggests we continue to take personal inventory and continue to set right any new mistakes as we go along.  We vigorously commenced this way of living as we cleaned up the past.  We have entered the world of the Spirit.  Our next function is to grow in understanding and effectiveness.  This is not an overnight matter.  It should continue for our lifetime.  Continue to watch for selfishness, dishonesty, resentment, and fear.  When these crop up, we ask God at once to remove them.  We discuss them with someone immediately and make amends quickly if we have harmed anyone.  Then we resolutely turn our thoughts to someone we can help.  Love and tolerance of others is our code.
And we have ceased fighting anything or anyone-even alcohol.  For by this time sanity will have returned.  We will seldom be interested in liquor.  If tempted, we recoil from it as from a hot flame.  We react sanely and normally, and we will find that this has happened automatically.  We will see that our new attitude toward liquor has been given us without any thought or effort on our part.  It just comes!  That is the miracle of it.  We are not fighting it, neither are we avoiding temptation.  We feel as though we had been placed in a position of neutrality-safe and protected.  We have not even sworn off.  Instead, the problem has been removed.  It does not exist for us.  We are neither cocky nor are we afraid.  That is our experience.  That is how we react so long as we keep in fit spiritual condition.
It is easy to let up on the spiritual program of action and rest on our laurels.  We are headed for trouble if we do, for alcohol is a subtle foe.  We are not cured of alcoholism.  What we really have is a daily reprieve contingent on the maintenance of our spiritual condition.  Every day is a day when we must carry the vision of God's will into all of our activities.  "How can I best serve Thee-Thy will (not mine) be done."  These are thoughts which must go with us constantly.  We can exercise our will power along this line all we wish.  It is the proper use of the will.
Much has already been said about receiving strength, inspiration, and direction from Him who has all knowledge and power.  If we have carefully followed directions, we have begun to sense the flow of His Spirit into us.  To some extent we have become God-conscious.  We have begun to develop this vital sixth sense.  But we must go further and that means more action.
Step Eleven suggests prayer and meditation.  We shouldn't be shy on this matter of prayer.  Better men than we are using it constantly.  It works, if we have the proper attitude and work at it.  It would be easy to be vague about this matter.  Yet, we believe we can make some definite and valuable suggestions.
When we retire at night, we constructively review our day.  Were we resentful, selfish, dishonest or afraid?  Do we owe an apology?  Have we kept something to ourselves which should be discussed with another person at once?  Were we kind and loving toward all?  What could we have done better?  Were we thinking of ourselves most of the time?  Or were we thinking of what we could do for others, of what we could pack into the stream of life?  But we must be careful not to drift into worry, remorse or morbid reflection, for that would diminish our usefulness to others.  After making our review we ask God's forgiveness and inquire what corrective measures should be taken.
On awakening let us think about the twenty-four hours ahead.  We consider our plans for the day.  Before we begin, we ask God to direct our thinking, especially asking that it be divorced from self-pity, dishonest or self-seeking motives.  Under these conditions we can employ our mental faculties with assurance, for after all God gave us brains to use.  Our thought-life will be placed on a much higher plane when our thinking is cleared of wrong motives. 
In thinking about our day we may face indecision.  We may not be able to determine which course to take.  Here we ask God for inspiration, an intuitive thought or a decision.  We relax and take it easy.  We don't struggle.  We are often surprised how the right answers come after we have tried this for a while.  What used to be the hunch or the occasional inspiration gradually becomes a working part of the mind.  Being still inexperienced and having just made conscious contact with God, it is not probable that we are going to be inspired at all times.  We might pay for this presumption in all sorts of absurd actions and ideas.  Nevertheless, we find that our thinking will, as time passes, be more and more on the plane of inspiration.  We come to rely upon it.
We usually conclude the period of meditation with a prayer that we be shown all through the day what our next step is to be, that we be given whatever we need to take care of such problems.  We ask especially for freedom from self-will, and are careful to make no request for ourselves only.  We may ask for ourselves, however, if others will be helped.  We are careful never to pray for our own selfish ends.  Many of us have wasted a lot of time doing that and it doesn't work.  You can easily see why.
If circumstances warrant, we ask our wives or friends to join us in morning meditation.  If we belong to a religious denomination which requires a definite morning devotion, we attend to that also.  If not members of religious bodies, we sometimes select and memorize a few set prayers which emphasize the principles we have been discussing.  There are many helpful books also.  Suggestions about these may be obtained from one's priest, minister, or rabbi.  Be quick to see where religious people are right.  Make use of what they offer.
As we go through the day we pause, when agitated or doubtful, and ask for the right thought or action.  We constantly remind ourselves we are no longer running the show, humbly saying to ourselves many times each day "Thy will be done."  We are then in much less danger of excitement, fear, anger, worry, self-pity, or foolish decisions.  We become much more efficient.  We do not tire so easily, for we are not burning up energy foolishly as we did when we were trying to arrange life to suit ourselves.
It works-it really does.
We alcoholics are undisciplined.  So we let God discipline us in the simple way we have just outlined.
But this is not all.  There is action and more action.  "Faith without works is dead."  The next chapter is entirely devoted to Step Twelve.

\end{biblechapter}


\biblebook{Vision}

\bbChapterPreamble

\innerVerseHeading{'A VISION FOR YOU' p. 151-164}

\begin{biblechapter}
\verseWithHeading{Introduction}
FOR MOST normal folks, drinking means conviviality, companionship and colorful imagination.  It means release from care, boredom and worry.  It is joyous intimacy with friends and a feeling that life is good.  But not so with us in those last days of heavy drinking.  The old pleasures were gone.  They were but memories.  Never could we recapture the great moments of the past.  There was an insistent yearning to enjoy life as we once did and a heartbreaking obsession that some new miracle of control would enable us to do it.  There was always one more attempt-and one more failure.

The less people tolerated us, the more we withdrew from society, from life itself.  As we became subjects of King Alcohol, shivering denizens of his mad realm, the chilling vapor that is loneliness settled down.  It thickened, ever becoming blacker.  Some of us sought out sordid places, hoping to find understanding companionship and approval.  Momentarily we did-then would come oblivion and the awful awakening to face the hideous Four Horsemen-Terror, Bewilderment, Frustration, Despair.  Unhappy drinkers who read this page will understand!

Now and then a serious drinker, being dry at the moment says, "I don't miss it at all.  Feel better.  Work better.  Having a better time."  As ex-problem drinkers, we smile at such a sally.  We know our friend is like a boy whistling in the dark to keep up his spirits.  He fools himself.  Inwardly he would give anything to take half a dozen drinks and get away with them.  He will presently try the old game again, for he isn't happy about his sobriety.  He cannot picture life without alcohol.  Some day he will be unable to imagine life either with alcohol or without it.  Then he will know loneliness such as few do.  He will be at the jumping-off place.  He will wish for the end.

We have shown how we got out from under.  You say, "Yes, I'm willing.  But am I to be consigned to a life where I shall be stupid, boring and glum, like some righteous people I see?  I know I must get along without liquor, but how can I?  Have you a sufficient substitute?"

Yes, there is a substitute and it is vastly more than that.  It is a fellowship in Alcoholics Anonymous.  There you will find release from care, boredom and worry.  Your imagination will be fired.  Life will mean something at last.  The most satisfactory years of your existence lie ahead.  Thus we find the fellowship, and so will you.

"How is that to come about?" you ask.  "Where am I to find these people?"

You are going to meet these new friends in your own community.  Near you, alcoholics are dying helplessly like people in a sinking ship.  If you live in a large place, there are hundreds.  High and low, rich and poor, these are future fellows of Alcoholics Anonymous.  Among them you will make lifelong friends.  You will be bound to them with new and wonderful ties, for you will escape disaster together and you will commence shoulder to shoulder your common journey.  Then you will know what it means to give of yourself that others may survive and rediscover life.  You will learn the full meaning of "Love thy neighbor as thyself."

It may seem incredible that these men are to become happy, respected, and useful once more.  How can they rise out of such misery, bad repute and hopelessness?  The practical answer is that since these things have happened among us, they can happen with you.  Should you wish them above all else, and be willing to make use of our experience, we are sure they will come.  The age of miracles is still with us.  Our own recovery proves that!

Our hope is that when this chip of a book is launched on the world tide of alcoholism, defeated drinkers will seize upon it, to follow its suggestions.  Many, we are sure, will rise to their feet and march on.  They will approach still other sick ones and fellowships of Alcoholics Anonymous may spring up in each city and hamlet, havens for those who must find a way out.

In the chapter "Working With Others" you gathered an idea of how we approach and aid others to health.  Suppose now that through you several families have adopted this way of life.  You will want to know more of how to proceed from that point.  Perhaps the best way of treating you to a glimpse of your future will be to describe the growth of the fellowship among us.  Here is a brief account:

Years ago, in 1935, one of our number made a journey to a certain western city.  From a business standpoint, his trip came off badly.  Had he been successful in his enterprise, he would have  been set on his feet financially which, at the time, seemed vitally important.  But his venture wound up in a law suit and bogged down completely.  The proceeding was shot through with much hard feeling and controversy.

Bitterly discouraged, he found himself in a strange place, discredited and almost broke.  Still physically weak, and sober but a few months, he saw that his predicament was dangerous.  He wanted so much to talk with someone, but whom?

One dismal afternoon he paced a hotel lobby wondering how his bill was to be paid.  At one end of the room stood a glass covered directory of local churches.  Down the lobby a door opened into an attractive bar.  He could see the gay crowd inside.  In there he would find companionship and release.  Unless he took some drinks, he might not have the courage to scrape an acquaintance and would have a lonely week-end.

Of course he couldn't drink, but why not sit hopefully at a table, a bottle of ginger ale before him?  After all, had he not been sober six months now?  Perhaps he could handle, say, three drinks-no more!  Fear gripped him.  He was on thin ice.  Again it was the old, insidious insanity-that first drink.  With a shiver, he turned away and walked down the lobby to the church directory.  Music and gay chatter still floated to him from the bar.

But what about his responsibilities-his family and the men who would die because they would not know how to get well, ah-yes, those other alcoholics?  There must be many such in this town.  He would phone a clergyman.  His sanity returned and he thanked God.  Selecting a church at random from the directory, he stepped into a booth and lifted the receiver.

His call to the clergyman led him presently to a certain resident of the town, who, though formerly able and respected, was then nearing the nadir of alcoholic despair.  It was the usual situation: home in jeopardy, wife ill, children distracted, bills in arrears and standing damaged.  He had a desperate desire to stop, but saw no way out, for he had earnestly tried many avenues of escape.  Painfully aware of being somehow abnormal, the man did not fully realize what it meant to be alcoholic.

When our friend related his experience, the man agreed that no amount of will power he might muster could stop his drinking for long.  A spiritual experience, he conceded, was absolutely necessary, but the price seemed high upon the basis suggested.  He told how he lived in constant worry about those who might find out about his alcoholism.  He had, of course, the familiar alcoholic obsession that few knew of his drinking.  Why, he argued, should he lose the remainder of his business, only to bring still more suffering to his family by foolishly admitting his plight to people from whom he made his livelihood?  He would do anything, he said, but that.

Being intrigued, however, he invited our friend to his home.  Some time later, and just as he thought  he was getting control of his liquor situation, he went on a roaring bender.  For him, this was the spree that ended all sprees.  He saw that he would have to face his problems squarely that God might give him mastery.

One morning he took the bull by the horns and set out to tell those he feared what his trouble had been.  He found himself surprisingly well received, and learned that many knew of his drinking.  Stepping into his car, he made the rounds of people he had hurt.  He trembled as he went about, for this might mean ruin, particularly to a person in his line of business.

At midnight he came home exhausted, but very happy.  He has not had a drink since.  As we shall see, he now means a great deal to his community, and the major liabilities of thirty years of hard drinking have been repaired in four.


But life was not easy for the two friends.  Plenty of difficulties presented themselves.  Both saw that they must keep spiritually active.  One day they called up the head nurse of a local hospital.  They explained their need and inquired if she had a first class alcoholic prospect.

She replied, "Yes, we've got a corker.  He's just beaten up a couple of nurses.  Goes off his head completely when he's drinking.  But he's a grand chap when he's sober, though he's been in here eight times in the last six months.  Understand he was once a well-known lawyer in town, but just now we've got him strapped down tight." 

Here was a prospect all right but, by the description, none too promising.  The use of spiritual principles in such cases was not so well understood as it is now.  But one of the friends said, "Put him in a private room.  We'll be down."

Two days later, a future fellow of Alcoholics Anonymous stared glassily at the strangers beside his bed.  "Who are you fellows, and why this private room?  I was always in a ward before."

Said one of the visitors, "We're giving you a treatment for alcoholism."

Hopelessness was written large on the man's face as he replied, "Oh, but that's no use.  Nothing would fix me.  I'm a goner.  The last three times, I got drunk on the way home from here.  I'm afraid to go out the door.  I can't understand it."

For an hour, the two friends told him about their drinking experiences.  Over and over, he would say: "That's me.  That's me.  I drink like that."

The man in the bed was told of the acute poisoning from which he suffered, how it deteriorates the body of an alcoholic and warps his mind.  There was much talk about the mental state preceding the first drink.

"Yes, that's me," said the sick man, "the very image.  You fellows know your stuff all right, but I don't see what good it'll do.  You fellows are somebody.  I was once, but I'm a nobody now.  From what you tell me, I know more than ever I can't stop."  At this both the visitors burst into a laugh.  Said the future Fellow Anonymous: "Damn little to laugh about that I can see."

The two friends spoke of their spiritual experience and told him about the course of action they carried out.

He interrupted: "I used to be strong for the church, but that won't fix it.  I've prayed to God on hangover mornings and sworn that I'd never touch another drop but by nine o'clock I'd be boiled as an owl."

Next day found the prospect more receptive.  He had been thinking it over.  "Maybe you're right," he said.  "God ought to be able to do anything."  Then he added, "He sure didn't do much for me when I was trying to fight this booze racket alone."

On the third day the lawyer gave his life to the care and direction of his Creator, and said he was perfectly willing to do anything necessary.  His wife came, scarcely daring to be hopeful, though she thought she saw something different about her husband already.  He had begun to have a spiritual experience.

That afternoon he put on his clothes and walked from the hospital a free man.  He entered a political campaign, making speeches, frequenting men's gathering places of all sorts, often staying up all night.  He lost the race by only a narrow margin.  But he had found God-and in finding God had found himself.

That was in June, 1935.  He never drank again.  He too, has become a respected and useful member of his community.  He has helped other men recover, and is a power in the church from which he was long absent.

So, you see, there were three alcoholics in that town, who now felt they had to give to others what they had found, or be sunk.  After several failures to find others, a fourth turned up.  He came through an acquaintance who had heard the good news.  He proved to be a devil-may-care young fellow whose parents could not make out whether he wanted to stop drinking or not.  They were deeply religious people, much shocked by their son's refusal to have anything to do with the church.  He suffered horribly from his sprees, but it seemed as if nothing could be done for him.  He consented, however, to go to the hospital, where he occupied the very room recently vacated by the lawyer.

He had three visitors.  After a bit, he said, "The way you fellows put this spiritual stuff makes sense.  I'm ready to do business.  I guess the old folks were right after all."  So one more was added to the Fellowship.

All this time our friend of the hotel lobby incident remained in that town.  He was there three months.  He now returned home, leaving behind his first acquaintance, the lawyer and the devil-may-care chap.  These men had found something brand new in life.  Though they knew they must help other alcoholics if they would remain sober, that motive became secondary.  It was transcended by the happiness they found in giving  themselves for others.  They shared their homes, their slender resources, and gladly devoted their spare hours to fellow-sufferers.  They were willing, by day or night, to place a new man in the hospital and visit him afterward.  They grew in numbers.  They experienced a few distressing failures, but in those cases they made an effort to bring the man's family into a spiritual way of living, thus relieving much worry and suffering.

A year and six months later these three had succeeded with seven more.  Seeing much of each other, scarce an evening passed that someone's home did not shelter a little gathering of men and women, happy in their release, and constantly thinking how they might present their discovery to some newcomer.  In addition to these casual get-togethers, it became customary to set apart one night a week for a meeting to be attended by anyone or everyone interested in a spiritual way of life.  Aside from fellowship and sociability, the prime object was to provide a time and place where new people might bring their problems.

Outsiders became interested.  One man and his wife placed their large home at the disposal of this strangely assorted crowd.  This couple has since become so fascinated that they have dedicated their home to the work.  Many a distracted wife has visited this house to find loving and understanding companionship among women who knew her problem, to hear from the lips of their husbands what had happened to them, to be advised how her own wayward mate might be hospitalized and approached when next he stumbled.

Many a man, yet dazed from his hospital experience, has stepped over the threshold of that home into freedom.  Many an alcoholic who entered there came away with an answer.  He succumbed to that gay crowd inside, who laughed at their own misfortunes and understood his.  Impressed by those who visited him at the hospital, he capitulated entirely when, later, in an upper room of this house, he heard the story of some man whose experience closely tallied with his own.  The expression on the faces of the women, that indefinable something in the eyes of the men, the stimulating and electric atmosphere of the place, conspired to let him know that here was haven at last.

The very practical approach to his problems, the absence of intolerance of any kind, the informality, the genuine democracy, the uncanny understanding which these people had were irresistible.  He and his wife would leave elated by the thought of what they could now do for some stricken acquaintance and his family.  They knew they had a host of new friends; it seemed they had known these strangers always.  They had seen miracles, and one was to come to them.  They had visioned the Great Reality-their loving and All Powerful Creator.

Now, this house will hardly accommodate its weekly visitors, for they number sixty or eighty as a rule.  Alcoholics are being attracted from far and near.  From surrounding towns, families drive long distances to be present.  A community thirty miles away has fifteen fellows of Alcoholics Anonymous.  Being a large place, we think that some day its Fellowship will number many hundreds.

But life among Alcoholics Anonymous is more than attending gatherings and visiting hospitals.  Cleaning up old scrapes, helping to settle family differences, explaining the disinherited son to his irate parents, lending money and securing jobs for each other, when justified-these are everyday occurrences.  No one is too discredited or has sunk too low to be welcomed cordially-if he means business.  Social distinctions, petty rivalries and jealousies-these are laughed out of countenance.  Being wrecked in the same vessel, being restored and united under one God, with hearts and minds attuned to the welfare of others, the things which matter so much to some people no longer signify much to them.  How could they?

Under only slightly different conditions, the same thing is taking place in many eastern cities.  In one of these there is a well-known hospital for the treatment of alcoholic and drug addiction.  Six years ago one of our number was a patient there.  Many of us have felt, for the first time, the Presence and Power of God within its walls.  We are greatly indebted to the doctor in attendance there, for he, although it might prejudice his own work, has told us of his belief in ours.

Every few days this doctor suggests our approach to one of his patients.  Understanding our work, he can do this with an eye to selecting those who are willing and able to recover on a spiritual basis.  Many of us, former patients, go there to help.  Then, in this eastern city, there are informal meetings such as we have described to you, where you may now see scores of members.  There are the same fast friendships, there is the same helpfulness to one another as you find among our western friends.  There is a good bit of travel between East and West and we foresee a great increase in this helpful interchange.

Some day we hope that every alcoholic who journeys will find a Fellowship of Alcoholics Anonymous at his destination.  To some extent this is already true.  Some of us are salesmen and go about.  Little clusters of twos and threes and fives of us have sprung up in other communities, through contact with our two larger centers.  Those of us who travel drop in as often as we can.  This practice enables us to lend a hand, at the same time avoiding certain alluring distractions of the road, about which any traveling man can inform you.

Thus we grow.  And so can you, though you be but one man with this book in your hand.  We believe and hope it contains all you will need to begin.

We know what you are thinking.  You are saying to yourself: "I'm jittery and alone.  I couldn't do that."  But you can.  You forget that you have just now tapped a source of power much greater than yourself.  To duplicate, with such backing, what we have accomplished is only a matter of willingness, patience and labor.

We know of an A.A. member who was living in a large community.  He had lived there but a few weeks when he found that the place probably contained more alcoholics per square mile than any city in the country.  This was only a few days ago at this writing.  (1939)  The authorities were much concerned.  He got in touch with a prominent psychiatrist who had undertaken certain responsibilities for the mental health of the community.  The doctor proved to be able and exceedingly anxious to adopt any workable method of handling the situation.  So he inquired, what did our friend have on the ball?

Our friend proceeded to tell him.  And with such good effect that the doctor agreed to a test among his patients and certain other alcoholics from a clinic which he attends.  Arrangements were also made with the chief psychiatrist of a large public hospital to select still others from the stream of misery which flows through that institution.

So our fellow worker will soon have friends galore.  Some of them may sink and perhaps never get up, but if our experience is a criterion, more than half of those approached will become fellows of Alcoholics Anonymous.  When a few men in this city have found themselves, and have discovered the joy of helping others to face life again, there will be no stopping until everyone in that town has had his opportunity to recover-if he can and will.

Still you may say: "But I will not have the benefit of contact with you who write this book."  We cannot be sure.  God will determine that, so you must remember that your real reliance is always upon Him.  He will show you how to create the fellowship you crave.

Our book is meant to be suggestive only.  We realize we know only a little.  God will constantly disclose more to you and to us.  Ask Him in your morning meditation what you can do each day for the man who is still sick.  The answers will come, if your own house is in order.  But obviously you cannot transmit something you haven't got.  See to it that your relationship with Him is right, and great events will come to pass for you and countless others.  This is the Great Fact for us.

Abandon yourself to God as you understand God.  Admit your faults to Him and to your fellows.  Clear away the wreckage of your past.  Give freely of what you find and join us.  We shall be with you in the Fellowship of the Spirit, and you will surely meet some of us as you trudge the Road of Happy Destiny.

May God bless you and keep you-until then.

\end{biblechapter}

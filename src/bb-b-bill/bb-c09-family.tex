\biblebook{Family}

\bbChapterPreamble

\innerVerseHeading{'THE FAMILY AFTERWORD' p. 122-135}

\begin{biblechapter}
\verseWithHeading{Introduction}
OUR WOMEN FOLK have suggested certain attitudes a wife may take with the husband who is recovering.  Perhaps they created the impression that he is to be wrapped in cotton wool and placed on a pedestal.  Successful readjustment means the opposite.  All members of the family should meet upon the common ground of tolerance, understanding and love.  This involves a process of deflation.  The alcoholic, his wife, his children, his "in-laws," each one is likely to have fixed ideas about the family's attitude towards himself or herself.  Each is interested in having his or her wishes respected.  We find the more one member of the family demands that the others concede to him, the more resentful they become.  This makes for discord and unhappiness.

And why?  Is it not because each wants to play the lead?  Is not each trying to arrange the family show to his liking?  Is he not unconsciously trying to see what he can take from the family life rather than give?

Cessation of drinking is but the first step away from a highly strained, abnormal condition.  A doctor said to us, "Years of living with an alcoholic is almost sure to make any wife or child neurotic.  The entire family is, to some extent, ill."  Let families realize, as they start their journey, that all will not be fair weather.  Each in his turn may be footsore and may straggle.  There will be alluring shortcuts and by-paths down which they may wander and lose their way.

Suppose we tell you some of the obstacles a family will meet; suppose we suggest how they may be avoided-even converted to good use for others.  The family of an alcoholic longs for the return of happiness and security.  They remember when father was romantic, thoughtful and successful.  Today's life is measured against that of other years and, when it falls short, the family may be unhappy.

Family confidence in dad is rising high.  The good old days will soon be back, they think.  Sometimes they demand that dad bring them back instantly!  God, they believe, almost owes this recompense on a long overdue account.  But the head of the house has spent years in pulling down the structures of business, romance, friendship, health-these things are now ruined or damaged.  It will take time to clear away the wreck.  Though old buildings will eventually be replaced by finer ones, the new structures will take years to complete.

Father knows he is to blame; it may take him many seasons of hard work to be restored financially, but he shouldn't be reproached.  Perhaps he will never have much money again.  But the wise family will admire him for what he is trying to be, rather than for what he is trying to get.

Now and then the family will be plagued by spectres from the past, for the drinking career of almost every alcoholic has been marked by escapades, funny, humiliating, shameful or tragic.  The first impulse will be to bury these skeletons in a dark closet and padlock the door.  The family may be possessed by the idea that future happiness can be based only upon forgetfulness of the past.  We think that such a view is self-centered and in direct conflict with the new way of living.

Henry Ford once made a wise remark to the effect that experience is the thing of supreme value in life.  That is true only if one is willing to turn the past to good account.  We grow by our willingness to face and rectify errors and convert them into assets.  The alcoholic's past thus becomes the principal asset of the family and frequently it is almost the only one!

This painful past may be of infinite value to other families still struggling with their problem.  We think each family which has been relieved owes something to those who have not, and when the occasion requires, each member of it should be only too willing to bring former mistakes, no matter how grievous, out of their hiding places.  Showing others who suffer how we were given help is the very thing which makes life seem so worth while to us now.  Cling to the thought that, in God's hands, the dark past is the greatest possession you have-the key to life and happiness for others.  With it you can avert death and misery for them.

It is possible to dig up past misdeeds so they become a blight, a veritable plague.  For example, we know of situations in which the alcoholic or his wife have had love affairs.  In the first flush of spiritual experience they forgave each other and drew closer together.  The miracle of reconciliation was at hand.  Then, under one provocation or another, the aggrieved one would unearth the old affair and angrily cast its ashes about.  A few of us have had these growing pains and they hurt a great deal.  Husbands and wives have sometimes been obliged to separate for a time until new perspective, new victory over hurt pride could be re-won.  In most cases, the alcoholic survived this ordeal without relapse, but not always.  So we think that unless some good and useful purpose is to be served, past occurrences should not be discussed.

We families of Alcoholics Anonymous keep few skeletons in the closet.  Everyone knows about the others' alcoholic troubles.  This is a condition which, in ordinary life, would produce untold grief; there might be scandalous gossip, laughter at the expense of other people, and a tendency to take advantage of intimate information.  Among us, these are rare occurrences.  We do talk about each other a great deal, but we almost invariably temper such talk by a spirit of love and tolerance.

Another principle we observe carefully is that we do not relate intimate experiences of another person unless we are sure he would approve.  We find it better, when possible, to stick to our own stories.  A man may criticize or laugh at himself and it will affect others favorably, but criticism or ridicule coming from another often produces the contrary effect.  Members of a family should watch such matters carefully, for one careless, inconsiderate remark has been known to raise the very devil.  We alcoholics are sensitive people.  It takes some of us a long time to outgrow that serious handicap.

Many alcoholics are enthusiasts.  They run to extremes.  At the beginning of recovery a man will take, as a rule, one of two directions.  He may either plunge into a frantic attempt to get on his feet in business, or he may be so enthralled by his new life that he talks or thinks of little else.  In either case certain family problems will arise.  With these we have had experience galore.

We think it dangerous if he rushes headlong at his economic problem.  The family will be affected also, pleasantly at first, as they feel their money troubles are about to be solved, then not so pleasantly as they find themselves neglected.  Dad may be tired at night and preoccupied by day.  He may take small interest in the children and may show irritation when reproved for his delinquencies.  If not irritable, he may seem dull and boring, not gay and affectionate as the family would like him to be.  Mother may complain of inattention.  They are all disappointed, and often let him feel it.  Beginning with such complaints, a barrier arises.  He is straining every nerve to make up for lost time.  He is striving to recover fortune and reputation and feels he is doing very well.

Sometimes mother and children don't think so.  Having been neglected and misused in the past, they think father owes them more than they are getting.  They want him to make a fuss over them.  They expect him to give them the nice times they used to have before he drank so much, and to show his contrition for what they suffered.  But dad doesn't give freely of himself.  Resentment grows.  He becomes still less communicative.  Sometimes he explodes over a trifle.  The family is mystified.  They criticize, pointing out how he is falling down on his spiritual program.

This sort of thing can be avoided.  Both father and the family are mistaken, though each side may have some justification.  It is of little use to argue and only makes the impasse worse.  The family must realize that dad, though marvelously improved, is still convalescing.  They should be thankful he is sober and able to be of this world once more.  Let them praise his progress.  Let them remember that his drinking wrought all kinds of damage that may take long to repair.  If they sense these things, they will not take so seriously his periods of crankiness, depression, or apathy, which will disappear when there is tolerance, love, and spiritual understanding.

The head of the house ought to remember that he is mainly to blame for what befell his home.  He can scarcely square the account in his lifetime.  But he must see the danger of over-concentration on financial success.  Although financial recovery is on the way for many of us, we found we could not place money first.  For us, material well-being always followed spiritual progress; it never preceded.

Since the home has suffered more than anything else, it is well that a man exert himself there.  He is not likely to get far in any direction if he fails to show unselfishness and love under his own roof.  We know there are difficult wives and families, but the man who is getting over alcoholism must remember he did much to make them so.

As each member of a resentful family begins to see his shortcomings and admits them to the others, he lays a basis for helpful discussion.  These family talks will be constructive if they can be carried on without heated argument, self-pity, self-justification or resentful criticism.  Little by little, mother and children will see they ask too much, and father will see he gives too little.  Giving, rather than getting, will become the guiding principle.

Assume on the other hand that father has, at the outset, a stirring spiritual experience.  Overnight, as it were, he is a different man.  He becomes a religious enthusiast.  He is unable to focus on anything else.  As soon as his sobriety begins to be taken as a matter of course, the family may look at their strange new dad with apprehension, then with irritation.  There is talk about spiritual matters morning, noon and night.  He may demand that the family find God in a hurry, or exhibit amazing indifference to them and say he is above worldly considerations.  He may tell mother, who has been religious all her life, that she doesn't know what it's all about, and that she had better get his brand of spirituality while there is yet time.

When father takes this tack, the family may react unfavorably.  They may be jealous of a God who has stolen dad's affections.  While grateful that he drinks no more, they may not like the idea that God has accomplished the miracle where they failed.  They often forget father was beyond human aid.  They may not see why their love and devotion did not straighten him out.  Dad is not so spiritual after all, they say.  If he means to right his past wrongs, why all this concern for everyone in the world but his family?  What about his talk that God will take care of them?  They suspect father is a bit balmy!

He is not so unbalanced as they might think.  Many of us have experienced dad's elation.  We have indulged in spiritual intoxication.  Like a gaunt prospector, belt drawn in over the last ounce of food, our pick struck gold.  Joy at our release from a lifetime of frustration knew no bounds.  Father feels he has struck something better than gold.  For a time he may try to hug the new treasure to himself.  He may not see at once that he has barely scratched a limitless lode which will pay dividends only if he mines it for the rest of his life and insists on giving away the entire product. 

If the family cooperates, dad will soon see that he is suffering from a distortion of values.  He will perceive that his spiritual growth is lopsided, that for an average man like himself, a spiritual life which does not include his family obligations may not be so perfect after all.  If the family will appreciate that dad's current behavior is but a phase of his development, all will be well.  In the midst of an understanding and sympathetic family, these vagaries of dad's spiritual infancy will quickly disappear.

The opposite may happen should the family condemn and criticize.  Dad may feel that for years his drinking has placed him on the wrong side of every argument, but that now he has become a superior person with God on his side.  If the family persists in criticism, this fallacy may take a still greater hold on father.  Instead of treating the family as he should, he may retreat further into himself and feel he has spiritual justification for so doing.

Though the family does not fully agree with dad's spiritual activities, they should let him have his head.  Even if he displays a certain amount of neglect and irresponsibility towards the family, it is well to let him go as far as he likes in helping other alcoholics.  During those first days of convalescence, this will do more to insure his sobriety than anything else.  Though some of his manifestations are alarming and disagreeable, we think dad will be on a firmer foundation than the man who is placing business or professional success ahead of spiritual development.  He will be less likely to drink again, and anything is preferable to that.

Those of us who have spent much time in the world of spiritual make-believe have eventually seen the childishness of it.  This dream world has been replaced by a great sense of purpose, accompanied by a growing consciousness of the power of God in our lives.  We have come to believe He would like us to keep our heads in the clouds with Him, but that our feet ought to be firmly planted on earth.  That is where our fellow travelers are, and that is where our work must be done.  These are the realities for us.  We have found nothing incompatible between a powerful spiritual experience and a life of sane and happy usefulness.

One more suggestion:  Whether the family has spiritual convictions or not, they may do well to examine the principles by which the alcoholic member is trying to live.  They can hardly fail to approve these simple principles, though the head of the house still fails somewhat in practicing them.  Nothing will help the man who is off on a spiritual tangent so much as the wife who adopts a sane spiritual program, making a better practical use of it.

There will be other profound changes in the household.  Liquor incapacitated father for so many years that mother became head of the house.  She met these responsibilities gallantly.  By force of circumstances, she was often obliged to treat father as a sick or wayward child.  Even when he wanted to assert himself he could not, for his drinking placed him constantly in the wrong.  Mother made all the plans and gave the directions.  When sober, father usually obeyed.  Thus mother, through no fault of her own, became accustomed to wearing the family trousers.  Father, coming suddenly to life again, often begins to assert himself.  This means trouble, unless the family watches for these tendencies in each other and comes to a friendly agreement about them.

Drinking isolates most homes from the outside world.  Father may have laid aside for years all normal activities-clubs, civic duties, sports.  When he renews interest in such things, a feeling of jealousy may arise.  The family may feel they hold a mortgage on dad, so big that no equity should be left for outsiders.  Instead of developing new channels of activity for themselves, mother and children demand that he stay home and make up the deficiency.

At the very beginning, the couple ought to frankly face the fact that each will have to yield here and there if the family is going to play an effective part in the new life.  Father will necessarily spend much time with other alcoholics, but this activity should be balanced.  New acquaintances who know nothing of alcoholism might be made and thoughtful consideration given their needs.  The problems of the community might engage attention.  Though the family has no religious connections, they may wish to make contact with or take membership in a religious body.

Alcoholics who have derided religious people will be helped by such contacts.  Being possessed of a spiritual experience, the alcoholic will find he has much in common with these people, though he may differ with them on many matters.  If he does not argue about religion, he will make new friends and is sure to find new avenues of usefulness and pleasure.  He and his family can be a bright spot in such congregations.  He may bring new hope and new courage to many a priest, minister, or rabbi, who gives his all to minister to our troubled world.  We intend the foregoing as a helpful suggestion only.  So far as we are concerned, there is nothing obligatory about it.  As non-denominational people, we cannot make up others' minds for them.  Each individual should consult his own conscience.

We have been speaking to you of serious, sometimes tragic things.  We have been dealing with alcohol in its worst aspect.  But we aren't a glum lot.  If newcomers could see no joy or fun in our existence, they wouldn't want it.  We absolutely insist on enjoying life.  We try not to indulge in cynicism over the state of the nations, nor do we carry the world's troubles on our shoulders.  When we see a man sinking into the mire that is alcoholism, we give him first aid and place what we have at his disposal.  For his sake, we do recount and almost relive the horrors of our past.  But those of us who have tried to shoulder the entire burden and trouble of others find we are soon overcome by them.

So we think cheerfulness and laughter make for usefulness.  Outsiders are sometimes shocked when we burst into merriment over a seemingly tragic experience out of the past.  But why shouldn't we laugh?  We have recovered, and have been given the power to help others.

Everybody knows that those in bad health, and those who seldom play, do not laugh much.  So let each family play together or separately, as much as their circumstances warrant.  We are sure God wants us to be happy, joyous, and free.  We cannot subscribe to the belief that this life is a vale of tears, though it once was just that for many of us.  But it is clear that we made our own misery.  God didn't do it.  Avoid then, the deliberate manufacture of misery, but if trouble comes, cheerfully capitalize it as an opportunity to demonstrate His omnipotence.

Now about health:  A body badly burned by alcohol does not often recover overnight nor do twisted thinking and depression vanish in a twinkling.  We are convinced that a spiritual mode of living is a most powerful health restorative.  We, who have recovered from serious drinking, are miracles of mental health.  But we have seen remarkable transformations in our bodies.  Hardly one of our crowd now shows any mark of dissipation.

But this does not mean that we disregard human health measures.  God has abundantly supplied this world with fine doctors, psychologists, and practitioners of various kinds.  Do not hesitate to take your health problems to such persons.  Most of them give freely of themselves, that their fellows may enjoy sound minds and bodies.  Try to remember that though God has wrought miracles among us, we should never belittle a good doctor or psychiatrist.  Their services are often indispensable in treating a newcomer and in following his case afterward.

One of the many doctors who had the opportunity of reading this book in manuscript form told us that the use of sweets was often helpful, of course depending upon a doctor's advice.  He thought all alcoholics should constantly have chocolate available for its quick energy value at times of fatigue.  He added that occasionally in the night a vague craving arose which would be satisfied by candy.  Many of us have noticed a tendency to eat sweets and have found this practice beneficial.

A word about sex relations.  Alcohol is so sexually stimulating to some men that they have over-indulged.  Couples are occasionally dismayed to find that when drinking is stopped the man tends to be impotent.  Unless the reason is understood, there may be an emotional upset.  Some of us had this experience, only to enjoy, in a few months, a finer intimacy than ever.  There should be no hesitancy in consulting a doctor or psychologist if the condition persists.  We do not know of many cases where this difficulty lasted long.

The alcoholic may find it hard to re-establish friendly relations with his children.  Their young minds were impressionable while he was drinking.  Without saying so, they may cordially hate him for what he has done to them and to their mother.  The children are sometimes dominated by a pathetic hardness and cynicism.  They cannot seem to forgive and forget. This may hang on for months, long after their mother has accepted dad's new way of living and thinking.

In time they will see that he is a new man and in their own way they will let him know it.  When this happens, they can be invited to join in morning meditation and then they can take part in the daily discussion without rancor or bias.  From that point on, progress will be rapid.  Marvelous results often follow such a reunion.

Whether the family goes on a spiritual basis or not, the alcoholic member has to if he would recover.  The others must be convinced of his new status beyond the shadow of a doubt.  Seeing is believing to most families who have lived with a drinker.

Here is a case in point:  One of our friends is a heavy smoker and coffee drinker.  There was no doubt he over-indulged.  Seeing this, and meaning to be helpful, his wife commenced to admonish him about it.  He admitted he was overdoing these things, but frankly said that he was not ready to stop.  His wife is one of those persons who really feels there is something rather sinful about these commodities, so she nagged, and her intolerance finally threw him into a fit of anger.  He got drunk.

Of course our friend was wrong-dead wrong.  He had to painfully admit that and mend his spiritual fences.  Though he is now a most effective member of Alcoholics Anonymous, he still smokes and drinks coffee, but neither his wife nor anyone else stands in judgment.  She sees she was wrong to make a burning issue out of such a matter when his more serious ailments were being rapidly cured.

We have three little mottoes which are apropos.  Here they are:

First Things First

Live and Let Live

Easy Does It.

\end{biblechapter}


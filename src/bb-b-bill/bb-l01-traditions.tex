\biblebook{Traditions}

\innverVerseHeading{'THE A.A. TRADITION'}

\begin{biblechapter}
\verseWithHeadings{Origins}
    To those now in its fold, 
    Alcoholics Anonymous has made the difference 
    between misery and sobriety, 
    and often the difference between life and death. 
\verse A.A. can, of course, 
    mean just as much to uncounted alcoholics not yet reached.
  
\verse Therefore, no society of men and women 
    ever had a more urgent NEED for continuous effectiveness 
    and permanent unity.
\verse We alcoholics see that we must work together and hang together, 
    else most of us will finally die alone.

\verse The "12 Traditions" of Alcoholics Anonymous are, 
    we A.A.'s believe, 
    the best answers that our experience has yet given 
    to those ever urgent questions, 
    "How can A.A. best function?" 
    and, "How can A.A. best stay whole 
    and so survive?"
  
\verse On the next page, 
    A.A.'s "12 Traditions" are seen in their so-called "short form," 
    the form in general use today.
\verse This is a condensed version of the original 
    "long form" A.A. Traditions as first printed in 1946.
\verse Because the "long form" is more explicit 
    and of possible historic value, it is also reproduced.
\end{biblechapter}

\begin{biblechapter}
\verseWithHeadings{The Twelve Traditions}
     
    One--Our common welfare should come first; 
    personal recovery depends upon A.A. unity.

\verse Two--For our group purpose there is but one ultimate 
    authority--a loving God as He may express Himself 
    in our group conscience.
    Our leaders are but trusted servants;
    they do not govern.

\verse Three--The only requirement for A.A. membership 
    is a desire to stop drinking.

\verse Four--Each group should be autonomous 
    except in matters affecting other groups, 
    or A.A. as a whole.

\verse Five--Each group has but one primary purpose--to
    carry its message to the alcoholic who still suffers.

\verse Six--An A.A. group ought never endorse, finance or lend
    the A.A. name to any related facility or outside enterprise;
    lest problems of money, property and prestige 
    divert us from our primary purpose.

\verse Seven--Every A.A. group ought to be fully self-supporting, 
    declining outside contributions.

\verse Eight--Alcoholics Anonymous should remain forever nonprofessional, 
    but our service centers may employ special workers.

\verse Nine--A.A., as such, ought never be organized; 
    but we may create service boards or committees 
    directly responsible to those they serve.

\verse Ten--Alcoholics Anonymous has no opinion on outside issues; 
    hence the A.A. name ought never be drawn into public controversy.

\verse Eleven--Our public relations policy is based upon attraction 
    rather than promotion; 
    we need always maintain personal anonymity 
    at the level of press, radio and films.

\verse Twelve--Anonymity is the spiritual foundation of all our Traditions, 
    ever reminding us to place principles before personalities.
\end{biblechapter}


\begin{biblechapter}
\verseWithHeadings{The Long Form}
    Our A.A. experience has taught us that:
  
    1.--Each member of Alcoholics Anonymous 
    is but a small part of a great whole.
    A.A. must continue to live or most of us will surely die.
    Hence our common welfare comes first.
    But individual welfare follows close afterward.

\verse 2.--For our group purpose there is but one ultimate authority--a 
    loving God as He may express Himself in our group conscience.

\verse 3.--Our membership ought to include all who suffer from alcoholism.
    Hence we may refuse none who wish to recover.
    Nor ought A.A. membership ever depend upon money or conformity.
    Any two or three alcoholics gathered together for sobriety 
    may call themselves an A.A. Group, provided that, as a group, 
    they have no other affiliation.

\verse 4.--With respect to its own affairs, 
    each A.A. group should be responsible to no other authority 
    than its own conscience.
    But when its plans concern the welfare of neighboring groups also, 
    those groups ought to be consulted.
    And no group, regional committee, or individual 
    should ever take any action that might greatly affect A.A. as a whole 
    without conferring with the Trustees of the General Service Board.
    On such issues our common welfare is paramount.

\verse 5.--Each Alcoholics Anonymous group 
    ought to be a spiritual entity HAVING BUT ONE PRIMARY PURPOSE--that 
    of carrying its message to the alcoholic who still suffers.

\verse 6.--Problems of money, property, and authority 
    may easily divert us from our primary spiritual aim.
    We think, therefore, 
    that any considerable property of genuine use to A.A. 
    should be separately incorporated and managed,
    thus dividing the material from the spiritual.
    An A.A. group, as such, should never go into business.
    Secondary aids to A.A., such as clubs or hospitals 
    which require much property or administration, 
    ought to be incorporated and so set apart that, if necessary, 
    they can be freely discarded by the groups.
    Hence such facilities ought not to use the A.A. name.
    Their management should be the sole responsibility 
    of those people who financially support them.
    For clubs, A.A. managers are usually preferred.
    But hospitals, as well as other places of recuperation, 
    ought to be well outside A.A.--and medically supervised.
    While an A.A. group may cooperate with anyone, 
    such cooperation ought never go so far as affiliation or endorsement, 
    actual or implied.
    An A.A. group can bind itself to no one.

\verse 7.--The A.A. group themselves ought to be fully supported 
    by the voluntary contributions of their own members.
    We think that each group should soon achieve this ideal; 
    that any public solicitation of funds 
    using the name of Alcoholics Anonymous is highly dangerous, 
    whether by groups, clubs, hospitals, or other outside agencies; 
    that acceptance of large gifts from any source, 
    or of contributions carrying any obligation whatever, is unwise.
    Then too, we view with much concern 
    those A.A. treasuries which continue, beyond prudent reserves, 
    to accumulate funds for no stated A.A. purpose.
    Experience has often warned us that nothing can so surely destroy our spiritual heritage as futile disputes over 
    property, money, and authority.

\verse 8.--Alcoholics Anonymous should remain forever nonprofessional.
    We define professionalism as the occupation of 
    counseling alcoholics for fees or hire.
    But we may employ alcoholics 
    where they are going to perform those services for which 
    we might otherwise have to engage nonalcoholics.  
    Such special services may be well recompensed.
    But our usual A.A. "12th Step" work is never to be paid for.

\verse 9.--Each A.A. groups needs the least possible organization.
    Rotating leadership is the best.
    The small group may elect its Secretary, 
    the large group its Rotating Committee, 
    and the groups of a large Metropolitan area their Central or Intergroup Committee, 
    which often employs a full-time Secretary.
    The trustees of the General Service Board are, in effect, 
    our A.A. General Service Committee.
    They are the custodians of our A.A. Tradition 
    and the receivers of voluntary A.A. contributions 
    by which we maintain our A.A. General Service Office at New York.
    They are authorized by the groups 
    to handle our over-all public relations 
    and they guarantee the integrity of our principle newspaper, 
    "The A.A. Grapevine."
    All such representatives are to be guided in the spirit of service, 
    for true leaders in A.A. are but 
    trusted and experienced servants of the whole.
    They derive no real authority from their titles; 
    they do not govern.
    Universal respect is the key to their usefulness.

\verse 10.--No A.A. group or member should ever, 
    in such a way as to implicate A.A., 
    express any opinion on outside controversial issues--particularly 
    those of politics, alcohol reform, or sectarian religion.
    The Alcoholics Anonymous groups oppose no one.
    Concerning such matters they can express no views whatever.

\verse 11.--Our relations with the general public 
    should be characterized by personal anonymity.
    We think A.A. ought to avoid sensational advertising.
    Our names and pictures as A.A. members 
    ought not be broadcast, filmed, or publicly printed.
    Our public relations should be guided by the principle of attraction 
    rather than promotion.
    There is never need to praise ourselves.
    We feel it better to let our friends recommend us.

\verse 12.--And finally, we of Alcoholics Anonymous 
    believe that the principle of Anonymity 
    has an immense spiritual significance.
    It reminds us that we are to place principles before personalities; 
    that we are actually to practice a genuine humility.
    This to the end that our great blessings may never spoil us; 
    that we shall forever live in thankful contemplation 
    of Him who presides over us all.
\end{biblechapter}


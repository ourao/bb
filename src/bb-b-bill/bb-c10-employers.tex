\biblebook{Employers}

\bbChapterPreamble

\innerVerseHeading{'TO EMPLOYERS' p. 136-150}

\begin{biblechapter}
\verseWithHeading{Introduction}

AMONG MANY employers nowadays, we think of one member who has spent much of his life in the world of big business.  He has hired and fired hundreds of men.  He knows the alcoholic as the employer sees him.  His present views ought to prove exceptionally useful to business men everywhere. 

But let him tell you:

I was at one time assistant manager of a corporation department employing sixty-six hundred men.  One day my secretary came in saying that Mr. B- insisted on speaking with me.  I told her to say that I was not interested.  I had warned him several times that he had but one more chance.  Not long afterward he had called me from Hartford on two successive days, so drunk he could hardly speak.  I told him he was through-finally and forever.

My secretary returned to say that it was not Mr. B- on the phone; it was Mr. B-'s brother, and he wished to give me a message.  I still expected a plea for clemency, but these words came through the receiver:  "I just wanted to tell you Paul jumped from a hotel window in Hartford last Saturday.  He left us a note saying you were the best boss he ever had, and that you were not to blame in any way."

Another time, as I opened a letter which lay on my desk, a newspaper clipping fell out.  It was the obituary of one of the best salesmen I ever had.  After two weeks of drinking, he had placed his toe on the trigger of a loaded shotgun-the barrel was in his mouth.  I had discharged him for drinking six weeks before.

Still another experience:  A woman's voice came faintly over long distance from Virginia.  She wanted to know if her husband's company insurance was still in force.  Four days before he had hanged himself in his woodshed.  I had been obliged to discharge him for drinking, though he was brilliant, alert, and one of the best organizers I have ever known.

Here were three exceptional men lost to this world because I did not understand alcoholism as I do now.  What irony-I became an alcoholic myself!  And but for the intervention of an understanding person, I might have followed in their footsteps.  My downfall cost the business community unknown thousands of dollars, for it takes real money to train a man for an executive position.  This kind of waste goes on unabated.  We think the business fabric is shot through with a situation which might be helped by better understanding all around. 

Nearly every modern employer feels a moral responsibility for the well-being of his help, and he tries to meet these responsibilities.  That he has not always done so for the alcoholic is easily understood.  To him the alcoholic has often seemed a fool of the first magnitude.  Because of the employee's special ability, or of his own strong personal attachment to him, the employer has sometimes kept such a man at work long beyond a reasonable period.  Some employers have tried every known remedy.  In only a few instances has there been a lack of patience and tolerance.  And we, who have imposed on the best of employers, can scarcely blame them if they have been short with us.

Here, for instance, is a typical example:  An officer of one of the largest banking institutions in America knows I no longer drink.  One day he told me about an executive of the same bank who, from his description, was undoubtedly alcoholic.  This seemed to me like an opportunity to be helpful, so I spent two hours talking about alcoholism, the malady, and described the symptoms and results as well as I could.  His comment was, "Very interesting.  But I'm sure this man is done drinking.  He has just returned from a three-months leave of absence, has taken a cure, looks fine, and to clinch the matter, the board of directors told him this was his last chance."

The only answer I could make was that if the man followed the usual pattern, he would go on a bigger bust than ever.  I felt this was inevitable and wondered if the bank was doing the man an injustice.  Why not bring him into contact with some of our alcoholic crowd?  He might have a chance.  I pointed out that I had had nothing to drink whatever for three years, and this in the face of difficulties that would have made nine out of ten men drink their heads off.  Why not at least afford him an opportunity to hear my story?  "Oh no," said my friend, "this chap is either through with liquor, or he is minus a job.  If he has your will power and guts, he will make the grade."

I wanted to throw up my hands in discouragement, for I saw that I had failed to help my banker friend understand.  He simply could not believe that his brother-executive suffered from a serious illness.  There was nothing to do but wait.

Presently the man did slip and was fired.  Following his discharge, we contacted him.  Without much ado, he accepted the principles and procedure that had helped us.  He is undoubtedly on the road to recovery.  To me, this incident illustrates lack of understanding as to what really ails the alcoholic, and lack of knowledge as to what part employers might profitably take in salvaging their sick employees.

If you desire to help it might be well to disregard your own drinking, or lack of it.  Whether you are a hard drinker, a moderate drinker or a teetotaler, you may have some pretty strong opinions, perhaps prejudices.  Those who drink moderately may be more annoyed with an alcoholic than a total abstainer would be.  Drinking occasionally, and understanding your own reactions, it is possible for you to become quite sure of many things which, so far as the alcoholic is concerned, are not always so.  As a moderate drinker, you can take your liquor or leave it alone.  Whenever you want to, you control your drinking.  Of an evening, you can go on a mild bender, get up in the morning, shake your head and go to business.  To you, liquor is no real problem.  You cannot see why it should be to anyone else, save the spineless and stupid.

When dealing with an alcoholic, there may be a natural annoyance that a man could be so weak, stupid and irresponsible.  Even when you understand the malady better, you may feel this feeling rising.

A look at the alcoholic in your organization is many times illuminating.  Is he not usually brilliant, fast-thinking, imaginative and likeable?  When sober, does he not work hard and have a knack of getting things done?  If he had these qualities and did not drink would he be worth retaining?  Should he have the same consideration as other ailing employees?  Is he worth salvaging?  If your decision is yes, whether the reason be humanitarian or business or both, then the following suggestions may be helpful.

Can you discard the feeling that you are dealing only with habit, with stubbornness, or a weak will?  If this presents difficulty, re-reading chapters two and three, where the alcoholic sickness is discussed at length might be worth while. You, as a business man, want to know the necessities before considering the result.  If you concede that your employee is ill, can he be forgiven for what he has done in the past?  Can his past absurdities be forgotten?  Can it be appreciated that he has been a victim of crooked thinking, directly caused by the action of alcohol on his brain?

I well remember the shock I received when a prominent doctor in Chicago told me of cases where pressure of the spinal fluid actually ruptured the brain.  No wonder an alcoholic is strangely irrational.  Who wouldn't be, with such a fevered brain?  Normal drinkers are not so affected, nor can they understand the aberrations of the alcoholic.

Your man has probably been trying to conceal a number of scrapes, perhaps pretty messy ones.  They may be disgusting.  You may be at a loss to understand how such a seemingly above-board chap could be so involved.  But these scrapes can generally be charged, no matter how bad, to the abnormal action of alcohol on his mind.  When drinking, or getting over a bout, an alcoholic, sometimes the model of honesty when normal, will do incredible things.  Afterward, his revulsion will be terrible.  Nearly always, these antics indicate nothing more than temporary conditions.

This is not to say that all alcoholics are honest and upright when not drinking.  Of course that isn't so, and such people often may impose on you.  Seeing your attempt to understand and help, some men will try to take advantage of your kindness.  If you are sure your man does not want to stop, he may as well be discharged, the sooner the better.  You are not doing him a favor by keeping him on.  Firing such an individual may prove a blessing to him.  It may be just the jolt he needs.  I know, in my own particular case, that nothing my company could have done would have stopped me for, so long as I was able to hold my position, I could not possibly realize how serious my situation was.  Had they fired me first, and had they then taken steps to see that I was presented with the solution contained in this book, I might have returned to them six months later, a well man.

But there are many men who want to stop, and with them you can go far.  Your understanding treatment of their cases will pay dividends.

Perhaps you have such a man in mind.  He wants to quit drinking and you want to help him, even if it be only a matter of good business.  You now know more about alcoholism.  You can see that he is mentally and physically sick.  You are willing to overlook his past performances.  Suppose an approach is made something like this:

State that you know about his drinking, and that it must stop.  You might say you appreciate his abilities, would like to keep him, but cannot if he continues to drink.  A firm attitude at this point has helped many of us.

Next he can be assured that you do not intend to lecture, moralize, or condemn; that if this was done formerly, it was because of misunderstanding.  If possible express a lack of hard feeling toward him.  At this point, it might be well to explain alcoholism, the illness.  Say that you believe he is a gravely ill person, with this qualification-being perhaps fatally ill, does he want to get well?  You ask, because many alcoholics, being warped and drugged, do not want to quit.  But does he?  Will he take every necessary step, submit to anything to get well, to stop drinking forever?

If he says yes, does he really mean it, or down inside does he think he is fooling you, and that after rest and treatment he will be able to get away with a few drinks now and then?  We believe a man should be thoroughly probed on these points.  Be satisfied he is not deceiving himself or you.

Whether you mention this book is a matter for your discretion.  If he temporizes and still thinks he can ever drink again, even beer, he might as well be discharged after the next bender which, if an alcoholic, he is almost certain to have.  He should understand that emphatically.  Either you are dealing with a man who can and will get well or you are not.  If not, why waste time with him?  This may seem severe, but it is usually the best course.

After satisfying yourself that your man wants to recover and that he will go to any extreme to do so, you may suggest a definite course of action.  For most alcoholics who are drinking, or who are just getting over a spree, a certain amount of physical treatment is desirable, even imperative.  The matter of physical treatment should, of course, be referred to your own doctor.  Whatever the method, its object is to thoroughly clear mind and body of the effects of alcohol.  In competent hands, this seldom takes long nor is it very expensive.  Your man will fare better if placed in such physical condition that he can think straight and no longer craves liquor.  If you propose such a procedure to him, it may be necessary to advance the cost of treatment, but we believe it should be made plain that any expense will later be deducted from his pay.  It is better for him to feel fully responsible.

If your man accepts your offer, it should be pointed out that physical treatment is but a small part of the picture.  Though you are providing him with the best possible medical attention, he should understand that he must undergo a change of heart.  To get over drinking will require a transformation of thought and attitude.  We all had to place recovery above everything, for without recovery we would have lost both home and business.

Can you have every confidence in his ability to recover?  While on the subject of confidence, can you adopt the attitude that so far as you are concerned this will be a strictly personal matter, that his alcoholic derelictions, the treatment about to be undertaken, will never be discussed without his consent?  It might be well to have a long chat with him on his return.

To return to the subject matter of this book:  It contains full suggestions by which the employee may solve his problem.  To you, some of the ideas which it contains are novel.  Perhaps you are not quite in sympathy with the approach we suggest.  By no means do we offer it as the last word on this subject, but so far as we are concerned, it has worked with us.  After all, are you not looking for results rather than methods?  Whether your employee likes it or not, he will learn the grim truth about alcoholism.  That won't hurt him a bit, even though he does not go for this remedy.

We suggest you draw the book to the attention of the doctor who is to attend your patient during treatment.  If the book is read the moment the patient is able, while acutely depressed, realization of his condition may come to him.

We hope the doctor will tell the patient the truth about his condition, whatever that happens to be.  When the man is presented with this volume it is best that no one tell him he must abide by its suggestions.  The man must decide for himself.

You are betting, of course, that your changed attitude plus the contents of this book will turn the trick.  In some cases it will, and in others it may not.  But we think that if you persevere, the percentage of successes will gratify you.  As our work spreads and our numbers increase, we hope your employees may be put in personal contact with some of us.  Meanwhile, we are sure a great deal can be accomplished by the use of the book alone.

On your employee's return, talk with him.  Ask him if he thinks he has the answer.  If he feels free to discuss his problems with you, if he knows you under-stand and will not be upset by anything he wishes to say, he will probably be off to a fast start.

In this connection, can you remain undisturbed if the man proceeds to tell you shocking things?  He may, for example, reveal that he has padded his expense account or that he has planned to take your best customers away from you.  In fact, he may say almost anything if he has accepted our solution which, as you know, demands rigorous honesty.  Can you charge this off as you would a bad account and start fresh with him?  If he owes you money you may wish to make terms.

If he speaks of his home situation, you can undoubtedly make helpful suggestions.  Can he talk frankly with you so long as he does not bear business tales or criticize his associates?  With this kind of employee such an attitude will command undying loyalty.

The greatest enemies of us alcoholics are resentment, jealousy, envy, frustration, and fear.  Wherever men are gathered together in business there will be rivalries and, arising out of these, a certain amount of office politics.  Sometimes we alcoholics have an idea that people are trying to pull us down.  Often this is not so at all.  But sometimes our drinking will be used politically.

One instance comes to mind in which a malicious individual was always making friendly little jokes about an alcoholic's drinking exploits.  In this way he was slyly carrying tales.  In another case, an alcoholic was sent to a hospital for treatment.  Only a few knew of it at first but, within a short time, it was billboarded throughout the entire company.  Naturally this sort of thing decreased the man's chance of recovery.  The employer can many times protect the victim from this kind of talk.  The employer cannot play favorites, but he can always defend a man from needless provocation and unfair criticism.

As a class, alcoholics are energetic people.  They work hard and they play hard.  Your man should be on his mettle to make good.  Being somewhat weakened, and faced with physical and mental readjustment to a life which knows no alcohol, he may overdo.  You may have to curb his desire to work sixteen hours a day.  You may need to encourage him to play once in a while.  He may wish to do a lot for other alcoholics and something of the sort may come up during business hours.  A reasonable amount of latitude will be helpful.  This work is necessary to maintain his sobriety.

After your man has gone along without drinking for a few months, you may be able to make use of his services with other employees who are giving you the alcoholic run-around-provided, of course, they are willing to have a third party in the picture.  An alcoholic who has recovered, but holds a relatively unimportant job, can talk to a man with a better position.  Being on a radically different basis of life, he will never take advantage of the situation.

Your man may be trusted.  Long experience with alcoholic excuses naturally arouses suspicion.  When his wife next calls saying he is sick, you might jump to the conclusion he is drunk.  If he is, and is still trying to recover, he will tell you about it even if it means the loss of his job.  For he knows he must be honest if he would live at all.  He will appreciate knowing you are not bothering your head about him, that you are not suspicious nor are you trying to run his life so he will be shielded from temptation to drink.  If he is conscientiously following the program of recovery he can go anywhere your business may call him.

In case he does stumble, even once, you will have to decide whether to let him go.  If you are sure he doesn't mean business, there is no doubt you should discharge him.  If, on the contrary, you are sure he is doing his utmost, you may wish to give him another chance.  But you should feel under no obligation to keep him on, for your obligation has been well discharged already.

There is another thing you might wish to do.  If your organization is a large one, your junior executives might be provided with this book.  You might let them know you have no quarrel with the alcoholics of your organization.  These juniors are often in a difficult position.  Men under them are frequently their friends.  So, for one reason or another, they cover these men, hoping matters will take a turn for the better.  They often jeopardize their own positions by trying to help serious drinkers who should have been fired long ago, or else given an opportunity to get well.

After reading this book, a junior executive can go to such a man and say approximately this, "Look here, Ed.  Do you want to stop drinking or not?  You put me on the spot every time you get drunk.  It isn't fair to me or the firm.  I have been learning something about alcoholism.  If you are an alcoholic, you are a mighty sick man.  You act like one.  The firm wants to help you get over it, and if you are interested, there is a way out.  If you take it, your past will be forgotten and the fact that you went away for treatment will not be mentioned.  But if you cannot or will not stop drinking, I think you ought to resign."

Your junior executive may not agree with the contents of our book.  He need not, and often should not show it to his alcoholic prospect.  But at least he will understand the problem and will no longer be misled by ordinary promises.  He will be able to take a position with such a man which is eminently fair and square.  He will have no further reason for covering up an alcoholic employee.

It boils right down to this:  No man should be fired just because he is alcoholic.  If he wants to stop, he should be afforded a real chance.  If he cannot or does not want to stop, he should be discharged.  The exceptions are few.

We think this method of approach will accomplish several things.  It will permit the rehabilitation of good men.  At the same time you will feel no reluctance to rid yourself of those who cannot or will not stop.  Alcoholism may be causing your organization considerable damage in its waste of time, men and reputation.  We hope our suggestions will help you plug up this sometimes serious leak.  We think we are sensible when we urge that you stop this waste and give your worthwhile man a chance.

The other day an approach was made to the vice president of a large industrial concern.  He remarked:  "I'm mighty glad you fellows got over your drinking.  But the policy of this company is not to interfere with the habits of our employees.  If a man drinks so much that his job suffers, we fire him.  I don't see how you can be of any help to us for, as you see, we don't have any alcoholic problem."  This same company spends millions for research every year.  Their cost of production is figured to a fine decimal point.  They have recreational facilities.  There is company insurance.  There is a real interest, both humanitarian and business, in the well-being of employees.  But alcoholism-well, they just don't believe they have it.

Perhaps this is a typical attitude.  We, who have collectively seen a great deal of business life, at least from the alcoholic angle, had to smile at this gentleman's sincere opinion.  He might be shocked if he knew how much alcoholism is costing his organization a year.  That company may harbor many actual or potential alcoholics.  We believe that managers of large enterprises often have little idea how prevalent this problem is.  Even if you feel your organization has no alcoholic problem, it might pay to take another look down the line.  You may make some interesting discoveries.

Of course, this chapter refers to alcoholics, sick people, deranged men.  What our friend, the vice president, had in mind was the habitual or whoopee drinker.  As to them, his policy is undoubtedly sound, but he did not distinguish between such people and the alcoholic.

It is not to be expected that an alcoholic employee will receive a disproportionate amount of time and attention.  He should not be made a favorite.  The right kind of man, the kind who recovers, will not want this sort of thing.  He will not impose.  Far from it.  He will work like the devil and thank you to his dying day.

Today I own a little company.  There are two alcoholic employees, who produce as much as five normal salesmen.  But why not?  They have a new attitude, and they have been saved from a living death.  I have enjoyed every moment spent in getting them straightened out.

\end{biblechapter}

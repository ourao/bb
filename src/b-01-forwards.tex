
\section{The Word}


ALCOHOLICS 


ANONYMOUS


The Story of


How Many Thousands of Men and Women


Have Recovered from Alcoholism


\subsection{PREFACE}

This is the fourth edition of the book “Alcoholics Anonymous.”
The First edition appeared in April 1939, and in the following sixteen years, more than 300,000 copies went into circulation.
The second edition, published in 1955, reached a total of more than 1,150,500 copies.
The third edition, which came off press in 1976, achieved a circulation of approximately 19,550,000 in all formats.

Because this book has become the basic text for our Society and has helped such large numbers of alcoholic men and women to recovery, there exists strong sentiment against any radical changes being made in it.
Therefore, the first portion of this volume, describing the A.A. recovery program, has been left largely untouched in the course of revisions made for the second, third, and fourth editions.
The section called “The Doctor’s Opinion” has been kept intact, just as it was originally written in 1939 by the late Dr. William D. Silkworth, our Society’s great medical benefactor.
The second edition added the appendices, the Twelve Traditions, and the directions for getting in touch with A.A.
But the chief change was in the section of personal stories, which was expanded to reflect the Fellowship’s growth.
“Bill’s Story,” “Doctor Bob’s Nightmare,” and one other personal history from the first edition were retained intact; 
three were edited and one of these was retitled; 
new versions of two stories were written, with new titles; 
thirty completely new stories were added; 
and the story section was divided into three parts, under the same headings that are used now.

In the third edition, Part I (“Pioneers of A.A.”) was left unchanged.
Nine of the stories in Part II (“They Stopped in Time”) were carried over from the second edition; 
eight new stories were added.
In Part III (“They Lost Nearly All”), eight stories were retained; five new ones were added. 

This fourth edition includes the Twelve Concepts for World Service and revises the three sections of personal stories as follows.
One new story has been added to Part I, and two that originally appeared in Part III have been repositioned there; 
six stories have been deleted.
Six of the stories in Part II have been carried over, eleven new ones have been added, and eleven taken out.
Part III now includes twelve new stories; 
eight were removed (in addition to the two that were transferred to Part I).

All changes made over the years in the Big Book (A.A. members’ fond nickname for this volume) have had the same purpose: 
to represent the current membership of Alcoholics Anonymous more accurately, and thereby to reach more alcoholics.
If you have a drinking problem, we hope that you may pause in reading one of the forty-two personal stories and think: 
“Yes, that happened to me”; 
or, more important, “Yes, I’ve felt like that”; 
or, most important, “Yes, I believe this program can work for me too.”

%% xii


\subsection{FOREWORD TO SECOND EDITION}

Figures given in this foreword describe the
Fellowship as it was in 1955.

Since the original Foreword to this book was written in 1939, a wholesale miracle has taken place. 
Our earliest printing voiced the hope “that every alcoholic who journeys will find the Fellowship of Alcoholics Anonymous at his destination. Already,”
continues the early text “twos and threes and fives of us have sprung up in other communities.”

Sixteen years have elapsed between our first printing of this book and the presentation in 1955 of our second edition.
In that brief space, Alcoholics Anonymous has mushroomed into nearly 6,000 groups whose membership is far above 150,000 recovered alcoholics.
Groups are to be found in each of the United States and all of the provinces of Canada.
A.A. has flourishing communities in the British Isles, the Scandinavian countries, South Africa, South America, Mexico, Alaska, Australia and Hawaii. 
All told, promising beginnings have been made in some 50 foreign countries and U. S. possessions. 
Some are just now taking shape in Asia. 
Many of our friends encourage us by saying that this is but a beginning, only the augury of a much larger future ahead.

The spark that was to flare into the first A.A. group was struck at Akron, Ohio, in June 1935, during a talk between a New York stockbroker and an Akron physician. 
Six months earlier, the broker had been relieved of his drink obsession by a sudden spiritual
%% xv
experience, following a meeting with an alcoholic friend who had been in contact with the Oxford Groups of that day. 
He had also been greatly helped by the late Dr. William D. Silkworth, a New York specialist in alcoholism who is now accounted no less than a medical saint by A.A. members, and whose story of the early days of our Society appears in the next pages. 
From this doctor, the broker had learned the grave nature of alcoholism. 
Though he could not accept all the tenets of the Oxford Groups, he was convinced of the need for moral inventory, confession of personality defects, restitution to those harmed, helpfulness to others, and the necessity of belief in and dependence upon God.

Prior to his journey to Akron, the broker had worked hard with many alcoholics on the theory that only an alcoholic could help an alcoholic, but he had succeeded only in keeping sober himself. 
The broker had gone to Akron on a business venture which had collapsed, leaving him greatly in fear that he might start drinking again. 
He suddenly realized that in order to save himself he must carry his message to another alcoholic. 
That alcoholic turned out to be the Akron physician.

This physician had repeatedly tried spiritual means to resolve his alcoholic dilemma but had failed. 
But when the broker gave him Dr. Silkworth’s description of alcoholism and its hopelessness, the physician began to pursue the spiritual remedy for his malady with a willingness he had never before been able to muster.
He sobered, never to drink again up to the moment of his death in 1950. 
This seemed to prove that one alcoholic could affect another as no nonalcoholic
%% xvi
could.
It also indicated that strenuous work, one alcoholic with another, was vital to permanent recovery.

Hence the two men set to work almost frantically upon alcoholics arriving in the ward of the Akron City Hospital. 
Their very first case, a desperate one, recovered immediately and became A.A. number three. 
He never had another drink. 
This work at Akron continued through the summer of 1935. 
There were many failures, but there was an occasional heartening success. 
When the broker returned to New York in the fall of 1935, the first A.A. group had actually been formed, though no one realized it at the time.

A second small group promptly took shape at New York, to be followed in 1937 with the start of a third at Cleveland. 
Besides these, there were scattered alcoholics who had picked up the basic ideas in Akron or New York who were trying to form groups in other cities. 
By late 1937, the number of members having substantial sobriety time behind them was sufficient to convince the membership that a new light had entered the dark world of the alcoholic.

It was now time, the struggling groups thought, to place their message and unique experience before the
world. 
This determination bore fruit in the spring of 1939 by the publication of this volume. 
The membership had then reached about 100 men and women. 
The fledgling society, which had been nameless, now began to be called Alcoholics Anonymous, from the title of its own book. 
The flying-blind period ended and A.A. entered a new phase of its pioneering time.
With the appearance of the new book a great deal began to happen. Dr. Harry Emerson Fosdick, the
%% xvii
noted clergyman, reviewed it with approval. 
In the fall of 1939 Fulton Oursler, then editor of Liberty, printed a piece in his magazine, called “Alcoholics and God.” 
This brought a rush of 800 frantic inquiries into the little New York office which meanwhile had
been established. 
Each inquiry was painstakingly answered; 
pamphlets and books were sent out. 
Businessmen, traveling out of existing groups, were referred to these prospective newcomers. 
New groups started up and it was found, to the astonishment of everyone, that A.A.’s message could be transmitted in the mail as well as by word of mouth. 
By the end of 1939 it was estimated that 800 alcoholics were on their way to recovery.

In the spring of 1940, John D. Rockefeller, Jr. gave a dinner for many of his friends to which he invited A.A. members to tell their stories. 
News of this got on the world wires; 
inquiries poured in again and many people went to the bookstores to get the book “Alcoholics Anonymous.’’ 
By March 1941 the membership had shot up to 2,000. 
Then Jack Alexander wrote a feature article in the Saturday Evening Post and placed such a compelling picture of A.A. before the general public that alcoholics in need of help really deluged us. 
By the close of 1941, A.A. numbered 8,000 members. 
The mushrooming process was in full swing.
A.A. had become a national institution.

Our Society then entered a fearsome and exciting adolescent period. 
The test that it faced was this:
Could these large numbers of erstwhile erratic alcoholics successfully meet and work together? 
Would there be quarrels over membership, leadership, and money? 
Would there be strivings for power and
%% xviii
prestige? 
Would there be schisms which would split A.A. apart? 
Soon A.A. was beset by these very problems on every side and in every group. 
But out of this frightening and at first disrupting experience the conviction grew that A.A.’s had to hang together or die separately. 
We had to unify our Fellowship or pass off the scene.

As we discovered the principles by which the individual alcoholic could live, so we had to evolve principles by which the A.A. groups and A.A. as a whole could survive and function effectively. 
It was thought that no alcoholic man or woman could be excluded from our Society; 
that our leaders might serve but never govern; 
that each group was to be autonomous and there was to be no professional class of therapy.
There were to be no fees or dues; 
our expenses were to be met by our own voluntary contributions. 
There was to be the least possible organization, even in our service centers. 
Our public relations were to be based upon attraction rather than promotion. 
It was decided that all members ought to be anonymous at the level of press, radio, TV and films. 
And in no circumstances should we give endorsements, make alliances, or enter public controversies.

This was the substance of A.A.’s Twelve Traditions, which are stated in full on page 561 of this book.
Though none of these principles had the force of rules or laws, they had become so widely accepted by 1950 that they were confirmed by our first International Conference held at Cleveland. 
Today the remarkable unity of A.A. is one of the greatest assets that our Society has.
While the internal difficulties of our adolescent
%% xix
period were being ironed out, public acceptance of A.A. grew by leaps and bounds.
For this there were two principal reasons:
the large numbers of recoveries, and reunited homes.
These made their impressions everywhere.
Of alcoholics who came to A.A. and really tried, 50\% got sober at once and remained that way;
25\% sobered up after some relapses, and among the remainder, those who stayed on with A.A. showed improvement.
Other thousands came to a few A.A. meet ings and at first decided they didn’t want the program. 
But great numbers of these—about two out of three—began to return as time passed.

Another reason for the wide acceptance of A.A. was the ministration of friends - friends in medicine, religion, and the press, together with innumerable others who became our able and persistent advocates.
Without such support, A.A. could have made only the slowest progress.
Some of the recommendations of A.A.’s early medical and religious friends will be found further on in this book.

Alcoholics Anonymous is not a religious organization.
Neither does A.A. take any particular medical point of view, though we cooperate widely with the men of medicine as well as with the men of religion.
Alcohol being no respecter of persons, we are an accurate cross section of America, and in distant lands, the same democratic evening-up process is now going on. 
By personal religious affiliation, we include Catholics, Protestants, Jews, Hindus, and a sprinkling of Moslems and Buddhists. 
More than 15\% of us are women.

At present, our membership is pyramiding at the rate of about twenty per cent a year.
So far, upon the
%% xx
total problem of several million actual and potential alcoholics in the world, we have made only a scratch.
In all probability, we shall never be able to touch more than a fair fraction of the alcohol problem in all its ramifications.
Upon therapy for the alcoholic himself, we surely have no monopoly. 
Yet it is our great hope that all those who have as yet found no answer may begin to find one in the pages of this book and will presently join us on the high road to a new freedom.

%% xxi


\subsection{FOREWORD TO THIRD EDITION}

By March 1976, when this edition went to the printer, the total worldwide membership of Alcoholics Anonymous was conservatively estimated at more than 1,000,000, with almost 28,000 groups meeting in over 90 countries.

Surveys of groups in the United States and Canada indicate that A.A. is reaching out, not only to more and more people, but to a wider and wider range.
Women now make up more than one-fourth of the membership; 
among newer members, the proportion is nearly one-third. 
Seven percent of the A.A.’s surveyed are less than 30 years of age—among them, many in their teens.

The basic principles of the A.A. program, it appears, hold  good  for  individuals  with  many  different lifestyles, just as the program has brought recovery to those of many different nationalities. 
The Twelve Steps that summarize the program may be called los Doce Pasos in one country, les Douze Etapes in another, but they trace exactly the same path to recovery that was blazed by the earliest members of Alcoholics Anonymous.

In spite of the great increase in the size and the span of this Fellowship, at its core it remains simple and personal. Each day, somewhere in the world, recovery begins when one alcoholic talks with another alcoholic, sharing experience, strength, and hope.

xxii


\subsection{FOREWORD TO FOURTH EDITION}

This fourth edition of “Alcoholics Anonymous” came off press in November 2001, at the start of a new millennium. Since the third edition was published in 1976, worldwide membership of A.A. has just about doubled, to an estimated two million or more, with nearly 100,800 groups meeting in approximately 150 countries around the world. 
Literature has played a major role in A.A.’s growth, and a striking phenomenon of the past quarter-century has been the explosion of translations of our basic literature into many languages and dialects. 
In country after country where the A.A. seed was planted, it has taken root, slowly at first, then growing by leaps and bounds when literature has become available. 
Currently, “Alcoholics Anonymous” has been translated into forty-three* languages.

As the message of recovery has reached larger numbers of people, it has also touched the lives of a vastly greater variety of suffering alcoholics. 
When the phrase “We are people who normally would not mix” (page 17 of this book) was written in 1939, it referred to a Fellowship composed largely of men (and a fewwomen) with quite similar social, ethnic, and economic backgrounds. 
Like so much of A.A.’s basic text, those words have proved to be far more visionary than the founding members could ever have imagined. 
The stories added to this edition represent a membership whose characteristics—of age, gender, race, and culture—have widened and have deepened to encompass

%%xxiii
%(todo - make into footnote - *In 2013, Alcoholics Anonymous is in seventy languages.)
virtually everyone the first 100 members could have hoped to reach.

While our literature has preserved the integrity of the A.A. message, sweeping changes in society as a whole are reflected in new customs and practices within the Fellowship. 
Taking advantage of technological advances, for example, A.A. members with computers can participate in meetings online, sharing with fellow alcoholics across the country or around the world. 
In any meeting, anywhere, A.A.’s share experience, strength, and hope with each other, in order to stay sober and help other alcoholics. 
Modem-to-modem or face-to-face, A.A.’s speak the language of the heart in all its power and simplicity.

%% xxiv


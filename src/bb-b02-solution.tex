
\biblebook{Solution}

Bigbook "There Is A Solution"; Pages 17-29


\begin{biblechatper}
\verseWithHeading{Who Are We}

We, of Alcoholics Anonymous, know thousands of men and women who were once just as hopeless as Bill. 
Nearly all have recovered. 
They have solved the drink problem.

\textbf{Who we are}

We are average Americans. 
All sections of this country and many of its occupations are represented, as well as many political, economic, social, and religious backgrounds. 
We are people who normally would not mix. 
But there exists among us a fellowship, a friendliness, and an understanding which is indescribably wonderful. 
We are like the passengers of a great liner the moment after rescue from shipwreck when camaraderie, joyousness and democracy pervade the vessel from steerage to Captain's table. 
Unlike the feelings of the ship's passengers, however, our joy in escape from disaster does not subside as we go our individual ways. 
The feeling of having shared in a common peril is one element in the powerful cement which binds us. 
But that in itself would never have held us together as we are now joined.

\textbf{Solution}

The tremendous fact for every one of us is that we have discovered a common solution. 
We have a way out on which we can absolutely agree, 
and upon which we can join in brotherly and harmonious action. 
This is the great news this book carries to those who suffer from alcoholism. 
An illness of this sort and we have come to believe it an illness involves those about us in a way no other human sickness can. 
If a person has cancer all are sorry for him and no one is angry or hurt. 
But not so with the alcoholic illness, for with it there goes annihilation of all the things worth while in life. 
It engulfs all whose lives touch the sufferer's. 
It brings misunderstanding, fierce resentment, financial insecurity, disgusted friends and employers, 
warped lives of blameless children, sad wives and parents anyone can increase the list.

We hope this volume will inform and comfort those who are, or who may be affected. 
There are many.
\end{biblechapter}


\begin{biblechatper}
\verseWithHeading{Shared Experience}

Highly competent psychiatrists who have dealt with us have found it sometimes impossible to persuade an alcoholic to discuss his situation without reserve. 
Strangely enough, wives, parents and intimate friends usually find us even more unapproachable than do the psychiatrist and the doctor.

But the ex-problem drinker who has found this solution, who is properly armed with facts about himself, 
can generally win the entire confidence of another alcoholic in a few hours. 
Until such an understanding is reached, little or nothing can be accomplished.

That the man who is making the approach has had the same difficulty, that he obviously knows what he is talking about, 
that his whole deportment shouts at the new prospect that he is a man with a real answer, 
that he has no attitude of Holier Than Thou, nothing whatever except the sincere desire to be helpful; 
that there are no fees to pay, no axes to grind, no people to please, no lectures to be endured 
these are the conditions we have found most effective. 
After such an approach many take up their beds and walk again.
\end{biblechapter}


\begin{biblechatper}
\verseWithHeading{Carrying The Message}

None of us makes a sole vocation of this work, 
nor do we think its effectiveness would be increased if we did. 
We feel that elimination of our drinking is but a beginning. 
A much more important demonstration of our principles lies before us in our respective homes, occupations and affairs. 
All of us spend much of our spare time in the sort of effort which we are going to describe. 
A few are fortunate enough to be so situated that they can give nearly all their time to the work.

If we keep on the way we are going there is little doubt that much good will result, 
but the surface of the problem would hardly be scratched. 
Those of us who live in large cities are overcome by the reflection that close by hundreds are dropping into oblivion every day. 
Many could recover if they had the opportunity we have enjoyed. 
How then shall we present that which has been so freely given us?

We have concluded to publish an anonymous volume setting forth the problem as we see it. 
We shall bring to the task our combined experience and knowledge. 
This should suggest a useful program for anyone concerned with a drinking problem.

Of necessity there will have to be discussion of matters medical, psychiatric, social, and religious. 
We are aware that these matters are from their very nature, controversial. 
Nothing would please us so much as to write a book which would contain no basis for contention or argument. 
We shall do our utmost to achieve that ideal. 
Most of us sense that real tolerance of other people's shortcomings and viewpoints and a respect for their opinions are attitudes which make us more useful to others. 
Our very lives, as ex-problem drinkers, depend upon our constant thought of others and how we may help meet their needs.
\end{biblechapter}


\begin{biblechatper}
\verseWithHeading{Who Me?}

You may already have asked yourself why it is that all of us became so very ill from drinking. 
Doubtless you are curious to discover how and why, in the face of expert opinion to the contrary, we have recovered from a hopeless condition of mind and body. 
If you are an alcoholic who wants to get over it, you may already be asking 
"What do I have to do?"
\end{biblechapter}


\begin{biblechatper}
\verseWithHeading{Non-Alcoholics}

It is the purpose of this book to answer such questions specifically. 
We shall tell you what we have done. 
Before going into a detailed discussion, it may be well to summarize some points as we see them.

How many time people have said to us: 
"I can take it or leave it alone. Why can't he?" 
"Why don't you drink like a gentleman or quit?" 
"That fellow can't handle his liquor." 
"Why don't you try beer and wine?" 
"Lay off the hard stuff." 
"His will power must be weak." 
"He could stop if he wanted to." 
"She's such a sweet girl, I should think he'd stop for her sake." 
"The doctor told him that if he ever drank again it would kill him, but there he is all lit up again."

Now these are commonplace observations on drinkers which we hear all the time. 
Back of them is a world of ignorance and misunderstanding. 
We see that these expressions refer to people whose reactions are very different from ours.

Moderate drinkers have little trouble in giving up liquor entirely if they have good reason for it. 
They can take it or leave it alone.

Then we have a certain type of hard drinker. 
He may have the habit badly enough to gradually impair him physically and mentally. 
It may cause him to die a few years before his time. 
If a sufficiently strong reason ill health, falling in love, change of environment, or the warning of a doctor becomes operative, 
this man can also stop or moderate, 
although he may find it difficult and troublesome and may even need medical attention.
\end{biblechapter}


\begin{biblechatper}
\verseWithHeading{The Alcoholic}

But what about the real alcoholic? 
He may start off as a moderate drinker; 
he may or may not become a continuous hard drinker; 
but at some stage of his drinking career he begins to lose all control of his liquor consumption, 
once he starts to drink.

Here is a fellow who has been puzzling you, especially in his lack of control. 
He does absurd, incredible, tragic things while drinking. 
He is a real Dr. Jekyll and Mr. Hyde. 
He is seldom mildly intoxicated. 
He is always more or less insanely drunk. 
His disposition while drinking resembles his normal nature but little. 
He may be one of the finest fellows in the world. 
Yet let him drink for a day, and he frequently becomes disgustingly, and even dangerously anti-social. 
He has a positive genius for getting tight at exactly the wrong moment, 
particularly when some important decision must be made or engagement kept. 
He is often perfectly sensible and well balanced concerning everything except liquor, 
but in that respect he is incredibly dishonest and selfish. 
He often possesses special abilities, skills, and aptitudes, 
and has a promising career ahead of him. 
He uses his gifts to build up a bright outlook for his family and himself, 
and then pulls the structure down on his head by a senseless series of sprees. 
He is the fellow who goes to bed so intoxicated he ought to sleep the clock around. 
Yet early next morning he searches madly for the bottle he misplace the night before. 
If he can afford it, he may have liquor concealed all over his house to be certain no one gets his entire supply away from him to throw down the wastepipe. 
As matters grow worse, he begins to use a combination of high-powered sedative and liquor to quiet his nerves so he can go to work. 
Then comes the day when he simply cannot make it and gets drunk all over again. 
Perhaps he goes to a doctor who gives him morphine or some sedative with which to taper off. 
Then he begins to appear at hospitals and sanitariums.

This is by no means a comprehensive picture of the true alcoholic, as our behavior patterns vary. 
But this description should identify him roughly.
\end{biblechapter}


\begin{biblechatper}
\verseWithHeading{Powerlessness}

Why does he behave like this? 
If hundreds of experiences have shown him that one drink means another debacle with all its attendant suffering and humiliation, 
why is it he takes that one drink? 
Why can't he stay on the water wagon? 
What has become of the common sense and will power that he still sometimes displays with respect to other matters?

Perhaps there never will be a full answer to these questions. 
Opinions vary considerably as to why the alcoholic reacts differently from normal people. 
We are not sure why, once a certain point is reached, little can be done for him. 
We cannot answer the riddle.

We know that while the alcoholic keeps away from drink, as he may do for months or years, he reacts much like other men. 
We are equally positive that once he takes any alcohol whatever into his system, something happens, 
both in the bodily and mental sense, 
which makes it virtually impossible for him to stop. 
The experience of any alcoholic will abundantly confirm this.

These observations would be academic and pointless if our friend never took the first drink, thereby setting the terrible cycle in motion. 
Therefore, the main problem of the alcoholic centers in his mind, rather than in his body. 
If you ask him why he started on that last bender, the chances are he will offer you any one of a hundred alibis. 
Sometimes these excuses have a certain plausibility, 
but none of them really makes sense in the light of the havoc an alcoholic's drinking bout creates. 
They sound like the philosophy of the man who, having a headache, beats himself on the head with a hammer so that he can't feel the ache. 
If you draw this fallacious reasoning to the attention of an alcoholic, he will laugh it off, 
or become irritated and refuse to talk.

Once in a while he may tell the truth. 
And the truth, strange to say, is usually that he has no more idea why he took that first drink than you have. 
Some drinkers have excuses with which they are satisfied part of the time. 
But in their hearts they really do not know why they do it. 
Once this malady has a real hold, they are a baffled lot. 
There is the obsession that somehow, someday, they will beat the game. 
But they often suspect they are down for the count.

How true this is, few realize. 
In a vague way their families and friends sense that these drinkers are abnormal, 
but everybody hopefully awaits the day when the sufferer will rouse himself from his lethargy and assert his power of will.
\end{biblechapter}


\begin{biblechatper}
\verseWithHeading{Unmanageability}

The tragic truth is that if the man be a real alcoholic, the happy day may not arrive. 
He has lost control. 
At a certain point in the drinking of every alcoholic, he passes into a state where the most powerful desire to stop drinking is of absolutely no avail. 
This tragic situation has already arrived in practically every case long before it is suspected.

The fact is that most alcoholics, for reasons yet obscure, have lost the power of choice in drink. 
Our so called will power becomes practically nonexistent. 
We are unable, at certain times, to bring into our consciousness with sufficient force the memory of the suffering and humiliation of even a week or a month ago. 
We are without defense against the first drink.

The almost certain consequences that follow taking even a glass of beer do not crowd into the mind to deter us. 
If these thoughts occur, they are hazy and readily supplanted with the old threadbare idea that this time we shall handle ourselves like other people. 
There is a complete failure of the kind of defense that keeps one from putting his hand on a hot stove.

The alcoholic may say to himself in the most casual way, 
"It won't burn me this time, so here's how!". 
Or perhaps he doesn't think at all. 
How often have some of us begun to drink in this nonchalant way, 
and after the third or fourth, pounded on the bar and said to ourselves, 
"For God's sake, how did I ever get started again?". 
Only to have that thought supplanted by 
"Well, I'll stop with the sixth drink." 
Or "What's the use anyhow?"

When this sort of thinking is fully established in an individual with alcoholic tendencies, he has probably placed himself beyond human aid, 
and unless locked up, may die or go permanently insane. 
These stark and ugly facts have been confirmed by legions of alcohoholics throughout history. 
But for the grace of God, there would have been thousands more convincing demonstrations. 
So many want to stop but cannot.
\end{biblechapter}


\begin{biblechatper}
\verseWithHeading{There is a Solution}

There is a solution. 
Almost none of us liked the self-searching, 
the leveling of our pride, the confession of shortcomings 
which the process requires for its successful consummation. 
But we saw that it really worked in others, 
and we had come to believe in the hopelessness and futility of life as we had been living it. 
When, therefore, we were approached by those in whom the problem had been solved, 
there was nothing left for us but to pick up the simple kit of spiritual tools laid at out feet. 
We have found much of heaven 
and we have been rocketed into a fourth dimension of existence of which we had not even dreamed.

The great fact is just this, and nothing less: 
That we have had deep and effective spiritual experiences* 
which have revolutionized our whole attitude toward life, 
toward our fellows 
and toward God's universe. 
The central fact of our lives today is the absolute certainty that our Creator has entered into our hearts and lives 
in a way which is indeed miraculous. 
He has commenced to accomplish those things for us which we could never do by ourselves.

If you are as seriously alcoholic as we were, we believe there is no middle-of-the-road solution. 
We were in a position where life was becoming impossible, 
and if we had passed into the region from which there is no return through human aid, we had but two alternatives: 
One was to go on to the bitter end, blotting out the consciousness of our intolerable situation as best we could; 
and the other, to accept spiritual help. 
This we did because we honestly wanted to, and were willing to make the effort.
\end{biblechapter}


\begin{biblechatper}
\verseWithHeading{Hazard}

A certain American business man had ability, good sense, and high character. 
For years he had floundered from one sanitarium to another. 
He had consulted the best known American psychiatrists. 
Then he had gone to Europe, placing himself in the care of a celebrated physician (the psychiatrist, Dr. Jung) who prescribed for him. 
Though experience had made him skeptical, he finished his treatment with unusual confidence. 
His physical and mental condition were unusually good. 
Above all, he believed he had acquired such a profound knowledge of the inner workings of his mind and its hidden springs that relapse was unthinkable. 
Nevertheless, he was drunk in a short time. 
More baffling still, he could give himself no satisfactory explanation for his fall.

So he returned to this doctor, whom he admired, 
and asked him point-blank why he could not recover. 
He wished above all things to regain self-control. 
He seemed quite rational and well-balanced with respect to other problems. 
Yet he had no control whatever over alcohol. 
Why was this?

He begged the doctor to tell him the whole truth, and he got it. 
In the doctor's judgment he was utterly hopeless; 
he could never regain his position in society 
and he would have to place himself under lock and key or hire a bodyguard if he expected to live long. 
That was a great physician's opinion.

But this man still lives, and is a free man. 
He does not need a bodyguard nor is he confined. 
He can go anywhere on this earth where other free men may go without disaster, 
provided he remains willing to maintain a certain simple attitude.
\end{biblechapter}


\begin{biblechatper}
\verseWithHeading{Jung}

Some of our alcoholic readers may think they can do without spiritual help. 
Let us tell you the rest of the conversation our friend had with his doctor.

The doctor said: 
\begin{quote}
    "You have the mind of a chronic alcoholic. 
    I have never seen one single case recover, 
    where that state of mind existed to the extent that it does in you." 
    Our friend felt as though the gates of hell had closed on him with a clang.
\end{quote}

He said to the doctor, "Is there no exception?"

"Yes," replied the doctor, 
\begin{quote}
    "there is. 
    Exceptions to cases such as yours have been occurring since early times. 
    Here and there, once in a while, alcoholics have had what are called vital spiritual experiences. 
    To me these occurrences are phenomena. 
    They appear to be in the nature of huge emotional displacements and rearrangements. 
    Ideas, emotions, and attitudes which were once the guiding forces of the lives of these men are suddenly cast to one side, 
    and a completely new set of conceptions and motives begin to dominate them. 
    In fact, I have been trying to produce some such emotional rearrangement within you. 
    With many individuals the methods which I employed are successful, 
    but I have never been successful with an alcoholic of your description."
\end{quote}

Upon hearing this, our friend was somewhat relieved, for he reflected that, after all, he was a good church member. 
This hope, however, was destroyed by the doctor's telling him that while his religious convictions were very good, 
in his case they did not spell the necessary vital spiritual experience.

Here was the terrible dilemma in which our friend found himself when he had the extraordinary experience, 
which as we have already told you, made him a free man.
\end{biblechapter}


\begin{biblechatper}
\verseWithHeading{James}

We, in our turn, sought the same escape with all the desperation of drowning men. 
What seemed at first a flimsy reed, has proved to be the loving and powerful hand of God. 
A new life has been given us or, if you prefer, 
"a design for living" that really works.

The distinguished American psychologist, William James, in his book "Varieties of Religious Experience," indicates a multitude of ways in which men have discovered God. 
We have no desire to convince anyone that there is only one way by which faith can be acquired. 
If what we have learned and felt and seen means anything at all, it means that all of us, whatever our race, creed, or color are the children of a living Creator 
with whom we may form a relationship upon simple and understandable terms 
as soon as we are willing and honest enough to try. 
Those having religious affiliations will find here nothing disturbing to their beliefs or ceremonies. 
There is no friction among us over such matters.

We think it no concern of ours what religious bodies our members identify themselves with as individuals. 
This should be an entirely personal affair which each one decides for himself 
in the light of past associations, or his present choice. 
Not all of us join religious bodies, but most of us favor such memberships.
\end{biblechapter}


\begin{biblechatper}
\verseWithHeading{Bigbook Arc}

In the following chapter, there appears an explanation of alcoholism, as we understand it, 
then a chapter addressed to the agnostic. 
Many who once were in this class are now among our members. 
Surprisingly enough, we find such convictions no great obstacle to a spiritual experience.

Further on, clear-cut directions are given showing how we recovered. 
These are followed by three dozen personal experiences.

Each individual, in the personal stories, describes in his own language and from his own point of view the way he established his relationship with God. 
These give a fair cross section of our membership 
and a clear-cut idea of what has actually happened in their lives.

We hope no one will consider these self-revealing accounts in bad taste. 
Our hope is that many alcoholic men and women, desperately in need, will see these pages, 
and we believe that it is only by fully disclosing ourselves and our problems that they will be persuaded to say, 
"Yes, I am one of them too; I must have this thing."
\end{biblechapter}


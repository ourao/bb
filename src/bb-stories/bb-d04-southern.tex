\biblebook{Southern}

OUR SOUTHERN FRIEND

Pioneer A.A., minister’s son, and southern farmer,

“Who am I,” said he, “to say there is no God?”

      FATHER IS AN Episcopal minister and his work takes him over long drives on bad roads. His parishioners are limited in number, but his friends are many, for to him race, creed, or social position make no difference. It is not long before he drives up in the buggy. Both he and old Maud are glad to get home. The drive was long and cold but he was thankful for the hot bricks which some thoughtful person had given him for his feet. Soon supper is on the table. Father says grace, which delays my attack on the buckwheat cakes ans sausage.

Bed-time comes. I climb to my room in the attic. It is cold so there is no delay. I craw under a pile of blankets and blow out the candle. The wind is rising and howls around the house. But I am safe and warm. I fall into a dreamless sleep.

I am in church. Father is delivering his sermon. A wasp is crawling up the back of the lady in front of me. I wonder if it will reach her neck. Shucks! It has flown away. At last! The message has been delivered.

“Let your light so shine before men that they may see your good works-.” I hunt for my nickel to drop in the plate so that mine will be seen.

I am in another fellow’s room at colledge. “Fresh-

man,” said he to me, “do you ever take a drink?” I hesitated. Father had never directly spoken to me about drinking but he never drank any, so far as I knew. Mother hated liquor and feared a drunken man. Her brother had been a drinker and had died in a state hospital for the insane. But his life was unmentioned, so far as I was concerned. I had never had a drink, but I had seen enough merriment in the boys who were drinking to be interested. I would never be like the village drunkard at home.

“Well,” said the older boy, “Do you?”

“Once in a while,” I lied. I could not let him think I was a sissy.

He poured out two drinks. “Here’s looking at you,” said he. I gulped it down and choked. I didn’t like it, but I would not say so. A mellow glow stole over me. This wasn’t so bad after all. Sure I’d have another. The glow increased. Other boys came in. My tongue loosened. Everyone laughed loudly. I was witty. I had no inferiorities. Why, I wasn’t even ashamed of my skinny legs! This was the real thing!

A haze filled the room. The electric light began to move. Then two bulbs appeared. The faces of the other boys grew dim. How sick I felt. I staggered to the bathroom. Shouldn’t have drunk so much or so fast. But I knew how to handle it now. I’d drink like a gentleman after this.

And so I met John Barleycorn. The grand fellow who at my call made me “a hale fellow, well met,” who gave me such a fine voice, as we sang, “Hail, hail, the gang’s all here,” and “Sweet Adeline,” who gave me freedom from fear and feelings of inferiority. Good old John! He was my pal, all right.

      Final exams of my Senior year and I may somehow graduate. I would never have tried, but mother counts on it so. A case of measles saved me from being kicked out during my Sophomore year.

But the end is in sight. My last exam and an easy one. I gaze at the board with its questions. Can’t remember the answer to the first. I’ll try the second. No soap there. I don’t seem to remember anything. I concentrate on one of the questions. I don’t seem to be able to keep my mind on what I am doing. I get uneasy. If I don’t get started soon, I won’t have time to finish. No use. I can’t think.

I leave the room, which the honor system allows. I go to my room. I pour out half a tumbler of grain alcohol and fill it with ginger ale. Now back to the exam. My pen moves rapidly. I know enough of the answers to get by. Good old John Barleycorn! He can certainly be depended on. What a wonderful power he has over the mind! He has given me my diploma!

Underweight! How I hate that word. Three attempts to enlist in the service, and three failures because of being skinny. True, I have recently recovered from pneumonia and have an alibi, but my friends are in the war or going, and I am not. I visit a friend who is awaiting orders. The atmosphere of “eat, drink, and be merry” prevails and I absorb it. I drink a lot every night. I can hold a lot now, more than the others.

I am examined for the draft and pass the physical test. I am to go to camp on November 13th. The Armistice is signed on the 11th and the draft is called off. Never in the service! The war leaves me with a pair of blankets, a toilet kit, a sweater knit by my sister, and a still greater inferiority.

      It is ten o’clock of a Saturday night. I am working hard on the books of a subsidiary company of a large corporation. I have had experience in selling, collecting, and accounting, and am on my way up the ladder.

Then the crack-up. Cotton struck the skids and collections went cold. A twenty three million dollar surplus wiped out. Offices closed up and workers discharged. I, and the books of my division, have been transferred to the head office. I have no assistance and am working nights, Saturdays and Sundays. My salary has been cut. My wife and new baby are fortunately staying with relatives. I feel exhausted. The doctor has told me that if I don’t give up inside work, I’ll have tuberculosis. But what am I to do? I have a family to support and have no time to be looking for another job.

I reach for the bottle which I just got from George, the elevator boy.

I am a traveling salesman. The day is over and business has been not so good. I’ll go to bed. I wish I were home with the family and not in this dingy hotel.

Well-well-look who’s here! Good old Charlie! It’s great to see him. How’s the boy? A drink? You bet your life! We buy a gallon of “corn” because it is so cheap. Yet I am fairly steady when I go to bed.

Morning comes. I feel horribly. A little drink will put me on my feet. But it takes others to keep me there.

I become a teacher in a boy’s school. I am happy in my work. I like the boys and we have lots of fun, in class and out.

      The doctors bills are heavy and the bank account is low. My wife’s parents come to our assistance. I am filled with hurt pride and self-pity. I seem to get no sympathy for my illness and have no appreciation of the love behind the gift.

I call the bootlegger and fill up my charred keg. But I do not wait for the charred keg to work. I get drunk. My wife is extremely unhappy. Her father comes to sit with me. He never says an unkind word. He is a real friend but I do not appreciate him.

We are staying with my wife’s father. Her mother is in critical condition at a hospital. I cannot sleep. I must get myself together. I sneak down stairs and get a bottle of whiskey from the cellaret. I pour drinks down my throat. My father-in-law appears. “Have a drink?” I ask. He makes no reply, and hardly seems to see me. His wife dies that night.

Mother has been dying of cancer for a long time. She is near the end now and is in a hospital. I have been drinking a lot, but never get drunk. Mother must never know. I see her about to go.

I return to the hotel where I am staying and get gin from the bellboy. I drink and go to bed; I take a few the next morning and go see my mother once more. I cannot stand it. I go back to the hotel and get more gin. I drink steadily. I come to at three in the morning. The indescribable torture has me again. I turn on the light. I must get out of the room or I shall jump out of the window. I walk miles. No use. I go to the hospital, where I have made friends with the night superintendent. She puts me to bed and gives me a hypodermic.

I am at the hospital to see my wife. We have an-

other child. But she is not glad to see me. I have been drinking while the baby was arriving. Her father stays with her.

It is a cold, bleak day in November. I have fought hard to stop drinking. Each battle has ended in defeat. I tell my wife I cannot stop drinking. She begs me to go to a hospital for alcoholics which has been recommended. I say I will go. She makes the arrangements, but I will not go. I’ll do it all myself. This time I’m off of it for good. I’ll just take a few beers now and then.

It is the last day of the following October, a dark, rainy morning. I come to in a pile of hay in a barn. I look for liquor and can’t find any. I wander to a stable and drink five bottles of beer. I must get some liquor. Suddenly I feel hopeless, unable to go on. I go home. My wife is in the living room. She had looked for me last evening after I left the car and wandered off into the night. She had looked for me this morning. She has reached the end of her rope. There is no use trying any more, for there is nothing to try. “Don’t say anything,” I say to her. “I am going to do something.”

I am in the hospital for alcoholics. I am an alcoholic. The insane asylum lies ahead. Could I have myself locked up at home? One more foolish idea. I might go out West on a ranch where I couldn’t get anything to drink. I might do that. Another foolish idea. I wish I were dead, as I have often wished before. I am too yellow to kill myself.

Four alcoholics play bridge in a smoke-filled room. Anything to get my mind from myself. The game is over and the other three leave. I start to clean up the

debris. One man comes back, closing the door behind him.

He looks at me. “You think you are hopeless, don’t you?” he asks.

“I know it,” I reply.

“Well, you’re not,” says the man. “There are men on the streets of New York today who were worse than you, and they don’t drink anymore.”

“What are you doing here then?” I ask.

“I went out of here nine days ago saying that I was going to be honest, and I wasn’t,” he answers.

A fanatic, I thought to myself, but I was polite. “What is it?” I enquire.

Then he asks me if I believe in a power greater than myself, whether I call that power God, Allah, Confucius, Prime Cause, Divine Mind, or any other name. I told him that I believe in electricity and other forces of nature, but as for a God, if there is one, He has never done anything for me. Then he asks me if I am willing to right all the wrongs I have ever done to anyone, no matter how wrong I thought the others were. Am I willing to be honest with myself about myself and tell someone about myself, and am I willing to think of other people, of their needs instead of myself, in order to get rid of the drink problem?

“I’ll do anything,” I reply.

“Then all of your troubles are over,” says the man and leaves the room. The man is in bad mental shape certainly. I pick up a book and try to read, but cannot concentrate. I get in bed and turn out the light. But I cannot sleep. Suddenly a thought comes. Can all the worthwhile people I have known be wrong about God? Then I find myself thinking about myself,

and a few things that I had wanted to forget. I begin to see I am not the person I had thought myself, that I had judged myself by comparing myself to others, and always to my own advantage. It is a shock.

Then comes a thought that is like A Voice. “Who are you to say there is no God?” It rings in my head; I can’t get rid of it.

I get out of bed and go to the man’s room. He is reading. “I must ask you a question,” I say to the man. “How does prayer fit into this thing?”

“Well,” he answers, “you’ve probably tried praying like I have. When you’ve been in a jam you’ve said, ‘God, please do this or that,’ and if it turned out your way that was the last of it, and if it didn’t you’ve said ‘There isn’t any God’ or ‘He doesn’t do anything for me’. Is that right?”

“Yes” I reply.

“That isn’t the way” he continued. “The thing I do is to say ‘God here I am and here are all my troubles. I’ve made a mess of things and can’t do anything about it. You take me, and all my troubles, and do anything you want with me.’ Does that answer your question?”

“Yes, it does” I answer. I return to bed. It doesn’t make sense. Suddenly I feel a wave of utter hopelessness sweep over me. I am in the bottom of hell. And there a tremendous hope is born. It might be true.

I tumble out of bed onto my knees. I know not what I say. But slowly a great peace comes to me. I believe in God. I crawl back into bed and sleep like a child.

Some men and women come to visit my friend of the night before. He invites me to meet them. They are a joyous crowd. I have never seen people that joyous

before. We talk. I tell them of the peace, and that I believe in God. I think of my wife. I must write her. One girl suggests that I phone her. What a wonderful idea!

My wife hears my voice and knows I have found the answer to life. She comes to New York. I get out of the hospital and we visit some of these new-found friends.

I am home again. I have lost the fellowship. Those that understand me are far away. The same old problems and worries surround me. Members of my family annoy me. Nothing seems to be working out right. I am blue and unhappy. Maybe a drink—I put on my hat and dash off in the car.

Get into the lives of other people, is one thing the fellows in New York had said. I go to see a man I had been asked to visit and tell him my story. I feel much better! I have forgotten about a drink.

I am on a train, headed for a city. I have left my wife at home, sick, and I have been unkind to her in leaving. I am very unhappy. Maybe a few drinks when I get to the city will help. A great fear seizes me. I talk to the stranger in the seat beside me. The fear and the insane idea is taken away.

Things are not going so well at home. I am learning that I cannot have my own way as I used to. I blame my wife and children. Anger possesses me, anger such as I have never felt before. I will not stand for it. I pack my bag and leave. I stay with understanding friends.

I see where I have been wrong in some respects. I do not feel angry any more. I return home and say I am sorry for my wrong. I am quiet again. But I have

not seen yet that I should do some constructive acts of love without expecting any return. I shall learn this after some more explosions.

I am blue again. I want to sell the place and move away. I want to get where I can find some alcoholics to help, and where I can have some fellowship. A man calls me on the phone. Will I take a young fellow who has been drinking for two weeks to live with me? Soon I have others who are alcoholics and some who have other problems.

I begin to play God. I feel that I can fix them all. I do not fix anyone, but I am getting part of a tremendous education and I have made some new friends.

Nothing is right. Finances are in bad shape. I must find a way to make some money. The family seems to think of nothing but spending. People annoy me. I try to read. I try to pray. Gloom surrounds me. Why has God left me? I mope around the house. I will not go out and I will not enter into anything. What is the matter? I cannot understand. I will not be that way.

I’ll get drunk! It is a cold-blooded idea. It is premeditated. I fix up a little apartment over the garage with books and drinking water. I am going to town to get some liquor and food. I shall not drink until I get back to the apartment. Then I shall lock myself in and read. And as I read, I shall take little drinks at long intervals. I shall get myself “mellow” and stay that way.

I get in the car and drive off. Halfway down the driveway a thought strikes me. I’ll be honest anyway. I’ll tell my wife what I am going to do. I back up to the door and go into the house. I call my wife into a

room where we can talk privately. I tell her quietly what I intend to do. She says nothing. She does not get excited. She maintains a perfect calm.

When I am through speaking, the whole idea has become absurd. Not a trace of fear is in me. I laugh at the insanity of it. We talk of other things. Strength has come from weakness.

I cannot see the cause of this temptation now. But I am to learn later that it began with my desire for material success becoming greater than my interest in the welfare of my fellow man. I learn more of that foundation stone of character, which is honesty. I learn that when we act upon the highest conception of honesty which is given us, our sense of honesty becomes more acute.

I learn that honesty is truth, and the truth shall make us free!


\biblebook{Housewife}

 THE HOUSEWIFE WHO DRANK AT HOME

She hid her bottles in the clothes hampers and in

the dresser drawers. She realized what she was be-

coming. In A.A., she discovered she had lost nothing

and had found everything.

       MY STORY HAPPENS to be a particular kind of woman’s story; the story of the woman who drinks at home. I had to be at home. I had two babies. When alcohol took me over, my bar was my kitchen, my living room, my bedroom, the back bathroom, and the two hampers.

At one time, the admission that I was and am an alcoholic meant shame, defeat, and failure to me. But in the light of the new understanding that I have found in A.A., I have been able to interpret that defeat, and that failure, and that shame, as seeds of victory. Because it was only through feeling defeat and feeling failure, the inability to cope with my life and with alcohol, that I was able to surrender and accept the fact that I had this disease, and that I had to learn to live again without alcohol.

I was never a very heavy social drinker. But during a period of particular stress and strain about thirteen years ago, I resorted to alcohol in my home, alone, as a means of temporary release, as a means of getting a little extra sleep.

       I had problems. We all have them, and I thought a little brandy or a little wine now and then could certainly hurt no one. I don’t believe, when I started, that I even had in mind the thought that I was drinking. I had to sleep, I had to clear my mind and free it from worry, and I had to relax. But from one or two drinks of an afternoon or evening, my intake mounted, and mounted fast. It wasn’t long before I was drinking all day. I had to have that wine. The only incentive that I had, toward the end, for getting dressed in the morning was to get out and get supplies to help me get my day started. But the only thing that got started was my drinking.

I should have realized that alcohol was getting hold of me when I started to become secretive in my drinking. I began to have to have supplies on hand for the people who “might come in.” And of course a half empty bottle wasn’t worth keeping, so I finished it up and naturally had to get more in right away for the people who “might come in unexpectedly.” But I was always the unexpected person who had to finish the bottle. I couldn’t go to one wine store and look the man honestly in the face and buy a bottle, as I used to do when I had parties and entertained and did normal drinking. I had to give him a story and ask him the same question over and over again, “Well, now, how many will that bottle serve?” I wanted him to be sure that I wasn’t the one who was going to drink the whole bottle.

I had to hide, as a great many people in A.A. have had to do. I did my hiding in the hampers and in my dresser drawers. When we begin to do things like that with alcohol, something’s gone wrong. I needed it,

and I knew I was drinking too much, but I wasn’t conscious of the fact that I should stop. I kept on. My home at that time was a place to mill around in. I wandered from room to room, thinking, drinking, drinking, thinking. And the mops would get out, the vacuum would get out, everything would get out, but nothing would get done. Toward five o’clock, helter-skelter, I’d get everything put away and try to get supper on the table, and after supper I’d finish the job up and knock myself out.

I never knew which came first, the thinking or the drinking. If I could only stop thinking, I wouldn’t drink. If I could only stop drinking, maybe I wouldn’t think. But they were all mixed up together, and I was all mixed up inside. And yet I had to have that drink. You know the deteriorating effects, the disintegrating effects of chronic wine-drinking. I cared nothing about my personal appearance. I didn’t care nothing about my personal appearance. I didn’t care what I looked like, I didn’t care what I did. To me, taking a bath was just being in a place with a bottle where I could drink in privacy. I had to have it with me at night, in case I woke up and needed that drink.

How I ran my home, I don’t know. I went on, realizing what I was becoming, hating myself for it, bitter, blaming life, blaming everything else but the fact that I should turn about and do something about my drinking. Finally I didn’t care, I was beyond caring. I just wanted to live to a certain age, carry through with what I felt was my job with the children, and after that—no matter. Half a mother was better than no mother at all.

I needed that alcohol. I couldn’t live without it. I couldn’t do anything without it. But there came a

point when I could no longer live with it. And that came after a three-weeks’ illness of my son. The doctor prescribed brandy for the boy to help him through the night when he coughed, a teaspoon of brandy. Well, of course that was all I needed—to switch from wine to brandy for three weeks. I knew nothing about alcoholism or the D.T.’s, but when I woke up on this last morning of my son’s illness, I taped the keyhole on my door because “everyone was out there.” I paced back and forth in the apartment with the cold sweats. I screamed on the telephone for my mother to get up there; something was going to happen; I didn’t know what, but if she didn’t get there quick, I’d split wide open. I called my husband up and told him to come home.

After that I sat for a week, a body in a chair, a mind off in space. I thought the two would never get together. I knew that alcohol and I had to part. I couldn’t live with it any more. And yet, how was I going to live without it? I didn’t know. I was bitter, living in hate. The very person who stood with me through it all and has been my greatest help was the person that I turned against, my husband. I also turned against my family, my mother. The people who would have come to help me were just the people I would have nothing to do with.

Nevertheless, I began to try to live without alcohol. But I only succeeded in fighting it. And believe me, an alcoholic cannot fight alcohol. I had all kinds of reasons for my drinking. I had problems. I was a woman, tied to my home. What I needed was a change, mental relaxation, getting out and doing something. I thought that was my answer. I said to

my husband, “I’m going to try, every free moment that I have, to get interested in something outside, get myself over this rut I’m in.” I thought I was going out of my mind. If I didn’t have a drink, I had to do something.

I became one of the most active women in the community, what with P.T.A., other community organizations and drives. I’d go into an organization, and it wasn’t long before I was on the committee, then I was chairman of the committee; and if I was in a group, I’d soon be treasurer or secretary of the group. But I wasn’t happy. I became a Jekyll-and-Hyde person. As long as I worked, as long as I got out, I didn’t drink. But I had to get back to that first drink somehow. And when I took that first drink, I was off on the usual merry-go-round. And it was my home that suffered. My husband, my children saw the other side of me. So that didn’t work.

I figured I’d be all right if I could find something I liked to do. So when the children were in school from nine to three, I started up a nice little business and was fairly successful in it. But not happy. Because I found that everything I turned to became a substitute for drink. And when all of life is a substitute for drink, there’s no happiness, no peace. I still had to drink; I still needed that drink. Mere cessation from drinking is not enough for an alcoholic while the need for that drink goes on. I switched to beer. I had always hated beer, but now I grew to love it, bottle after bottle of it, warm or cold. So that wasn’t my answer either.

I went to my doctor again. He knew what I was

doing, how I was trying. I said, “I can’t find my middle road in life. I can’t find it. It’s either all work, or I drink.” He said, “Why don’t you try Alcoholics Anonymous?” I was willing to try anything. I was licked. For the second time, I was licked. The first time was when I knew I couldn’t live with alcohol. But this second time, I found I couldn’t live normally without it, and I was licked worse than ever.

The fellowship I found in A.A. enabled me to face my problem honestly and squarely. I couldn’t do it among my relatives, I couldn’t do it among my friends. No one likes to admit that they’re a drunk, that they can’t control this thing. But when we come into A.A., we can face our problem honestly and openly. I went to closed meetings and open meetings. And I took everything that A.A. had to give me. Easy does it, first things first, one day at a time. It was at that point that I reached surrender. I heard one very ill woman say that she didn’t believe in the surrender part of the A.A. program. My heavens! Surrender to me has meant the ability to run my home, to face my responsibilities as they should be faced, to take life as it comes to me day by day, and work my problems out. That’s what surrender has meant to me. I surrendered once to the bottle, and I couldn’t do these things. Since I gave my will over to A.A., whatever A.A. has wanted of me I’ve tried to do to the best of my ability. When I’m asked to go out on a call, I go. I’m not going; A.A. is leading me there. A.A. gives us alcoholics direction into a way of life without the need for alcohol. That life for me is lived one day at a time, letting the problems of the future rest with the future.

When the time comes to solve them, God will give me strength for that day.

I had been brought up to believe in God, but I know that until I found this A.A. program, I had never found or known faith in the reality of God, the reality of His power that is now with me in everything I do.


\chapter{Alcoholism}

BB 'More About Alcoholism'; Pages 30-43

\section{Unwilling}

MOST OF us have been unwilling to admit we were real alcoholics. 
No person likes to think he is bodily and mentally different from his fellows.  
Therefore, it is not surprising that our drinking careers 
have been characterized by countless vain attempts to prove we could drink like other people.  
The idea that somehow, someday 
he will control and enjoy his drinking is the great obsession of every abnormal drinker.  
The persistence of this illusion is astonishing.  
Many pursue it into the gates of insanity or death.

We learned that we had to fully concede to our innermost selves that we were alcoholics. 
This is the first step in recovery. 
The delusion that we are like other people, or presently may be, has to be smashed.


\section{A Progressive Illness}

We alcoholics are men and women who have lost the ability to control our drinking. 
We know that no real alcoholic ever recovers control. 
All of us felt at times that we were regaining control, 
but such intervals-usually brief-were inevitably followed by still less control, 
which led in time to pitiful and incomprehensible demoralization. 
We are convinced to a man that alcoholics of our type are in the grip of a progressive illness. 
Over any considerable period we get worse, never better.

We are like men who have lost their legs; they never grow new ones. 
Neither does there appear to be any kind of treatment which will make alcoholics of our kind like other men. 
We have tried every imaginable remedy. 
In some instances there has been brief recovery, followed always by a still worse relapse. 
Physicians who are familiar with alcoholism 
agree there is no such thing as making a normal drinker out of an alcoholic. 
Science may one day accomplish this, but it hasn't done so yet.


\section{Failure of The Will}

Despite all we can say, many who are real alcoholics are not going to believe they are in that class. 
By every form of self-deception and experimentation, 
they will try to prove themselves exceptions to the rule, therefore nonalcoholic. 
If anyone who is showing inability to control his drinking 
can do the right-about-face and drink like a gentleman, our hats are off to him. 
Heaven knows, we have tried hard enough and long enough to drink like other people!

Here are some of the methods we have tried: 
Drinking beer only, 
limiting the number of drinks, 
never drinking alone, 
never drinking in the morning, 
drinking only at home, 
never having it in the house, 
never drinking during business hours, 
drinking only at parties, 
switching from scotch to brandy, 
drinking only natural wines, 
agreeing to resign if ever drunk on the job, 
taking a trip, 
not taking a trip, 
swearing off forever (with and without a solemn oath), 
taking more physical exercise, 
reading inspirational books, 
going to health farms and sanitariums, 
accepting voluntary commitment to asylums 
-we could increase the list ad infinitum. 


\section{Drinking Experiments}

We do not like to pronounce any individual as an alcoholic, but you can quickly diagnose yourself. 
Step over to the nearest bar-room and try some controlled drinking. 
Try to drink and stop abruptly. 
Try it more than once. 
It will not take long for you to decide, if you are honest with yourself about it. 
It may be worth a bad case of jitters if you get a full knowledge of your condition.
Though there is no way of proving it, 
we believe that early in our drinking careers most of us could have stopped drinking. 
But the difficulty is that few alcoholics have enough desire to stop while there is yet time. 
We have heard of a few instances where people, who showed definite signs of alcoholism, 
were able to stop for a long period because of an overpowering desire to do so. 
Here is one.


\section{A Tragic Story}

A man of thirty was doing a great deal of spree drinking. 
He was very nervous in the morning after these bouts and quieted himself with more liquor. 
He was ambitious to succeed in business, 
but saw that he would get nowhere if he drank at all. 
Once he started, he had no control whatever. 
He made up his mind that until he had been successful in business and had retired, 
he would not touch another drop. 
An exceptional man, he remained bone dry for twenty-five years 
and retired at the age of fifty-five, 
after a successful and happy business career. 
Then he fell victim to a belief which practically every alcoholic has
-that his long period of sobriety and self-discipline had qualified him to drink as other men. 
Out came his carpet slippers and a bottle. 
In two months he was in a hospital, puzzled and humiliated. 
He tried to regulate his drinking for a while, 
making several trips to the hospital meantime. 
Then, gathering all his forces, he attempted to stop altogether and found he could not. 
Every means of solving his problem which money could buy was at his disposal. 
Every attempt failed. 
Though a robust man at retirement, 
he went to pieces quickly and was dead within four years.

This case contains a powerful lesson. 
Most of us have believed that if we remained sober for a long stretch, 
we could thereafter drink normally. 
But here is a man who at fifty-five years found he was just where he had left off at thirty. 
We have seen the truth demonstrated again and again: 
"Once an alcoholic, always an alcoholic." 
Commencing to drink after a period of sobriety, we are in a short time as bad as ever. 
If we are planning to stop drinking, there must be no reservation of any kind, 
nor any lurking notion that someday we will be immune to alcohol.

Young people may be encouraged by this man's experience to think that they can stop, 
as he did, on their own will power. 
We doubt if many of them can do it, because none will really want to stop, 
and hardly one of them, because of the peculiar mental twist already acquired, 
will find he can win out. 
Several of our crowd, men of thirty or less, had been drinking only a few years, 
but they found themselves as helpless as those who had been drinking twenty years.


\section{}

To be gravely affected, one does not necessarily have to drink a long time 
nor take the quantities some of us have. 
This is particularly true of women. 
Potential female alcoholics often turn into the real thing 
and are gone beyond recall in a few years. 
Certain drinkers, who would be greatly insulted if called alcoholics, 
are astonished at their inability to stop. 
We, who are familiar with the symptoms, 
see large numbers of potential alcoholics among young people everywhere. 
But try and get them to see it!  

As we look back, 
we feel we had gone on drinking many years beyond the point where we could quit on our will power. 
If anyone questions whether he has entered this dangerous area, 
let him try leaving liquor alone for one year. 
If he is a real alcoholic and very far advanced, there is scant chance of success. 
In the early days of our drinking we occasionally remained sober for a year or more, 
becoming serious drinkers again later. 
Though you may be able to stop for a considerable period, you may yet be a potential alcoholic. 
We think few, to whom this book will appeal, can stay dry anything like a year. 
Some will be drunk the day after making their resolutions; 
most of them within a few weeks.


\section{Relapse}

For those who are unable to drink moderately the question is how to stop altogether. 
We are assuming, of course, that the reader desires to stop. 
Whether such a person can quit upon a nonspiritual basis 
depends upon the extent to which he has already lost the power to choose whether he will drink or not. 
Many of us felt that we had plenty of character. 
There was a tremendous urge to cease forever. 
Yet we found it impossible. 
This is the baffling feature of alcoholism as we know it-this utter inability to leave it alone, 
no matter how great the necessity or the wish.

How then shall we help our readers determine, to their own satisfaction, 
whether they are one of us? 
The experiment of quitting for a period of time will be helpful, 
but we think we can render an even greater service to alcoholic sufferers 
and perhaps to the medical fraternity. 
So we shall describe some of the mental states that precede a relapse into drinking, 
for obviously this is the crux of the problem.

What sort of thinking dominates an alcoholic who repeats time after time the desperate experiment of the first drink? 
Friends who have reasoned with him after a spree which has brought him to the point of divorce or bankruptcy 
are mystified when he walks directly into a saloon. 
Why does he? 
Of what is he thinking?


\section{Jim}

Our first example is a friend we shall call Jim. 
This man has a charming wife and family. 
He inherited a lucrative automobile agency. 
He had a commendable World War record. 
He is a good salesman. 
Everybody likes him. 
He is an intelligent man, normal so far as we can see, 
except for a nervous disposition. 
He did no drinking until he was thirty-five. 
In a few years he became so violent when intoxicated that he had to be committed. 
On leaving the asylum he came into contact with us.

We told him what we knew of alcoholism and the answer we had found. 
He made a beginning. 
His family was re-assembled, 
and he began to work as a salesman for the business he had lost through drinking. 
All went well for a time, but he failed to enlarge his spiritual life. 
To his consternation, he found himself drunk half a dozen times in rapid succession. 
On each of these occasions we worked with him, reviewing carefully what had happened. 
He agreed he was a real alcoholic and in a serious condition. 
He knew he faced another trip to the asylum if he kept on. 
Moreover, he would lose his family for whom he had a deep affection.

Yet he got drunk again. 
We asked him to tell us exactly how it happened. 
This is his story: 
\begin{quote}
"I came to work on Tuesday morning. 
I remember I felt irritated that I had to be a salesman for a concern I once owned. 
I had a few words with the boss, but nothing serious. 
Then I decided to drive into the country and see one of my prospects for a car. 
On the way I felt hungry so I stopped at a roadside place where they have a bar. 
I had no intention of drinking. 
I just thought I would get a sandwich. 
I also had the notion that I might find a customer for a car at this place, 
which was familiar for I had been going to it for years. 
I had eaten there many times during the months I was sober. 
I sat down at a table and ordered a sandwich and a glass of milk. 
Still no thought of drinking. 
I ordered another sandwich and decided to have another glass of milk."
\end{quote}

\begin{quote}
"Suddenly the thought crossed my mind 
that if I were to put an ounce of whiskey in my milk 
it couldn't hurt me on a full stomach. 
I ordered a whiskey and poured it into the milk. 
I vaguely sensed I was not being any too smart, 
but felt reassured as I was taking the whiskey on a full stomach. 
The experiment went so well that I ordered another whiskey 
and poured it into more milk. 
That didn't seem to bother me so I tried another."
\end{quote}

Thus started one more journey to the asylum for Jim. 
Here was the threat of commitment, 
the loss of family and position, 
to say nothing of that intense mental and physical suffering 
which drinking always caused him. 
He had much knowledge about himself as an alcoholic. 
Yet all reasons for not drinking were easily pushed aside 
in favor of the foolish idea that he could take whiskey 
if only he mixed it with milk!


\section{Insanity}

Whatever the precise definition of the word may be, 
we call this plain insanity. 
How can such a lack of proportion, 
of the ability to think straight, 
be called anything else?

You may think this an extreme case. 
To us it is not far-fetched, 
for this kind of thinking has been characteristic of every single one of us. 
We have sometimes reflected more than Jim did upon the consequences. 
But there was always the curious mental phenomenon that parallel with our sound reasoning 
there inevitably ran some insanely trivial excuse for taking the first drink. 
Our sound reasoning failed to hold us in check. 
The insane idea won out. 
Next day we would ask ourselves, in all earnestness and sincerity, how it could have happened.

In some circumstances we have gone out deliberately to get drunk, 
feeling ourselves justified by nervousness, anger, worry, depression, jealousy or the like. 
But even in this type of beginning we are obliged to admit 
that our justification for a spree was insanely insufficient in the light of what always happened. 
We now see that when we began to drink deliberately, instead of casually, 
there was little serious or effective thought during the period of premeditation 
of what the terrific consequences might be.


\section{Jaywalkers}

Our behavior is as absurd and incomprehensible with respect to the first drink 
as that of an individual with a passion, say, for jay-walking. 
He gets a thrill out of skipping in front of fast-moving vehicles. 
He enjoys himself for a few years in spite of friendly warnings. 
Up to this point you would label him as a foolish chap having queer ideas of fun. 
Luck then deserts him and he is slightly injured several times in succession. 
You would expect him, if he were normal, to cut it out. 
Presently he is hit again and this time has a fractured skull. 
Within a week after leaving the hospital a fast-moving trolley car breaks his arm. 
He tells you he has decided to stop jay-walking for good, 
but in a few weeks he breaks both legs.

On through the years this conduct continues, 
accompanied by his continual promises to be careful 
or to keep off the streets altogether. 
Finally, he can no longer work, 
his wife gets a divorce and he is held up to ridicule. 
He tries every known means to get the jay-walking idea out of his head. 
He shuts himself up in an asylum, hoping to mend his ways. 
But the day he comes out he races in front of a fire engine, 
which breaks his back. 
Such a man would be crazy, wouldn't he?

You may think our illustration is too ridiculous. 
But is it? 
We, who have been through the wringer, 
have to admit if we substituted alcoholism for jay-walking, 
the illustration would fit us exactly. 
However intelligent we may have been in other respects, 
where alcohol has been involved, we have been strangely insane. 
It's strong language-but isn't it true?


\section{Halfway There}

Some of you are thinking: 
\begin{quote}
"Yes, what you tell us is true, but it doesn't fully apply. 
We admit we have some of these symptoms, 
but we have not gone to the extremes you fellows did, 
nor are we likely to, 
for we understand ourselves so well after what you have told us 
that such things cannot happen again. 
We have not lost everything in life through drinking and we certainly do not intend to. 
Thanks for the information."
\end{quote}

That may be true of certain nonalcoholic people who, 
though drinking foolishly and heavily at the present time, 
are able to stop or moderate, 
because their brains and bodies have not been damaged as ours were. 
But the actual or potential alcoholic, with hardly an exception, 
will be absolutely unable to stop drinking on the basis of self-knowledge. 
This is a point we wish to emphasize and re-emphasize, 
to smash home upon our alcoholic readers 
as it has been revealed to us out of bitter experience. 
Let us take another illustration.


\section{Fred}

Fred is partner in a well known accounting firm. 
His income is good, he has a fine home, 
is happily married and the father of promising children of college age. 
He has so attractive a personality that he makes friends with everyone. 
If ever there was a successful business man, it is Fred. 
To all appearance he is a stable, well balanced individual. 
Yet, he is alcoholic. 
We first saw Fred about a year ago in a hospital where he had gone to recover from a bad case of jitters. 
It was his first experience of this kind, and he was much ashamed of it. 
Far from admitting he was an alcoholic, he told himself he came to the hospital to rest his nerves. 
The doctor intimated strongly that he might be worse than he realized. 
For a few days he was depressed about his condition. 
He made up his mind to quit drinking altogether. 
It never occurred to him that perhaps he could not do so, in spite of his character and standing. 
Fred would not believe himself an alcoholic, much less accept a spiritual remedy for his problem. 
We told him what we knew about alcoholism. 
He was interested and conceded that he had some of the symptoms, 
but he was a long way from admitting that he could do nothing about it himself. 
He was positive that this humiliating experience, 
plus the knowledge he had acquired, 
would keep him sober the rest of his life. 
Self-knowledge would fix it.

We heard no more of Fred for a while. 
One day we were told that he was back in the hospital. 
This time he was quite shaky. 
He soon indicated he was anxious to see us. 
The story he told is most instructive, 
for here was a chap absolutely convinced he had to stop drinking, 
who had no excuse for drinking, 
who exhibited splendid judgment and determination in all his other concerns, 
yet was flat on his back nevertheless.


\section{Fred's Testimony}

Let him tell you about it: 

\begin{quote}
"I was much impressed with what you fellows said about alcoholism, 
and I frankly did not believe it would be possible for me to drink again. 
I rather appreciated your ideas about the subtle insanity which precedes the first drink, 
but I was confident it could not happen to me after what I had learned. 
I reasoned I was not so far advanced as most of you fellows, 
that I had been usually successful in licking my other personal problems, 
and that I would therefore be successful where you men failed. 
I felt I had every right to be self-confident, 
that it would be only a matter of exercising my will power and keeping on guard.
\end{quote}

\begin{quote}
"In this frame of mind, I went about my business and for a time all was well. 
I had no trouble refusing drinks, 
and began to wonder if I had not been making too hard work of a simple matter. 
One day I went to Washington to present some accounting evidence to a government bureau. 
I had been out of town before during this particular dry spell, 
so there was nothing new about that. 
Physically, I felt fine. 
Neither did I have any pressing problems or worries. 
My business came off well, I was pleased and knew my partners would be too. 
It was the end of a perfect day, not a cloud on the horizon."
\end{quote}


\begin{quote}
"I went to my hotel and leisurely dressed for dinner. 
As I crossed the threshold of the dining room, 
the thought came to mind that it would be nice to have a couple of cocktails with dinner. 
That was all. 
Nothing more. 
I ordered a cocktail and my meal. 
Then I ordered another cocktail. 
After dinner I decided to take a walk. 
When I returned to the hotel it struck me a highball would be fine before going to bed, 
so I stepped into the bar and had one. 
I remember having several more that night and plenty next morning. 
I have a shadowy recollection of being in an airplane bound for New York 
and of finding a friendly taxicab driver at the landing field instead of my wife. 
The driver escorted me about for several days. 
I know little of where I went or what I said and did. 
Then came the hospital with unbearable mental and physical suffering."
\end{quote}


\textbf{Admittance}

\begin{quote}
"As soon as I regained my ability to think, I went carefully over that evening in Washington. 
Not only had I been off guard, I had made no fight whatever against the first drink. 
This time I had not thought of the consequences at all. 
I had commenced to drink as carelessly as though the cocktails were ginger ale. 
I now remembered what my alcoholic friends had told me, 
how they prophesied that if I had an alcoholic mind, 
the time and place would come-I would drink again. 
They had said that though I did raise a defense, 
it would one day give way before some trivial reason for having a drink. 
Well, just that did happen and more, 
for what I had learned of alcoholism did not occur to me at all. 
I knew from that moment that I had an alcoholic mind. 
I saw that will power and self-knowledge would not help in those strange mental blank spots. 
I had never been able to understand people who said that a problem had them hopelessly defeated. 
I knew then. 
It was a crushing blow."
\end{quote}

\textbf{Powerlessness}

\begin{quote}
"Two of the members of Alcoholics Anonymous came to see me. 
They grinned, which I didn't like so much, 
and then asked me if I thought myself alcoholic and if I were really licked this time. 
I had to concede both propositions. 
They piled on me heaps of evidence to the effect that an alcoholic mentality, 
such as I had exhibited in Washington, was a hopeless condition. 
They cited cases out of their own experience by the dozen. 
This process snuffed out the last flicker of conviction that I could do the job myself."
\end{quote}

\textbf{Turning Over To God}

\begin{quote}
"Then they outlined the spiritual answer and program of action which a hundred of them had followed successfully. 
Though I had been only a nominal churchman, their proposals were not, intellectually, hard to swallow. 
But the program of action, though entirely sensible, was pretty drastic. 
It meant I would have to throw several lifelong conceptions out of the window. 
That was not easy. 
But the moment I made up my mind to go through with the process, 
I had the curious feeling that my alcoholic condition was relieved, 
as in fact it proved to be.
\end{quote}

\textbf{Radical Transformation}

\begin{quote}
"Quite as important was the discovery that spiritual principles would solve all my problems. 
I have since been brought into a way of living infinitely more satisfying and, 
I hope, more useful than the life I lived before. 
My old manner of life was by no means a bad one, 
but I would not exchange its best moments for the worst I have now. 
I would not go back to it even if I could."
\end{quote}
Fred's story speaks for itself. 
We hope it strikes home to thousands like him. 
He had felt only the first nip of the wringer. 
Most alcoholics have to be pretty badly mangled before they really commence to solve their problems.


\section{Spiritual Experience}

Many doctors and psychiatrists agree with our conclusions. 
One of these men, staff member of a world-renowned hospital, recently made this statement to some of us: 
\begin{quote}
    "What you say about the general hopelessness of the average alcoholic's plight is, in my opinion, correct.  
    As to two of you men, whose stories I have heard, there is no doubt in my mind that you were 100\% hopeless, apart from divine help.  
    Had you offered yourselves as patients at this hospital, I would not have taken you, if I had been able to avoid it. 
    People like you are too heartbreaking. 
    Though not a religious person, I have profound respect for the spiritual approach in such cases as yours. 
    For most cases, there is virtually no other solution."
\end{quote}

Once more: The alcoholic at certain times has no effective mental defense against the first drink. 
Except in a few rare cases, neither he nor any other human being can provide such a defense. 
His defense must come from a Higher Power.

